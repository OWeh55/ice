\nsection{Geometrische Transformationen}{geoTrafo}

\subsection{Die Klasse Trafo}
\label{Trafo}
\hypertarget{Trafo}{}

Projektive und affine Transformationen werden durch Objekte der
Klasse $Trafo$ beschrieben. Dabei wird intern mit homogenen
Koordinaten gerechnet und die Transformation durch die 
entsprechende Transformations-Matrix beschrieben.
Es existieren Methoden, die das Zusammensetzen einer Transformation 
aus speziellen affinen Transformationen erlauben. Dabei wird die 
hinzuzufügende Transformations-Matrix jeweils von links mit der
bereits vorhandenen Transformation multipliziert, was der nachfolgenden 
Ausführung der hinzuzufügenden Transformation entspricht.
\subsubsection{Konstruktoren und Destruktoren}
\proch{}{Trafo::Trafo}{void}{geo.h}
\descr{Initialisieren einer 2D-2D-Transformation mit der identischen
Transformation (Einheitsmatrix).}
\proch{}{Trafo::Trafo}{const Trafo \&}{geo.h}
\descr{Initialisieren einer Transformation mit einer gegebenen
Transformation (Copy-Konstruktor).}
\proch{}{Trafo::Trafo}{const Matrix \&}{geo.h}
\descr{Initialisieren einer Transformation anhand einer gegebenen
Transformations-Matrix (Klasse \see{Matrix}).}
\proch{}{Trafo::Trafo}{int dims,int dimd}{geo.h}
\descr{Initialisieren einer Transformation aus dem Raum der Dimension
$dims$ in den Raum der Dimension $dimd$.}
\subsubsection{Operatoren und elementare Funktionen}
\proch{Trafo \&}{Trafo::operator =}{const Trafo \&}{geo.h}
\descr{Zuweisungs-Operator}
\proch{Trafo \&}{Trafo::operator =}{const Matrix \&}{geo.h}
\descr{Umwandlung einer (Transformations-)Matrix in eine 
Transformation und Zuweisung an eine bestehende Transformation.}

\proch{double \&}{Trafo::operator ()}{int x,int y}{geo.h}
\descr{Zugriff auf das Element [i,j] der Transformationsmatrix
\see{Trafo::Tmatrix}.
}

\proch{Trafo}{operator *}{const Trafo \&,const Trafo \&}{geo.h}
\descr{Multiplikation zweier Transformationen = Hintereinander-Ausführung.
Das Ergebnis ist wieder eine Transformation (Klasse Trafo) }
\proch{Vector}{operator *}{const Trafo \&,const Vector \&}{geo.h}
\descr{Multiplikation einer Transformation mit einem $Vector$ = 
Anwendung der Transformation auf den Vektor. Das Ergebnis ist der
transformierte Vektor}
\proch{int}{Trafo::DimS}{void}{geo.h}
\descr{Liefert die Dimension des Ausgangs-Raumes der Transformation.}
\proch{int}{Trafo::DimD}{void}{geo.h}
\descr{Liefert die Dimension des Ziel-Raumes der Transformation.}
\proch{Matrix}{Trafo::Tmatrix}{void}{geo.h}
\descr{Liefert die Transformations-Matrix als Objekt der Klasse
\see{Matrix}.}
\proch{Matrix}{Trafo::Init}{void}{geo.h}
\descr{Setzt die Transformation zurück in den Zustand wie nach dem Anlegen
durch den Konstruktor.}
\proch{int}{Trafo::Invert}{}{geo.h}
\descr{Invertiert die Transformation.}
\proch{Trafo}{Trafo::Inverse}{}{geo.h}
\descr{Gibt die invertierte Transformation zurück.}
\subsubsection{Konstruktion allgemeiner Transformationen aus
elementaren Transformationen}
Die folgendenden Methoden fügen einer bestehenden Transformation eine weitere
Transformation hinzu (im Sinne der Hintereinanderausführung). Um eine bereits verwendete Transformation neu zu belegen, muß diese mit \see{Trafo::Init} zuückgesetzt werden.
\proch{int}{Trafo::Shift}{double x0,double y0}{geo.h}
\descr{Hinzufügen einer Translation um $(x0,y0)$
an eine xD-2D-Transformation.}
\proch{int}{Trafo::Shift}{double x0,double y0,double y0}{geo.h}
\descr{Hinzufügen einer Translation um $(x0,y0,y0)$
an eine xD-3D-Transformation.}
\proch{int}{Trafo::Shift}{vector3d v}{geo.h}
\descr{Hinzufügen einer Translation um $v$
an eine xD-3D-Transformation.}
\proch{int}{Trafo::Shift}{Vector v}{geo.h}
\descr{Hinzufügen einer Translation um $v$
an eine Transformation. Die Dimension des Ziel-Raumes der Transformation
muß gleich der Dimension des Verschiebungsvektors sein.}

\proch{int}{Trafo::Rotate}{double x0,double y0,double phi}{geo.h}
\descr{Hinzufügen einer Rotation um den Punkt $(x0,y0)$ um
den Winkel $phi$ (Bogenmaß) an eine xD-2D-Transformation.}
\proch{int}{Trafo::Rotate}{vector3d point,vector3d dir,double phi}{geo.h}
\descr{Hinzufügen einer Rotation um den Strahl vom Punkt $point$ 
mit der Richtung $dir$ um den Winkel $phi$ (Bogenmaß)an eine 
xD-3D-Transformation.}

\proch{int}{Trafo::RotateX}{double phi}{geo.h}
\proch{int}{Trafo::RotateY}{double phi}{geo.h}
\proch{int}{Trafo::RotateZ}{double phi}{geo.h}
\descr{Hinzufügen einer Rotation um die jeweilige Achse des
Koordinatensystems um den Winkel $phi$ (Bogenmaß)an eine 
xD-3D-Transformation.}

\proch{int}{Trafo::Flip}{int axis}{geo.h}
\descr{Hinzufügen einer Spiegelung, die die angebene Achse invertiert.}

\proch{int}{Trafo::ShearX}{double dxy}{geo.h}
\descr{Hinzufügen einer Scherung in Richtung der X-Achse an eine 
xD-2D-Transformation. Der Parameter ist der Quotient aus der Verschiebung
in X-Richtung und dem Y-Koordinatenwert.}
\proch{int}{Trafo::ShearY}{double dyx}{geo.h}
\descr{Hinzufügen einer Scherung in Richtung der Y-Achse an eine 
xD-2D-Transformation. Der Parameter ist der Quotient aus der Verschiebung
in Y-Richtung und dem X-Koordinatenwert.}
\proch{int}{Trafo::Scale}{double x0, double y0, double f}{geo.h}
\descr{Hinzufügen einer isotropen Skalierung um den 
Punkt $(x0,y0)$ um den Faktor $f$ an eine xD-2D-Transformation.}
\proch{int}{Trafo::Scale}{double x0, double y0, double fx,double fy}{geo.h}
\descr{Hinzufügen einer anisotropen Skalierung um den 
Punkt $(x0,y0)$ um die Faktor $fx$ bzw. $fy$ in X- bzw. Y-Richtung
an eine xD-2D-Transformation.}
\proch{int}{Trafo::Scale}{vector3d v, double f}{geo.h}
\descr{Hinzufügen einer isotropen Skalierung um den 
Punkt $v$ um den Faktor $f$ an eine xD-3D-Transformation.}
\proch{int}{Trafo::Scale}{vector3d v, double fx,double fy,double fz}{geo.h}
\descr{Hinzufügen einer anisotropen Skalierung um den 
Punkt $v$ um die Faktor $fx$, $fy$ bzw. $fz$ in X-, Y- bzw. Z-Richtung
an eine xD-3D-Transformation.}
\proch{int}{Trafo::Scale}{Vector v, double f}{geo.h}
\descr{Hinzufügen einer isotropen Skalierung um den 
Punkt $v$ um den Faktor $f$ an eine Transformation.}
\proch{int}{Trafo::Scale}{Vector v, Vector f}{geo.h}
\descr{Hinzufügen einer anisotropen Skalierung um den
Punkt $v$ an eine Transformation, wobei die Faktoren für die
verschiedenen Koordinatenrichtungen im Vektor $f$ abgelegt sind.}

\proch{int}{Trafo::Projective}{void}{geo.h}
\descr{Hinzufügen einer projektiven Abbildung, die die Dimension 
des Raumes um 1 vermindert. Die Abbildung erfolgt durch eine 
projektive Abbildung mit dem Ursprung als Zentrum in den Unter-Raum, bei 
dem die letzte Koordinate den Wert 1 hat.}

\subsubsection{Anwendung von Transformationen}

\proc{int}{Transform}{const Trafo \&,double \&x,double \&y}
\descr{Wendet die 2D-2D-Transformation auf $(x,y)$ an und
ändert die Variablen-Werte entsprechend.}

\proc{int}{Transform}{const Trafo \&,Point \&p}
\descr{Wendet die 2D-2D-Transformation auf $p$ an.}

\proc{int}{Transform}{const Trafo \&,double x,double y,double \&xt,double
\&yt}
\descr{Wendet die 2D-2D-Transformation auf $(x,y)$ an und
gibt das Ergebnis auf den Variablen $xt$ und $yt$ zurück.}

\proc{int}{Transform}{const Trafo \&,Point p1,Point \&p2}
\descr{Wendet die 2D-2D-Transformation auf $p$ an und gibt das Ergebnis auf $p2$ zurück.}

\proc{int}{TransformAndRound}{const Trafo \&,int \&x,int \&y}
\descr{Wendet die 2D-2D-Transformation auf $(x,y)$ an und
ändert die Variablen-Werte entsprechend. Die ganzzahligen Werte sind 
gegenüber den exakten Werten gerundet.}

\proc{int}{TransformAndRound}{const Trafo \&,int x,int y,int \&xt,int \&yt}
\descr{Wendet die 2D-2D-Transformation auf $(x,y)$ an und
gibt das Ergebnis auf den Variablen $xt$ und $yt$ zurück. Die ganzzahligen Werte sind 
gegenüber den exakten Werten gerundet.}

\proc{int}{Transform}{const Trafo \&,double \&x,double \&y,double \&z)}
\descr{Wendet die 3D-3D-Transformation auf $(x,y,z)$ an und
ändert die Variablen-Werte entsprechend.}

\proc{int}{Transform}{const Trafo \&,double x,double y,double z,double
\&xt,double \&yt,double \&zt}
\descr{Wendet die 3D-3D-Transformation auf $(x,y,z)$ an und
gibt das Ergebnis auf den Variablen $xt$, $yt$ und $zt$ zurück.}
\proc{int}{Transform}{const Trafo \&,double x,double y,double z,double
\&xt,double \&yt}

\descr{Wendet die 3D-2D-Transformation auf $(x,y,z)$ an und
gibt das Ergebnis auf den Variablen $xt$ und $yt$ zurück.}

\proc{int}{TransformList}{const Trafo \&tr,Matrix \&m}
\procf{int}{TransformList}{const Trafo \&tr,const Matrix \&m,Matrix \&m2}
\descr{Interpretiert die Matrix $m$ als Punktliste (jede Zeile
enthält die Koordinaten eines Punktes) und transformiert diese
mittels der Trafo $tr$. Das Ergebnis ist die transformierte Matrix $m$
beziehungsweise die Matrix $m2$.}

\proc{Contur}{Transform}{const Trafo \& tr,const Contur \& c}
\descr{Wendet die 2D-2D-Transformation auf die Kontur $c$ an 
und gibt die transformierte Kontur zurück}

\proc{int}{Transform}{const Trafo \&,const Image \&simg,Image \&dimg,int mode=DEFAULT}
\proc{int}{Transform}{const Trafo \&,const Image \&simg,Image \&dimg,int mode,
Image \&mark,int val=1}
\descr{Wendet die 2D-2D-Transformation auf das Bild $simg$ an 
und schreibt das Ergebnis in das Bild $dimg$. Da man bei der 
Transformation der Bildpunkte im allgemeinen keine ganzzahligen 
Koordinaten erhält, werden die Grauwerte für $mode$=INTERPOL 
bilinear interpoliert. Bei $mode$=DEFAULT werden die Koordinaten 
gerundet. Wenn der Originalpunkt zu einem Pixel des Zielbildes ausserhalb des
Quellbildes liegt, kann kein Grauwert berechnet werden und es wird der Wert
Null eingetragen. Wird als Parameter $mark$ ein Bild angegeben, so werden
diese Pixel in diesem Bild mit dem Wert $val$ belegt.
}
\seealso{TransImg} 

\subsection{Verzeichnungen}
\label{Distortion}
\hypertarget{Distortion}{}

Verzeichnungen von Kameras werden üblicherweise als radialsymmetrisch 
angenommen. Die Enstehung eines verzeichneten Bildes aus einem 
unverzerrten Bild wird als Änderung des Abstandes $r$ vom 
Symmetriezentrum (x0,y0) beschrieben. 

In ICE stehen die Klassen {\bf Distortion0}, {\bf Distortion1}, {\bf Distortion2} und 
{\bf Distortion3} zur Verfügung, die von der abstrakten Basisklasse  {\bf
Distortion} abgeleitet sind.

Diese Klassen realisieren die folgenden Verzeichnungs-Modelle:
\begin{itemize}
\item[Distortion0:] $r'=r \cdot (1 + d_2 \cdot r^2)$
\item[Distortion1:] $r'=r \cdot (1 + d_2 \cdot r^2 + d_4 \cdot r^4)$
\item[Distortion2:] $r'=r \cdot (1 + d_2 \cdot r^2 + d_3 \cdot r^3 + d_4 \cdot r^4)$
\item[Distortion3:] $r'=r \cdot (1 + d_2 \cdot r^2 + d_4 \cdot r^4 + d_6 \cdot r^6)$
\end{itemize}

\subsubsection{Konstruktoren}
Verzeichnungen können durch direkte Angabe von Verzeichnungsparametern oder
durch die Ermittlung aus Punktkorrespondenzen angelegt werden. 

\proch{}{Distortion0::Distortion0}{void}{distort.h}
\procf{}{Distortion1::Distortion1}{void}
\procf{}{Distortion2::Distortion2}{void}
\procf{}{Distortion3::Distortion3}{void}
\procf{}{Distortion0::Distortion0}{double x0,double y0,double d2=0.0}
\procf{}{Distortion1::Distortion1}{double x0,double y0,double d2=0.0,
double d4=0.0}
\procf{}{Distortion2::Distortion2}{double x0,double y0,double d2=0.0,
double d3=0.0,double d4=0.0}
\procf{}{Distortion3::Distortion3}{double x0,double y0,double d2=0.0,
double d4=0.0,double d6=0.0}
\descr{
Ein Verzeichnungs-Objekt wird angelegt und mit den gegebenen Werten
initialisiert.}

\proch{}{Distortion0::Distortion0}{const Matrix \&mark,const Matrix \&orig,Trafo \&tr,const Vector \&ImageCenter}{distort.h}
\procf{}{Distortion0::Distortion0}{const Matrix \&mark,const Matrix \&orig,const Vector \&ImageCenter}
\procf{}{Distortion0::Distortion0}{const Matrix \&mark,const Matrix \&orig}
\procf{}{Distortion1::Distortion1}{const Matrix \&mark,const Matrix \&orig,Trafo \&tr,const Vector \&ImageCenter}
\procf{}{Distortion1::Distortion1}{const Matrix \&mark,const Matrix \&orig,const Vector \&ImageCenter}
\procf{}{Distortion1::Distortion1}{const Matrix \&mark,const Matrix \&orig}
\procf{}{Distortion2::Distortion2}{const Matrix \&mark,const Matrix \&orig,Trafo \&tr,const Vector \&ImageCenter}
\procf{}{Distortion2::Distortion2}{const Matrix \&mark,const Matrix \&orig,const Vector \&ImageCenter}
\procf{}{Distortion2::Distortion2}{const Matrix \&mark,const Matrix \&orig}
\procf{}{Distortion3::Distortion3}{const Matrix \&mark,const Matrix \&orig,Trafo \&tr,const Vector \&ImageCenter}
\procf{}{Distortion3::Distortion3}{const Matrix \&mark,const Matrix \&orig,const Vector \&ImageCenter}
\procf{}{Distortion3::Distortion3}{const Matrix \&mark,const Matrix \&orig}
\descr{
$mark$ stellt eine Liste von im Bild vermessenen Markern bereit, $orig$
enthält eine 2D-Liste der originalen Position der Marker. Es wird die
Verzeichnung konstruiert. Das zugrunde liegende Modell ist eine projektive
2D-2D-Transformation gefolgt von der Verzeichnung. Dies entspricht der
Aufnahme einer ebenen Vorlage mit Markierungen mit einer Kamera. Wenn bereits
bekannt, kann für eine projektive 2D-2D-Transformation als Startwert $tr$
übergeben werden. Nach der Rückkehr ist diese Transformation auf die
ermittelte (optimierte) projektive Transformation gesetzt. $ImageCenter$
übergibt einen Startwert für die Lage des Symmetriezentrums.
}

\subsubsection{Zugriff auf die Daten der Verzeichnung}
\proch{double}{Distortion::X0}{void}{distort.h}
\descr{Es wird die X-Koordinate des Symmetriezentrums der Verzeichnung
zurückgegeben.}
\proch{double}{Distortion::Y0}{void}{distort.h}
\descr{Es wird die Y-Koordinate des Symmetriezentrums der Verzeichnung
zurückgegeben.}

\proch{double}{Distortion0::D2}{void}{distort.h}
\descr{Es wird der Wert von d2 zurückgegeben.}

\proch{double}{Distortion1::D2}{void}{distort.h}
\descr{Es wird der Wert von d2 zurückgegeben.}
\proch{double}{Distortion1::D4}{void}{distort.h}
\descr{Es wird der Wert von d4 zurückgegeben.}

\proch{double}{Distortion2::D2}{void}{distort.h}
\descr{Es wird der Wert von d2 zurückgegeben.}
\proch{double}{Distortion2::D3}{void}{distort.h}
\descr{Es wird der Wert von d3 zurückgegeben.}
\proch{double}{Distortion2::D4}{void}{distort.h}
\descr{Es wird der Wert von d4 zurückgegeben.}

\proch{double}{Distortion2::D2}{void}{distort.h}
\descr{Es wird der Wert von d2 zurückgegeben.}
\proch{double}{Distortion2::D4}{void}{distort.h}
\descr{Es wird der Wert von d4 zurückgegeben.}
\proch{double}{Distortion2::D6}{void}{distort.h}
\descr{Es wird der Wert von d6 zurückgegeben.}

\subsubsection{Anwendung und Invertierung von Verzeichnungen}

\proch{int}{Distortion::Distort}{double \&x,double \&y}{distort.h}
\procf{int}{Distortion::Distort}{double x,double y,double \&xd,double \&yd}
\procf{Vector}{Distortion::Distort}{const Vector \&p}
\procf{Point}{Distortion::Distort}{const Point \&p}
\descr{Anwendung der Verzeichnung auf einen Punkt. Der Punkt kann mit seinen
Koordinaten $x$ und $y$, als Punkt vom Typ \see{Point} oder als Vektor 
vom Typ \see{Vector} gegeben sein.
Die verzeichneten Kordinaten überschreiben $x$ und $y$ oder werden als 
$xd$ und $yd$, als Punkt oder als Vektor zurückgegeben.}

\proch{int}{Distortion::Rect}{double \&x,double \&y}{distort.h}
\procf{int}{Distortion::Rect}{double x,double y,double \&xr,double \&yr}
\procf{Vector}{Distortion::Rect}{const Vector \&p}
\procf{Point}{Distortion::Rect}{const Point \&p}
\descr{Invertierung der Verzeichnung eines Punktes. Der Punkt kann mit seinen
Koordinaten $x$ und $y$, als Punkt vom Typ \see{Point} oder als Vektor vom 
Typ \see{Vector} gegeben sein. Die unverzeichneten Kordinaten überschreiben 
$x$ und $y$ oder werden als $xd$ und $yd$, als Punkt oder als Vektor 
zurückgegeben.}

\proch{Image}{Distortion::RectImg}{Image source,int mode=DEFAULT}{distort.h}
\procf{Image}{Distortion::RectImg}{Image source,Image dest,int mode=DEFAULT}
\descr{Entzerrt das verzeichnete Bild $source$ und überschreibt $source$
beziehungsweise schreibt es als neues Bild $dest$. Punkte im Zielbild, zu
denen kein Punkt im Quellbild existiert, werden mit Null belegt. Die
Bestimmung der Grauwerte erfolgt je nach Modus:
\begin{itemize}
\item $mode=DEFAULT$ Es wird der Wert des nächstgelegenen Pixels verwendet.
\item $mode=INTERPOL$ Es werden die Werte der Nachbarn bilinear interpoliert.
\end{itemize}
Rückgabewert ist das Zielbild.}

%%% Local Variables: 
%%% mode: latex
%%% TeX-master: t
%%% End: 
