\lsection{Beschreibung von Modellen und Szenen}
\subsection{Datenstrukturen}

Ähnlich wie bei der Bildbeschreibung werden auch für die Beschreibung von
3D-Modellen und Szenen komplexe Datenstrukturen benötigt, die aus
untereinander verketteten Listen von Eckpunkten, Kanten und Oberflächen
bestehen. Übergeordnete Datenstruktur ist der Typ $ModelList$, der lediglich
Verweise auf das erste und letzte Element einer doppelt verketteten Liste von
Modellbeschreibungen vom Typ $Model$ enthält.
\begprogr\begin{verbatim}
typedef struct ModelList_
{
  struct Model_       *FirstModel;
  struct Model_       *LastModel;
}*ModelList;
\end{verbatim}\endprogr

Diese Liste kann sowohl als vogegebene Liste von CAD-Modellen als auch im
Zusammenhang mit der in den Modellbeschreibungen enthalten Lage und
Orientierung als Szenenbeschreibung genutzt werden. Durch die
Modellbeschreibung können Polyedrische Modelle beschrieben werden. Die
Modellbeschreibung enthält Felder mit Pointern auf die zum Modell gehörenden
Eckpunkte, Kanten und Oberflächen, ein Frame zur Beschreibung der Lage und
Felder für die Identifizierung des Objektes.
\begprogr\begin{verbatim}
typedef struct Model_
{
  struct Model_       *prev,*next;
  int                 index;                    /*Modellindex*/
  char                modelfile[80];            /*Modellfilename*/
  int                 vertices;                 /*Anzahl der Eckpunkte*/
  PointCorr           *pcorr;                   /*Punktkorrespondenzen*/
  struct Vertex3D_    **vx;                     /*Knoten (Eckpunkte)*/
  int                 edges;                    /*Anz. d. Kanten*/
  struct Edge3D_      **edg;                    /*Kantenliste*/
  int                 meshes;                   /*Anz. d. Oberflächen*/
  struct Mesh3D_      **msh;                    /*Liste der Oberflächen*/
  Frame               frm;                      /*Lageframe*/
  double              err;                      /*Rekonstruktionsfehler*/
  int                 matchedvx;                /*Anz. d. gematchten Knoten*/
}*Model;
\end{verbatim}\endprogr

Die Oberflächen polyedrischer Körper werden durch 3D-Maschen beschrieben, die
wiederum aus Kanten aufgebaut sind. Sie werden durch ihren Normalenvektor und
verschiedene Merkmale für das Matching mit Bildbeschreibungen
charakterisiert. Sie können außerdem ein Feld mit Verweisen auf Lochmaschen,
die sich in der Masche befinden, enthalten.
\begprogr\begin{verbatim}
typedef struct Mesh3D_
{
  int                 index;                   /*Index in Modellwelt*/
  double              n[3];                    /*Normalenvektor*/
  char                visible;                 /*Sichtbarkeitsflag*/
  struct Model_       *mod;                    /*Verweis auf zug. Modell*/
  int                 IsHole;                  /*Kennung Lochfl. (TRUE/FALSE)*/
  int                 edges;                   /*Anz. d. Außenkanten*/
  struct Edge3D_      **edg;                   /*Liste d. Außenkanten*/
  int                 *vxi;                    /*Knotenindizes*/
  int                 holes;                   /*Anz. d. Löcher*/
  struct Mesh3D_      **hol;                   /*Liste d. Lochflächen*/
  double              *inv1;                   /*InvariantenListen*/
  double              *inv2;
  int                 class;                   /*affine Klasse*/
  void                *Corr2D;                 /*korrespondierende Bildmasche*/
  void                *poly;                   /*z.B. für 2D-Punktliste*/
}*Mesh3D;
\end{verbatim}\endprogr

Die 3D-Kanten enthalten Verweise auf die Endknoten und die benachbarten Maschen.
\begprogr\begin{verbatim}
typedef struct Edge3D_
{
  int                 index;                   /*Index in Modellwelt*/
  struct Vertex3D_    *vx[2];                  /*inzidierende Knoten*/
  int                 vxi[2];                  /*Knotenindizes*/
  struct Mesh3D_      *msh[2];                 /*inzidierende Maschen*/
  void                *Corr2D;                 /*korrespondierende 2D-Kante*/
}*Edge3D;
\end{verbatim}\endprogr

Die 3D-Knoten enthalten die Koordinaten des zugehörigen Punktes und Verweise
auf die Kanten, mit denen der Knoten verbunden ist.
\begprogr\begin{verbatim}
typedef struct Vertex3D_
{
  int             index;
  double          p[3];                        /*Punktkoordinaten*/
  int             edges;                       /*Anzahl der Kanten*/
  struct Edge3D_  **edg;                       /*Kantenliste*/
  void            *Corr2D;                     /*korrespondierender Bildknoten*/
}*Vertex3D;
\end{verbatim}\endprogr


\subsection{Verwaltung der Modell- und Szenenbeschreibungen}

\proc{Model}{NewModel}{ModelList M,int index,int nvx,int nedg,int nmsh}
\descr{
Es wird ein Modell mit dem Index $index$ angelegt, das $nvx$ Knoten, $nedg$
Kanten und $nmsh$ Maschen aufnehmen kann.
}
\proc{void}{DeleteModel}{Model mod,ModelList M}
\descr{
Das Modell $mod$ wird aus der Modelliste $m$ entfernt und gelöscht.
}
\proc{Mesh3D}{NewMesh3D}{int nedg,int nhol,int index}
\descr{
Es wird eine Masche mit dem Index $index$ angelegt. Die Anzahl der Kanten
beträgt $nedg$ und die Anzahl der Lochmaschen $nhol$.
}
\proc{void}{DeleteMesh3D}{Mesh3D msh}
\descr{
Die Masche $msh$ wird gelöscht.
}
\proc{Edge3D}{NewEdge3D}{Model mod,int v1,int v2,int index}
\descr{
Es wird eine Kante zwischen den Knoten mit den Indizes $v1$ und $v2$ in der
Knotenliste des Modells angelegt. Der Kante wird der Index $indx$ zugeordnet.
}
\proc{int}{AddEdge3D}{Edge3D edg,Vertex3D vx}
\descr{
Die Kante $edg$ wird mit dem Knoten $vx$ verbunden.
}
\proc{Vertex3D}{NewVertex3D}{double p[3],int index}
\descr{
Es wird ein Knoten mit den Koordinaten des Punktes $p$ angelegt.
}
\proc{int}{DeleteModelList}{ModelList M}
\descr{
Die Modelliste $M$ wird gelöscht.
}


\subsection{Zugriff aud die Modellbeschreibungen}

\proc{int}{MeshOfEdge3D}{Mesh3D msh,Edge3D edg}
\descr{
Es wird der Index der Masche $msh$ in der Maschenliste der Kante $edg$ bestimmt.
}
\proc{Vertex3D}{NextVertex3D}{Vertex3D vx,int enr,int *dir}
\descr{
Zum Knoten $vx$ wird der über die Kante mit dem Index $enr$ benachbarte Knoten bestimmt.
}
\proc{Vertex3D}{StartVertex3D}{Mesh3D msh,Edge3D edg}
\descr{
Es wird der Anfangsknoten der Kante $edg$ beim Umlauf um die Masche $msh$ bestimmt.
}
\proc{Mesh3D}{NeighborMesh3D}{Edge3D edg,Mesh3D msh}
\descr{
Es wird die Masche bestimmt, die zur Masche $msh$ über die Kante $edg$
benachbart ist.
}
\proc{int}{EdgeOfMesh3D}{Edge3D edg,Mesh3D msh}
\descr{
Es wird der Index der Kante $edg$ in der Kantemöiste der Masche $msh$ bestimmt.
}
\proc{int}{VertexOfEdge3D}{Vertex3D vx,Edge3D edg}
\descr{
Es wird der Knotenindex des Knotens $vx$ in der Knotenliste der Kante $edg$ bestimmt.
}
\proc{int}{EdgeOfVertex3D}{Edge3D edg,Vertex3D vx}
\descr{
Es wird der Index der Kante $edg$ in der Kantenliste des Knotens $vx$ bestimmt.
}


\subsection{Erzeugung und Bearbeitung von Modellbeschreibungen}

\proc{Model}{ReadEPLYModel}{ModelList M,char *file,int index}
\descr{
Es wird ein Modell angelegt und die Daten aus der EPLY-Modellbeschreibung
$file$ eingelesen. Das Modell wird in die Modelliste $M$ eingefügt und erhält
den Index $index$.
}
Model ReadEPLYModel(ModelList M,char *file,int code);
\proc{ModelList}{CreateModelList}{char *file}
\descr{
In $file$ wird eine Liste von EPLY-Files erwartet, aus denen die
Modellbeschreibungen gelesen werden.
}
\proc{Model}{CopyModel}{Model mods,ModelList M}
\descr{
Es wird ein Modell angelegt und die Daten aus $mods$ kopiert.
}
\proc{Mesh3D}{CopyMesh3D}{Mesh3D mshs,Model mod}
\descr{
Es wird eine Masche mit den Daten der Masche $mshs$ angelegt und in das Modell
$mod$ eingeordnet.
}
\proc{int}{CalcSurfaceNormal}{Mesh3D msh}
\descr{
Für die Masche $msh$ wird die Oberflächennormale berechnet.
}
\proc{int}{CalcInvar}{Mesh3D msh}
\descr{
Für die Masche $msh$ werden die Fünf-Punkte-Invarianten berechnet.
}
\proc{void}{Delete2DCorrespondences}{ModelList M}
\descr{
Die den Modellbestandteilen zugeordneten 2D-Korrespondenzen werden zurückgesetzt.
}


\subsection{Zeichnen von Modellbeschreibungen}

\proc{int}{DrawVertex3D}{Vertex3D vx,camera cam,Frame *frm,int val,Image img}
\descr{
Der 3D-Knoten $vx$ wird mit $frm$ positioniert, mit $cam$ abgebildet und mit dem
Wert $val$ im Bild $img$ eingezeichnet.
}
\proc{int}{DrawEdge3D}{Edge3D edg,camera cam,Frame *frm,int val,Image img}
\descr{
Die 3D-Kante $edg$ wird mit $frm$ positioniert, mit $cam$ abgebildet und mit dem
Wert $val$ im Bild $img$ eingezeichnet.
}
\proc{int}{DrawSurface}{Mesh3D msh,camera cam,Frame *frm,int val,Image img}
\descr{
Die 3D-Oberfläche $msh$ wird mit $frm$ positioniert, mit $cam$ abgebildet und mit dem
Wert $val$ im Bild $img$ eingezeichnet.
}
\proc{int}{DrawModel}{Model mod,camera cam,Frame *frm,int val,Image img}
\descr{
Das 3D-Modell $mod$ wird mit $frm$ positioniert, mit $cam$ abgebildet und mit dem
Wert $val$ im Bild $img$ eingezeichnet.
}
\proc{void}{DrawSzene}{ModelList S,camera cam,Image img}
\descr{
Die Modelle der Szenenbeschreibung $S$ werden mit dem den jeweiligen Modellen
zugeordneten Lageframes positioniert, mit der Kamera $cam$ abgebildet und in
das Bild $img$ eingezeichnet.
}
\proc{int}{DrawFrame}{Frame *frm,camera cam, int val,int axelng,Image img}
\descr{
Es wird ein Dreibein mit mit der Achsenlänge $axelng$ mit $frm$ positioniert,
mit $cam$ abgebildet und mit dem Wert $val$ im Bild $img$ eingezeichnet. Die
Endpunkte der Achsen werden mit dem Wert 1, 2 bzw 3 für die x-, y-
bzw. z-Achse markiert.
}


