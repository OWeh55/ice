\nsection{Grafik}{graphics}

\subsection{Zeichnen}
\subsubsection{Grafische Primitiva}

\proc{int}{Marker}{int mode, int x,int y,int val,int size,Image img}
\procf{int}{Marker}{int mode, const IVector \&p,int val,int size,Image img}
\procf{int}{Marker}{int mode, const Point \&p,int val,int size,Image img}
\descr{
In das Bild $img$ wird an der Stelle $(x,y)$ ein Markierungspunkt mit dem Wert
$val$ und der Größe $size$ eingetragen. Die Form der Markierung wird durch
$mode$ festgelegt.}
\begin{quote}
\begin{tabular}{ll}
DEFAULT&senkrechtes Kreuz\\
1&      schräges Kreuz\\
2&      leerer Kreis\\
3&      voller Kreis\\
4&      leeres achsparalleles Quadrat\\
5&      volles achsparalleles Quadrat\\
6&      leeres schräges Quadrat\\
7&      volles schräges Quadrat\\
\end{tabular}
\end{quote}

\proc{int}{Text}{const string \&s,IPoint p,int val,int exp,Image \& img}
\proc{int}{Text}{const string \&s,int x,int y,int val,int exp,Image \& img}
\proc{int}{Text}{const char *s,int x,int y,int val,int exp,Image \& img}
\descr{
In das Bild $img$ wird der Text $s$ mit dem Wert $val$ eingetragen.
Die linke obere Ecke des Textes liegt auf dem gegebenen Punkt.
Der Parameter $exp$ ($0 \le exp \le 4$) bewirkt eine Vergrößerung des
Textes um $2^{exp}$.
}

\subsubsection{Zeichnen geometrischer Figuren}

\proc{int}{Line}{int x1,int y1,int x2,int y2,int val,int mode,Image img}
\procf{int}{Line}{int x1,int y1,int x2,int y2,int val,Image img}
\procf{int}{Line}{const IVector \&p1,const IVector \&p2,int val,int mode,Image
img}
\procf{int}{Line}{const IVector \&p1,const IVector \&p2,int val,Image img}
\procf{int}{Line}{const Point \&p1,const Point \&p2,int val,int mode,Image img}
\procf{int}{Line}{const Point \&p1,const Point \&p2,int val,Image img}
\descr{
In das Bild $img$ wird mit dem Wert $val$ ein Geradensegment von $(x1,y1)$ 
bis $(x2,y2)$ ($mode$=DEFAULT) bzw. eine Gerade durch die beiden Punkte 
($mode$=1) gezeichnet. Alternativ können die Punkte als \see{Point} oder
als Integer-Vektoren vom Typ \see{IVector} angegeben werden.}
\proc{int}{HesseLine}{double p,double phi,int val,Image img}
\procf{int}{HesseLine}{const Vector \&p,int val,Image img}
\descr{
In das Bild $img$ wird mit dem Wert $val$ eine Gerade gezeichnet, deren
Hessesche Normalform die Parameter $p$ und $phi$ hat. Alternativ können die
Parameter p und phi als 2-elementiger Vektor vom Typ \see{Vector} angegeben 
werden.}
\proc{int}{DrawCircle}{double par[3],int val1,int val2,int mode,Image img}
\descr{
In das Bild $img$ wird der Kreis mit dem Mittelpunkt $(par[0],par[1])$ und dem Radius
$par[2]$ mit $val1$ gezeichnet. Für $mode$=DEFAULT wird die Kreisfläche mit dem
Wert $val2$ gefüllt. Bei $mode$=NOFILL wird nur der Kreis gezeichnet und der
Parameter $val2$ ist redundant.}
\proc{int}{DrawEllipse}{double par[5],int val1,int val2,int mode,Image img}
\descr{ 
In das Bild $img$ wird mit dem Wert $val1$ eine Ellipse gezeichnet mit dem
Mittelpunkt $(par[0],par[1])$, den Halbachsen $par[2]$ bzw. $par[3]$ und einer Drehung um den
Winkel $par[4]$. Für $par[4]=0$ ist die Halbachse $par[2]$ parallel zur x-Achse. Wenn
$mode$=DEFAULT ist, wird die Ellipse mit dem Wert $val2$ gefüllt. Bei
$mode$=NOFILL wird nur die Ellipse gezeichnet und der Parameter $val2$ ist
redundant.}
\proc{void}{DrawSuperEllipse}{double c, double tr[3][3], int val,Image img}
\descr{ 
In das Bild $img$ wird mit dem Wert $val$ eine SuperEllipse gezeichnet. Der
Wert $c$ ist der Exponent in Zentrallage und $tr[3][3]$ ist eine affine
Transformation zur Abbildung der Zentrallage.}

\proc{int}{DrawCircleSegment}{double par[5],int val1,int val2,int mode,Image img}
\descr{Wie \bsee{DrawCircle}, es wird aber nur das Segment vom Winkel 
$par[3]$ bis zum Winkel $par[4]$ gezeichnet.}
\proc{int}{DrawEllipseSegment}{double *par[7],int val1,int val2,int mode,Image img}
\descr{Wie \bsee{DrawEllipse}, es wird aber nur das Segment vom Winkel 
$par[5]$ bis zum Winkel $par[6]$ gezeichnet.}
\proc{void}{DrawPolygon}{PointList pl,int val,Image img}
\descr{
Die Punktliste $pl$ wird als Folge von Eckpunkten eines Polygons interpretiert
und mit dem Wert $val$ in das Bild $img$ eingezeichnet.
}

\subsubsection{Geometrische Figuren als Konturen}

\proc{Contur}{LineContur}{Contur c,int p1[2][,int p2[2]]}
\descr{
Ablegen eines Geradensegmentes als Kontur. Es ist ein zwei- und ein
dreiparametriger Aufruf vorgesehen. Wird für $c$ der NULL-Pointer verwendet,
so wird intern eine neue Kontur angelegt. Die Funktion gibt den Pointer auf
diese neue Kontur zurück. In ihr wird eine Kontur von $p1$ nach $p2$ abgelegt
(dreiparametriger Aufruf). Existiert die Kontur $c$ bereits ($c$!=NULL) so sind
für diesen Funktionsaufruf nur zwei Parameter relevant ($p2$ redundant).
An die Kontur $c$ wird eine Linienkontur vom Endpunkt der alten Kontur nach
$p1$ angefügt. Der Pointer auf die neue Kontur wird als Funktionswert zurückgegeben. 
}
\proc{Contur}{CircleContur}{double par[3]}
\descr{
Es wird eine geschlossene Kontur angelegt, die einen Kreis mit dem Radius $par[2]$
um den Punkt $(par[0],par[1])$ beschreibt. Die Parameter sind so zu
wählen, daß alle Konturpunkte positive Koordinaten haben. Bei erfolgreicher
Ausführung wird der Pointer auf die Kreiskontur zurückgegeben, im Fehlerfall
der NULL-Pointer.
}
\proc{Contur}{EllipseContur}{double par[5]}
\descr{
Es wird eine geschlossene Kontur angelegt, die eine Ellipse mit dem
Mittelpunkt $(par[0],par[1])$, den Halbachsen $par[2]$ bzw. $par[3]$ und einer Drehung um den
Winkel $par[4]$ beschreibt. Für $par[4]=0$ ist die Halbachse $par[2]$ parallel zur x-Achse.
}
\proc{Contur}{CircleSegmentContur}{double par[5]}
\descr{
Es wird eine Kontur angelegt, die ein Segment eines Kreises mit dem Radius $par[2]$
um den Punkt $(par[0],par[1])$ beschreibt. Das Segment beginnt bei dem Winkel $par[3]$,
endet bei dem Winkel $par[4]$ und verläuft im positiven Drehsinn. Der Betrag der
Differenz zwischen den beiden Winkeln $par[3]$ und $par[4]$ darf nicht größer
als $2\pi$ sein.
}
\proc{Contur}{EllipseSegmentContur}{double par[7]}
\descr{
Es wird eine Kontur angelegt, die ein Segment einer Ellipse mit dem
Mittelpunkt $(par[0],par[1])$, den Halbachsen $par[2]$ bzw. $par[3]$ und einer Drehung um den
Winkel $par[4]$ beschreibt. Für $par[4]=0$ ist die Halbachse $par[2]$ parallel zur
x-Achse. Das Segment beginnt bei dem Winkel $par[5]$, endet bei dem Winkel
$par[6]$ und verläuft im positiven Drehsinn. Der Betrag der Differenz zwischen
den beiden Winkeln $par[5]$ und $par[6]$ darf nicht größer als $2\pi$ sein.
}

\subsubsection{Zeichnen von Kurven und Flächen}

\proc{int}{MarkContur}{Contur c,int val,Image img}
\descr{
Die Kontur $c$ wird mit dem Wert $val$ in das Bild $img$ eingezeichnet.
}
\proc{int}{FillRegion}{Contur c,int val,Image img}
\descr{
Die durch die {\em geschlossene} Kontur $c$ eingeschlossene Fläche wird mit
dem Wert $val$ im Bild $img$ gefüllt.
}

\subsection{Interaktive Zeichenfunktionen}

Zur Benutzung der interaktiven Zeichenfunktionen muß das zughörige Bild
visualisiert werden.

\proch{\bsee{LineSeg}}{SelLine}{Image img}{polygon\_vis.h}
\procf{int}{SelLine}{Image img,\bsee{IPoint} \&p1,IPoint p2}
\procf{int}{SelLine}{Image img,int p0[2],int p[2]}
\descr{
Interaktive Selektion einer Linie durch Auswahl der Endpunkte mit der Maus.
Abbruch durch Drücken einer Maustaste. Der Rückgabewert hängt von der zum 
Abbruch gedrücken Maustaste ab:
Links 1, Mitte 0, Rechts -1. Übrige Funktionalität wie bei \see{SelPoint}.
}

%\proch{PointList}{SelPolygon}{Image img}{polygon\_vis.h}
%\descr{
%Interaktive Selektion eines Polygons im Bild $img$. Auswahl der 
%Eckpunkte mit der linken
%Maustaste, Beenden mit der rechten Maustaste.
%}

\proch{Contur}{SelContur}{Image img,int force\_close=FALSE}{contool\_vis.h}
\descr{
Interaktives Zeichnen einer Kurve mit der Maus. Bei erstmaliger 
Betätigung der linken Maustaste wird der Startpunkt der Kontur 
festgelegt. Bei gedrückter linker Maustaste kann ``gezeichnet''
werden. Wenn die Maus bewegt wird, ohne daß eine Taste gedrückt
ist, wird ausgehend vom aktuellen Endpunkt der Kontur ein 
Liniensegment ``aufgezogen'', das durch Drücken der linken 
Maustaste bestätigt wird. Es besteht die Möglichkeit, das 
Selektieren der Kontur ohne Schließen (Betätigen der rechten
Maustaste) bzw. mit Schließen der Kontur (Betätigen der linken 
und rechten Maustaste) zu beenden. Ist der Parameter $force\_close$
auf TRUE gesetzt, so wird die Kontur immer geschlossen, indem
an die selektierte Kontur eine geradlinige Verbindung mit dem 
Startpunkt angehängt wird.
}

%%% Local Variables: 
%%% mode: latex
%%% TeX-master: "icedoc_base"
%%% End: 
