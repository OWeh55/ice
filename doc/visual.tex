\nsection{Visualisierung von Bildern und interaktive Auswahl}{Visualisierung}

\subsection{Displaysteuerung}

Die darzustellenden Bilder müssen mit der Funktion \see{Show} zur 
Visualisierung angemeldet werden. Sie werden dann in einem Fenster 
dargestellt. Bilder können als einfaches Grauwertbild, 
als Grauwertbild mit Overlays und als Echtfarbbilder dargestellt
werden. Veränderungen der Bilder werden im Normalfall simultan dargestellt.

Die Bilddarstellung wird durch einen gesonderten Prozess realisiert. Die
Aktualisierung erfolgt in kurzen Abständen, gesteuert durch einen Zeitgeber.
Falls diese Abstände - z.B. für Interaktionen - zu groß sind, kann
mit der Funktion \see{ForceUpdate} ein sofortiges Update erzwungen werden.

\proch{int}{Show}{int mode,Image img,const string \& title=''''}{visual.h}
\procf{int}{Show}{int mode,Image img1,Image img2,const string \& title=''''}
\procf{int}{Show}{int mode,Image img1,Image img2,Image img3,const string \&title=''''}
\descr{
  Es werden Bilder vom Type \class{Image} für die Visualisierung angemeldet.
  Die Zahl der Parameter ist abhängig vom gewünschten Darstellungs-Modus ab. 
  Wenn mehrere Bilder übergeben werden, müssen diese jeweils die gleiche 
  Größe haben. Der Parameter $title$ gibt den Titel des zugehörigen Fensters
  an. Wird $title$ nicht angegeben, wird der Name des ersten Bildes verwendet.
  Der Modus $mode$ kann eine der folgenden Konstanten sein:\\
  \begin{tabular}{|c|l|} \hline
    ON & 
    \begin{minipage}{0.6\textwidth}
      Das Bild $img1$ wird als einfaches Grauwertbild dargestellt.
    \end{minipage} \\ \hline
    \begin{minipage}{0.2\textwidth}
      OVERLAY, OVERLAY1, OVERLAY2, OVERLAY3 
    \end{minipage}
    & 
    \begin{minipage}{0.6\textwidth}
      Das Bild $img1$ wird als Grauwertbild dargestellt, das Bild $img2$ als
      Overlaybild, d.h, dort, wo im Bild $img2$ der Wert Null steht, ist der
      Grauwert des Bildes $img1$ sichtbar, ansonsten die zum Wert aus $img2$
      gehörende Farbe. In der Reihenfolge OVERLAY, OVERLAY1, OVERLAY2, OVERLAY3
      steigt die Durchsichtigkeit des Overlays, so das das Grauwertbild unter
      den Markierungen erkannt werden kann.
    \end{minipage} \\ \hline
    \_RGB & 
    \begin{minipage}{0.6\textwidth}
      Es wird ein Farbbild dargestellt, wobei die Bilder $img1$, $img2$ 
      und $img3$ als Rot-, Grün- und Blauauszug interpretiert werden.
    \end{minipage} \\ \hline
    OFF &
    \begin{minipage}{0.6\textwidth}
      Das Bild $img1$ und die zugehörigen Overlaybilder bzw. Farbauszüge werden von
      der Visualisierung abgemeldet. 
    \end{minipage} \\ \hline
  \end{tabular} 
}

\proch{int}{Show}{const ColorImage \& img, int mode=ON}{visual/ColorImage\_vis.h}
\descr{
Das Farbbild $img$ wird zur Visualisierung als Farbbild angemeldet ($mode=ON$) 
oder abgemeldet ($mode=OFF$).
}

\proch{int}{Zoom}{Image img,int val,int x,int y}{visual/visual.h}
\procf{int}{Zoom}{Image img}
\descr{Die Göße der Darstellung des Bildes $img$ wird geändert.
Wenn $val$ größer als Null ist, wird jeder Bildpunkt in $val$-facher 
Größe dargestellt. Wenn $val$ kleiner als Null ist, wird nur jeder 
$val$-te Bildpunkt dargestellt wird. 
}

\proch{int}{Cursor}{int mode,Image img,int x,int y}{visual/visual.h}
\descr{
Die Kursorkoordinaten für das Bild $img$ werden gesetzt und die Visualisierung
des Kursors kann ein- und ausgeschaltet werden. Der Kursor wird generell nur
dann sichtbar, wenn das Bild zur Darstellung angemeldet und die Darstellung
eingeschaltet ist. Der Modus kann durch Konstanten mit der folgenden Bedeutung
festgelegt werden.
\begin{tabular}{|c|l|} \hline
SET&
\begin{minipage}[t]{0.6\textwidth}
Der Kursor wird an der Stelle $(x,y)$ dargestellt. Ein evtl. vorher an anderer
Stelle des Bildes gesetzter Kursor wird vorher gelöscht.
\end{minipage}\\ \hline
OFF&
\begin{minipage}[t]{0.6\textwidth}
Der Kursor wird gelöscht, wobei aber die Koordinaten intern erhalten bleiben,
so daß der Kursor mit ON an gleicher Stelle wieder gesetzt werden kann.
\end{minipage}\\ \hline
ON&
\begin{minipage}[t]{0.6\textwidth}
Der Kursor wird dargestellt, wobei die intern verwalteten Koordinaten
verwendet werden.
\end{minipage}\\ \hline
\end{tabular}
}

\subsection{Farbtabellen}

Die Darstellung der Werte von Grauwert- und Overlaybildern kann durch
Farbtabellen beeinflußt werden. Für die Darstellung von Grauwerten werden
standardmäßig die verfügbaren Intensitätswerte den Werten von Null bis zum
größten zulässigen Grauwert linear zugeordnet. Wenn gleichzeitig Bilder mit
unterschiedlichem maximalem Grauwert dargestellt werden, kann das dazu führen,
daß die Bilder mit dem kleineren Grauwertumfang flau dargestellt werden.

Für die Overlaydarstellung sind den Werten 1\dots 8 die Farben rot, grün,
blau, gelb, rosa, cyan, purpur und braun zugeordnet. Bei höheren Werten
wiederholen sich diese Farben zyklisch. Die letzten drei Einträge der
Overlay-Farbtabelle legen die Farben für die Umrandung der Bilder und 
den Kursor fest.

\proch{int}{SetGrayColor}{int val,int red,int green,int blue}{visual/visual.h}
\descr{
Dem Grauwert $val$ wird ein Farbwert mit den Intensitäten $red$, $green$ und
$blue$ zugeordnet, die zwischen 0 und 255 liegen können.
}
\proch{int}{SetGrayLUT}{int val1,int val2}{visual/visual.h}
\descr{
Die verfügbaren Intensitäten werden linear den Grauwerten zwischen val1 und
val2 zugeordnet. Die unterhalb bzw. oberhalb dieses Intervalls liegenden
Grauwerte werden weiß bzw. schwarz dargestellt.
}
\proch{int}{SetOverlayColor}{int val,int red,int green,int blue}{visual/visual.h}
\descr{
Dem Overlay-Wert $val$ wird ein Farbwert mit den Intensitäten $red$, $green$ und
$blue$ zugeordnet, die zwischen 0 und 255 liegen können.
}

\subsection{Informationsabfrage zur Visualisierung}

Einige Eigenschaften der Bilddarstellung können in Abhängigkeit von der
Hardware und vom Betriebssystem unterschiedlich realisiert sein. Zur
Unterstützung einer portablen Programmierung werden durch die Funktion
InfVis() einige Information zur Bilddarstellung bereitgestellt.

\proch{int}{InfVis}{int code}{visual/visual.h}
\descr{
Es können Informationen zur Bilddarstellung abgefragt werden, die als
Funktionswert zurückgegeben werden. Als $code$
können die folgenden vordefinierten Konstanten verwendet werden:
}
\begin{quote}
\begin{tabular}{ll}
MAXX&
\begin{minipage}[t]{9cm}
x-Ausdehnung des Grafikbildschirms bzw. des Grafikfensters
\end{minipage}\\
MAXY&
\begin{minipage}[t]{9cm}
y-Ausdehnung des Grafikbildschirms bzw. des Grafikfensters
\end{minipage}\\
TABLE\_LEN&
\begin{minipage}[t]{9cm}
Anzahl der zur Verfügung stehenden Farbtabelleneinträge
\end{minipage}\\
REAL\_TIME\_COLOR&
\begin{minipage}[t]{9cm}
TRUE, wenn die Farbtabellen in Echzeit modifiziert werden können, sonst FALSE.
\end{minipage}\\
REAL\_TIME\_ZOOM&
\begin{minipage}[t]{9cm}
TRUE, wenn die Zoomfunktionen in Echzeit ausgeführt werden, sonst FALSE.
\end{minipage}\\
VIRTUAL\_X&
\begin{minipage}[t]{9cm}
Aktuelle x-Ausdehnung des virtuellen Bildspeichers.
\end{minipage}\\
VIRTUAL\_Y&
\begin{minipage}[t]{9cm}
Aktuelle y-Ausdehnung des virtuellen Bildspeichers.
\end{minipage}\\
IMAGES&
\begin{minipage}[t]{9cm}
Anzahl der zur Visualisierung angemeldeten Bilder. Bilder mit Overlay und
Echtfarbbilder werden jeweils als ein Bild gezählt.
\end{minipage}\\
\end{tabular}
\end{quote}

\subsection{Interaktive Auswahlfunktionen}

\proch{int}{Mouse}{Image img,int *x,int *y}{visual/mouse.h}
\procf{int}{Mouse}{Image img,int \&x,int \&y}
\descr{
Auf $x$ und $y$ werden die entsprechend den Mausbewegungen veränderten
Koordinaten eingetragen. Die Window-Systemen sind die Koordinaten an die
Position des System-Mauszeigers gekoppelt. Der Rückgabewert ergibt sich aus
einer ODER-Verknüpfung der Werte für die folgenden Mausereignisse:
}
\begin{quote}
\begin{tabular}{ll}
M\_LEFT\_DOWN&\begin{minipage}[t]{9cm}linke Maustaste ist gedrückt\end{minipage}\\
M\_RIGHT\_DOWN&\begin{minipage}[t]{9cm}rechte Maustaste ist gedrückt\end{minipage}\\
M\_MIDDLE\_DOWN&\begin{minipage}[t]{9cm}mittlere Maustaste ist gedrückt\end{minipage}\\
M\_LEFT\_PRESSED&\begin{minipage}[t]{9cm}linke Maustaste wurde seit der letzten
Abfrage gedrückt\end{minipage}\\
M\_RIGHT\_PRESSED&\begin{minipage}[t]{9cm}rechte Maustaste wurde seit der letzten
Abfrage gedrückt\end{minipage}\\
M\_MIDDLE\_PRESSED&\begin{minipage}[t]{9cm}mittlere Maustaste wurde seit der
letzten Abfrage gedrückt\end{minipage}\\
M\_LEFT\_RELEASED&\begin{minipage}[t]{9cm}linke Maustaste wurde seit der letzten
Abfrage losgelassen\end{minipage}\\
M\_RIGHT\_RELEASED&\begin{minipage}[t]{9cm}rechte Maustaste wurde seit der letzten
Abfrage losgelassen\end{minipage}\\
M\_MIDDLE\_RELEASED&\begin{minipage}[t]{9cm}mittlere Maustaste wurde seit der
letzten Abfrage losgelassen\end{minipage}\\
M\_MOVED&\begin{minipage}[t]{9cm}Maus wurde seit der letzten Abfrage bewegt\end{minipage}\\
\end{tabular}
\end{quote}

Die Funktionen zur interaktiven Auswahl eines Punktes werden wie folgt
bedient:\\
Im Bild $img$ wird bei eingeschalteter Visualisierung ein Kursor eingeblendet,
der mit der Maus bewegt werden kann. Bei $mode=1$ werden in einer Textzeile 
jeweils die aktuellen Koordinaten und der aktuelle Grauwert ausgegeben. 
Für $mode$=DEFAULT erfolgt keine Einblendung der Koordinaten. 
Durch Drücken der linken Maustaste oder \enter wird die Funktion beendet und 
der aktuelle Punkt ausgewählt. Wenn die rechte Maustaste gedrückt wird oder 
\esc, erfolgt ebenfalls ein Abbruch der Funktion.
Durch Drücken der Leertaste wird der sichtbare Bereich so verschoben, dass 
die aktuelle Kursorposition in der Mitte des Darstellungsfesters liegt. Durch
Drücken der ``+''-Taste bzw. der ``-''-Taste kann zusätzlich die Vergrößerung
eingestellt werden. Drücken von ``.'' stellt eine Vergrößerung ein, die das
gesamte Bild sichtbar macht.

\proch{int}{SelPoint}{int mode,const Image \&img,int p[2]}{visual/pointsel.h}
\descr{
Auswahl eines Punktes im Bild $img$ und Rückgabe der Koordinaten in $p$.
Der Rückgabewert ist der selektierte Grauwert oder -1 bei Abbruch mit rechter
Maustaste beziehungsweise \esc.
}

\proch{IPoint}{SelPoint}{int mode,const Image \&img}{visual/pointsel.h}
\procf{IPoint}{SelPoint}{int mode,const Image \&img,int \&rc}
\procf{Point}{SelPoint}{const Image \&img,int \&rc}
\procf{Point}{SelPoint}{const Image \&img}
\descr{
Auswahl eines Punktes im Bild $img$ und Rückgabe als \see{Point}. 
$rc$ enthält nach der Rückkehr den Grauwert an der ausgewählte Stelle oder -1
bei Abbruch mit rechter Maustaste oder \esc.
}

\proch{IVector}{SelVector}{int mode,const Image \&img,int \&rc}{visual/pointsel.h}
\procf{IVector}{SelVector}{int mode,const Image \&img}
\procf{IVector}{SelVector}{const Image \&img,int \&rc}
\procf{IVector}{SelVector}{const Image \&img}
\descr{
Auswahl eines Punktes im Bild $img$ und Rückgabe als \see{IVector}. 
$rc$ enthält nach der Rückkehr den Grauwert an der ausgewählte Stelle oder -1
bei Abbruch mit rechter Maustaste oder \esc.
}

\proch{int}{SelectWindow}{const Image \&img,int mode}{pointsel.h}
\descr{
In dem Bild $img$ wird bei eingeschalteter Visualisierung interaktiv mit der
Maus oder den Kursortasten ein Fenster festgelegt. Bei $mode$=0 kann das
vorher gesetzte Fenster lediglich verschoben werden, die Größe wird nicht
verändert. Mit $mode$=1 wird zunächst die linke obere Ecke angeklickt und dann
das Fenster ``aufgezogen''. Durch Drücken der rechten Maustaste oder \esc
wird die Funktion abgebrochen und das vorherige Fenster wieder gesetzt. Der
Rückgabewert ist in diesem Fall -1, sonst 0. Das folgende Programmbeispiel
demonstriert die Verwendung der Funktion.
}
\begprogr\begin{verbatim}
// Interaktive Fensterauswahl 
#include <stdio.h>
#include <image.h>
void main(void)
{
  int x,y;
  OpenAlpha("ICE Text Server");      // Textfenster
  SetAttribute(0,7,0,0);             // Attribute setzen
  ClearAlpha();                      // initialisieren 
  Image img=NewImg(768,512,255);     // Bilder anfordern
  Image imgo=NewImg(768,512,255);
  ClearImg(imgo);
  for (int y=0;y<img.ysize;y++)
    for (int x=0;x<img.xsize;x++)
      {
        PutVal(img,x,y,x&255);
        if (x & 8) PutVal(imgo,x,y,1);
      }  
  Show(OVERLAY, img, imgo);          // Bilddarstellung
  Display(ON);
  Printf("Linke obere Ecke anklicken und Fenster aufziehen\n");
  SelectWindow(img,1);               // Fenster "aufziehen" 
  Printf("Fenster verschieben                             \n");
  SelectWindow(img,0);               // Fenster verschieben
  Display(OFF);
}
\end{verbatim}\endprogr

%%% Local Variables: 
%%% mode: latex
%%% TeX-master: "icedoc_base"
%%% End: 
