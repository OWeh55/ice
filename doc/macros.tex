\newcommand\lt{\textless}
\newcommand\gt{\textgreater}
\newcommand\cb[1]{\textless{}#1\textgreater}

% <ESC> und <ENTER>
\def\esc{\cb{ESC}}
\def\enter{\cb{ENTER}}

\def\vector#1{vector\cb{#1}}
\def\cvector#1{const vector\cb{#1} \&}

\newcommand{\nextlabel}[3]{
  \stepcounter{referencecounter}
  \glossary{#1;#2;#3;\arabic{referencecounter}}
  \hypertarget{reflabel\arabic{referencecounter}}{}
}

%******************************************************
%Makro für Kapitelüberschrift
%******************************************************
\newcommand{\nsection}[2]{\section{#1}
  \label{#2}
  \hypertarget{SECTION:#2}{}
  % includefile ungültig machen
  \renewcommand{\includefile}{}
  \nextlabel{SECTION}{#2}{#1}
}

\newcommand{\subtitle}[1]{{\noindent\bf#1}}
%******************************************************
%Makro für Prozedurkopf
%******************************************************
% erster Header
\newcommand{\proc}[3]{
  \vspace*{0.5cm}
  \par\noindent#1 \hypertarget{#2}{{\bf #2}}{\small (#3)}\hspace{0.5cm} \includefile 
  \nextlabel{#1}{#2}{#3} \\
}

% erster Header mit Include-Fileangabe
\newcommand{\proch}[4]{
  % includefilename für nachfolgende Funktionen merken
%  \renewcommand{\includefile}{\hspace*{\fill} ({\it \#include \ls #4\gt})}
  \renewcommand{\includefile}{\hspace*{\fill} ({\it \#include~\cb{#4}})}
  \vspace*{0.5cm}
  \par\noindent#1 \hypertarget{#2}{{\bf #2}}{\small (#3)} \includefile 
  \nextlabel{#1}{#2}{#3} \\
}

% nachfolgende Header 
\newcommand{\procf}[3]{
  #1 \hypertarget{#2}{{\bf #2}}{\small (#3)}
  \nextlabel{#1}{#2}{#3} \\
}

\newcommand{\ctor}[4][]{
\renewcommand{\classname}{#2}
\proch{#1}{#2::#2}{#3}{#4}
}

\newcommand{\ctorf}[2]{
\renewcommand{\classname}{#1}
\procf{}{#1::#1}{#2}
}

\newcommand{\method}[3]{
\proc{#1}{\classname::#2}{#3}
}

\newcommand{\methodf}[3]{
\procf{#1}{\classname::#2}{#3}
}

%******************************************************
%Makro für Prozedurbeschreibung
%******************************************************
\newcommand{\descr}[1]{\nopagebreak
  \vspace*{-0.6cm}
  \begin{quote}
    %  \par\noindent
    #1
  \end{quote}
}

%******************************************************
% allgemeine Verweise
%******************************************************
\newcommand{\seealso}[1]{
  \vspace{0.1cm}
  Siehe auch: \hyperlink{#1}{#1}
}

\newcommand{\seealsonext}[1]{
  \hyperlink{#1}{#1}
}

\def\see#1{$\to$ #1}
\def\bsee#1{#1}

%******************************************************
% Verweis auf Klassendefinition
%******************************************************

\newcommand{\class}[1]{{\bf \bsee{#1}}}

\newcommand{\seebaseclass}[1]{Siehe auch die geerbten Variablen 
und Methoden der Basisklasse \class{#1}}

%******************************************************
%Definition für Prozedurliste
%******************************************************

\def\functionlistentry#1#2#3#4#5#6{\noindent #1 {\bf #2}(#3) \dotfill #6\\}

\def\letterref#1{}
\def\letterlabel#1{\vspace{0.5cm}\centerline{\Large #1}}
\def\letterlabelend#1{}

%******************************************************
%Makro für Programmbeginn
%******************************************************
\def\begprogr{\par\noindent\begin{minipage}{\textwidth}\medskip\hrule\medskip}

%******************************************************
%Makro für Programmende
%******************************************************
\def\endprogr{\nopagebreak\hrule\end{minipage}\par\medskip\noindent}

%******************************************************
%Prozedurliste erzeugen
%******************************************************
\makeglossary
