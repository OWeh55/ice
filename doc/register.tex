\nsection{Merkmale, Ähnlichkeiten und Abstandsmaße}{features}
\label{Merkmale}

Zur Klassifikation, zur Bestimmung von Ähnlichkeiten und zur Lösung von 
Zuordnungsproblemen werden Merkmale von Bildern, Regionen und Objekten 
ermittelt. Typischerweise sind Merkmale numerische Werte, die 
charakteristische Eigenschaften beschreiben. Gewöhnlich kommen
mehrere Merkmale zum Einsatz, die in ICE in einem \class{Vector} 
oder als \vector{double} abgelegt werden. 

\subsection{Abstandsmaße}
\label{VectorDistance}
Die (abstrakte) Klasse \verb+VectorDistance+ dient in ICE als Abstandsmaß und 
kann bei der Klassifikation, bei Zuordnungsproblemen und zur Bestimmung von 
Ähnlichkeitsmaßen verwendet werden. Die Klasse (und damit die abgeleiteten 
Klassen für spezielle Abstandsmaße) definiert die Methode 
\verb+Distance+ und den Operator \verb+operator()+, die den Abstand ihrer 
Argumente ermitteln:

\proch{double}{VectorDistance::Distance}{const vector$<$double$>$ \&x,const vector$<$double$>$ \&x}{vectordistance.h}
\procf{double}{VectorDistance::Distance}{const Vector \&x,const Vector \&y}
\procf{double}{VectorDistance::operator()}{const vector$<$double$>$ \&x,const vector$<$double$>$ \&y}
\procf{double}{VectorDistance::operator()}{const Vector \&x,const Vector \&y}
\descr{Es wird der Abstand der übergebenen Argumente ermittelt und zurückgegeben.}

\subsubsection{EuclidianVectorDistance}
\label{EuclidianVectorDistance}
\verb+EuclidianVectorDistance+ ist eine von \bsee{VectorDistance} abgeleitete Klasse, die 
die Distanz als Euklidische Distanz $ \sqrt{\sum_{i} (x_{i} - y_{i} )^2} $  berechnet.

\subsubsection{CityBlockVectorDistance}
\label{CityBlockVectorDistance}
\verb+CityBlockVectorDistance+ ist eine von \verb+VectorDistance+ abgeleitete Klasse, die 
die Distanz als CityBlock-Distanz $ \sum_{i} |x_{i} - y_{i}| $  berechnet.

\subsubsection{QuadraticFormVectorDistance}
\label{QuadraticFormVectorDistance}
\verb+QuadraticFormVectorDistance+ ist eine von \verb+VectorDistance+ abgeleitete Klasse, die 
die Distanz als Quadratische Form $ \sum_{i,j} a_{i,j} \cdot (x_{i} - y_{i} ) \cdot (x_{j} - y_{j} ) $  
berechnet. Die Matrix $a$ wird über den Konstruktor vorgegeben:

\proch{}{QuadraticFormVectorDistance::QuadraticFormVectorDistance}{const Matrix \&a}{vectordistance.h}
\descr{Erzeugt eine QuadraticFormVectorDistance unter Verwendung der gegebenen Matrix $a$.}

\subsubsection{EarthMoverVectorDistance}
\label{EarthMoverVectorDistance}
\verb+EarthMoverVectorDistance+ ist eine von \verb+VectorDistance+ 
abgeleitete Klasse, die die Earth Mover-Distanz berechnet. Diese Distanz 
ist sinnvoll einsetzbar, wenn benachbarte/aufeinderfolgende Merkmale in 
Zusammenhang stehen, wie es bei den Häufigkeiten von Histogrammklassen 
der Fall ist.

Beispiel für die Berechnung der Distanzen zweier Bilder anhand des 
Histogramms:
\nopagebreak
\begprogr
\begin{verbatim}
Image img1;
Image img2;
...
Histogramm h1(img1);
Histogramm h2(img2);

vector<double> hvec1=h1.Rel();
vector<double> hvec2=h2.Rel();
EuklidianVectorDistanz ed;
CityBlockVectorDistanz md;
EarthMoverVectorDistanz emd;

cout << "Euklidische Distanz: " << ed(hvec1,hvec2) << endl;
cout << "City Block Distanz: " << md(hvec1,hvec2) << endl;
cout << "Earth Mover Distanz: " << emd(hvec1,hvec2) << endl;
\end{verbatim}
\endprogr

\subsection{Distanz-Matrix}
\proch{Matrix}{DistanceMatrix}{const Matrix \&feat1, const Matrix \&feat2,
  int mode=D\_EUKLID}{assignment.h}
\procf{Matrix}{DistanceMatrix}{const Matrix \&feat1, int mode=D\_EUKLID}
\procf{Matrix}{DistanceMatrix}{const Matrix \&feat1, const Matrix \& feat2, const VectorDistance \&dist}
\procf{Matrix}{DistanceMatrix}{const Matrix \&feat1, const VectorDistance \&dist}
\descr{
Aus zwei Merkmalslisten $feat1$ und $feat2$ wird eine Kostenmatrix 
aufgestellt, die geeignet ist, die Zuordnung zu ermitteln. Die Zahl 
der Merkmale in beiden Listen muss gleich sein (gleiche Spaltenzahl).
Die Zeilenzahl der Ergebnismatrix ist gleich der Zeilenzahl von $feat1$,
die Spaltenzahl ist gleich der Zeilenzahl von $feat2$.\\
Der Modus $mode$ gibt die zu verwendende Metrik an:\\
\begin{tabular}{ll}
D\_EUCLID    & Euklidischer Abstand\\
D\_SQUARE    & Quadrat des euklidischen Abstands\\
D\_CITYBLOCK & city block-Metrik (L1-Norm)\\
\end{tabular}
}
\seealso{Hungarian}
\seealsonext{TimeWarp}

\nsection{Registrierung, Matching}{registration}

Registrierung ist die Bestimmung von Transformationen, die Bilder oder
Objekte ineinander überführen.

\hypertarget{Trafo-Konstanten}{}
Die Spezifikation der zu ermittelnden Transformation erfolgt in ICE 
über Konstanten. Dabei nimmt die Spiegelung eine Sonderstellung ein.
Da bei vielen Bildverarbeitungsaufgaben die Spiegelung auszuschließen ist,
gibt es von den Transformations-Gruppen, die auch eine Spiegelung beinhalten,
auch jeweils eine Variante ohne Spiegelung.

\begin{tabular}{|l|l|l|} \hline
* & mit Spiegelung & ohne Spiegelung \\ \hline
projektive Transformation & TRM\_PROJECTIVE & - \\
affine Transformation & TRM\_AFFINE & TRM\_AFFINE\_NOR \\
Ähnlichkeitstransformation & TRM\_SIMILARITY & TRM\_SIMILARITY\_NOR \\
Euklidische Transformation & TRM\_EUCLIDEAN & TRM\_EUCLIDEAN\_NOR \\
Translation & - & TRM\_SHIFT \\ \hline
\end{tabular}

\subsection{Schätzung geometrischer Transformationen
aus Punktkorrespondenzen}

\proch{Trafo}{MatchPointlists}{const Matrix \&p1,const Matrix \&p2,int
  mode=TRM\_PROJECTIVE}{geo.h}
\procf{Trafo}{MatchPointlists}{const Matrix \&p1,const Matrix \&p2,
int mode,const Vector \&w}
\descr{Bestimmung einer an die gegebenen Punktreferenzen angepassten
Transformation mittels Ausgleichsrechnung, bei der der quadratische 
Fehler $\| T(p_1) - p_2\|^2$ minimiert wird.
Die Punktreferenzen müssen als Punktlisten in den zwei Matrizen 
$p1$ und $p2$ bereitgestellt werden, wobei in jeder Zeile ein Punkt 
gespeichert sein muss. Die Spaltenzahl von p1 und p2 bestimmen die
Diomension des Quell- bzw. Zielraumes.\\
In einem (optionalen) Vektor $w$ ist für jedes Punktpaar ein 
Gewicht anzugeben. Ohne $w$ werden alle Punktpaare gleich 
gewichtet.\\
Der Parameter $mode$ (\hyperlink{Trafo-Konstanten}{Trafo-Konstanten}) 
bestimmt den Typ der geometrischen Transformation.
\begin{itemize}
\item TRM\_SHIFT - Translation
\item TRM\_SIMILARITY\_NOR - Ähnlichkeits-Transformationen (ohne Spiegelung)
\item TRM\_AFFINE - Affine Transformationen
\item TRM\_PROJECTIVE - Projektive Transformationen
\end{itemize}
}
\seealso{AffinFitMoments}
\seealso{estimateTransformation}

\proch{Trafo}{MatchPointlists}{const \vector{Point} \&pl1,
  const \vector{Point} \&pl2, int mode = TRM\_PROJECTIVE}{geo.h}
\procf{Trafo}{MatchPointlists}{const \vector{Point} \&pl1,
  const \vector{Point} \&pl2, int mode = TRM\_PROJECTIVE, 
  const \vector{double} \&weight}
\descr{Spezialfall von MatchPointlists, der Punktlisten als 
\vector{Point} akzeptiert. Diese Funktionen funktionieren genauso, 
sind aber auf 2D-2D-Transformationen beschränkt.}

\proc{Trafo}{MatchPointlists}{PointList pl1,PointList pl2,int mode}
\descr{Spezialfall von MatchPointlists, bei dem die Punktlisten
als Datentyp $PointList$ übergeben werden. 
Dieser Typ ist veraltet und sollte nicht mehr genutzt werden. 
Als Gewicht der 
Punktpaare wird das in der Liste pl2 gespeicherte Gewicht 
verwendet.}

\proc{Trafo}{MatchPointlistsLinOpt}{const Matrix \&p1,const Matrix \&p2,int
mode=TRM\_AFFINE}
\procf{Trafo}{MatchPointlistsLinOpt}{const Matrix \&p1,const Matrix \&p2,int mode,const Vector \&w,double limit=1000000.0}
\descr{Bestimmung einer an die gegebenen Punktreferenzen angepassten
Transformation. Die Punktreferenzen müssen als Punktlisten in den 
zwei Matrizen $p1$ und $p2$ bereitgestellt werden, wobei in jeder 
Zeile ein Punkt gespeichert sein muß.
In einem (optionalen) Vektor $w$ kann für jedes Punktpaar ein 
Gewicht angegeben werden. Ohne $w$ werden alle Punktpaare gleich gewichtet.
Die Transformation wird mittels linearer Optimierung (Simplexmethode) 
bestimmt. Anders als bei der Methode der kleinsten Quadrate werden
hier die Beträge des Fehlers $\|T(p_1) - p_2\|$ minimiert, wodurch Ausreißer
das Ergebnis weniger verfälschen.
Diese Funktion ist zur Zeit auf 2D-2D-Transformationen beschränkt.
Der Parameter $limit$  wird im linearen Optimierungsmodell in einer 
zusätzlichen Beschränkungsungleichung verwendet. Der Default-Wert ist
1000000.0. Gegebenenfalls muss dieser Wert vergroessert werden.\\
Der Parameter $mode$ bestimmt den Typ der geometrischen 
Transformation (\hyperlink{Trafo-Konstanten}{Transformations-Konstanten}).
\begin{itemize}
\item TRM\_SHIFT - Translation (2 Parameter)
\item TRM\_SIMILARITY\_NOR - Ähnlichkeits-Transformationen (4 Parameter)
\item TRM\_AFFINE - Affine Transformationen (6 Parameter)
\item TRM\_PROJECTIVE - Projektive Transformationen (8 Parameter)
\end{itemize}
}

\proc{int}{MatchPointlistsLinOpt}{PointList pl1,PointList pl2,double
tr[][3],int mode=TRM\_AFFINE,double limit=1000000.0}
\descr{Analog MatchPointlistsLinOpt, die Punktlisten werden
als Datentyp $PointList$ übergeben werden.
Dieser Typ ist veraltet und sollte nicht mehr genutzt werden. 
}
\seealso{NewPointList}

\subsection{Signalbasierte Methoden}
Die implementierten signalbasierten Methoden der Registrierung 
beruhen auf der Detektion von Verschiebungen mittels
Shift Detection by Restoration (SDR), beschrieben zum Beispiel in \\

{
\noindent
K.Voss,H.Süße,W.Ortmann,T.Baumbach: Shift detection by restoration\\
Pattern Recognition 32(1999), Elsevier, pp. 2067-2068\\
}

SDR basiert auf dem Modell, eine Verschiebung durch eine 
Faltung mit einem verschobenen Einheitsimpuls zu beschreiben. 
Die Rekonstruktion dieser Faltungsmaske erlaubt die Bestimmung 
der Lage des verschobenen Einheitsimpulses als Peak.

Die elementaren Funktionen zur Rekonstruktion der Faltungsmaske
der Verschiebung sind im Abschnitt Faltung und Kreuzkorrelation 
dokumentiert (\see{InvConvolution}). Die Ermittlung der Peaks
in den erhaltenen Bildern erfolgt mit der Funktion \see{PeakValuation}.

\see{DetectShift} verwendet diese Funktionen um die Verschiebung im Bild 
bzw. in einem Teil-Bild zu detektieren. Durch eine Blockmatching-artige
Verwendung des Verfahrens kann \see{DetectTrafo} allgemeine 
Transformationen bestimmen, wenn lokal die Transformation 
einer Verschiebung nahekommt.. 
Zur Unterdrückung des Einflusses globaler Helligkeitsschwankungen (Shading) 
kann die Vorverarbeitung der Bilder mit \see{PreprocessImg} sinnvoll sein.

\proc{int}{DetectShift}{Image img1,Image img2,
       double \&dx,double \&dy,double \&val,double beta=0.1}
\descr{
In zwei Bilder wird mittels der SDR-Methode eine Verschiebung
detektiert. Dazu werden die Funktionen zur Invertierung der
Faltung (\see{InvConvolution}) und zur Peaksuche (\see{PeakValuation})
verwendet. Es wird die Verschiebung ($dx$,$dy$) ermittelt, die das 
Bild $img1$ in das Bild $img2$ überführt. Der zurückgegebene Wert 
der Variable $val$ stellt ein Maß für die Güte dar. Der Parameter 
$beta$ optimiert den Prozess und muss abhängig von Rauschen und 
Störungen gewählt werden.}

\proc{int}{DetectTrafo}{Image img1,Image img2,Trafo \&tr,
double beta=0.1,int ct=3,int mode=DT\_NEARLY\_SHIFT}
\descr{
Es wird eine Transformation $tr$ ermittelt, die das Bild $img1$
in das Bild $img2$ überführt. Dies geschieht durch Ermittlung 
eines lokalen Verschiebungsvektors durch Anwendung
der SDR-Methode auf Teilblöcke der Bilder und Anpassung einer
projektiven 2D-2D-Transformation an die ermittelten 
Verschiebungsvektoren. Der Parameter $beta$ optimiert den Prozess
und muss abhängig von Rauschen und Störungen gewählt werden.\\
Im ersten Schritt wird die globale Verschiebung anhand des 
gesamten Bildes ermittelt. \\
Ist der Parameter $ct>0$, so erfolgt danach eine $ct$-fache 
iterative Anwendung der Verschiebungsdetektion in den Teilblöcken
des Bildes.\\
Der Parameter $mode$ steuert die Art der zu detektierenden 
Transformation. Einziger zulässiger Wert ist zur Zeit 
DT\_NEARLY\_SHIFT für die Detektion von Transformationen, die 
näherungsweise eine Verschiebung darstellen.
}

\proc{int}{PreprocessImg}{(Image imgs,Image imgd)}
\descr{
Oft treten bei der Fouriertransformation von Bildern die sogenannten
Fourierkreuze auf, die bei manchen Anwendungen störend sein können. Oft möchte
man auch hohe Frequenzen abschwächen. Sollte dies in einer Anwendung nötig
sein, so kann man das Bild vorher mit dieser Funktion vorverarbeiten. 
Es hängt von der jeweiligen Anwendung ab, ob dies sinnvoll ist. Dabei ist 
$imgs$ das Ausgangsbild, $imgd$ das Zielbild. Diese Funktion ist z.B. 
sinnvoll, wenn man aus dem Amplitudenspektrum weitere Transformationen 
berechnen will (Log-Polar-Transform).
}

\subsubsection{Peaksuche}

\proc{double}{PeakValuation}{Image img,Image mark,double\& x0,double\& y0,
        int mode=PN\_CONVOLUTION,int gnull=0,int zykl=TRUE}
\descr{
Es erfolgt die Suche nach dem globalen Maximum der Grauwertfunktion im 
Bild $img$ und dessen Bewertung. Wird ein Markierungsbild $mark!=NULL$ 
angegeben, so werden nur die unmarkierten Bildpunkte einbezogen. Die 
zurückgegebene Bewertung $B$ liegt im Intervall $[0,1]$ und ist ein Maß 
für den Abstand des Bildes vom Bild eines beliebig verschobenen 
Einheitsimpulses. Für $B=1.0$ liegt ein idealer verschobener 
Einheitsimpuls vor.\\
Der Parameter $gnull>=0$ legt den Grauwert-Nullpunkt des Bildsignales 
fest. Dieser liegt im allgemeinen bei $0$ oder $g_max/2$. 
Mit $zykl=TRUE$ wird das Bild als zweifach zyklisch fortgesetzt 
angesehen. Der Parameter $mode$ hat momentan keine Bedeutung.
}

\proc{int*}{Peak1D}{double* values,int anz,int\& panz,int zykl,int noise}
\procf{IVector}{Peak1D}{const Vector\& vec,int panz,int zykl,int noise}
\descr{
Es erfolgt die Suche nach lokalen Maxima in der durch das Feld $values$
oder den Vektor $vec$ repräsentierten eindimensionalen, diskreten 
Funktion. Im Falle eines Feldes beschreibt der Parameter $anz$ die
Anzahl der Funktionswerte (Größe des Feldes $values$). Die Anzahl der 
zu suchenden Maxima wird durch $panz$ (>=0) festgelegt. Mit $panz=0$ 
werden alle lokalen Maxima bestimmt.\\
Mit $zykl=TRUE$ wird die Funktion als eine zyklisch fortgesetzte Funktion
angesehen, was die Erkennung von Peaks an den Intervallrändern beeinflußt.
Solche Funktionen ergeben sich zum Beispiel als Ergebnis der Entfaltung.\\
Der Parameter $noise$ steuert die Rauschempfindlichkeit der Maximumsuche.
Es werden alle Nebenpeaks, die sich maximal $noise$ Grauwertstufen über
das größere der zwei direkt benachbarten lokalen Minima erheben, ignoriert.\\
In der ersten Aufrufvariante ist der Rückgabeparameter ein neu angelegtes 
Integer-Feld. Auf $panz$ wird dann die gefundene Anzahl der lokalen 
Maxima (Größe des Rückgabefeldes) abgelegt. Dieses Feld ist vom Nutzer 
freizugeben, sobald es nicht mehr benötigt wird. Im Falle eines Fehlers 
wird NULL zurückgegeben und $panz$ wird auf $-1$ gesetzt.
Die Rückgabe ist ebenfalls NULL, wenn die Funktion konstant ist.\\
In der Vektorvariante wird ein neu angelegter Vektor zurückgegeben.
}

Im Falle der Suche mehrerer Peaks wird eine Datenstruktur vom Typ
\see{PeakList} verwendet.

\proc{PeakList}{ImgPeakList}{Image img,Image mark,int panz=0,int mingrw=0,
        int zykl=FALSE, int noise=0,int feat=IPL\_NOFEAT,int gnull=0}
\descr{
Es erfolgt die Suche nach lokalen Extremwerten im Bild $img$. Wird ein
Markierungsbild $mark!=NULL$ angegeben, so werden in die Extremwertsuche
nur die unmarkierten Bildpunkte einbezogen. Die Anzahl der zu suchenden
Extremwerte wird durch $panz$ ($>=0$) festgelegt. Mit $panz=0$ werden alle
lokalen Extrema bestimmt. Für die Extremwerte kann mit $mingrw!=0$ ein
Mindestgrauwert gefordert werden.\\
Mit $zykl=TRUE$ wird das Bild als zweifach zyklisch fortgesetzt angesehen,
was die Erkennung von Maxima/Minima an den Bildrändern beeinflußt.
Solche Bilder ergeben sich zum Beispiel als Ergebnis der Entfaltung.\\
Der Parameter $noise$ steuert die Rauschempfindlichkeit der Maximumsuche.
Es werden alle Nebenpeaks, die sich maximal $noise$ Grauwertstufen über
den maximalen Grauwert des Peak Tales erheben, ignoriert.\\
Mit dem Parameter $feat$ werden die zurückgelieferten Daten festgelegt:
\begin{itemize}
\item IPL\_NOFEAT Koordinaten der Extrema
\item IPL\_MINCONTUR Koordinaten und Kontur des minimalen Einzugsbereiches
\item IPL\_MAXCONTUR Koordinaten und Kontur des maximalen Einzugsbereiches
\item IPL\_STANDARD Koordinaten und Mittelwert/Streuung
\item IPL\_ALL alle obigen Merkmale werden berechnet
\end{itemize}
Der Parameter $gnull>=0$ legt den Grauwert-Nullpunkt des Bildsignales fest.
Dieser liegt im allgemeinen bei $0$ oder $g_{max}/2$.
Die Rückgabe der Merkmale erfolgt als verkettete Liste von Elementen, die
durch einen neuen Datentyp PeakList definiert sind.\\
}

\label{PeakList}
Für die Beschreibung von Peakmerkmalen eines Bildes wird die Datenstruktur 
$PeakList$ verwendet, die den Aufbau von verketteten Listen erlaubt.

\begprogr\begin{verbatim}
typedef struct _PeakList {
  int        grw;
  ConturList contur;
  ConturList contur_max;
  double     sx,sy,sxy,x,y;
  _PeakList  *prev,*next;
} *PeakList;
\end{verbatim}\endprogr

\proc{PeakList}{NewPeakList}{}
\descr{
Es wird eine neue, leere Peakliste angelegt.
}

\proc{}{FreePeakList}{PeakList pl}
\descr{
Sämtlicher von pl belegter Speicherplatz (inklusive aller eventuell 
darin gespeicherten Konturen) wird freigegeben.
}

Beispiel für die Erzeugung einer Peakliste aus einem Bild:
\begprogr
\begin{verbatim}
PeakList pl;
pl=NewPeakList();
if (ImgPeakList(img,20)==NULL) exit(-1);
// .. bearbeiten/auslesen der Peakliste
FreePeakList(pl);
\end{verbatim}
\endprogr

\subsection{Merkmalsbasierte Methoden}

Zuordnungsprobleme entstehen im Rahmen der Registrierung als Zuordnung von
Punkten oder Objekten eines Bildes zu solchen des anderen
Bildes. Ausgangspunkt für die Zuordnung sind oft (invariante) Merkmale der 
Objekte. Es ist möglich, eine Kostenmatrix zu erstellen, die die Kosten der
Zuordnung zweier Objekte zueinander bewertet. Dies wird oft die Distanz der
Merkmale der Objekte sein. Je nach Randbedingungen ist die Zuordnung der
Punkte zueinander ansonsten beliebig (\see{Hungarian}) oder es gibt 
wie bei Konturpunkten eine vorgegebene Ordnung (\see{TimeWarp}).

Nach Ermittlung der Zuordnung kann die Transformation zwischen den Objekten
beziehungsweise den Bildern ermittelt werden (\see{MatchPointlists}..).

\subsubsection{Zuordnungsprobleme}
\proc{int}{AssignFunction}{const Matrix \&cost,IMatrix \&pairs,int mode}
\descr{
Dies ist eine Typdefinition für Funktionen, die anhand einer Kostenmatrix
Zuordnungen herstellen. Implementierte Funktionen dieses Types sind
\see{Hungarian} und \see{TimeWarp}.
}

\proc{double}{ReferenceCosts}{const Matrix \&costs,const IMatrix \&pairs}
\descr{
Anhand einer gegebenen Liste von Referenzen $pairs$ werden die Kosten 
entspechend der Werte in der Kostenmatrix $costs$ ermittelt und zurückgegeben.
Die Referenzen müssen als Paare von Indizes angegeben sein.
}

\proc{int}{Hungarian}{const Matrix \&cost,IMatrix \&reference\_pairs,
double \&min\_cost}
\procf{int}{Hungarian}{const Matrix \&cost,IMatrix \&reference\_pairs,
int mode=0}
\descr{
Es wird ein allgemeines Zuordnungsproblem (m Quellen, n Ziele) mit der
Ungarischen Methode gelöst. Das Problem wird allgemein durch eine 
Kostenmatrix $cost$ beschrieben. $cost$ is eine Matrix mit m Zeilen 
und n Spalten. Das Matrixelement cost[i][j] stellt die Kosten für 
die Zuordnung der Quelle i zum Ziel j dar. Ein Beispiel einer Kostenmatrix 
ist die Distanzmatrix (\see{DistanceMatrix}). 
{\bf Intern wird nur mit ganzzahligen Kosten gerechnet}, wobei durch eine 
interne Normierung der Kostenbereich auf den Integerbereich abgebildet
wird. Der Parameter $mode$ steuert die Art der Normalisierung. Bei mode=0
erfolgte eine lineare Transformation der Kostenwerte auf den Integer-Bereich,
wobei der maximale Kostenwert auf einen festgelegten Integer-Wert abgebildet
wird. $mode=1$ bewirkt eine Abbildung, die linear den minimalen Kostenwert auf
die Null und den nächstgrößeren Kostenwert auf die 1 abbildet. Werte, die bei
dieser Transformation den möglichen Maximalwert überschreiten, werden auf den
Maximalwert gesetzt.\\ 
Berechnet wird eine optimale Zuordnung im Sinne einer
minimalen Gesamtkostensumme. In der Integer-Matrix reference\_pairs
sind alle ermittelten optimalen Indexpaare enthalten. Diese Matrix 
besitzt 2 Spalten und höchstens min(m,n) Zeilen. Ist die Anzahl der 
Indexpaare kleiner als min(m,n) (return-Wert 1),so können für die nicht 
belegten Indizes beliebige Zuordnungen vom Nutzer selbst getroffen werden, 
da diese dann willkürlich sind. Auf min\_cost steht die optimale 
Gesamtkostensumme zur Verfügung.
}

\proc{int}{TimeWarp}{const Matrix \&cost,IMatrix \&reference\_pairs,
int mode=TW\_NORMAL}
\descr{
Es wird ein Zuordnungsproblem zwischen zwei Mengen gelöst, bei dem 
die Ordnung der Objekte beider Mengen vorgegeben ist. Beispiele für 
solche Mengen sind Funktionswerte diskreter Funktionen von der Zeit, 
wie sie z.B. für Sprachanalyse typisch sind. In der Bildverarbeitung 
treten dagegen oft als zyklische Funktionen die Merkmale von 
Konturpunkten auf.
Das Problem wird allgemein durch eine m*n-Kostenmatrix $cost$ beschrieben. 
Das Matrixelement cost[i][j] stellt die Kosten für die Zuordnung der 
Quelle i zum Ziel j dar. Lösung des Problems ist ein Weg in der Matrix 
mit minimaler Kostensumme, wobei jeder Punkt des Weges einer Zuordnung eines
Quellpunktes zu einem Zielpunkt entspricht.
TimeWarping ermittelt eine Zuordnung in Form eines Weges vom 
Startpunkt (0,0) zum Endpunkt (m,n).
Dieser Weg ist optimal im Sinne minimaler Gesamtkosten. 
In der Integer-Matrix reference\_pairs sind alle ermittelten optimalen
Indexpaare enthalten. Diese Matrix besitzt 2 Spalten. Jede Zeile enthält
ein Index-Paar der Zuordnung. Die Zuordnung kann mehrdeutig sein, ein
Objekt der einen Menge kann mehreren Objekten der zweiten Menge zugeordnet
sein.\\
Der Parameter $mode$ beschreibt die Vorgehensweise:\\
\begin{itemize}
\item mode==TW\_NORMAL \\
Die Wegsuche erfolgt vom Startpunkt (0,0) zum Endpunkt (m,n).
Zusätzlich können folgende Modifikationen angegeben werden: 
\begin{itemize}
\item mode==TW\_NORMAL||TW\_BIDIRECTIONAL \\
Es erfolgt eine zusätzliche Suche, bei der die Ordnung einer Menge 
invertiert (enstpricht einer Spiegelung) wird. Ergebnis ist das Minimum 
beider Suchschritte.
\item mode==TW\_NORMAL||TW\_SEARCHSTART\\
Die Funktionen werden als zyklisch angenommen (Konturen..). Jede Kombination
von Startpunkten wird untersucht. (Vorsicht, zeitaufwendig!)
\end{itemize}
\item mode==TW\_REDUCED\\
Die sehr zeitaufwendige Wegsuche für jeden möglichen Startpunkt bei 
zyklischen Funktionen kann in gutartigen Fällen drastisch verkürzt 
werden, wenn der feste Start- und Endpunkt aufgegeben wird. Dies 
geschieht durch Angabe von TW\_REDUCED. Es ist hier nicht mehr 
garantiert, dass sich ''der Kreis schließt''. Das bedeutet, dass eventuell 
eine Funktion nur einem Teil der anderen zugeordnet wird. 
Zusätzlich kann TW\_BIDIRECTIONAL angegeben werden. Dann erfolgt eine 
zusätzliche Suche in entgegengesetzter Richtung. In jedem Fall erfolgt 
die Wegsuche zyklisch.
\end{itemize}
\seealso{MatchObject}
}
\subsubsection{Objekt-Matching}
\proc{Trafo}{MatchObject}{(const Contur \&cont1,const Contur \&cont2,int tmode,double \&eval,int mmode=MM\_SQUARE,int anz=0)}
\procf{Trafo}{MatchObject}{(const Contur \&cont1,const Contur \&cont2,int tmode=TRM\_AFFINE)}
\procf{Trafo}{MatchObject}{(const Contur \&cont1,const Matrix \&pl2,int tmode,double \&eval,int mmode=MM\_SQUARE,int anz=0)}
\procf{Trafo}{MatchObject}{(const Contur \&cont1,const Matrix \&pl2,int tmode=TRM\_AFFINE)}
\procf{Trafo}{MatchObject}{(const Matrix \&pl1,const Matrix \&pl2,int tmode,double \&eval,int mmode=MM\_SQUARE,int anz=0)}
\procf{Trafo}{MatchObject}{(const Matrix \&pl1,const Matrix \&pl2,int tmode=TRM\_AFFINE)}
\procf{Trafo}{MatchObject}{(const Matrix \&pl1,const Contur \&cont2,int tmode,double \&eval,int mmode=MM\_SQUARE,int anz=0)}
\procf{Trafo}{MatchObject}{(const Matrix \&pl1,const Contur \&cont2,int tmode=TRM\_AFFINE)}
\descr{
Es wird eine Transformation (Klasse \see{Trafo}) ermittelt, die die Kontur 
$cont1$ in die Kontur $cont2$ überführt. 
Für den Standardfall $tmode$=TRM\_AFFINE wird eine affine
Transformation der Klasse $Trafo$ zurückgegeben. Für jeden Konturpunkt beider
Konturen werden aus Momenten affin invariante Merkmale berechnet und
dem Konturpunkt zugeordnet. Es wird die Abstandsmatrix aller Punkte der
einen zur anderen Kontur bezüglich dieser Invarianten berechnet. Bezüglich
dieser Abstandsmatrix werden mittels dynamischer Programmierung 
(\see{TimeWarp}) die Konturpunkte beider Konturen einander zugeordnet, 
es wird eine Liste von Punkt-Referenzen ermittelt. Aus dieser Liste 
wird die Transformation mittels der kleinsten Quadrate-Methode 
($mmode$=MM\_SQUARE, \see{MatchPointlists}) oder mit den kleinsten 
Absolutbeträgen (lineare Optimierung, $mmode$=MM\_LINEAR, 
\see{MatchPointlistsLinOpt}) ermittelt.\\
Die Bewertung des optimalen Weges in der Abstandsmatrix wird auf 
$eval$ zurückgeliefert und kann als Güte der Anpassung benutzt werden. \\
Wird die Methode der kleinsten Absolutbeträge benutzt, so kann die 
Rechenzeit wesentlich steigen. Durch $anz$ kann man vorgeben, wieviele
Konturpunkte in die Rechnung eingehen sollen, wobei $anz=0$ bedeutet, 
daß alle Konturpunkte benutzt werden. \\
Der gesamte Algorithmus ist äußerst robust gegenüber Störungen und 
Rauschen der Kontur. Auch Symmetrien
der Objekte haben keinen Einfluß auf die Güte. Der Algorithmus kann nur
versagen, wenn die Objekte nicht durch eine affine Transformation ineinander
überführbar sind, sondern z.B. durch starke projektive Verzerrungen.\\
Alternativ kann MatchObject auch mit Punklisten (Typ \see{Matrix}) aufgerufen 
werden. Dabei ist zu beachten, daß die Punkte in den Listen dicht 
liegen müssen, wie zum Beispiel in Listen, die durch Konvertierung 
aus Konturen entstanden sind.
}
\subsubsection{Winkelbestimmung von Objekten mit einer geschlossenen Kontur}
\procf{int}{OrientationMoments}{double moment[15],double \&angle}
\descr{Aus den (Flächen-)Momenten bis vierter Ordnung $moment[15]$ 
eines Objektes mit einer geschlossenen Kontur wird eine Orientierung 
berechnet. Zurückgegeben wird der Winkel $angle$ im Bogenmaß. 
Es werden zunächst die Momente zweiter Ordnung benutzt (Rückgabewert 1)
(Orientierung der Trägheitsellipse), wenn diese instabil sind 
(nahe dem Trägheitskreis) die Momente dritter Ordnung (Rückgabewert 2), 
wenn diese instabil sind, die Momente vierter Ordnung (Rückgabewert 3). 
Wenn alles instabil sein sollte, dann wird kein Winkel berechnet und 
es wird -1 zurückgegeben.  
}

\seealso{AffinFitMoments}
\seealsonext{MatchPointlists}
\seealsonext{MatchPointlistsLinOpt}

%%% Local Variables: 
%%% mode: latex
%%% TeX-master: "icedoc_base"
%%% End: 
