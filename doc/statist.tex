\nsection{Statistik}{statistics}
\subsection{Erzeugung von Zufallszahlen}

\proc{void}{Randomize}{}
\descr{Initialisierung des Zufallszahlengenerators}
\proc{int}{Random}{int val}
\descr{Es wird eine Realisierung einer im Intervall [0,$val$] gleichverteilten
Zufallsvariablen erzeugt.}
\proc{double}{RandomD}{}
\descr{Es wird eine Realisierung einer im Intervall [0,1] gleichverteilten
Zufallsvariablen erzeugt.}
\proc{double}{GaussRandom}{double sigma}
\descr{Es wird eine Realisierung einer normalverteilten Zufallsvariablen 
mit dem Erwartungswert Null und der Standardabweichung $sigma$ erzeugt}

\subsection{Multivariate Statistik}
\label{Statistics}
\hypertarget{Statistics}{}

Zur Ermittlung statistischer Maßzahlen dient die Klasse $Statistics$,
die einen Datenbereich und Methoden zur Erfassung statistischer 
Eigenschaften bereitstellt.

\proc{}{Statistics::Statistics}{}
\descr{
Standard-Konstruktor. Bevor ein derart angelegter Statistikbereich
genutzt werden kann, muß dieser mit Init() initialisiert werden. 
}
\proc{}{Statistics::Statistics}{int dim}
\descr{
Ein Statistikbereich für $dim$ Zufallsvariablen wird angelegt 
und initialisiert.
}
\proc{int}{Statistics::Init}{}
\descr{
Reinitialisieren eines Statistikbereiches. Alle akkumulierten
Werte werden zurückgesetzt.
}
\proc{int}{Statistics::Init}{int dim}
\descr{
Der Statistikbereich wird für $dim$ Zufallsvariablen eingerichtet
und initialisiert.
}

\proc{int}{Put}{Statistics \&st,const Vector \&val,double weight=1.0}
\descr{
Der Zufallsvariablenvektor $val$ wird in den Statistikbereich $st$
mit einem Gewicht von $weight$ aufgenommen
}

\proc{Vector}{Mean}{const Statistics \&st}
\descr{
Liefert den Mittelwert-Vektor des Statistikbereiches $st$.
}

\proc{double}{Weight}{const Statistics \&st}
\descr{
Liefert das Gesamtgewicht aller eingtragenen Werte des 
Statistikbereiches $st$.
}

\proc{Matrix}{Covariance}{const Statistics \&st}
\descr{
Liefert die Covarianz-Matrix des Statistikbereiches $st$.
}

\proc{Matrix}{Correlation}{const Statistics \&st}
\descr{
Liefert die Korrellations-Matrix des Statistikbereiches $st$.
}

\proc{int}{Write}{const Statistics \&st,const string \&fn}
\descr{
Schreibt den Statistikbereich $st$ in die Datei $fn$.
}

\proc{int}{Read}{Statistics \&st,const string \&fn}
\descr{
Lädt den Statistikbereich $st$ aus der Datei $fn$.
}

\proc{ostream \&}{operator <<}{ostream \&s,const Statistics \&t}
\descr{
Schreibt den Statistikbereich $st$ in den Stream $s$.
}

\proc{istream \&}{operator >>}{istream \&s,Statistics \&st}
\descr{
Liest den Statistikbereich $st$ aus dem Stream $s$.
}

Das folgende Beispiel bestimmt die Statistik der Farbwerte in einem
Farbbild:

\begprogr\begin{verbatim}
void Stat(Image ri,Image gi,Image bi)
{
 int x,y, r,g,b, cl;
 Statistics st(3); // Statistikbereich mit 3 Zufalls-Größen
 wloop(mi,x,y)
  {
    r=GetVal(ri,x,y); // Größen abfragen
    g=GetVal(gi,x,y);
    b=Getval(bi,x,y);
    Put(st,Vector(r,g,b)); // Eintragen (mit Gewicht 1)
  }
 // Ergebnisse ausgeben
 cout << "Mittelwerte" << endl << Mean(st) << endl;
 cout << "Kovarianz" << endl << Covariance(st) << endl;
 cout << "Korrelation" << endl << Correlation(st) << endl;
 return clf;
}
\end{verbatim}\endprogr

\subsection{Karhunen-Loewe-Transformation}

Die Karhunen-Loewe-Transformation kann aus einer bereits ermittelten
Statistik berechnet werden oder auch anhand einer Liste von
Merkmalsvektoren.

\proc{Matrix}{KLT}{const Matrix \&m}
\procf{Matrix}{KLT}{const Matrix \&m,Vector \&r}
\descr{Für die Stichprobe von Merkmalsvektoren (Zeilenvektoren der 
Matrix $m$) wird die Karhunen-Loewe-Transformation berechnet. Wird
der Vektor $r$ als Parameter angegeben, so wird auf dem Vektor $r$
für jedes transformierte Merkmal ein Relevanzmaß zurückgegeben.
}

\proc{Matrix}{KLT}{const Statistics \&st}
\procf{Matrix}{KLT}{const Statistics \&st,Vector \&r}
\descr{Aus der Information im Statistikbereich $st$ wird die 
Karhunen-Loewe-Transformation berechnet. Wird
der Vektor $r$ als Parameter angegeben, so wird auf dem Vektor $r$
für jedes transformierte Merkmal ein Relevanzmaß zurückgegeben.
}

\subsection{Histogramme}
Die Klassen \class{Histogram} und \class{Hist} dienen der 
Berechnung von Histogrammen für eindimensionale Zufallsgrößen.
Während die Klasse \class{Histogram} eine direkt Zuordnung der Werte
im Interval 0 .. nClasses-1 zu den Klassen-Nummern verwendet, erlaubt 
die Klasse \class{Hist} die Aufteilung eines Intervalls in jeweils
gleichgroße Klassen.

\subsubsection{Die Klasse {\bf Histogram}}
\label{Histogram}
\hypertarget{Histogram}{}
Die Klasse \class{Histogram} stellt eine einfach und schnelle 
Histogramm-Klasse zur Verfügung. In die knumber Klassen des Histogrammes 
werden ganzzahlige Werte von 0 .. {\bf knumber-1} eingeordnet, wobei 
jedem Zahlenwert eine Klasse entspricht. Typische Anwendungen sind 
Histogramme von Grauwerten und Markierungen. Eine variablere 
Zuordnung von Werten zu Klassen ermöglicht die Klasse \class{Hist}.

\subtitle{Konstruktoren}

\ctor{Histogram}{}{histogram.h}
\descr{Legt ein Histogramm an. Dieses ist noch ungültig, da keine
Klassenanzahl bekannt ist (\see{Histogram::isValid}).}

\ctor{Histogram}{int knumber}{histogram.h}
\descr{Legt ein Histogramm mit $knumber$ Klassen an, wobei den Klassen
die ganzzahligen Werte 0..knumber zugeordnet sind.}

\ctor{Histogram}{const Image \&img,int diff=1}{histogram.h}
\descr{Erzeugt ein Histogramm des Bildes $img$. Zur Beschleunigung ermöglicht
  es der Parameter $diff$, in x- und y-Richtung nur jeden diff-ten Wert zu 
berücksichtigen.}

\subtitle{Elementare Methoden}
\method{int}{reset}{}
\descr{Setzt das Histogram zurück, es ist danach wieder ungültig
(\see{Histogram::isValid}).}

\method{int}{reset}{int knumber}
\descr{Setzt das Histogram zurück und legt $knumber$ Klassen neu an.}

\method{int}{isValid}{}
\descr{Gibt wahr zurück, wenn das Histogramm gültig ist. Ein Histogramm ist
  ungültig, wenn es mit dem Standard-Konstruktor angelegt wurde oder mittels
  parameterlosem \see{reset} zurückgesetzt wurde.}

\method{int}{nClasses}{}
\descr{Gibt die Zahl der Klassen im Histogramm zurück.
  Der Wert ist undefiniert für ungültige Histogramme (\see{Histogram::isValid}).}

\method{int}{addValue}{int val,int count = 1}
\descr{Trägt den Wert $val$ in das Histogramm ein. Der optionale 
  Parameter $count$ erlaubt das mehrfache Eintragen eines Wertes. 
  Durch negative Werte von $count$ können Werte wieder ausgetragen werden.
Werte außerhalb des Intervalles $[0,nClasses-1]$ werden ignoriert.
}

\method{int}{getCount}{int nr}
\methodf{int}{operator []}{int nr}
\descr{Gibt der Zahl der in die Klasse $nr$ eingetragenen Werte zurück.}

\method{\vector{int}}{getCount}{}
\methodf{void}{getCount}{\vector{int} \&v}
\descr{Gibt die gezählten Werte pro Klasse in einem int-Vektor zurück.} 

\method{double}{getRelative}{int nr}
\descr{Gibt den Anteil der in die Klasse $nr$ eingetragenen Werte zurück.}

\method{\vector{double}}{getRelative}{}
\methodf{void}{getRelative}{\vector{double} \&v}
\descr{Gibt den Anteil der in die Klassen eingetragenen Werte als 
double-Vektor zurück.}

\subtitle{Komplexe Abfragen}

\method{int}{getLimits}{int \&min, int \&max}
\descr{Ermittelt den kleinsten und den größten Wert, für den Eintrage
  im Histogramms existieren.}

\method{int}{getLimits}{int \&min, int \&max, double quantil}
\descr{Ermittelt den kleinsten und den größten Wert, für den Eintrage
  im Histogramms existieren. Dabei bleiben die jeweils 
  $quantil \cdot sum$ größten und kleinsten Werte unberücksichtigt (Ausreißer).
}

\method{int}{Statistic}{int \&sum}
\methodf{int}{Statistic}{int \&sum,double \&xm,double \&xs}
\methodf{int}{Statistic}{int \&sum,double \&xm,double \&xs,double \&skew}
\descr{Ermittelt eine Schätzung der statistischem Maßzahlen für die in das
Histogramm eingetragenen Werte.
\begin{itemize}
\item sum - Anzahl der eingetragenen Werte
\item xm - Mittelwert
\item xs - Standardabweichung
\item skew - Schiefe
\end{itemize}
}

\proch{double}{Distance}{const Histogram \&h1,const Histogram \&h2}{histogram.h}
\descr{Berechnet die ''Earth Mover's Distance'' zwischen den beiden gegebenen
  Histogrammen. Die Implementierung ist beschränkt auf Histogramme mit gleicher
  Klassen-Anzahl.}

\subtitle{Ausgabe}

\method{int}{Vis}{int val,const Image \&img}
\descr{Stellt das Histogram im Bild $img$ mit dem Wert $val$ grafisch dar.}

\proch{int}{PrintHistogram}{const Histogram \& h}{histogram\_vis.h}
\descr{Gibt das Histogramm $h$ auf dem alphanumerischen Bildschirm aus.}

\subsubsection{Die Klasse {\bf Hist}}
\label{Hist}
\hypertarget{Hist}{}

Im Unterschied zu \class{Histogram} verwendet die Klasse \class{Hist}
Gleitkommawerte für die in die Klassen einzutragenden Werte. Dies ermöglicht
eine variablere Zuordnung. Wo dies nicht nötig ist, sollte aus
Effizienzgründen die Klasse \class{Histogram} verwendet werden.

\proch{}{Hist::Hist}{}{hist.h}
\descr{
  Es wird ein Histogramm angelegt. Dieses ist wegen fehlender
  Initialisierung noch ungültig ($valid=false$).
}

\proc{}{Hist::Hist}{const Hist \&h}
\descr{
  Es wird ein Histogramm als Kopie des Histogramms $h$ angelegt. 
}

\proc{}{Hist::Hist}{int knumber,double diff=1.0,double lower=-0.5}
\descr{
  Es wird ein Histogramm mit $knumber$ Klassen, der Klassenbreite $diff$ und der
  unteren Schranke $lower$ der untersten Histogrammklasse angelegt und
  initialisiert.
}

\proc{int}{Hist::Reset}{}
\descr{
  Das Histogramm wird zurückgesetzt und Speicher wird freigegeben. Es 
  ist damit (wieder) ungültig ($valid=false$).
}

\proc{int}{Hist::Reset}{int knumber,double diff=1.0,double lower=-0.5}
\descr{
  Das Histogramm wird zurückgesetzt und neu mit $knumber$ Klassen, der 
  Klassenbreite $diff$ und der unteren Schranke $lower$ der untersten 
Histogrammklasse angelegt und initialisiert.
}

\proc{int}{Hist::Add}{double val}
\descr{
Der Wert $val$ wird in das Histogramm eingetragen.
}

\proc{int}{Hist::Count}{int index}
\descr{
Die Zahl der in die Klasse $index$ einsortierten Werte
wird zurückgegeben.
}

\proc{double}{Hist::Rel}{int index}
\descr{
Die relative Zahl der in die Klasse $index$ einsortierten Werte
wird zurückgegeben.
}

\proc{int}{Hist::Limits}{double \&min,double \&max}
\descr{Der kleinste und der größte in das Histogramm eingetragene
Wert werden auf $min$ bzw. $max$ zurückgegeben.
}

\proc{int}{Hist::Limits}{double \&min,double \&max,double quant}
\descr{
Es wird ein Intervall von Klassen ermittelt, so daß der Anteil
der unterhalb bzw. oberhalb liegenden Werte im Histogram einen 
Anteil unter quant an der Gesamtzahl der Werte hat. Der mittlere
Wert der minimalen bzw. maximalen Klasse wird auf $min$ bzw. $max$
zurückgegeben.
}

\proc{Hist}{HistImg}{Image img,int diff=1,int clw=1}
\descr{
Es wird ein Grauwerthistogramm für das Bild $img$ angelegt, wobei für jeden
möglichen Grauwert eine Klasse vorgesehen wird. Für das Histogramm werden
Bildpunkte aus einem Raster mit dem Abstand $diff$ ausgewertet. Jede Klasse
des Histogramms hat die Breite $clw$.
}

\proc{int}{Hist::Statistic}{int \&nbr,double \&mean,double \&dispers}
\descr{
Für das Histogramm wird die Anzahl der eingetragenen Werte $nbr$, 
der Mittelwert $mean$ und die empirische Standardabweichung $dispers$
bestimmt.
}

\proch{int}{PrintHist}{const Hist \& h}{hist\_vis.h}
\descr{Das Histogramm $h$ wird auf dem Text-Bildschirm ausgegeben.}

\proc{int}{Hist::Vis}{int val,Image img}
\descr{
In das Bild $img$ wird das Histogramm $h$ eingezeichnet.
}

Die folgenden Werte können zur Abfrage der Histogramm-Eigenschaften
verwendet werden:

\begin{tabular}{ll}
int valid & Boolscher Wert Gültigkeit des Histogramms\\
int number & Zahl der Klassen des Histogramms \\
double minval & Untere Schranke der ersten Klasse \\
double diff & Klassenbreite \\
\end{tabular}

\subsection{Akkumulation}
Unter Akkumulation werde hier verstanden, dass eine große Zahl von Werten erfasst und
am Ende der häufigste Wert ermittelt werde. Dies ähnelt der Eintragung von Werten 
in ein Histogramm und nachfolgender Maximumsuche. Die Akkumulatorklassen enthalten 
eine effizientere Implementierung für die spezielle Aufgabe und reduzieren die 
Quantisierungseffekte: Dazu werden Einträge neben der eigentlichen Klasse mit 
geringerem Gewicht in die Nachbarklassen eingetragen.

Die Klasse {\bf accu1} realisiert eine eindimensionale Akkumulation über einen skalaren Wert. 
{\bf accu2} akkumuliert über 2-dimensionale Werte.

\subtitle{Konstruktoren und Initialisierung}
\label{accumulation}
\hypertarget{accumulation}{}

\proch{}{accu1::accu1}{}{accu.h}
\procf{}{accu2::accu2}{}
\descr{Der Standardkonstruktor legt einen noch ungültigen Akkumulator an. 
Vor einer weiteren Nutzung müssen die Parameter mittels \bsee{accu1::setDim} 
festgelegt werden.}

\proch{}{accu1::accu1}{int n,double min=0.0,double max=1.0,bool mod=false}{accu.h}
\procf{}{accu2::accu2}{int n1,double min1,double max1,bool mod1,
int n1,double min2, double max2,bool mod2,int smear=1}
\descr{Diese Konstruktoren legen ein Akkumulatorobjekt an und initialisieren es.
Die Parameter beschreiben die Zahl der Klassen ($n$ bzw. $n1$ und $n2$), den Wertebereich 
der zu berücksichtigenden Werte ($min$ ... $max$) und legen fest, ob die Eingangsgröße
zyklisch behandelt werden soll, wie zum Beispiel Winkelgrößen.
Der Parameter smear legt fest, dass ein Wert neben der eigenen Klasse 
auch in die Nachbarklassen in einer smear*smear-Umgebung einzutragen ist. Das Gewicht 
fällt dabei proportional zum Abstand ab.
}

\proch{void}{accu1::setDim}{int n,double min=0.0,double max=1.0,bool mod=false}{accu.h}
\descr{Diese Methode legt die Parameter für die Eingangsgröße fest.
Die Parameter beschreiben die Zahl der Klassen ($n$), den Wertebereich 
der zu berücksichtigenden Werte ($min$ ... $max$) und legen fest, ob die Eingangsgröße
zyklisch behandelt werden soll, wie zum Beispiel Winkelgrößen.
}

\proch{void}{accu2::setDim}{int dim,int n,double min=0.0,double max=1.0,bool mod=false}{accu.h}
\descr{Diese Methode legt die Parameter für eine Dimension der Eingangsgröße fest.
Für dim sind die Werte 0 und 1 möglich.
Die Parameter beschreiben die Zahl der Klassen ($n$), den Wertebereich 
der zu berücksichtigenden Werte ($min$ ... $max$) und legen fest, ob die Eingangsgröße
zyklisch behandelt werden soll, wie zum Beispiel Winkelgrößen.
}

\subtitle{Eintrag von Werten}

\proch{void}{accu1::Add}{double val}{accu.h}
\descr{Trägt den Wert $val$ in das Akkumulatorobjekt ein.}

\proch{void}{accu2::Add}{double val1,double val2}{accu.h}
\procf{void}{accu2::Add}{const Vector \&v}
\procf{void}{accu2::Add}{Point v}
\descr{Trägt das Wertpaar $val1$/$val2$ in das Akkumulatorobjekt ein. Alternativ können die Werte 
als \bsee{Vector} oder als \bsee{Point} angegeben werden.}

\subtitle{Abfrage}

\proch{void}{accu1::getMax}{double \&val}{accu.h}
\procf{void}{accu1::getMax}{double \&val,double \&ct}
\descr{Ermittelt den am häufigsten aufgetretenen Wert und gibt ihn auf $val$ zurück. $ct$ wird
auf den akkumulierten Wert, ein Maß für die Häufigkeit, gesetzt.}

\proch{void}{accu2::getMax}{double \&val1,double \&val2}{accu.h}
\procf{void}{accu2::getMax}{double \&val1,double \&val2,double \&ct}
\procf{Point}{accu2::getMax}{}
\procf{Point}{accu2::getMax}{double \&ct}
\descr{Ermittelt den am häufigsten aufgetretenen Wert und gibt ihn auf $val1$/$val2$ zurück. $ct$ wird
auf den akkumulierten Wert, ein Maß für die Häufigkeit, gesetzt. Alternativ kann der Wert (das Wertepaar) 
als \bsee{Point} zurückgegeben werden.}

\subsection{Geradendetection durch Akkumulation}
Bei der Geradendetektion durch \hyperlink{accumulation}{Akkumulation} wird eine Liste von Punkten ausgewertet, die Kandidaten für 
Geradenpunkte darstellen. Für alle bzw. eine gegebene Anzahl von Paaren von diesen Punkten werden die 
Parameter der Gerade berechnet, die durch diese Punkte geht. Mittels \hyperlink{accumulation}{Akkumulation} wird 
die Gerade ermittelt, die am häufigsten aufgetreten ist und die damit durch die meisten Punkte geht.

\proch{LineSeg}{DetectLine}{const vector<Point> \&pointlist}{accu.h}
\procf{LineSeg}{DetectLine}{const vector<Point> \&pointlist,int pairs}
\procf{LineSeg}{DetectLine}{const Matrix \&pointlist}
\procf{LineSeg}{DetectLine}{const Matrix \&pointlist,int pairs}
\descr{Ermittelt durch Akkumulation eine Gerade, die durch möglichst viele Punkte der übergebenen 
Punktliste $pointlist$ geht. Im ersten Fall werden alle möglichen Punktpaare verwendet, im zweiten Fall wird die
Auswertung auf $pairs$ zufällig gewählte Punktpaare beschränkt.}

%%% Local Variables: 
%%% mode: latex
%%% TeX-master: "icedoc_base"
%%% End: 
