\section{Steuerung für Roboter RX90L}
\label{RoboterSteuerung}

Diese Funktionen dienen zur Kommunikation mit dem Controller des Stäubli
RX90. Auf dem Kontroller muß vorher ein Interpreter gestartet werden, der die
Kommandos, die durch die folgenden Funktionen gesendet werden, ausführt
(inter.v2, Programm ``interpreter''). Der Start des Interpreters erfolgt
im Normalfall automatisch beim Start des Roboters.

{\em Es kann nicht überprüft werden, ob die angegebenen Positionen
kollisionsfrei erreicht werden können!!! Die Bewegungsbefehle sind deshalb mit
entsprechender Vorsicht zu verwenden. Ein Not-Aus-Schalter (Teach-Panel oder
Taster-Box) muß immer unmittelbar erreichbar sein.}

Bewegungen werden nur ausgeführt, wenn die ``Tote Mann''-Taste gedrückt
wird (Taste 1 am Teach-Panel bzw. grüne Taste an der Taster-Box). Ist 
die Bewegung des Roboters durch die ``Tote Mann''-Taste angehalten, kann 
diese Bewegung durch das Drücken der Taste STOP am Teach-Panel abgebrochen 
werden. Ist dies nicht erfolgreich, sollte die Notaus-Taste verwendet 
werden. Dann ist aber ein Neustart des Interpreters erforderlich.

\proch{int}{RobOpen}{void}{robcntl.h}
\descr{
Die serielle Schnittstelle zum Roboter wird zum Lesen und Schreiben 
geöffnet.
}
\proc{void}{RobClose}{void}
\descr{
  Die serielle Verbindung zum Roboter wird geschlossen.
}
\proc{int}{RobInit}{}
\descr{
Es wird eine Initialisierung der Controller-Schnittstelle durchgeführt und eine
definierte Konfiguration des Roboters eingestellt (lefty, above, noflip).
}
\proc{int}{RobReset}{}
\descr{
Es wird eine Initialisierung der Controller-Schnittstelle durchgeführt und eine
definierte Konfiguration des Roboters eingestellt (lefty, above, noflip). Der
Roboter bewegt sich zu einer Startposition.
}
\proc{int}{RobSetSpeed}{int speed}
\descr{
Die Geschwindigkeit der Roboterbewegungen wird festgelegt. Der Wert $speed$
muß im Intervall $[0,100]$ liegen und gibt einen Prozentwert der 
einstellbaren Maximalgeschwindigkeit an.
}
\proc{int}{RobGripper}{int stat}
\descr{
Mit $stat=0$ wird der Greifer geöffnet, mit $stat=1$ kann der Greifer
geschlossen werden.Mit $stat=-1$ greift der Greifer ``nach aussen''.
}
\proc{int}{RobTeach}{const string \&prompt}
\descr{
Es wird ein Umschalten in den Teach-Modus ermöglicht. Die Zeichenkette
$prompt$ wird auf dem Teach-Pendant ausgegeben. Es kann mit der ``MAN''-Taste
das Teach-Pendant aktiviert und der Roboter manuell bewegt werden. 
Durch Drücken der ``COMP''-Taste wird die Steuerung an das Programm 
zurückgegeben.
}
\proc{int}{RobMove}{double par[6],int mode}
\procf{int}{RobMove}{const Vector \&v,int mode}
\descr{
  Ein Movebefehl wird an den Roboter gesendet.
  $par$ bzw. $v$ enthält die drei Koordinaten X, Y, Z und 
  die drei Winkel yaw, pitch und roll in dieser
  Reihenfolge. Mit $mode==0$ wird die Bewegung im Achsinterpolationsmodus 
  ausgeführt, mit $mode!=0$ im Linearmodus (Bewegung auf einer Strecke).
  Bei fehlerfreier Ausführung des  MOVE-Befehls ist der Rückgabewert 0, 
  negative Werte signalisieren fatale Fehler, positive Werte ein 
  Überschreiten des Arbeitsbereiches, wobei die
  Ursache binär kodiert ist. Die Bits 1 bis 6 stehen für eine
  Bereichsüberschreitung der Achsen 1 bis 6, Bit 14 für eine Position 
  zu nahe an der Roboterbasis und Bit 15 für eine Position zu weit außerhalb.
}
\proc{int}{RobMoveFrame}{Frame *f,int mode}
\descr{Ein Movebefehl wird an den Roboter gesendet. $f$ ist eine homogene
  Transformationsmatrix, die die gewünschte Lage und Orientierung des
  Werkzeugkoordinatensystems im Basiskoordinatensystem beschreibt. Mit
  $mode==0$ wird die Bewegung im Achsinterpolationsmodus ausgeführt, mit
  $mode!=0$ im Linearmodus (Bewegung auf einer Strecke).
 Bei fehlerfreier Ausführung des MOVE-Befehls ist der Rückgabewert 0,
negative Werte signalisieren fatale Fehler, positive Werte ein Überschreiten
des Arbeitsbereiches, wobei die Ursache binär kodiert ist. Die Bits 1 bis 6
stehen für eine Bereichsüberschreitung der Achsen 1 bis 6, Bit 14 für eine
Position zu nahe an der Roboterbasis und Bit 15 für eine Position zu weit
außerhalb. 
}
\proc{int}{RobCP}{int on}
\descr{
  Die Funktion schaltet das Verschleifen der Bahn (Continous Path) ein
  ($on=1$) oder aus ($on=0$). Grundzustand nach RobInit ist
  CP=on. Im Fehlerfall ist der Funktionswert ERROR, sonst OK.
}

\proc{int}{RobWaitStop}{int total}
\proc{int}{RobWaitStop}{}
\descr{
  Die Funktion wartet, bis die laufende Bewegung des Roboters beendet ist.
  Ist $total==1$ oder erfolgt der Aufruf ohne diesen Parameter, so wird der
  vollständige Stillstand nach Erreichen der Position abgewartet. 
  Ist $total==0$, so wird abgebrochen, wenn die Abweichung von der 
  Sollposition klein genug ist, um sicher greifen 
  zu können, ohne daß der Roboter vollständig zum Stillstand gekommen sein 
  muß. 
  Im Fehlerfall ist der Funktionswert ERROR, sonst OK. 
}

\proc{int}{RobGetPos}{double par[6]}
\procf{int}{RobGetPos}{Vector \&v}
\descr{
  Die aktuelle Position des Roboters wird abgefragt. $par$ wird mit den drei
  Koordinaten X, Y, Z und den drei Winkeln yaw, pitch und roll in dieser
  Reihenfolge belegt. Im Fehlerfall ist der Funktionswert ERROR, sonst OK. 
}
\proc{int}{RobGetFrame}{Frame *f}
\descr{
  Die aktuelle Position des Roboters wird abgefragt. $f$ wird mit einer
  homogenen Transformationsmatrix belegt, die die Lage und Orientierung des
  Werkzeugkoordinatensystems im Basiskoordinatensystem beschreibt. Im
  Fehlerfall ist der Funktionswert ERROR, sonst OK. 
}
\proc{int}{RobSetTool}{Frame *f}
\descr{
  Das Werkzeugkoordinatensystem wird entsprechend der homogenen
  Transformationsmatrix $f$ festgelegt, die die Lage und Orientierung
  bezüglich des Werkzeugflansches des Roboters angibt. Im Fehlerfall ist der
  Funktionswert ERROR, sonst OK. 
}
