\lsection{Projektive Abbildungen und Kalibrierung}

\subsection{Punktkorrespondenzen}

Für die Kamerakalibrierung müssen Korrespondenzpaare von Raum- und Bildpunkten
vorgegeben werden. Zur Handhabung solcher Punktkorrespondenzen dient die
Datenstruktur $PointCorr$:
\begprogr\begin{verbatim}
typedef struct
{
double pw[3];   /* Raumpunktkoordinaten */
double pc[2];   /* Bildebenenkoordinaten */
double w;       /* Gewichtskoeffizient */
int status;     /* Status der Bildebenenkoordinaten
                   OK      - gültiger Abbildungspunkt
                   INVALID - ungültigen Abbildungspunkt */
} PointCorr;
\end{verbatim}\endprogr

\proc{int}{WritePointCorr}{PointCorr *pcr,int n,char *file}
\descr{
Die ersten $n$ Einträge der Punktkorrespondenzliste $pcorr$ werden auf das
File $file$ ausgegeben.
}
\proc{PointCorr*}{ReadPointCorr}{char *pcfile,int *n}
\descr{
Vom File $pcfile$ wird eine Liste von Punktkorrespondenzen eingelesen. Die
Anzahl der gelesenen Einträge wird auf $n$ zurückgegeben.
}
\proc{int}{CopyPointCorr}{PointCorr *pcs,PointCorr *pcd}
\descr{
Die Punktkorrespondenzliste $pcs$ wird nach $pcd$ kopiert.
}
\proc{int}{GetTriangle}{PointCorr *pcorr,int n,PointCorr pc[3]}
\descr{
Aus der Punktkorrespondenzliste $pcorr$ mit $n$ Elementen wird ein Dreieck mit
möglichst großer Fläche ausgewählt. Die Korrespondenzen für dessen Eckpunkte
werden auf $pc$ zurückgegeben.
}


\subsection{Kamerakalibrierung}

Kameras werden durch ihre projektiven Parameter beschrieben. Auf diese intern
verwalteten Parameter kann über einen Pointer vom Typ $camera$ zugegriffen
werden.
\begprogr\begin{verbatim}
typedef struct Camera_
{
  double a[3][4];             /* projektive Kameraparameter */
  double ac[3][3];            /* inverse Projektionsmatrix */
  double c[3];                /* Zentrum der Projektion */
  double pp[2];               /* Hauptpunkt */
  int has_distortion;         /* Flag Verzeichnungparameter */
  double k2,k4;               /* Verzeichnungsparameter */
} *camera;
\end{verbatim}\endprogr

\proc{camera}{CalibCamera}{camera cam,PointCorr *pc,int nbr,int flag,double
*mse}
\descr{
Aus den Punktkorrespondenzen $pc$ werden die Kameraparameter berechnet und in
die Struktur $cam$ eingetragen. Wenn für $cam$ der NULL-Pointer übergeben
wird, wird die Kamerastruktur implizit angelegt und als Rückgabewert
übergeben. $nbr$ gibt die Anzahl der Punktkorrespondenzen an, die mindestens 6
betragen muß. Durch $flag$=TRUE kann die Berechnung von
Verzeichnungsparametern erzwungen werden.
}
\proc{camera}{CopyCamera}{camera cam}
\descr{
Es wird eine Kamerastruktur angelegt, die mit den Daten von $cam$ gefüllt
wird.
}
\proc{int}{WriteCamera}{camera cam, char name[]}
\descr{
Die Parameter der Kamera $cam$ werden in ein Textfile $name$ geschrieben.
}
\proc{camera}{ReadCamera}{char name[]}
\descr{
Es wird eine Kamerastruktur angelegt, deren Parameter aus dem File $name$
gelesen werden.
}
\proc{int}{DistortPoint}{camera cam,double p[2]}
\descr{
Zu dem Bildpunkt $p$ wird mit den Verzeichnungsparametern aus $cam$ der
zugehörige verzeichnete Punkt berechnet.
}
\proc{int}{RectifyPoint}{camera cam,double p[2]}
\descr{
Zu dem Bildpunkt $p$ wird mit den Verzeichnungsparametern aus $cam$ der
zugehörige unverzeichnete (rektifizierte) Punkt berechnet.
}
\proc{int}{CalcPhysicalParameters}{camera cam,double par[5]}
\descr{
Aus den projektiven Parametern der Kamera werden die inneren Kameraparameter
berechnet. $par[0],par[1]$ sind die Hauptpunktkoordinaten, $par[2]$ und
$par[3]$ sind die Produkte aus Brennweite und Skalierung in x- bzw. y-Richtung
und $par[4]$ ist der Winkel zwischen der x- und y-Achse der Bildebene.
}
\proc{camera}{CalcProjectiveCamera}{double par[5]}
\descr{
Es wird eine Kamerastruktur angelegt, deren projektive Parameter aus den
inneren Kameraparametern $par$ berechnet werden. Die erzeugte Kamera befindet
in einer Normallage sich mit ihrem Projektionszentrum im Ursprung des
Weltkoordinatensystems.
} 
\proc{double}{CalibHandCamera}{Frame *hb,int n,PointCorr **c,int m,double p[5],Frame *ch,Frame *bw}
\descr{
Mit Hilfe der Felder $hb[n]$ für die Hand-Basis-Transformationen und c[n][m] für
die Punktkorrespondenzen aus $n$ verschiedenen Kamerapositionen werden durch
Ausgleichsrechnung die inneren Kameraparameter $p$, die
Kamera-Hand-Transformation $ch$ und die Basis-Welt-Transformation $bw$ berechnet.
}
Für die Kalibrierung mit dem vorhandenen Kalibrierkörper sind Funktionen
vorbereitet, die an die spezielle Form und Lage der Paßpunktmarken
(Doppelkreuze) angepaßt sind.
\proc{camera}{InitManCamera}{Image imgv,Image imgo}
\descr{
Interaktive Initialisierung der Kamerakalibrierung. Das Bild $imgv$ muß eine
Abbildung des Kalibrierkörpers enthalten. Zu den vorgegebenen Paßpunktmarken
(Nummer) muß die zugehörige Abbildung ausgewählt werden (durch Klicken in das
Innere des Doppelkreuzes). Wenn auf diese Weise sechs Punktkorrespondenzen
gefunden sind, wird eine projektive Abbildung angepaßt und als Kamerastruktur
zurückgegeben.
}
\proc{int}{Calibrate}{Image imgv,Image imgo,camera cam,int cnt,int distort}
\descr{
Mit Hilfe der initialen Kamera $cam$ werden im Bild $imgv$ alle Paßpunktmarken
gesucht und zugeordnet. Es wird vorausgesetzt, daß $imgv$ eine Abbildung des
Kalibrierkörpers enthält. Durch $cnt$ kann eine Anzahl von Aufnahmen angegeben
werden, über die die Paßpunkte gemittelt werden sollen. Dazu werden intern vom
aktuell ausgewählten Videokanal (s. SelChannel) Bilder eingelesen. Mit
$distort=TRUE$ werden zusätzlich Parameter für radialsymmetrische
Verzeichnungen bestimmt. Die aktualisierten Kameraparameter werden auf $cam$
bereitgestellt.
}
\proc{int}{CalibrateImg}{Image imgv,Image imgo,camera cam,int distort}
\descr{
Mit Hilfe der initialen Kamera $cam$ werden im Bild $imgv$ alle Paßpunktmarken
gesucht und zugeordnet. Es wird vorausgesetzt, daß $imgv$ eine Abbildung des
Kalibrierkörpers enthält. Mit $distort=TRUE$ werden zusätzlich Parameter für
radialsymmetrische Verzeichnungen bestimmt. Die aktualisierten Kameraparameter
werden auf $cam$ bereitgestellt.
}


\subsection{Abbildung, Sehstrahlen, Raumpunkte}

\proc{int}{MapPoint}{camera cam,double pw[3],double pc[2]}
\descr{
Der Raumpunkt $pw$ wird mit der Kamera $cam$ in den Bildpunkt $pc$ abgebildet.
}
\proc{int}{ViewRay}{camera cam, double p[2], double t[3]}
\descr{
Es wird ein Einheitsvektor $t$ berechnet, der die Richtung des Sehstrahls vom
Bildpunkt $p$ durch das Projektionszentrum der Kamera $cam$ hat.
}
\proc{int}{ViewPlane}{camera cam,double p,double phi,double n[3]}
\descr{
Es wird der Normalenvektor $n$ der Raumebene berechnet, die durch das
Projektionszentrum der Kamera $cam$ und die durch die Hesseschen Parameter $p$ und $phi$
charakterisierte Bildgerade bestimmt ist.
}
\proc{int}{SpacePoint}{camera cam[],double pl[][2],int nbr,double pw[3],double *mse}
\descr{
Durch Ausgleichsrechnung wird der Schnittpunkt der Sehstrahlen berechnet, die
jeweils vom Bildpunkt $p_i$ der Kamera $cam_i$ ausgehen.
}

