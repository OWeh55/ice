\nsection{Signal-Transformationen}{signalTrafo}

\subsection{Fourier- und Hartleytransformation - Operationen im Frequenzbereich}

Hauptanwendungen der Fourier- und der Hartleytransformation sind 
Operationen im Zusammenhang mit Faltung und Korrelation. Diese 
Operationen sind im Frequenzbereich einfacher auszuführen. Während
die Fouriertransformation die Arbeit mit komplexen Zahlen und Bildern
erfordert, ist die Hartleytransformation eine Transformation aus dem
reellen Zahlenbereich in den reellen Zahlenbereich. Zu den im 
Frequenzbereich möglichen Operationen auf Fourierspektren gibt es 
äquivalente Operationen für die Hartleytransformierte.

\vspace{1em}
\noindent Allgemeine Hinweise:
\begin{itemize}
\item Alle Funktionen dieser Gruppe arbeiten mit Gleitkommazahlen bzw.
Bildern mit Gleitkommawerten.
\item Die schnellen Algorithmen für die Fourier- und Hartleytransformation
erfordern Werte-Anzahlen, die Potenzen von 2 darstellen (2,4,8,16..).
Wenn diese Bedingung erfüllt ist, arbeiten die Funktionen wesentlich 
schneller.
\item Aus Geschwindigkeits- und Speichergründen ist die Hartley-Transformation
vorzuziehen, wenn das Problem damit lösbar ist.
\end{itemize}


\subsubsection{Fourier- und Hartleytransformation}

\proc{int}{Fourier}{const Vector \&re1,const Vector \&im1,int mode,
Vector \&re2,Vector \&im2}
\procf{int}{Fourier}{Vector \&re,Vector \&im,int mode=NORMAL}
\descr{
Die durch die Vektoren $re1$ und $im1$ gegebene Folge von komplexen 
Zahlen wird transformiert ($mode$=NORMAL: Hintransformation, $mode$=INVERS:
Rücktransformation). Das Ergebnis der Transformation wird auf den Vektoren
$re2$ und $im2$ bzw. den Ausgangvektoren bereitgestellt. Der Wert bei 
$re2[0]$ und $im2[0]$ ist der Gleichanteil.
}

\proc{int}{Fourier}{const Matrix \&src,int mode,Matrix \&dst}
\procf{int}{Fourier}{Matrix \&m,int mode=NORMAL}
\descr{
Die durch die Matrix gegebene Folge von komplexen Zahlen wird 
transformiert ($mode$=NORMAL: Hintransformation, $mode$=INVERS:
Rücktransformation). Das Ergebnis der Transformation wird auf der Matrix
$dst$ bzw. der Ausgangsmatrix bereitgestellt. Die Matrix muß 2 Zeilen
oder 2 Spalten aufweisen. Die erste Zeile (Spalte) enthält die Real-Teile,
die zweite Zeile (Spalte) die Imaginär-Teile. Der Gleichanteil liegt 
beim Index 0.
}

%\proc{int}{FourierD}{double *re1,double *im1,int nbr,int mode,double
%*re2,double *im2}
%\descr{
%Die durch die Felder $re1$ und $im1$ gegebene Folge von komplexen Zahlen wird
%transformiert ($mode$=NORMAL: Hintransformation, $mode$=INVERS:
%Rücktransformation). Das Ergebnis der Transformation wird auf den Feldern
%$re2$ und $im2$ bereitgestellt. Der Wert bei $re2[0]$ und $im2[0]$ ist der
%Gleichanteil.
%}

\proc{int}{Hartley}{const Vector \&src,const Vector \&dst}
\procf{int}{Hartley}{Vector \&src}
\descr{
Die durch den Vektor $src$ gegebene Folge von reellen Zahlen wird
transformiert. Das Ergebnis der Transformation wird auf dem Vektor
$dst$ bzw. dem Ausgangsvektor bereitgestellt. Der Wert bei $dst[0]$ 
ist der Gleichanteil.
}

%\proc{int}{HartleyD}{double *src,int nbr,double *dst}
%\descr{
%Die durch das Feld $src$ gegebene Folge von reellen Zahlen wird
%transformiert. Das Ergebnis der Transformation wird auf dem Feld
%$dst$ bereitgestellt. Der Wert bei $dst[0]$ ist der Gleichanteil.
%}

\proc{void}{FourierImgD}{ImageD re1,ImageD im1,int mode,ImageD re2,ImageD
  im2,int dir=XY}
\descr{
  Das durch Realteil $re1$ und Imaginärteil $im1$ gegebene komplexe Bild wird
  transformiert ($mode$=NORMAL: Hintransformation, $mode$=INVERS: Rücktransformation)
  und in den Bildern $re2$ und $im2$ bereitgestellt. Die Koordinaten werden
  hier auf den Wertebereich $[-\frac{n}{2},\frac{n}{2}-1]$ abgebildet, so 
  daß die Frequenz 0 in der Mitte des Bildes liegt.
  \noindent Folgende Werte für $dir$ sind möglich:
  \begin{itemize}
  \item XY - Die komplette 2D-Transformation (in X- und Y-Richtung) wird
    durchgeführt
  \item X\_ONLY - Es wird nur die Transformation in X-Richtung durchgeführt.
  \item Y\_ONLY - Es wird nur die Transformation in Y-Richtung durchgeführt.
  \end{itemize}
}

\proc{int}{HartleyImgD}{ImageD src,ImageD dest,int dir=XY}
\descr{
  Das durch $src$ gegebene Bild wird transformiert und im Bild $dest$ 
  bereitgestellt. Die Koordinaten werden hier auf den Wertebereich 
  $[-\frac{n}{2},\frac{n}{2}-1]$ abgebildet, so daß die Frequenz 0 
  in der Mitte des Bildes liegt.
  \noindent Folgende Werte für $dir$ sind möglich:
  \begin{itemize}
  \item XY - Die komplette 2D-Transformation (in X- und Y-Richtung) wird
    durchgeführt
  \item X\_ONLY - Es wird nur die Transformation in X-Richtung durchgeführt.
  \item Y\_ONLY - Es wird nur die Transformation in Y-Richtung durchgeführt.
  \end{itemize}
}

\subsubsection{Abgeleitete Spektren und Darstellungen}

\proc{int}{PowerSpectrumFImgD}{ImageD re,ImageD im,ImageD imgd,int mode=MD\_POWER}
\descr{
  Es wird das Leistungsspektrum eines Bildes aus dessen fouriertransformierten 
  Bild berechnet. 
  Dazu wird punktweise das Betragsquadrat der komplexen Bildpunkte berechnet.
  Es sind folgende Modi möglich:
  \begin{itemize}
  \item MD\_POWER Leistungsspektrum
  \item MD\_MAGNITUDE Amplitudenspektrum (Quadratwurzel des Leistungsspektrums)
  \item MD\_LOG logarithmiertes Leistungspektrum
  \end{itemize}
}

\proc{int}{PowerSpectrumHImgD}{ImageD src,ImageD imgd,int mode=MD\_POWER}
\descr{
  Es wird das Leistungsspektrum eines Bildes aus dessen Hartleytransformierten 
  Bild berechnet.
  Es sind folgende Modi möglich:
  \begin{itemize}
  \item MD\_POWER Leistungsspektrum
  \item MD\_MAGNITUDE Amplitudenspektrum (Quadratwurzel des Leistungsspektrums)
  \item MD\_LOG logarithmiertes Leistungspektrum
  \end{itemize}
}

\proc{int}{PowerSpectrumImgD}{ImageD imgs,ImageD imgd,int mode=MD\_POWER}
\descr{
  Es wird das Leistungsspektrum des Bildes $imgs$ berechnet.
}

\proc{int}{MPSpectrumFImgD}{ImageD re,ImageD im,ImageD mag,ImageD phase}
\descr{
  Aus dem komplexen Fourierspektrum in den Bildern $re$ und $im$ wird 
  die Betrags-/Phasen-Darstellung berechnet und in den Bildern $mag$ (Betrag)
  und $phase$ (Phase) gepeichert.
}

\proc{int}{MPSpectrumHImgD}{ImageD sp,ImageD mag,ImageD phase}
\descr{
  Aus der Hartleytransformierten $sp$ wird die Betrags-/Phasen-Darstellung 
  berechnet und in den Bildern $mag$ (Betrag) und $phase$ (Phase) gepeichert.
}

\proc{int}{MPSpectrumImgD}{ImageD im,ImageD mag,ImageD phase}
\descr{
  Aus dem Bild $im$ wird die Betrags-/Phasen-Darstellung der 
  Fouriertransformierten berechnet und in den Bildern $mag$ (Betrag) 
  und $phase$ (Phase) gepeichert.
}

\proc{int}{FourierMPImgD}{ImageD mag,ImageD phase,ImageD re,ImageD im}
\descr{
  Die Betrags-/Phasen-Darstellung der Fouriertransformierten in den 
  Bildern $mag$ und $phase$ wird in die komplexe Darstellung in den Bildern 
  $re$ (Realteil) und $im$ (Imaginärteil) umgerechnet.
}

\proc{int}{HartleyMPImgD}{ImageD mag,ImageD phase,ImageD dst}
\descr{
  Aus der Betrags-/Phasen-Darstellung der Fouriertransformierten in den 
  Bildern $mag$ und $phase$ wird die reellwertige Hartleytransformierte
  im Bild $dst$ umgerechnet. Dabei wird vorausgesetzt, daß das 
  das Ausgangsbild das Spektrum eines reellen Bildes ist. Die andernfalls
  entstehenden imaginären Anteile werden ignoriert.
}

\proc{int}{CepstrumImgD}{ImageD imgs,ImageD imgd}
\descr{
  Es wird das Cepstrum des Bildes $imgd$ berechnet. 
  Das Cepstrum ist das Leistungsspektrum des logarithmierten
  Leistungsspektrums).
}

\subsubsection{Faltung und Kreuzkorrelation}

Es stehen jeweils Varianten der Faltung, inversen Faltung und 
Kreuzkorrelation zur Verfügung, die die entspechende Operation
im Ortsbereich(Grauwertbild mit Gleitkommawerten), im Frequenzraum 
der Fouriertransformation und im Frequenzraum der Hartleytransformation
ausführen. Bei den Operationen im Ortsbereich erfolgt intern eine
Transformation in den Frequenzraum. Die bei der Fouriertransformation
gemachten Aussagen bezüglich der Wahl der Bildgrößen für eine optimale
Abarbeitungsgeschwindigkeit gelten deshalb auch hier.

\proc{int}{Convolution}{const Vector \&src1,const Vector \&src2,Vector \&dst}
\descr{
  Die Funktion im Vektor $src1$ wird mit dem Vektor $src2$ gefaltet und 
  das Ergebnis im Vektor $dst$ gespeichert.
}

\proc{int}{ConvolutionImgD}{ImageD im1,ImageD im2,ImageD dst,int mode=MD\_USE\_BIAS}
\descr{
  Das Bild $im1$ wird mit dem Bild $im2$ gefaltet und das Ergebnis in Bild
  $dst$ gespeichert.
}

\proc{int}{ConvolutionFImgD}{ImageD re1,ImageD im1,
  ImageD re2,ImageD im2,ImageD re3,ImageD im3}
\descr{
  Die der Faltung entsprechende Operation wird auf den komplexen
  Fouriertransformierten ($re1$,$im1$) und ($re2$,$im2$) ausgeführt und
  das Ergebnis (Fouriertransformierte des gefalteten Bildes) in den Bildern
  $re3$,$im3$ gespeichert.
}
\proc{int}{ConvolutionHImgD}{ImageD im1,ImageD im2,ImageD im3}
\descr{
  Die der Faltung entsprechende Operation wird auf den  
  Hartleytransformierten $im1$ und $im2$ ausgeführt und
  das Ergebnis (Hartleytransformierte des gefalteten Bildes) im Bild
  $im3$ gespeichert.
}

\proc{int}{ConvolutionImg}{const Image \&im1,const Image \&im2,
  Image \&dst,double factor=0.0,int mode=MD\_USE\_BIAS}
\descr{
  Das Bild $im1$ wird mit dem Bild $im2$ gefaltet und das Ergebnis in Bild
  $dst$ gespeichert. Um eine optimale Umwandlung in Integer-Werte zu
  ermöglichen, kann der Faktor $factor$ angegeben werden, mit dem alle Werte
  vor der Konvertierung multipliziert werden. Ist $factor=0.0$, so wird eine
  automatische Grauwert-Skalierung so vorgenommen, daß der Wert mit dem
  maximalen Betrag den Werte-Bereich genau ausschöpft.
}

\proc{int}{ConvolutionImg}{const Image \&im1,const Image \&im2,
ImageD dst, int mode=MD\_USE\_BIAS}
\descr{
  Das Bild $im1$ wird mit dem Bild $im2$ gefaltet und das Ergebnis im 
Gleitkomma-Bild $dst$ gespeichert. 
}

\proc{int}{InvConvolution}{const Vector \&src1,const Vector \&src2,double beta,Vector \&dst}
\descr{
  Es wird der Vektor $dst$ berechnet, der gefaltet mit Vektor $src1$ den 
  Vektor $src2$ ergibt (pseudo-inverse Faltung). Der vorzugebende 
  Parameter $beta$ beschreibt den Rauschanteil in den Ausgangsfunktionen. 
}

\proc{int}{InvConvolutionImgD}{ImageD im1,ImageD im2,ImageD dst,double beta=0.000001,int mode=MD\_USE\_BIAS}
\descr{
  Es wird das Bild $dst$ berechnet, das gefaltet mit Bild $im1$ das Bild $im2$
  ergibt (pseudo-inverse Faltung). Der vorzugebende Parameter $beta$ 
  beschreibt den Rauschanteil in den Bildern. 
}

\proc{int}{InvConvolutionFImgD}{ImageD re1,ImageD im1,
        ImageD re2,ImageD im2,ImageD re3,ImageD im3,double beta}
\descr{
Die der (pseudo-)inversen Faltung entsprechende Operation wird auf 
den komplexen Fouriertransformierten ($re1$,$im1$) und ($re2$,$im2$) 
ausgeführt und das Ergebnis (Fouriertransformierte des rekonstruierten Bildes) 
in den Bildern $re3$,$im3$ gespeichert.
}

\proc{int}{InvConvolutionHImgD}{ImageD im1,ImageD im2,ImageD im3}
\descr{
  Die der (pseudo-)inversen Faltung entsprechende Operation wird auf den  
  Hartleytransformierten $im1$ und $im2$ ausgeführt und
  das Ergebnis (Hartleytransformierte des rekonstruierten Bildes) im Bild
  $im3$ gespeichert.
}

\proc{int}{InvConvolutionImg}{const Image \&im1,const Image \&im2,
  Image \&dst,
  double factor=0.0,double beta=0.000001,int mode=MD\_USE\_BIAS}
\descr{
  Es wird das Bild $dst$ berechnet, das gefaltet mit Bild $im1$ das Bild $im2$
  ergibt (pseudo-inverse Faltung). Der vorzugebende Parameter $beta$
  beschreibt den Rauschanteil in den Bildern. Um eine optimale Umwandlung in
  Integer-Werte zu ermöglichen, kann der Faktor $factor$ angegeben werden,
  mit dem alle Werte vor der Konvertierung multipliziert werden. Ist
  $factor=0.0$, so wird eine automatische Grauwert-Skalierung so vorgenommen,
  daß der Wert mit dem maximalen Betrag den Werte-Bereich genau ausschöpft.
}

\proc{int}{InvConvolutionImg}{const Image \&im1,const Image \&im2,
  ImageD dst,
  double beta=0.000001,int mode=MD\_USE\_BIAS}
\descr{
  Es wird das Gleitkomma-Bild $dst$ berechnet, das gefaltet mit Bild 
  $im1$ das Bild $im2$ ergibt (pseudo-inverse Faltung). Der 
  vorzugebende Parameter $beta$ beschreibt den Rauschanteil in den 
  Bildern. 
}

\proc{int}{CrossCorrelationImgD}{ImageD im1,ImageD im2,ImageD dst}
\descr{
  Die Kreuzkorrelation der Bilder $im1$ und $im2$ wird berechnet und 
  das Ergebnis in Bild $dst$ gespeichert.
}

\proc{int}{CrossCorrelationFImgD}{ImageD re1,ImageD im1,
  ImageD re2,ImageD im2,ImageD re3,ImageD im3}
\descr{
  Die der Kreuzkorrelation entsprechende Operation wird auf den komplexen
  Fouriertransformierten ($re1$,$im1$) und ($re2$,$im2$) ausgeführt und
  das Ergebnis (Kreuzkorrelation des gefalteten Bildes) in den Bildern
  $re3$,$im3$ gespeichert.
}

\proc{int}{CrossCorrelationHImgD}{ImageD im1,ImageD im2,ImageD im3}
\descr{
  Die der Kreuzkorrelation entsprechende Operation wird auf den  
  Hartleytransformierten $im1$ und $im2$ ausgeführt und
  das Ergebnis (Hartleytransformierte des gefalteten Bildes) im Bild
  $im3$ gespeichert.
}

\subsubsection{Manipulation im Frequenzbereich}

\proch{int}{WhiteningImgD}{ImageD img1,ImageD img2,double beta=0}{fourier.h}
\descr{Das Bild $img1$ wird ''weiß gemacht'' (whitening), das heißt dass die
  Amplituden der Fourierkoeffizienten zu 1.0 normiert werden. Diese Funktion
  beinhaltet die implizite Transformation in den Frequenzbereich und zurück. 
  Die Normierung der Fourierkoeffizienten erfolgt nach:
  \[
  F_{i,j} = \frac{F_{i,j}}{| F_{i,j} |+\beta^2}
  \]
}

\proch{int}{WhiteningFImgD}{ImageD re1,ImageD im1,ImageD re2,ImageD im2,double
  beta=0}{fourier.h}
\descr{Auf der komplexen Fouriertransformierten wird die Operation des
  ''Whitening'' (\see{WhiteningImgD}) ausgeführt. Die Transformation in den
    Frequenzraum und zurück muss hier explizit vorher bzw. hinterher
    vorgenommen werden. 
}

\proch{int}{WhiteningHImgD}{ImageD imgs,ImageD imgd,double beta=0}{fourier.h}
\descr{Auf der Hartleytransformierten wird die Operation des ''Whitening'' 
(\see{WhiteningImgD}) ausgeführt. Die Hartley-Transformation muss hier explizit 
vorher und hinterher vorgenommen werden. 
}

\subsection{Haartransformation}

\proc{int}{HaarImg}{Image imgs,int depth,int mode,Image imgs}
\descr{
  Haartransformation des Bildes $imgs$ in das Bild $imgd$ mit der
  Auflösungsstufe $depth$ ($mode$=NORMAL: Hintransformation, $mode$=INVERS:
  Rücktransformation).
}

\subsection{Momenttransformation}

\proc{ImageD}{MomentImg}{Image imgs,int p,int q,int n,ImageD imgd}
\descr{Es wird punktweise aus einer $n \cdot n$-Umgebung das Moment
  $m_{pq}$ über dem Bild $imgs$ berechnet und im Bild $imgd$ abgelegt.
}

\subsection{Radon-Transformation}

\proch{int}{RadonImg}{const Image \&src,Image \&radon}{radon.h}
\descr{Berechnet die Radontransformierte des Bildes $src$ und legt das Ergebnis in
$radon$ ab.}

\proc{int}{InvRadonImg}{const Image \&radon,Image \&res,int fmax=-1}
\descr{Berechnet aus einem radontransformierten Bild $radon$ das Originalbild $res$. Ist $fmax$ größer
als Null, so werden Frequenzen größer $fmax$ bei der Rücktransformation nicht berücksichtigt.}

% Local Variables: 
%%% mode: latex
%%% TeX-master: "icedoc_base"
%%% End: 
