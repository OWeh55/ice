\nsection{Fehlerbehandlung}{errorHandling}

\subsection{Grundsätzliches}
\label{Fehlerbehandlung}
Bei fehlerhafter Abarbeitung (z.B. durch falsche Parameter) geben Funktionen
normalerweise eine Fehlermeldung aus. Die Abarbeitung des Programms kann dann
interaktiv abgebrochen oder fortgesetzt werden. Um eine Fehlerbehandlung durch
das Anwenderprogramm zu ermöglichen, kann diese Form der Fehlermeldung durch
spezielle Funktionen ein- und ausgeschaltet werden. Eine Fehlerdiagnose ist
mit Hilfe der Funktion GetError() möglich, mit der der Fehlercode abgefragt 
werden kann.

\noindent Bezüglich der Rückgabewerte bei Fehlern existieren in ICE drei 
verschiedene Funktionstypen:
\begin{itemize}
\item Funktionen, die keinen funktionsspezifischen Rückgabewert liefern: Der
Rückgabewert ist hier vom Typ $int$ und enthält den Fehlercode. Der Wert
$OK$ steht hier für ``kein Fehler''.
\item Funktionen, die Zeiger zurückliefern: Bei Fehler liefern diese
Funktionen den Nullpointer als Rückgabewert. Die genaue Fehlerbeschreibung 
(der Fehlercode) ist mit der Funktion GetError() abzufragen.
\item Funktionen mit einem funktionsspezifischen Rückgabewert: Der
Rückgabewert liefert hier keine Information über aufgetretene Fehler. 
Hier besteht nur die Möglichkeit, mit \bsee{GetError} abzufragen, ob und 
welcher Fehler aufgetreten ist.
\end{itemize}
\subsection{Fehlerbehandlungsfunktionen}
\proc{void}{Message}{char *name, char *msg, int code}
\descr{Ausgabe einer Fehlermeldung. $name$ ist der Funktionsname, {\em
msg} ist der Text der Fehlernachricht und $code$ ist ein Fehlercode, der
intern gesetzt wird und mit GetError() abgefragt werden kann. 
Vordefinierte Fehlermeldungen und Rückkehrcodes befinden sich im defs.h.
Eine besondere Bedeutung besitzt die Meldung (der Zeiger) $M\_0$. Er bewirkt, 
dass die von einer untergeordneten, gerufenen Funktion erzeugte Fehlermeldung
ausgegeben wird. }
\proc{void}{OffMessage}{void}
\descr{Ausschalten der Fehlermeldungen. Treten bei einem Funktionsaufruf
Fehler auf, so werden diese nicht angezeigt und können von der rufenden
Funktion selber ausgewertet werden. Um die Funktion des gesamten Systems 
zu gewährleisten, ist sicherzustellen, daß die Fehlermeldungen vor dem 
Verlassen der Funktion wieder eingeschaltet werden. Dies muß vor 
der Ausgabe eigener Fehlermeldungen geschehen, damit diese angezeigt werden. }
\proc{void}{OnMessage}{void}
\descr{Einschalten der Fehlermeldungen.}
\proc{void}{SetOk}{void}
\descr{Der interne Fehlercode wird zurückgesetzt. Damit kann die
Programmabarbeitung fortgesetzt werden.}
\proc{int}{GetError}{void}
\descr{Als Funktionswert wird der interne Fehlercode zurückgegeben.}
\subsection{Beispiele für Standardfehlerbehandlung in ICE-Funktionen}
\begin{itemize}
\item Funktion in der ein Fehler auftreten kann.\\
In diesem Beispiel kann ein Fehler durch falsche Parameterangabe bzw. 
bei Speicheranforderungen auftreten.
\begprogr
\begin{verbatim}
//  Anlegen eines neuen Bildes 
// Beispiel veraltet !!!
#define FNAME "NewImg"

Image NewImg(int xsize, int ysize, int maxval)
{
 Image img;
 if ((xsize<=0)||(ysize<=0)||(maxval<=0)); // Parametertestung
    {
        Message(FNAME,M_WRONG_PARAM,WRONG_PARAM); // Fehlermeldung
        return NULL; // Rückgabewert NULL als Zeichen für Fehler
    }
 img=(Image )malloc(sizeof(struct Image_)); // Speicheranforderung
 if (img==NULL)
    {
        Message(FNAME,M_NO_MEM,NO_MEM); // Nicht genug Speicher melden
        return NULL; // Rückgabewert NULL als Zeichen für Fehler
    }
// Etliche weitere Schritte zur Verwaltung der Bildspeicher 
// durch ICE folgen
...
//
 return img; // Bei Erfolg Bildzeiger zurückgeben
}
#undef FNAME
\end{verbatim}
\endprogr
\item Funktion, die eine andere ICE-Funktion aufruft.\\
Das Durchreichen der Fehlermeldungen bis zur obersten Ebene hat den 
Vorteil, daß der Anwender aus der Fehlermeldung leichter die Fehlerursache 
erkennt, da er ja die intern indirekt aufgerufenen Funktionen nicht kennt.
\begprogr
\begin{verbatim}
// Anlegen eines neuen Bildes mit einem (einfachen) Testmuster
#define FNAME "GenImg"
Image GenImg(int xsize,int ysize,int maxval)
{
  Image img;
  // Die Parametertestung wird durch NewImg vorgenommen
  OffMessage(); // Fehlermeldungen deaktivieren
  img = NewImg(xsize,ysize,maxval); // Bild neu anlegen
  OnMessage();  // Fehlermeldungen wieder einschalten !
  if (!img.isValid()) // Ist ein Fehler aufgetreten ?
    {
      int errorcode=GetError();
      Message(FNAME,M_0,errorcode); // Fehler von NewImg weiterreichen (M_0)
                                    // ErrorCode ebenfalls wie gemeldet 
      return img; // Rückkehr mit ungültigem (invalid) Bild
    }
  for (int y=0; y < ysize; y++)
    for (int x=0;x < xsize; x++)
      PutVal(img, x, y, (x + y) % (maxval + 1)); // Bild beschreiben

  return img; // Zeiger auf neues Bild zurückgeben
}
#undef FNAME
\end{verbatim}\endprogr
\end{itemize}
\subsection{Makros für Standardfälle der Fehlerbehandlung}
Zur Erleichterung der Behandlung von Fehlern sind Makros definiert, die
einige Standardfälle abdecken. Diese setzen die Definition des Makros FNAME 
voraus, welches den Namen der Funktion als Zeichenkette enthält.
\proc{}{ReturnNullIfFailed}{function}
\descr{
Bricht bei Fehler bei der Abarbeitung von function ab und kehrt mit
dem Nullpointer zur aufrufenden Funktion zurück. Dieses Makro ist geeignet
für Funktionen, die normalerweise einen Zeiger zurückgeben.
}
Unter Verwendung von ReturnNullIfFailed vereinfacht sich das obige Beispiel 
stark:
\begprogr\begin{verbatim}
// Anlegen eines neuen Bildes mit einem (einfachen) Testmuster
#define FNAME "GenImg"
Image GenImg(int xsize,int ysize,int maxval)
{
Image img;
// Aufruf NewImg und Fehlerbehandlung
ReturnNullIfFailed(img=NewImg(xsize,ysize,maxval)); 

for (int y=0;y<ysize;y++)
 for (int x=0;x<xsize;x++)
   PutVal(img,x,y,(x+y)%(maxval+1)); // Bild beschreiben

return img; // Zeiger auf neues Bild zurückgeben
}
#undef FNAME
\end{verbatim}\endprogr

\proc{}{ReturnErrorIfFailed}{function}
\descr{
Bricht bei Fehler bei der Abarbeitung von $function$ ab und kehrt mit
dem Wert des Fehlers zur aufrufenden Funktion zurück. Dieses Makro ist 
geeignet für Funktionen vom typ $int$, die keinen funktionsspezifischen
Wert zurückgeben, sondern einen Fehlerkode.
}
\proc{}{IfFailed}{function}
\descr{
Aufruf der Funktion $function$ und Fehlerauswertung.
Dieses Makro stellt eine if-Anweisung dar. Die Bedingung ist erfüllt,
wenn ein Fehler aufgetreten ist. Im Körper der if-Anweisung ist 
entsprechend auf den Fehler zu reagieren. 
}

\noindent Das Makro IfFailed ist in folgender Form anwendbar:
\begin{verbatim}
IfFailed(img=NewImg(x,y,v))
    \{
        // Reaktion auf Fehler, z.B. files schließen ..
        ...
        Message(FNAME,M_0,ERROR);
        return NULL;
    \}

\end{verbatim}
Dieses Makro ist geeignet, wenn bei Fehlern individuell reagiert werden muß. 
Gründe dafür können sein:
\begin{itemize}
\item Ein spezieller Rückgabewert
\item Freigabe von Resourcen vor Fehlermeldung und Rückkehr
\item Tolerierung von Fehlern, wenn die Funktion trotzdem erfolgreich
beendet werden kann. Im Block nach IfFailed ist die spezielle Behandlung 
bei Fehler zu realisieren und der Fehlerzustand mit SetOK() zurückzusetzen,
damit die Abarbeitung ohne weitere Fehlermeldungen forgesetzt werden kann.
\end{itemize}

%%% Local Variables: 
%%% mode: latex
%%% TeX-master: "icedoc_base"
%%% TeX-master: "icedoc_base"
%%% End: 
