%%% Einleitung

\nsection{Einführung}{introduction}

ICE ist eine C-Bibliothek mit Werkzeugen für die Bildanalyse. Durch 
ein weitgehend hardware- und betriebssystemunabhängiges
Konzept der Bildverwaltung und Visualisierung ist eine Verwendung
unter Unix/X-Window und MS-Windows möglich. 

Als Compiler muß ein C++-Compiler verwendet werden. Erprobt wurde der 
Einsatz von GNU-C++ (unter Linux und Windows). 

Bilddaten sind Datenstrukturen im Hauptspeicher. Es können gleichzeitig
beliebig viele unterschiedlich große Bilder verwaltet werden. 
Für die Darstellung der Bilddaten werden Funktionen des Frameworks {\bf wxWidgets}
verwendet.

Die Funktionsprototypen der ICE-Funktionen und die benötigten Datenstrukturen
sind in dem Headerfile {\em image.h} zusammengefaßt. Damit kann nach Einschluss 
einer Datei die gesamte Funktionalität von ICE verwendet werden. Vor allem 
in größeren Projekten sollte jedoch dem Einschließen der Einzel-Headerfiles 
der Vorzug gegeben werden, da das Bilden des Projekts dadurch schneller erfolgt.

Funktionen geben bei Fehlern bei Abarbeitung (z.B. durch falsche Parameter) 
normalerweise eine Fehlermeldung aus. Die Abarbeitung des Programms kann dann
interaktiv abgebrochen oder, falls dies sinnvoll ist, fortgesetzt werden. Um 
eine Fehlerbehandlung durch das Anwenderprogramm zu ermöglichen, kann diese 
Form der Fehlermeldung durch spezielle Funktionen (\see{OffMessage}) ein- und 
ausgeschaltet werden.

Das folgende Beispielprogramm soll die Verwendung der grundlegenden
ICE-Funktionen demonstrieren.

\begprogr
\begin{verbatim}
/* Erzeugung eines Graukeiles und Binarisierung */
#include <stdio.h>
#include <image.h>               /* ICE-Headerfile */
void main(void)
{
  const int sx = 800;            /* Bildgroesse in x-Richtung */
  const int sy = 600;            /* Bildgroesse in y-Richtung */
  const int maxv = 255;          /* maximal einzutragender Datenwert */
  Image img;                     /* Deklaration eines Bildes */
  int thr = 100;                 /* Schwellwert für Binarisierung */

  OpenAlpha("ICE Text Window");  /* Textfenster öffen */
  SetAttribute(0,7,0,0);         /* Attribute setzen */
  ClearAlpha();                  /* initialisieren */

  img=NewImg(sx, sy, maxv)       /* Bild anfordern */
  Show(ON, img);                 /* Bild zur Darstellung anmelden */
  Printf("Graukeil\n");          /* Ausgabe in Alphanumerikfenster */
                                 /* Graukeil erzeugen */
  for(int y = 0; y < img->ysize; y++)
    for(int x = 0; x < img->xsize; x++)
      PutVal(img,x,y,x);         /* Grauwert schreiben */
  GetChar();                     /* Tastendruck in Text- oder Bildfenster abwarten */

  Printf("Binarisierung mit Schwellwert %d\n", thr);
  for(int y = 0; y < img->ysize; y++)
    for(int x = 0; x < img->xsize; x++)
      {
        int val = GetVal(img,x,y); /* Grauwert lesen */
        if (val < thr) 
          PutVal(img,x,y,0);
        else 
          PutVal(img,x,y,img->maxval);
  }
  GetChar();                     /* Tastendruck in Text- oder Bildfenster abwarten */
  FreeImg(img);                  /* Freigabe des Bildes */ 
}
\end{verbatim}
\endprogr

%%% Local Variables: 
%%% mode: latex
%%% TeX-master: t
%%% End: 
