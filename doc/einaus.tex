\nsection{Ein- und Ausgabe von Bilddaten}{imageio}

\subsection{Bild-Dateien}

Bilder können von Bildfiles in verschiedenen Formaten eingelesen und 
auch in diesen Formaten ausgegeben werden. Das Format
des Bildfiles wird anhand des Dateinamen erkannt:
\begin{itemize}
\item Hat die Datei die Standard-Namens-Endung (extension) für dieses 
Format, so wird dieser Typ angenommen.
\item Wenn die Datei eine andere Namens-Endung besitzt, kann der Typ
der Datei durch einen durch Doppelpunkt abgetrennten Vorsatz vor
dem Dateinamen angegeben werden.\\
\end{itemize}
Beispiel: Bei Angabe von $TIFF:test.jpg$ behandelt ICE die Datei als
Bild im TIF-Format, auch wenn die Erweiterung etwas anderes nahelegt.

\begin{tabular}{|l|c|c|c|c|} \hline
 Typ & Prefix & Extension & Grauwert & Farbe \\ \hline \hline
TIFF & TIF: & .tif & Ja & Nein \\ 
 & TIFF: & .tiff & & \\ \hline
JPEG & JPG: & .jpg & Ja & Ja \\ 
 & JPEG: & .jpeg & & \\ \hline
PCX & PCX: & .pcx & Ja & Nein \\ \hline
Targa & TGA: & .tga & Ja & Ja \\ \hline
Windows-Bitmap & BMP: & .bmp & Ja & Ja \\ \hline
Portable Bitmap & PBM: &  & Ja & Ja \\ \hline
 & PPM: & .ppm & Ja & Ja \\ \hline
 & PGM: & .pgm & Ja & Ja \\ \hline
Photoshop & PSD: & .psd & Ja & Ja \\ \hline
\end{tabular}

Beim Lesen von Bildern im Portablen Bitmap-Format (PBM) kann als weitere
Besonderheit direkt die Ausgabe eines Programmes eingelesen werden. Damit wird
es möglich, Programme wie $convert$ (ImageMagick) oder $dcraw$ zum Einlesen von
Fremdformaten zu verwenden:
\begin{verbatim}
ReadImg("PGM:|convert ente.ras PPM:-",img);
\end{verbatim}
Das pipe-Symbol '|' als erstes Zeichen des Dateinamen kennzeichnet den Aufruf
eines Programmes, hier $convert$. Der Aufruf von {\bf convert} mit
\verb+convert ente.ras PPM:-+ bewirkt die Konvertierung der Datei 
{\bf ente.ras} in eine portable Bitmap und deren Ausgabe auf der 
Standard-Ausgabe. ReadImg liest diese Ausgabe in das Bild img ein.
Dies kann auch dazu verwendet werden, mittels $scanimage$ (sane) Bilder von
Scannern einzulesen.
Eine weitere spezielle Möglichkeit ist das Einlesen eines Bildes von der
Standardeingabe. Dazu ist als Filename ''-'' einzugeben:
\begin{verbatim}
ReadImg("PGM:-",img);
\end{verbatim}
Ein Programm {\bf myprog}, welches auf diese Art ein Bild einliest, kann mit
\verb+convert ente.ras PPM:- | myprog+ aufgerufen werden.

\proch{Image}{ReadImg}{const string \&filename,Image \&img,int flag=IB\_SCALE}{picio.h}
\descr{
Von dem Bildfile $file$ wird ein Grauwert-Bild in $img$ eingelesen.
Überschreitet die Bildgröße in der Datei die Größe des Bildes $img$, so
wird je nach dem Wert von $flag$ das eingelesene Bild ganzzahlig skaliert
($flag=IB\_SCALE$) oder auf die Größe von $img$ beschnitten ($flag=IB\_CROP$).
Wenn die Grauwertbereiche nicht übereinstimmen, wird eine lineare 
Skalierung vorgenommen. Wird als Parameter $img$ ein ungültiges Bild
übergeben, so wird ein Bild in der passenden Größe angelegt.
}

\proc{int}{WriteImg}{const Image \&img,const string \&filename}
\descr{
Das Grauwertbild $img$ wird in die Datei $filename$ geschrieben.
}

\proc{int}{ReadImg}{const string \&file,Image \&imgr,Image \&imgg,Image \&imgb,
int flag=IB\_SCALE}
\descr{
Von dem Bildfile $file$ wird ein Farbbild eingelesen und die
Farbauszüge Rot, Grün und Blau werden in die Bilder $imgr$, $imgg$ und
$imgb$ eingetragen. Die Bilder $imgr$, $imgg$ und $imgb$ müssen die
gleiche Größe besitzen.  Überschreitet die Bildgröße in der Datei 
die Größe der Bilder $imgr$, $imgg$ und $imgb$, so
wird je nach dem Wert von $flag$ das eingelesene Bild ganzzahlig skaliert
($flag=IB_SCALE$) oder auf die notwendige Größe beschnitten ($flag=IB_CROP$).
Wenn die Grauwertbereiche nicht übereinstimmen, wird eine lineare 
Skalierung vorgenommen. Wenn statt der Bilder $imgr$, $imgg$ und $imgb$ 
der Null-Pointer übergeben wird, wird ein Bild in der passenden Größe 
angelegt.
}
\proc{int}{WriteImg}{const Image \&ir,const Image \&ig,const Image \&ib,const string \&filename}
\descr{
Das durch die Farbauszüge in den Bildern $ir$, $ig$ und $ib$ übergebene
Farbbild wird in die Datei $filename$ ausgeschrieben.
}

\proc{int}{InfImgFile}{const string \&filename, int \&xsize, int \&ysize, 
  int \&maxval,int \&nr}
\procf{int}{InfImgFile}{const string \&filename, int \&xsize, int \&ysize, 
  int \&maxval}
\descr{
  Die Informationen zum Bildfile $filename$ werden ermittelt. Zurückgegeben
  werden die Bildgröße ($xsize$,$ysize$), der Grauwertumfang ($maxval$) und 
  die Zahl der Teilbilder bzw. Farbebenen ($nr$).
}

\subsection{Video-Dateien}
\label{VideoFile}
\hypertarget{VideoFile}
Bildsequenzen können aus Video-Dateien gelesen und in Video-Dateien 
geschrieben werden. Voraussetzung dafür ist die Installation der 
Bibliothek \verb+libffmpeg+.

Video-Dateien können nur streng sequentiell gelesen und geschrieben
werden. Dazu muss ein \class{VideoFile} angelegt werden. Dabei kann 
bereits beim Anlegen der Name angegeben werden und die Entscheidung 
über Lesen oder Schreiben gefällt werden. Wenn dies nicht hier 
erfolgt, muss die Datei später mit \bsee{VideoFile::open} geöffnet werden. 

Danach kann fortlaufend aus der Datei gelesen werden
(\bsee{VideoFile::Read}) oder in die Datei geschrieben werden 
(\bsee{VideoFile::Write}). 
Die Datei wird im Destruktor von \class{VideoFile} automatisch geschlossen.

Das Setzen von Parametern mit \bsee{VideoFile::setPara}
und \bsee{VideoFile::setCPara} ist nur vor dem ersten Schreiben möglich.

\proch{}{VideoFile::VideoFile}{}{videoio.h}
\descr{Standard-Konstruktor. Die Datei muss mit \bsee{VideoFile::open} 
geöffnet werden.}

\proc{}{VideoFile::VideoFile}{const string \&fn,ios\_base::openmode mode =
  ios\_base::in}
\descr{
Konstruktor, der den Dateinamen und die Schreib-Lese-Richtung
festlegt: Lesen - ios\_base::in, Schreiben - ios\_base::out. 
}

\proc{int}{VideoFile::open}{const string \&fn,ios\_base::openmode mode =
  ios\_base::in}
\descr{Legt den Dateinamen und die Schreib-Lese-Richtung fest: Lesen -
  ios\_base::in, Schreiben - ios\_base::out. }

\proc{void}{VideoFile::getPara}{int \&xs,int \&ys,int \&mv,int \&fps}
\descr{Fragt die Parameter Bildgröße (x,y), maximaler Grauwert und
  Bildfrequenz (fps) einer Video-Datei ab. Die Werte sind bei einer 
  zu lesenden Datei nach dem Öffnen gültig.}

\proc{void}{VideoFile::setPara}{int xs,int ys,int mv,int fps,int bitrate=0}
\descr{Legt die Parameter einer zu schreibenden Video-Datei fest. Dies muss
  vor dem ersten Schreibvorgang mit \bsee{VideoFile::Write} erfolgen. Die 
Bildgröße $x*y$ muss exakt der Größe der zu schreibenden Bilder entsprechen.
Als maximaler Grauwert ist z.Z. nur der Wert 255 möglich. Die Bitrate 
$bitrate$ (Bit pro Sekunde) kann auch mit dem Wert 0 vorgegeben werden. Dann 
wird ein geeigneter Wert aus Bildgröße und Framerate $fps$ ermittelt. }

\proc{void}{VideoFile::setCPara}{const string \&p}
\descr{Definiert optional für Ausgabe-Dateien einen Kompressionstyp.
Die zu verwendenden Kürzel entsprechen der Codec-Kennung in der 
libffmpeg oder dem MIME-Type der Dateien. Wird dies nicht gesetzt, wird 
der Dateityp aus der Datei-Erweiterung ermittelt.}

\proc{bool}{VideoFile::Read}{Image \&img,int ch=3}
\procf{bool}{VideoFile::Read}{Image \&ir,Image \&ig,Image \&ib}
\procf{bool}{VideoFile::Read}{ColorImage \&irgb}
\descr{Liest ein Bild aus einer Video-Datei. Die Methode gibt bei Erfolg
den Wert {\bf true} zurück.\\
Beim Lesen eines Grauwertbildes kann mittels des Parameters $ch$ entschieden 
werden, welcher Kanal in das Bild $img$ geschrieben wird.\\
\begin{tabular}{ll}
ch = 0 & Rot \\
ch = 1 & Grün \\
ch = 2 & Blau \\
ch = 3 & Grauwert aus R, G und B\\
\end{tabular}
}

\proc{bool}{VideoFile::Write}{const Image \&img}
\procf{bool}{VideoFile::Write}{const Image \&ir,const Image \&ig,const Image \&ib}
\procf{bool}{VideoFile::Write}{const ImageRGB \&irgb}
\descr{Schreibt ein Bild in eine Video-Datei.}

Im folgendem Beispiel wird eine Video-Datei ''kopiert'': Wegen der Dekodierung
beim Lesen und neuen Kodierung beim Schreiben ist dies keine echte Kopie.
\begprogr
\begin{verbatim}
void Copy(const string &infilename, const string &outfilename)
{
 VideoFile in(infilename,ios_base::in);
 VideoFile out(outfilename,ios_base::out);

 int xs, ys, mv, fps;
 in.getPara(xs, ys, mv, fps); // Parameter der Quelldatei abfragen

 out.setPara(xs, ys, mv, fps); // Parameter der Zieldatei setzen

 ColorImage rgb;

 while (in.Read(rgb))
 {
     out.Write(rgb);
 }
// Schließen der Dateien erfolgt im Destruktor 
// von in und out
}
\end{verbatim}
\endprogr

\subsection{Datei-Eingabe-Puffer (Template)}

Die Templates \verb+ReadImageCache+ und \verb+ReadColorImageCache+ 
ermöglichen die Pufferung der Eingabe aus einer Quelle, die Bilder 
sequentiell liest (typisch für \bsee{VideoFile}).
Der Puffer ermöglicht in begrenztem Umfang, in der Eingabedatei 
vorwärts und rückwärts zu gehen.

Die Templates erfordern als Template-Parameter eine Klasse, die eine 
Methode zum sequentiellen Lesen von Bildern besitzt.

\begin{tabular}{ll}
\verb+ReadImageCache<T>+&\verb+Read(const Image &img)+ \\
\verb+ReadColorImageCache<T>+&\verb+Read(const Image &imgr,const Image &imgg,const Image &imgb)+ \\
\end{tabular}

\proch{}{ReadImageCache$<$T$>$::ReadImageCache}{T \&tr,int xsize,int ysize,int maxval,int csize}{readcache.h}
\procf{}{ReadColorImageCache$<$T$>$::ReadColorImageCache}{T \&tr,int xsize,int ysize,int maxval,int csize}
\descr{Legt einen ReadImageCache an. Es müssen die Quelle $tr$ für das Lesen, die Parameter der
zu lesenden Bilder $xsize$, $ysize$, $maxval$ und die Cache-Größe $csize$ angegeben werden.
Die Quelle muss für die Lebensdauer des Caches existieren.}

\proc{bool}{ReadImageCache$<$T$>$::Read}{const Image \&img,int frame=ReadImagCache$<$T$>$::next}
\procf{bool}{ReadColorImageCache$<$T$>$::Read}{const ColorImage \&img,int frame}
\procf{bool}{ReadColorImageCache$<$T$>$::Read}{const Image \&imgr,const Image \&imgg,const Image \&imgb,int frame}
\descr{Liest den Frame $frame$ ein. Der Rückgabewert signalisiert, ob das 
Einlesen efolgreich war.
Die Framenummer ist eine nichtnegative Zahl, die die Nummer des Frames 
in der Quelle darstellt. Die speziellen Werte ReadImageCache$<$T$>$::next 
und ReadImageCache$<$T$>$::prev spezifizieren den nächsten 
beziehungsweise vorigen Frame (bezogen auf die vorherige Lese-Operation).}

\proc{int}{ReadImageCache$<$T$>$::FrameNumber}{}
\proc{int}{ReadColorImageCache$<$T$>$::FrameNumber}{}
\descr{Liefert die Nummer des zuletzt erfolgreich gelesenen Frames.}

\proc{int}{ReadImageCache$<$T$>$::getError}{}
\proc{int}{ReadColorImageCache$<$T$>$::getError}{}
\descr{Gibt das Ergebnis des letzten Lesevorganges an.}
\begin{tabular}{ll}
Fehlerwert&Bedeutung\\
\verb+ReadImageCache<T>::no_error+&Kein Fehler \\
\verb+ReadImageCache<T>::before+&Lesen vor Anfang des Puffers \\
\verb+ReadImageCache<T>::past_eof+&Lesen nach Dateiende \\
\end{tabular}

\begprogr
\begin{verbatim}
...
  VideoFile vi("test.mpg",ios::in);

  int xs,ys,mv,fps;
  vi.getPara(xs,ys,mv,fps);

  ReadImageCache cvi(vi,xs,ys,mv,300);

  Image img=NewImg(xs,ys,mv);
  Show(ON,img);

  cvi.Read(img); // ersten Frame lesen
  int c;
  while ((c=GetChar())!='x')
    {
      switch(c)
        {
        case 'n': vi.Read(img, cvi.next); break;
        case 'p': vi.Read(img, cvi.prev); break;
        }
      Printf("Aktueller Frame: %d , Error: %d\n",cvi.FrameNumber(),cvi.getError());
    }
...
\end{verbatim}
\endprogr

\subsection{Gepufferte Video-Dateien}
\label{VideoFileCached}
\hypertarget{VideoFileCached}{}
Gepufferte Video-Dateien lockern die Beschränkung der Video-Dateien 
({\see{VideoFile}) auf rein sequentiellen Zugriff. 
Im Rahmen der Größe des Puffers ist es in einer gepufferten Video-Datei 
auch möglich, rückwärts zu gehen. Gepufferte Video-Dateien können nur 
gelesen werden.

\proch{}{VideoFileCached::VideoFileCached}{const string \&fn,int buffersize}{videoio.h}
\descr{Konstruktor, der den Dateinamen und die Puffergröße festlegt und die 
Video-Datei öffnet. Die Puffergröße ist die Zahl der gespeicherten 
Bilder/Frames. Diese bestimmt, wieweit in der Datei rückwärts gegangen werden kann.}

\proc{void}{VideoFileCached::getPara}{int \&xs,int \&ys,int \&mv,int \&fps}
\descr{Fragt die Parameter Bildgröße (x,y), maximaler Grauwert und
  Bildfrequenz (fps) einer gepufferten Video-Datei ab.}

\proc{bool}{VideoFileCached::Read}{Image \&ir,Image \&ig,Image \&ib,int frame=VideoFileCached::next}{videoio.h}
\procf{bool}{VideoFileCached::Read}{ColorImage \&img,int frame=VideoFileCached::next}
\procf{bool}{VideoFileCached::Read}{ImageRGB \&img,int frame=VideoFileCached::next}
\descr{Liest ein Bild aus einer Video-Datei. Die Methode gibt bei Erfolg den Wert {\bf true} zurück.
Die Framenummer ist eine nichtnegative Zahl, die die Nummer des Frames 
in der Videodatei darstellt. Die speziellen Werte 
VideoFileCached::next und VideoFileCached::prev spezifizieren den nächsten 
beziehungsweise vorigen Frame (bezogen auf die vorherige Lese-Operation).}

\proc{int}{VideoFileCached::FrameNumber}{}
\descr{Liefert die Nummer des zuletzt erfolgreich gelesenen Frames.}

\proc{int}{VideoFileCached::getError}{}
\descr{Gibt das Ergebnis des letzten Lesevorganges an.}
\begin{tabular}{ll}
Fehlerwert&Bedeutung\\
\verb+ReadImageCache<T>::no_error+&Kein Fehler \\
\verb+ReadImageCache<T>::before+&Lesen vor Anfang des Puffers \\
\verb+ReadImageCache<T>::past_eof+&Lesen nach Dateiende \\
\end{tabular}

Beispiel:
\begprogr
\begin{verbatim}
...
  VideoFileCached vi("test.mpg",100);
  int xs,ys,mv,fps;
  vi.getPara(xs,ys,mv,fps);
  Image ri=NewImg(xs,ys,mv);
  Image gi=NewImg(xs,ys,mv);
  Image bi=NewImg(xs,ys,mv);
  Show(_RGB,ri,gi,bi);
  vi.Read(ri,gi,bi); // ersten Frame lesen
  int c;
  while ((c=GetChar())!='x')
    {
      switch(c)
        {
        case 'n': vi.Read(ri, gi, bi, vi.next); break;
        case 'p': vi.Read(ri, gi, bi, vi.prev); break;
        }
      Printf("Aktueller Frame: %d , Error: %d\n",vi.FrameNumber(),vi.getError());
    }
...
\end{verbatim}
\endprogr

\subsection{Bildaufnahme}

Wenn entsprechende Hardware vorhanden ist, können Bilder digitalisiert
(abgetastet) werden. Die auf dem jeweiligen Rechner vorhandenen
Eingabemöglichkeit werden als Kanäle konfiguriert und beim Funktionsaufruf 
durch eine Kanalnummer unterschieden.

\proch{void}{ScanWindow}{int ch,int x1,int y1,int x2,int y2}{scan.h}
\descr{
Für den Kanal $ch$ wird ein Ausschnitt für die Bildabtastung festgelegt. 
Die Koordinaten $(x1,y1)$ und $(x2,y2)$ geben die linke obere und die rechte 
untere Ecke des festzulegenden Bildausschnitts an. Der Bildauschnitt bezieht 
sich immer auf die Bildgröße des Eingabe-Gerätes.
}

\proc{int}{ScanImg}{int ch,Image \&pi,int interactive=TRUE}
\descr{
Es wird ein Bild vom Kanal $ch$ abgetastet. Bei $interactive$=TRUE wird vor 
dem eigentlichen Abtasten auf eine Bestätigung durch Tastendruck gewartet, 
so daß z.B. die Kameraeinstellung noch korrigiert werden kann. Bei 
$mode$=FALSE wird sofort abgetastet.
}

\proc{int}{ScanImg}{int ch,Image \&pr,Image \&pg,Image \&pb,int interactive=TRUE}
\descr{
Es wird ein Farb-Bild vom Kanal $ch$ abgetastet und die Farbauszüge in den
Bildern $pr$, $pg$ und $pb$ abgespeichert. Bei $interactive$=TRUE wird vor 
dem eigentlichen Abtasten auf eine Bestätigung durch Tastendruck gewartet, 
so daß z.B. die Kameraeinstellung noch korrigiert werden kann. 
Bei $mode$=FALSE wird sofort abgetastet.
}

\proc{int}{ScanInfo}{int ch,int \&xm,int \&ym,int \&maxval,int \&channels,
            int \&flags,string \&descr}
\descr{
Es werden grundlegende Informationen über den Kanal $ch$ bereitgestellt.
\begin{itemize}
\item xm,ym,maxval - Größe und maximaler Grauwert des abgetasteten Bildes
\item channels - Zahl aller Kanäle (ist unabhängig vom angegebenen Kanal)
\item flags - Eigenschaften des Kanals
\item descr - textuelle Beschreibung des Kanals
\end{itemize}
Die Werte für Bildgröße und maximalen Grauwert können den Wert -1 zurückgeben,
wenn der Treiber vor dem eigentlichen Scan-Vorgang keine Information darüber
besitzt.\\
Folgende Bit-Werte werden (wenn vorhanden) in der Variable $flag$ ODER-verknüpft:
\begin{tabular}{|l|c|c|} \hline
 SC\_SCAN & 1 & Es können Bilder aufgenommen werden \\  \hline
 SC\_RGB & 2 & Es können Farb-Bilder aufgenommen werden \\ \hline
 SC\_PREVIEW & 4 & Es gibt eine Vorschau-Funktion \\ \hline
 SC\_DIALOG & 8 & Es gibt einen interaktiven Einstellungs-Dialog \\ \hline
 SC\_EXTDATA & 16 & Es können Parameter eingestellt werden \\ \hline
\end{tabular} \\
{\it Es sollten nur die symbolischen Konstanten der ersten Spalte verwendet werden,
die konkreten Werte in der zweiten Spalte dienen nur der Information und können sich ändern.}
}

\subsubsection{Konfiguration der Bildaufnahme unter Unix/Linux}
Die Konfiguration der Bildaufnahme erfolgt durch Setzen von
Environment-Variablen. Die Variablen ICESCANDEVICE0 .. ICSCANDEVICE9
korrespondieren mit den 10 möglichen Kanälen 0..9. Die Einträge haben die
folgende Form:
\begin{verbatim}
ICESCANDEVICE0=fw,0,MODE_640x480_YUV422,30
\end{verbatim}
Der erste Wert (hier: fw) spezifiziert die Art des Eingabe-Gerätes, den
zuständigen Treiber, die darauffolgenden Parameter sind jeweils
Geräte-spezifisch.

\begin{itemize}
\item {\bf fw} - IEEE1394-Kameras (IIDC-kompatibel)
\begin{verbatim}
ICESCANDEVICE0=fw,0,MODE_640x480_YUV422,30
\end{verbatim}
Die Parameter nach {\bf fw} haben die folgende Bedeutung:
\begin{itemize}
\item Nummer der Kamera\\
Die vorhandenen Kameras werden von 0 an durchnummeriert.
\item Modus\\
Modus, in dem die Kamera betrieben werden soll.
\item Bildwiederhol-Rate\\
Die Bildwiederhol-Rate in fps (Frames per second).
\end{itemize}
%\item {\bf v4l} - Video for Linux
%\begin{verbatim}
%ICESCANDEVICE0=v4l,0,0,740,556
%\end{verbatim}
%Die Parameter nach {\bf v4l} haben die folgende Bedeutung:
%\begin{itemize}
%\item Schnittstelle\\
%Die Zahl entspricht der Gerätedatei /dev/videoX: 0 = /dev/video0 ..
%\item Kanalnummer\\
%Kanalnummer in Video for Linux.
%\item Bildgröße Breite, Höhe
%\end{itemize}

\item {\bf file} - Einlesen von einer Datei
\begin{verbatim}
ICESCANDEVICE0=file,/tmp/img.jpg,640,480,255
\end{verbatim}
Der erste Parameter nach der Kennung {\bf file } gibt den Namen der Datei an. 
Die folgenden optionalen Parameter beschreiben die Bildgröße (x,y) und den 
maximalen Grauwert. Werden diese nicht angegeben, so werden sie zur Laufzeit 
aus der Bilddatei ermittelt, was allerdings zeitaufwendiger ist.

\item {\bf program} - Ein Programm, welches eine Bilddatei bereitstellt
\begin{verbatim}
ICESCANDEVICE0=program,igen /tmp/x.jpg,/tmp/x.jpg,640,480,255
\end{verbatim}
Die Parameter nach {\bf program} haben die folgende Bedeutung:
\begin{itemize}
\item Programmaufruf\\
Aufruf des Programmes mit Parametern.
\item Bilddateiname\\
Name der erstellten Bilddatei.
\end{itemize}
Die folgenden optionalen Parameter beschreiben die Bildgröße (x,y) und 
den maximalen Grauwert. Werden diese nicht angegeben, so werden sie 
vom System als ``unbekannt'' gemeldet. Es ist dann Sache des
Anwendungsprogrammes, eine geeignete Bildgröße zu wählen.

\end{itemize}
%%% Local Variables: 
%%% mode: latex
%%% TeX-master: "icedoc_base"
%%% End: 
