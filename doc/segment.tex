\nsection{Segmentierung}{segmentation}
\subsection{Wasserscheidentransformation (WST)}

Die Idee der Wasserscheidentransformation ist recht einfach: Ein Grauwertbild
wird als topographisches Relief interpretiert und jeder Punkt dieses
digitalen Bildes einem regionalen Minimum zugewiesen.
Dazu wird ein Flutungsprozess simuliert, bei dem Wasser aus den
regionalen Minima str\"omt und damit die Becken (Minima) füllt.
Punkte werden einem Minimum zugeordnet, indem sie mit der entsprechenden 
Staubeckennummer markiert werden.

Es werden durch den steigenden Wasserstand Einflusszonen der Minima
berechnet. Treffen zwei Becken aufeinander, werden die Punkte an den 
Ber\"uhrungsstellen speziell als D\"amme markiert. Diese D\"amme sind 
die Wasserscheidenlinien, die m\"ogliche Objektgrenzen darstellen.

Um in konstanter Zeit auf die Pixel einer H\"ohenstufe zugreifen zu 
k\"onnen, werden die Bildpixel ihrem Grauwert entsprechend in ein Array 
von FIFO-Schlangen einsortiert. Die Gr\"o\ss e des Array ist durch die 
Anzahl der unterschiedlichen Grauwerte im Bild bestimmt.

Die Anwendung der Wasserscheidentransformation resultiert oft in starker
\"Ubersegmentierung. Diese kann durch Vorverarbeitungsmaßnahmen gemindert 
werden. Um die \"Ubersegmentierung nachtr\"aglich zu reduzieren, k\"onnen 
die Regionen nach bestimmten Kriterien verschmolzen werden.

Wenn Objekte mit homogenen Grauwerten segmentiert werden sollen, wird die
Wasserscheidentransformation auf das Gradientenbild angewendet. F\"ur die
Segmentierung von texturierten Objekten eignet sich dieses Vorgehen nicht.

\subsection{WST-Algorithmen und Hilfsfunktionen}

\proch{int}{WST}{Image in, Image WSImg, int mode=1}{WaterShed.h}
\descr{Anwendung der Wasserscheidentransformation auf Bild $in$. Im Bild
$WSImg$ werden die Minima und ihre Einzugsgebiete numeriert.
$mode = 1$ : Flutungsansatz, neue Minima in h\"oheren Grauwertleveln zulassen}
%$mode = 2$ : Markeransatz, nur von den Initial-Markern fluten

5 Schritte bei $mode=1$:
\begin{quote}
\begin{tabular}{ll}
1.&     Sortieren der Grauwertpixel nach H\"ohenstufen.\\
2.&     Punkte, die sich an Regionengrenzen befinden, in FIFO-Schlange aufnehmen.\\
3.&     Diese Punkte als Wasserscheide markieren oder einem Becken zuordnen.\\
4.&     Isoliert liegende Punkte als neue Regionen markieren.\\
5.&     N\"achstes Grauwertlevel verarbeiten und Schritt 2 ausf\"uhren, bis Grauwert-Maximum erreicht ist.\\
\end{tabular}
\end{quote}

%5 Schritte bei $mode=1$:
%\begin{quote}
%\begin{tabular}{ll}
%1.&     Sortieren der Grauwertpixel nach H\"ohenstufen.\\
%2.&     Markerregionen mit Nummern etikettieren.\\
%3.&     Punkte, die sich an Regionengrenzen befinden, in FIFO-Schlange aufnehmen.\\
%4.&     Diese Punkte als Wasserscheide markieren oder einem Becken zuordnen.\\
%5.&     N\"achstes Grauwertlevel verarbeiten und wieder Schritt 3 bis Maximum erreicht.\\
%\end{tabular}
%\end{quote}

%\proch{void}{prune}{Image in, Image out, int tresh}{WaterShed.h}
%\descr{Grauwerte im Bild $in$, die kleiner als der Schwellwert $tresh$ sind,
%in Bild $out$ auf 0 setzen.}

%\proch{void}{imposeMinima}{Image in, Image out}{WaterShed.h}
%\descr{Minima aus Bild $in$ in Bild $out$ auferlegen. Ein Minimum ist ein Punkt
%mit Grauwert 0.}

\proch{void}{GraphIter}{Image Original, Image WSImg, Image \&GrwImg, int Threshold}{WaterShed.h}
\descr{Originalbild $Original$, Ergebnisbild der Wasserscheidentransformation $WSImg$ und das R\"uckgabebild
$GrwImg$ \"ubergeben. Der Parameter $Threshold$ steuert die Verschmelzung der Regionen. Es werden
nur Regionen vereinigt, bei denen die Differenz der durchschnittlichen Grauwerte gr\"o\ss er als $Threshold$ ist.}

Eine vollst\"andige Grapheniteration durchf\"uhren:
\begin{quote}
\begin{tabular}{ll}
1.&     Regionengraph berechnen.\\
2.&     In Kantengraph transformieren.\\
3.&     WST auf Kantengraph ausf\"uhren.\\
4.&     Kantengraph in Regionenbild \"uberf\"uhren.\\
\end{tabular}
\end{quote}

%% \proch{void}{DGrwImage}{Image Original, Image WSImg, Image \&GrwImg}{WaterShed.h}
%% \descr{Durchschnittlichen Grauwert der Regionen des Bildes $WSImg$ unter
%% Zuhilfenahme der Originalgrauwerte aus Bild $Original$ berechnen und in
%% Bild $GrwImg$ einzeichnen.}

%% \proch{void}{delWSHDPixels}{Image GrwImg, Image \&WSHEDDeleted}{WaterShed.h}
%% \descr{Wasserscheidenlinien l\"oschen und durch gemittelten Grauwert der
%% angrenzenden Regionen ersetzen. Als Eingabebild dient ein Regionenbild
%% mit eingezeichneten Wasserscheidenlinien und durchschnittlichen Grauwert
%% f\"ur die Punkte einer Region in $GrwImg$.}

%\proch{void}{LowerCompleteImg}{Image in, Image \&lcImg}{WaterShed.h}
%\descr{Bild $in$ wird in ein nach unten vollst\"andiges Bild transformiert.
%Zu nicht minimalen Plateaus wird ein behelfsm\"a\ss iges Relief hinzuaddiert.}

%% \proch{void}{deleteSmallMarkers}{Image i1, int threshold}{WaterShed.h}
%% \descr{Zusammenh\"angende Punktmengen, die als Marker fungieren und weniger
%% als $threshold$ Pixel besitzen, werden gel\"oscht.}

%% \subsection{Regiongraph und Kantengraph}

%% Methoden zum Aufbau von Regionengraph und Kantengraph, 
%% WST-Algorithmus zur Ausf\"uhrung auf dem Kantengraph

%% \proch{}{RegionGraph::RegionGraph}{Image source, Image labImg, Image \&retImg}{RegionGraph.h}
%% \descr{Regionengraph mit dem Originalbild $source$ und dem Wasserscheidenbild $labImg$ initialisieren.}

%% \proch{void}{RegionNode::DrawRegion}{Image working, int color}{RegionGraph.h}
%% \descr{Die Regionenpixel der Region mit der Farbe $color$ in das Bild $working$ einzeichnen.}

%% \proch{void}{RegionNode::DrawWSHDPixel}{Image working, int color}{RegionGraph.h}
%% \descr{Die Wasserscheidenpixel der Region mit der Farbe $color$ in das Bild $working$ einzeichnen.}

%% \proch{EdgeGraph*}{RegionGraph::ComputeEdgeGraph}{void}{RegionGraph.h}
%% \descr{Den Kantengraph aus dem Regionengraph erstellen.}

%% \proch{void}{EdgeGraph::ComputeWatershedTrans}{void}{RegionGraph.h}
%% \descr{Die Wasserscheidentransformation auf dem Kantengraph ausf\"uhren.}

%% \proch{void}{EdgeGraph::drawAllWSHEDs}{Image i, Image wsheds, int tresh}{RegionGraph.h}
%% \descr{Wasserscheidenkanten in Bild $wsheds$ einzeichnen, die nach der
%% Wasserscheidentransformation auf dem Kantengraphen als Wasserscheidenknoten
%% markiert wurden. Zus\"atzlich werden nur die Kanten eingezeichnet, bei denen
%% die Grauwertdifferenz der angrenzenden Regionen gr\"o\ss er als $tresh$ ist.}
%% \subsection{Beispielcode}

%% Algorithmus Teil 1: Vorverarbeitung
%% \begin{quote}
%% \begin{tabular}{ll}
%%  1.&       (wenn n\"otig) Shading entfernen (Zeile 4)\\
%%  2.&       Rangfilter oder Gaussfilter zur Rauschminderung anwenden (Zeile 5)\\
%%  3.&       Gradientenbild des gefilterten Bildes berechnen (Zeile 6)\\
%%  4.&       (nur f\"ur WST2:) Marker mittels prune-Funktion vergr\"o\ss ern (Zeile 7)\\
%%  5.&       (nur f\"ur WST2:) Marker auf WST-Eingangsbild auferlegen (Zeile 8)\\
%% \end{tabular}
%% \end{quote}


%% Algorithmus Teil 2: WST ausf\"uhren und Nachverarbeitung durch Grapheniteration
%% \begin{quote}
%% \begin{tabular}{ll}
%%  6.&       Initialwasserscheiden berechnen (Zeile 9)\\
%%  7.&       Regionengraph berechnen (Aufruf von GraphIter Zeile 10)\\
%%  8.&       Transformation zu Kantengraph\\
%%  9.&       Wasserscheidentransformation auf Kantengraph anwenden\\
%% 10.&       Regionengraph des Ergebnis der Transformation berechnen\\
%% 11.&       noch zweimal wie 7. bis 10. (Zeile 11 und 12)\\
%% 12.&       Wasserscheidenlinien l\"oschen (Zeile 13)\\
%% 13.&       als G\"utevergleich Differenzbild von 12 mit Originalbild berechnen (Zeile 14)\\
%% \end{tabular}
%% \end{quote}

%% \begprogr
%% \begin{verbatim}
%% /* Test der Funktion WST
%%    Vorverarbeitung: Shadingkorrektur, Gaussfilterung, Gradientenbild */
%% #include "WaterShed.h"
%% #include "RegionGraph.h"

%% 1.      void testWST2(Image Original) {
%% 2.              Image Gauss, Grad, GradPruned, GradMin;
%% 3.              Image WS, BlurPic, Grw1, Grw2, Grw3;

%% 4.              Shading(Original,Original);
%% 5.              GaussImg(Original,13,2,Gauss);

%% 6.              GradImg(Gauss,1,Grad);
%% 7.              prune(Grad,GradPruned,3);
%% 8.              imposeMinima(GradPruned,GradMin);

%% 9.              WST(GradMin,WS);

%% 10.             GraphIter(Original,WS,GrwBild1,10);
%% 11.             GraphIter(Original,GrwBild1,GrwBild2,10);
%% 12.             GraphIter(Original,GrwBild2,GrwBild3,10);

%% 13.             delWSHDPixels(GrwBild3,BlurPic);
%% 14.             Image subPic=NewImg(MaxImg);
%% 15.             SubImg(BlurPic,Original,subPic);
%% 16.     }
%% \end{verbatim}
%% \endprogr

\subsection{Regionenwachstum}

\proch{Region}{RegionGrow}{int x,int y,const Image \&orig,int maxsize=INT\_MAX,int refvalue=-1}{segment1.h}
\procf{int}{RegionGrow}{int x,int y,const Image \&orig,Image \&mark,int val=1,int maxsize=INT\_MAX,int refvalue=-1}
\descr{Ermittelt durch Regionenwachstum auf dem Bild $orig$ eine Region mit 
  maximal $maxSize$ Pixeln um den Punkt (x,y). Das Regionenwachstum erfolgt 
durch Hinzufügen des ``ähnlichsten'' Punktes. Der ``ähnlichste'' Punkt ist 
der Punkt, der die geringste Wert-Abweichung vom Startpunkt zeigt, 
beziehungsweise die geringste Abweichung vom übergebenen Referenzwert 
$refvalue$, wenn dieser nicht negativ ist. Durch Angabe von 0 oder 
orig.maxval kann RegionGrow zur Ermittlung von Minima- oder 
Maxima-Regionen genutzt werden. Abbruchbedingung ist das
  Maximum der Differenz des mittleren Grauwertes der Region zu den 
  benachbarten Punkten. Die zweite Aufruf-Form zeichnet die Region mit dem
  Wert $val$ in das Bild $mark$ ein.}

\proch{Region}{RegionGrowGrw}{int x,int y,const Image \&orig,double stdmax=3.0,int maxSize=INT\_MAX}{segment1.h}
\procf{int}{RegionGrowGrw}{int x,int y,const Image \&orig,Image \&mark,int
  val=1,double stdmax=3.0,int maxSize=INT\_MAX}
\descr{Ermittelt durch Regionenwachstum eine Region mit maximal $maxSize$
  Pixeln um den Punkt (x,y). Das Wachstum wird abgebrochen, wenn die
  Standardabweichung der Grauwerte der Region den Wert $stdmax$ übersteigt.
  Die zweite Aufruf-Form zeichnet die Region mit dem Wert $val$ in das Bild $mark$ ein.}

%%% Local Variables: 
%%% mode: latex
%%% TeX-master: "icedoc_base"
%%% End: 
