\lsection{Abbildung von Szenenbeschreibungen}

\subsection{Datenstrukturen}

Für die Handhabung von räumlichen Geradensegmenten wird die Datenstruktur
$ClipPointList$ verwendet, die eine verkettete Liste der Endpunkte von
kollinearen Geradensegmenten ist. Die einzelnen Elemente der Liste enthalten
jeweils sowohl die Raumkoordinaten als auch die Bildkoordinaten des jeweiligen
Punktes. Das Ausblenden nicht sichtbarer Teile von Geradensegmenten erfolgt
durch einfügen neuer Punkte in die Liste.
\begprogr\begin{verbatim}
typedef struct ClipPoint_
{
  struct ClipPoint_    *prev,*next;  /*Listenverkettung*/
  double               pw[3];        /*Raumpunkt*/
  double               pc[2];        /*Bildpunkt*/
  double               l;            /*Abstand vom Anfangspunkt*/
}ClipPoint;

typedef struct
{
  ClipPoint            *first,*last; /*erstes und letztes Element der Liste*/
  int                  n;            /*Anzahl der Elemente*/
}ClipPointList;
\end{verbatim}\endprogr


\subsection{Verwaltung der Punktlisten}

\proc{ClipPointList*}{NewClipPointList}{}
\descr{
Es wird eine leere Punktliste angelegt.
}
\proc{ClipPoint*}{InsertClipPoint}{ClipPointList *cl,ClipPoint *cp}
\descr{
Es wird ein Punkt $cp$ in die verkettete Punktliste $cl$ eingefügt.
}
\proc{void}{DeleteClipPoint}{ClipPoint *cp,ClipPointList *cl}
\descr{
Der Punkt $cp$ wird aus der verketteten Punktliste gelöscht.
}
\proc{void}{FreeClipPointList}{ClipPointList *cl}
\descr{
Die gesamte Punktliste $cl$ wird gelöscht.
}


\subsection{Bestimmung verdeckter Kanten}

\proc{PointList}{Mesh3DPointList}{Mesh3D msh}
\descr{
Es wird eine Punktliste mit den Bildpunkten der Eckpunkte der 3D-Masche $msh$
erzeugt.
}
\proc{int}{PlaneCrossing}{double p0[3],double p1[3],double pe[3],double
n[3],double p[3]}
\descr{
Der Schnittpunkt der durch die Punkte $p0,p1$ verlaufenden Geraden mit der
durch den Punkt $pe$ und den Normalenvektor $n$ gegebenen Ebene wird, falls er
existiert, auf $p$ zurückgegeben. 
}
\proc{int}{LineCrossing}{double p0[3],double v0[3],double p1[3],double v1[3],double p[3]}
\descr{
Durch Ausgleichsrechnung wird ein ``Schnittpunkt'' der durch die Punkte $p0$
bzw. $p1$ und die Richtungsvektoren $v0$ bzw. $v1$ gegebenen Raumgeraden bestimmt.
}
\proc{void}{MeshPlaneCrossing}{double p1[3],double p2[3],Mesh3D msh,double p[3]}
\descr{
Der Schnittpunkt der durch die Punkte $p1,p2$ verlaufenden Geraden mit der
durch die 3D-Masche $msh$ bestimmten Ebene wird, falls er existiert, auf $p$
zurückgegeben.
}
\proc{void}{MarkVisibleSurfaces}{ModelList S,camera cam}
\descr{
Alle der Kamera zugewandten Modellmaschen werden intern durch ein
entsprechendes Flag markiert.
}
\proc{char}{EdgeConcave}{Edge3D edg}
\descr{
Es wird TRUE zurückgegeben, wenn in der 3D-Kante $edg$ zwei Flächen mit einem
Winkel aufeinanderstoßen, der kleiner als 180 Grad ist.
}
\proc{int}{OnVisibleSideOfPlane}{double p[3],Mesh3D msh}
\descr{
Es wird TRUE zurückgegeben, wenn sich der Punkt $p$ auf der Seite der 3D-Masche
$msh$ befindet, die vollständig außerhalb des zugehörigen Modells liegt.
}
\proc{int}{PointInMesh3D}{double p[2],Mesh3D msh}
\descr{
Es wird TRUE zurückgegeben,wenn sich der Bildpunkt $p$ innerhalb der in die
Bildebene projizierten 3D-Masche $msh$ befindet.
}
\proc{int}{SegmentInMesh3D}{double p1[2],double p2[2],Mesh3D msh}
\descr{
Es wird TRUE zurückgegeben,wenn sich das Bildgeradensegment vollständig innerhalb der in die
Bildebene projizierten 3D-Masche $msh$ befindet.
}
\proc{int}{ClipStraightLine}{ClipPoint *cp1,ClipPoint *cp2,ClipPointList
*cl,Mesh3D msh,camera cam}
\descr{
Aus dem Liniensegment $cp1,cp2$ aus der verketteten Punktliste $cl$, werden
die Teile ausgeblendet, die durch die 3D-Masche $msh$ verdeckt werden.
}
\proc{int}{CutEdge}{ClipPointList *cl,ModelList S,camera cam}
\descr{
Aus der durch die verkettete Punktliste $cl$ beschriebenen Menge von
kollinearen Geradensegmenten werden die Teile ausgeblendet, die durch
Bestandteile der Szenenbeschreibung $S$ verdeckt werden.
}
\proc{Description}{MapSzene}{ModelList S,camera cam}
\descr{
Es wird eine Bildbeschreibung erzeugt, die die vom Standpunkt der Kamera $cam$
aus sichtbaren Teile der Szenenbeschreibung enthält.
}
\proc{void}{ChangeView}{ModelList S,camera cam}
\descr{
Nach einer Positionsveränderung der Kamera bzw. zur Initialisierung werden die
Punktkorrespondenzen der Modellpunkte, die Sichtbarkeitsflags der Oberflächen
und die Polygone der abgebildeten Modellflächen neu berechnet. Diese Funktion
muß vor der Verwendung von MapEdge() ausgeführt werden.
}
\proc{Description}{MapEdge}{Edge3D edg,Frame *f,ModelList S, camera
c,Description D,int *fl}
\descr{
Die 3D-Kante $edg$ in der Position $f$ wird in die Bildebene der Kamera $c$
abgebildet. Die Teile der Kante, die durch Objekte der Szenenbeschreibung $S$
verdeckt sind, werden ausgeblendet. Die verbleibenden Kantenstücken werden als
2D-Kanten in die Bildbeschreibung $D$ eingetragen. Wenn für $D$ der
NULL-Pointer übergeben wird, wird die Bildbeschreibung intern neu
angelegt. Rückgabewert ist die Bildbeschreibung oder der NULL-Pointer im
Fehlerfall. Auf $fl$ wird 0 zurückgegeben, wenn die Kante vollständig sichtbar
ist, sonst 1.
Vor der Verwendung von MapEdge() muß einmal bzw. erneut nach jeder
Positionsänderung der Kamera die Funktion ChangeView() aufgerufen werden.
}
