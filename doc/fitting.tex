\nsection{Fitting}{fitting}

Zusätzlich zu den hier vorgestellten Wegen kann das Fitting von Funktionen
auch durch Verwendung der Klassen der Hierarchie ein- und zweidimensionaler
Funktionen erfolgen (\see{Funktionen}).

\subsection{Fitting mehrdimensionaler linearer Funktionen}
\label{GaussFit}
\hypertarget{GaussFit}{}

Die Klasse $GaussFit$ stellt die Mittel zum Bestimmen der Koeffizienten $a_i$ 
einer Funktion der Form $F(x_1,x_2,..x_i) = \sum_{i=0}^n a_i \cdot x_i$ 
zur Verfügung.

\seealso{Funktionen}

Konstruktoren:
\proch{}{GaussFit::GaussFit}{int n=0}{fitgauss.h}
\descr{Legt einen Fit-Bereich mit der Dimension $n$ an.
Wird die Dimension $n$ mit 0 angegeben, so muss diese 
durch Aufruf von \see{GaussFit::Init} vor der Benutzung der
Instanz noch festgelegt werden.}

Methoden:

\proch{void}{GaussFit::Init}{int n=0}{fitgauss.h}
\descr{Initialisiert einen (neuen) Fit-Prozess. Ein Wert von $n$ ungleich
Null legt eine (neue) Dimension der Funktion fest. Diese Funktion muss
aufgerufen werden, wenn die Dimension im Konstruktor nicht festgelegt wurde
oder wenn nach einer Verwendung ein neuer Fitprozess initialisiert werden 
soll.}

\proch{void}{GaussFit::Value}{const \bsee{Vector} \&x,double y,double w=1.0}{fitgauss.h}
\procf{void}{GaussFit::Value}{\vector{double} \&x,double y,double w=1.0}
\descr{Übergibt einen Wert für das Fitting. Der Vektor $x$ enthält die x-Werte
  $x_i$, $y$ den zugehörigen Funktionswert. Mit $w$ kann ein Gewicht
  festgelegt werden.}

\proch{void}{Gaussfit::Finish}{}{fitgauss.h}
\descr{Schließt das Fitting ab. Muss vor der Abfrage des Resultats
aufgerufen werden. Nach dem Abschluss kann das Fitting durch Eingabe weiterer
Werte fortgesetzt werden, dann muss auch noch einmal {\bf Finish} 
aufgerufen werden.}

\proch{void}{GaussFit::Fit}{const \bsee{Matrix} \&xn,const \bsee{Vector} \&yn}{fitgauss.h}
\descr{Fitted eine Funktion an eine Liste von Werten.
Die Matrix $xn$ enthält zeilenweise x-Vektoren, der Vektor $yn$ die 
zugehörigen y-Werte. Dieser Aufruf erledigt den vollständigen Fitting-Prozess
mit {\bf Init}, {\bf Value} und {\bf Finish}.}

\proch{void}{GaussFit::getResult}{Vector \&v}{fitgauss.h}
\procf{void}{GaussFit::getResult}{\vector{double} \&v}
\descr{Schreibt die Koeffizienten $a_i$ des Ergebnisses auf v[i].}

\proch{double}{GaussFit::Variance}{}{fitgauss.h}
\descr{Gibt die Varianz der Funktionswerte zurück.}

\subsection{Adaptives Fitting geometrischer Elemente}
\label{Adaptives Fitting}

Bei den hier betrachteten Fittung-Funktionen werden geometrische
(Linien-)Objekte an eine gegebene Punktmenge angepasst. Die Punktmenge wird
als \vector{Point} oder als \class{Matrix} übergeben.

Wird die Punktmenge als \vector{Point} übergeben, so kann als zusätzlicher 
Parameter ein \vector{double} übergeben werden, der für jeden Punkt 
ein Gewicht festlegt. Fehlt dieser Parameter, werden alle Punkte gleich gewichtet.

Wird die Punktmenge als \class{Matrix} übergeben, so müssen die ersten 
beiden Spalten der Matrix die Koordinaten x und y der Punkte enthalten. 
Ist eine dritte Spalte vorhanden, wird diese als Gewicht des jeweiligen 
Punktes interpretiert, sonst werden 
alle Punkte gleich gewichtet. Alle weiteren Spalten werden ignoriert.

Die Struktur \class{PointList} ist veraltet und sollte nicht mehr 
verwendet werden. 

Soll die Anpassung an eine Kontur erfolgen, so kann diese einfach mittels 
\bsee{ConturPointlist} gewandelt werden, zum Beispiel:
\begin{verbatim}
Contur c=...;
Circle c=FitCircle(ConturPointlist(c));
...
\end{verbatim}

\subsubsection{Gerade}
An eine Punktmenge wird eine Gerade angepaßt. Die Punktmenge kann als
\verb+vector<Point>+ oder als Matrix vorgegeben sein (\see{Adaptives Fitting}).

Es ist möglich eine Anzahl $step$ adaptiver Iterationsschritte vorzugeben.
Bei den weiteren Iterationsschritten werden die Punkte, die weiter von der im 
vorhergehenden Schritt angepaßten Geraden entfernt liegen, mit einem 
geringeren Gewicht versehen.

\proch{LineSeg}{FitLine}{const \vector{Point} \&pl}{fit.h}
\procf{LineSeg}{FitLine}{const \vector{Point} \&pl,int step}
\procf{LineSeg}{FitLine}{const \vector{Point} \&pl,const \vector{double} \&weight}
\procf{LineSeg}{FitLine}{const \vector{Point} \&pl,\vector{double} \&weight,int step}
\procf{LineSeg}{FitLine}{const Matrix \&pl,int step=0}
\descr{Die gefundene Gerade wird als Instanz der Klasse \class{LineSeg} 
zurückgegeben.
}

\proc{int}{FitLine}{const Matrix \&pl,double \&r,double \&phi,int step=0}
\descr{Die Parameter $r$ und $phi$ der Hesseschen Normalform der gefundenen 
  Gerade werden zurückgegeben. 
  $x \cdot \cos(\phi) + y \cdot sin(\phi) = r$
}

\proc{int}{FitLine}{const Matrix \&m,double par[2],int step=0}
\descr{Die Parameter der Hesseschen Normalform der gefundenen Gerade werden im 
  Array $par$ (2 Elemente) zurückgegeben.
  $x \cdot \cos(par[1]) + y \cdot sin(par[1]) = par[0]$
}

\proch{int}{FitLine}{PointList pl,int a1,int a2,int step,double par[2],double *mdist,int *madr}{fit\_pointlist.h}
\descr{
An die in einer Punktliste $pl$ zwischen $a1$ und $a2$ angegebenen Punkte wird 
adaptiv eine Gerade angepaßt. Die Anzahl der Adaptionsschritte ist mit $step$ 
vorzugeben. In der Adaption werden Punkte, die weiter von der im vorhergenden 
Schritt angepaßten Geraden entfernt liegen, mit einem geringeren Gewicht 
versehen. Es werden die Parameter der Hesseschen Normalform 
$x \cdot \cos(par[1]) + y \cdot sin(par[1]) = par[0]$, der Abstand $mdist$ 
des am weitesten entfernt liegenden Punktes sowie dessen Adresse $madr$ in
der Punktliste bereitgestellt. 
}

\proc{int}{FitLineLinOpt}{PointList pl,int a1,int a2,int step,double
par[2],double limit=1000000} 
\descr{
An die in einer Punktliste $pl$ zwischen $a1$ und $a2$ mit der Schrittweite
$step$  angegebenen Punkte wird
mit den Gewichten der Punktliste eine
Gerade angepaßt. Die gewichtete Ausgleichsrechnung wird mittels linearer
Optimierung durchgeführt (Simplexmethode),d.h. es wird nicht die Summe der
Quadrate der Defekte minimiert sondern die Summe der Beträge der Defekte. Der
Parameter limit (Standard:1000000) wird in einer zusätzlichen
Beschränkungsrelation zur linearer Optimierung benutzt. Es ist
möglich, daß dieser Parameter gegebenenfalls vergrößert werden muß. Es werden die
Parameter der Hesseschen Normalform $x \cdot \cos(par[1]) + y \cdot
sin(par[1]) = par[0]$ der Geraden bereitgestellt. 
}

\subsubsection{Kreis}
An eine Punktmenge wird ein Kreis angepaßt. Die Punktmenge kann als
\verb+vector<Point>+ oder als Matrix vorgegeben sein (\see{Adaptives Fitting}).
Es ist möglich eine Anzahl $step$ adaptiver Iterationsschritte vorzugeben. 
Dabei werden Punkte, die weiter von dem im vorhergehenden Schritt angepassten 
Kreis entfernt liegen, mit einem geringeren Gewicht versehen.

\proch{Circle}{FitCircle}{const \vector{Point} \&pl}{fit.h}
\procf{Circle}{FitCircle}{const \vector{Point} \&pl,const \vector{double} \&weight}
\procf{Circle}{FitCircle}{const \vector{Point} \&pl,int step}
\procf{Circle}{FitCircle}{const \vector{Point} \&pl,\vector{double} \&weight,int step}
\proc{Circle}{FitCircle}{const Matrix \&pl,int step=0}
\descr{Der gefundene Kreis wird als Datenstruktur \see{Circle} zurückgegeben. 
}

\proch{int}{FitCircle}{const Matrix \&pl,double \&xm,double \&ym,double \&rad,int step=0}{fit.h}
\descr{Der Mittelpunkt $(xm,ym)$ und der Radius $rad$ des gefundenen Kreises werden 
  zurückgegeben.}

\proc{int}{FitCircle}{const Matrix \&pl,double par[3],int step=0}
\descr{Der Mittelpunkt (xm,ym) und der Radius rad des gefundenen Kreises werden 
  im Array $par$ zurückgegeben (3 Elemente). }

\proc{int}{FitCircle}{PointList pl,int a1,int a2,int step,double par[3],double *mdist,int *madr} 
\descr{
An die in einer Punktliste $pl$ zwischen $a1$ und $a2$ angegebenen Punkte wird
adaptiv ein Kreis angepaßt. In der Adaption werden Punkte, die weiter von dem
im vorhergenden Schritt angepaßten Kreis entfernt liegen, mit einem geringeren
Gewicht versehen. Es wird der Mittelpunkt $(par[0],par[1])$, der Radius des Kreises $par[2]$,
der Abstand $mdist$ des am weitesten entfernt liegenden Punktes sowie dessen
Adresse $madr$ in der Punktliste bereitgestellt.
}

\proc{int}{FitCircleLinOpt}{PointList pl,int a1,int a2,int step,double
par[3],double limit=1000000} 
\descr{
An die in einer Punktliste $pl$ zwischen $a1$ und $a2$ mit der Schrittweite
$step$ angegebenen Punkte wird mit den Gewichten der Punktliste
ein Kreis angepaßt. Die gewichtete Ausgleichsrechnung wird mittels linearer
Optimierung durchgeführt (Simplexmethode), d.h. es wird nicht die Summe der
Quadrate der Defekte minimiert sondern die Summe der Beträge der Defekte.  Der
Parameter limit (Standard:1000000) wird in einer
zusätzlichen Beschränkungsrelation zur linearer Optimierung benutzt. Es ist
möglich, daß dieser Parameter gegebenenfalls vergrößert werden muß. Es
werden der Mittelpunkt $(par[0],par[1])$ und der Radius des Kreises $par[2]$
bereitgestellt. Falls der Rückkehrkode nicht
Null ist, konnte kein Kreis angepasst werden.
}

\subsubsection{Ellipse}
An eine Punktmenge wird eine Ellipse angepaßt.  Die Punktmenge kann als
\verb+vector<Point>+ oder als Matrix vorgegeben sein (\see{Adaptives Fitting}).

Es werden die Quadrate der algebraischen Distanzen 
$distance = a \cdot x^2 + b \cdot x^2 + c \cdot y^2+d \cdot x + e \cdot y + f$ 
minimiert. 

Bei der Default-Methode ($mode=1$) wird die Restriktion $ a^2+b^2+c^2+d^2+e^2+f^2=1$ 
bei der Minimierung verwendet. 

Alternativ ($mode=2$) wird als Restriktion $a + c = 1$ verwendet. In diesem 
Falle ist das Fitting einer Ellipse kovariant bezüglich 
Ähnlichkeitstransformationen, d.h. transformiert man die Punktmengen, 
so sind die beiden gefitteten Ellipsen auch die Abbilder voneinander 
bezüglich der verwendeten Ähnlichkeitstransformation.

Es ist möglich eine Anzahl $step$ adaptiver Iterationsschritte vorzugeben. 
Dabei werden Punkte, die weiter von der im vorhergenden Schritt angepaßten 
Ellipse entfernt liegen, mit einem geringeren Gewicht versehen.

\proch{Ellipse}{FitEllipse}{const Matrix \&pl,int step=0,int mode=1}{fit.h}
\descr{Die gefundene Ellipse wird als Datenstruktur \see{Ellipse} zurückgegeben. }

\proch{int}{FitEllipse}{const Matrix \&pl,double par[5],int step=0,int mode=1}{fit.h}
\descr{Die Parameter Mittelpunkt $(par[0],par[1])$,
  die Halbachsen $par[2]$und $par[3]$, der Drehwinkel $par[4]$ der 
  gefundenen Ellipse werden im Array $par$ zurückgegeben. 
}

\proc{int}{FitEllipse}{PointList pl,int a1,int a2,int step,double
par[5],double *mdist,int *madr,int mode=1}
\descr{
An die in einer Punktliste $pl$ zwischen $a1$ und $a2$ angegebenen Punkte wird
adaptiv eine Ellipse angepaßt. In der Adaption werden Punkte, die weiter von 
der im vorhergenden Schritt angepaßten Ellipse entfernt liegen, mit einem 
geringeren Gewicht versehen. Es wird der Mittelpunkt $(par[0],par[1])$, 
die Halbachsen $par[2]$und $par[3]$, der Drehwinkel $par[4]$, 
der Abstand $mdist$ des am weitesten entfernt liegenden Punktes sowie 
dessen Index $madr$ in der Punktliste bereitgestellt.}

\proc{int}{FitEllipseLinOpt}{PointList pl,int a1,int a2,int step,double
par[5],double limit}
\descr{
An die in einer Punktliste $pl$ zwischen $a1$ und $a2$ mit der Schrittweite
$step$  angegebenen Punkte wird
eine Ellipse angepaßt. Die gewichtete Ausgleichsrechnung wird mittels linearer
Optimierung durchgeführt (Simplexmethode), d.h. es wird nicht die Summe der
Quadrate der Defekte minimiert sondern die Summe der Beträge der Defekte.  Der
Parameter limit (Standard:1000000) wird in einer
zusätzlichen Beschränkungsrelation zur linearer Optimierung benutzt. Es ist
möglich, daß dieser Parameter gegebenenfalls vergrößert werden muß. Es werden der Mittelpunkt $(par[0],par[1])$, die Halbachsen
$par[2]$und $par[3]$ und der Drehwinkel $par[4]$ der Ellipse
bereitgestellt. Falls der Rückkehrkode nicht 
OK ist, konnte keine Ellipse angepasst werden. 
}

%\proc{int}{FitQuadrFunc}{PointList pl,int a1,int a2,int step,double *par,double *mdist,int *madr}
%\descr{
%An die in einer Punktliste $pl$ zwischen $a1$ und $a2$ angegebenen Punkte wird
%adaptiv eine quadratische Funktion der Form $Ax^2+By^2+Cx+Dy+Exy+F=0$
%angepaßt. In der Adaption werden Punkte, die weiter von der im vorhergenden
%Schritt angepaßten Ellipse entfernt liegen, mit einem geringeren Gewicht
%versehen. Auf $param$ werden die Parameter $A, B, C, D, E, F$ der
%quadratischen Funktion bereitgestellt. Es werden der Abstand mdist des am
%weitesten entfernt liegenden Punktes sowie dessen Adresse madr in der
%Punktliste bereitgestellt. Der Parameter $mode$ spezifiziert das
%Approximationsmodell:\\
%$mode=0$  Lagrang'sche Ausgleichung\\
%$mode=1$  Normierung A=1\\
%$mode=2$  Normierung B=1\\
%$mode=3$  Normierung C=1\\
%$mode=4$  Normierung D=1\\
%$mode=5$  Normierung E=1\\
%$mode=6$  Normierung F=1
%}

\subsubsection{Segmente}

Für die Beschreibung von Geraden-, Kreis- und Ellipsensegmenten wird die
Datenstruktur $Segment$ verwendet, die den Aufbau von verketteten Listen
erlaubt.
\begprogr\begin{verbatim}
typedef struct Segment_
{
  double          p0[2],p1[2];     /*Anfangs- und Endpunkt*/
  int             typ;             /*Segmenttyp: 1 - Geradensegment*/
                                   /*            2 - Kreissegment*/
                                   /*            3 - Ellipsensegment*/
  double          par[7];          /*Geradensegment: p,phi*/
                                   /*Kreissegment:   xm,ym,r,psi1,psi2*/
                                   /*Ellipsensegment:xm,ym,a,b,phi,psi1,psi2*/
  struct Segment_ *prev,*next;     /*Listenverkettung*/
}*Segment;
\end{verbatim}\endprogr

\proc{Segment}{FitLineSegment}{PointList pl,int a1,int a2,int step,double *mdist,int *madr}
\descr{An die in der Punktliste $pl$ von Adresse $a1$ bis Adresse $a2$
abgelegten Punkte wird ein Geradensegment angepaßt. Der Punkt mit dem größten
Abstand $mdist$ zu dem Geradensegment hat die Adresse $madr$.}
\proc{Segment}{FitCircleSegment}{PointList pl,int a1,int a2,int step,double *mdist,int *madr}
\descr{An die in der Punktliste $pl$ von Adresse $a1$ bis Adresse $a2$
abgelegten Punkte wird ein Kreissegment angepaßt. Der Punkt mit dem größten
Abstand $mdist$ zu dem Kreissegment hat die Adresse $madr$.}
\proc{Segment}{FitEllipseSegment}{PointList pl,int a1,int a2,int step,double *mdist,int *madr}
\descr{An die in der Punktliste $pl$ von Adresse $a1$ bis Adresse $a2$
abgelegten Punkte wird ein Ellipsensegment angepaßt. Der Punkt mit dem größten
Abstand $mdist$ zu dem Ellipsensegment hat die Adresse $madr$.}


\subsection{Kontursegmentierung}
\proc{Segment}{LineSegContur}{Contur c,int mlng,double mdist}
\descr{
Die Kontur $c$ wird in Geradensegmente zerlegt. Die Mindestanzahl der zu einem
Segment gehörenden Punkte wird durch $mlng$ festgelegt, $mdist$ gibt den
maximal zulässigen Abstand eines Konturpunktes zu dem nächstliegenden
Geradensegment an. Rückgabewert ist das erste Element einer Liste von
Geradensegmenten, die einen Polygonzug bilden. Falls die Kontur geschlossen
ist, wird durch die Geradensegmente ein geschlossener Polygonzug gebildet.
}

\proc{PointList}{FitPolygonContur}{Contur c,int mlng,double mdist}
\descr{
An die Kontur $c$ wird ein Polygonzug angepaßt. Die Mindestanzahl der zu einer
Seite des Polygonzuges gehörenden Punkte wird durch $mlng$ festgelegt, $mdist$
gibt den maximal zulässigen Abstand eines Konturpunktes zum Polygonzug
an. Rückgabewert ist eine Punktliste mit den Eckpunkten des angepaßten
Polygons. Falls die Kontur $c$ geschlossen ist, wird der Endpunkt der letzten
Seite nicht in die Liste aufgenommen, da er mit dem Anfangspunkt der ersten
Seite identisch ist.
}

\subsection{Segmentierung von Punktlisten}
\proc{Segment}{SegmentPointList}{PointList pl, int mode}
\descr{
Die Punktliste $pl$ wird in Linien, Kreise und Ellipsen ($mode$=SPL\_ALL),
in Linien und Kreise ($mode$=SPL\_LINECIRC) oder nur in Linien 
($mode$=SPL\_LINE) segmentiert. Zurückgegeben wird eine
Liste vom Typ Segment mit den Parametern der gefundenen Segment oder im
Fehlerfall NULL. Es wird der Algorithmus aus ``P. L. Rosin: Nonparametric
Segmentation of Curves into Various Representations. PAMI-17, 1995,
pp. 1140-1153'' verwendet.\\
}

\proc{Segment}{SegmentPointList}{PointList pl, int mode, [double maxdev]}
\descr{
Die Punktliste $pl$ wird in Linien, Kreise und Ellipsen ($mode$=SPL\_ALL),
 in Linien
und Kreise ($mode$=SPL\_LINECIRC) oder nur in Linien ($mode$=SPL\_LINE) 
segmentiert. Zurückgegeben
wird eine Liste vom Typ Segment mit den Parametern der gefundenen Segment oder
im Fehlerfall NULL. Wird der Parameter $mode$ zusätzlich mit SPL\_NOCLOSE
ODER-verknüpft, so wird nicht versucht, das erste und letzte Segment zu
verbinden. 
}

\proc{Segment}{DetermineSegment}{PointList pl, int pa, int pe, int type, Segment sl, int* ma, double* md}
\descr{
An den Ausschnitt [$pa$,$pe$] der Punktliste $pl$ wird genau ein 
Segment angepaßt. Der Parameter type definiert die in Frage kommenden
Segmenttypen (ODER-Verknüpfung):\\
DS\_LINE - Liniensegment\\
DS\_CIRCLE - Kreissegment\\
DS\_ELLIPSE - Ellipsensegment\\
Wird eine bereits vorhandene Segmentliste $sl$ übergeben, so wird das
gefundene Segment angehängt (sonst $sl$=NULL). Der zurückgegebene
Zeiger $ma$ zeigt auf den Punkt mit dem größten Abstand $md$ vom
Segment. Rückgabewert ist ein Zeiger auf das gefundene Segment bzw. NULL
bei Fehler.
}

\proc{Segment}{LineSegPointList}{PointList pl,int closed,int mlng,double mdist}
\descr{
Die geordnete Punktliste $pl$ wird in Geradensegmente zerlegt. Die
Mindestanzahl der zu einem Segment gehörenden Punkte wird durch $mlng$
festgelegt, $mdist$ gibt den maximal zulässigen Abstand eines Punktes der Punktliste zu
dem nächstliegenden Geradensegment an. Rückgabewert ist das erste Element einer
Liste von Geradensegmenten, die einen Polygonzug bilden. Falls $closed$=TRUE
ist, wird durch die Geradensegmente ein geschlossener Polygonzug gebildet.
}
\proc{PointList}{FitPolygonPointList}{PointList pl,int closed,int mlng,double mdist}
\descr{
An die geordnete Punktliste $pl$ wird ein Polygonzug angepaßt. Die
Mindestanzahl der zu einer Seite des Polygonzuges gehörenden Punkte wird durch $mlng$
festgelegt, $mdist$ gibt den maximal zulässigen Abstand eines Punktes der
Punktliste zum Polygonzug an. Rückgabewert ist eine Punktliste mit den
Eckpunkten des angepaßten Polygons. Falls $closed$=TRUE ist, wird der Endpunkt
der letzten Seite nicht in die Liste aufgenommen, da er mit dem Anfangspunkt
identisch ist.
}

\subsection{Anpassung an Grauwertverläufe}

\proc{double}{FitGrayLine}{Image img, double lp[2][2],int dist, double *p, double *phi}
\descr{
Über das Geradensegment mit den Endpunkten $lp[0]$ und $lp[1]$ wird ein
Fenster der Breite $2 \cdot dist$ gelegt. An die Punkte aus diesem Fenster
wird eine Gerade angepaßt, wobei die Punkte mit ihrem Grauwert gewichtet
werden. Punkte, deren Grauwert unterhalb des Mittelwertes im Fenster liegt,
erhalten das Gewicht Null.
}
\proc{double}{FitGradLine}{Image img, double lp[2][2],int dist, double *p, double *phi}
\descr{
Über das Geradensegment mit den Endpunkten $lp[0]$ und $lp[1]$ wird ein
Fenster der Breite $2 \cdot dist$ gelegt. An die Punkte aus diesem Fenster
wird eine Gerade angepaßt, wobei die Punkte mit ihrem Gradientenbetrag 
gewichtet werden. Punkte, deren Gradientenbetrag unterhalb des Mittelwertes 
im Fenster liegt, erhalten das Gewicht 0.0 .
}
%%% Local Variables: 
%%% mode: latex
%%% TeX-master: "icedoc_base"
%%% End: 
