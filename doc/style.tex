\nsection{Richtlinien zum Schreiben von ICE-Funktionen}{styleGuide}

\subsection{Äußere Form}
Die Bibliothek {\bf ICE} unterliegt der LGPL, der GNU Lesser General Public
License. Neue Module, die Bestandteil der Bibliothek werden sollen, müssen
unter den gleichen Bedingungen freigegeben werden. Dazu muss am Anfang jeder
Datei der entsprechende Copyright-Eintrag vorgenommen werden.

\begin{verbatim}
/*
 * ICE - C++ - Library for image processing 
 *
 * Copyright (C) 2011 FSU Jena, Digital Image Processing Group
 * Contact: ice@uni-jena.de
 *
 * This library is free software; you can redistribute it and/or
 * modify it under the terms of the GNU Lesser General Public
 * License as published by the Free Software Foundation; either
 * version 3 of the License, or (at your option) any later version.
 *
 * This library is distributed in the hope that it will be useful,
 * but WITHOUT ANY WARRANTY; without even the implied warranty of
 * MERCHANTABILITY or FITNESS FOR A PARTICULAR PURPOSE.  See the GNU
 * Lesser General Public License for more details.
 *
 * You should have received a copy of the GNU Lesser General Public License 
 * along with this library; if not, see <http://www.gnu.org/licenses/>.
 */
\end{verbatim}

Es wird empfohlen, dass der jeweilige Autor sich darunter(!) mit der Jahreszahl
einträgt.

\begin{verbatim}
// Mark Schuster, 2012
\end{verbatim}

Ein Block-Kommentar der Form
\begin{verbatim}
/*
 * foo.cpp
 * Implementierung der Aumentisierung als ICE-Funktion
 */
\end{verbatim}
sollte die in der Datei implementierten Funktionen grundsätzlich erklären.

Die Länge der Zeilen im Quelltext sollte 78 Zeichen nicht überschreiten. Alle
Quellcodedateien müssen im utf8-Code kodiert sein. Um Kodierungsproblemen
generell aus dem Weg zu gehen, sollten Umlaute und andere Sonderzeichen nicht
verwendet werden. 

\subsection{Programm-Stil}
Unter einem Modul soll eine Sammlung von Funktionen verstanden werden, die
ähnliche Inhalte für verschiedene Datentypen bereitstellen (zum Beispiel
geometrische Transformationen für Punkte, Bilder und Vektoren) oder Funktionen,
deren unmittelbares Zusammenspiel zum Erreichen einer Wirkung erforderlich
ist (zum Beispiel Startpunktsuche und Konturfolge). 

Ein Modul kann ein oder mehrere .cpp-Dateien mit Quellcode und ein oder
mehrere .h-Header-Dateien umfassen. Oft ist es sinnvoll, {\bf eine} 
Header-Datei und {\bf mehrere} Quellcode-Dateien zu verwenden.

Jede Quellcode-Datei soll den zugehörigen Header einschließen. Dies ist 
notwendig und sinnvoll, weil
\begin{itemize}
\item in den Header-Dateien definierte Konstanten in der Quelldatei angewendet
  werden sollten,
\item Fehler bei Widersprüchen zwischen Deklaration in der Header-Datei und 
Implemetierung eher erkannt werden können.
\end{itemize}

Jede Datei soll genau die Header-Dateien einschließen, die zu ihrer
Übersetzung nötig sind. 
Die ``Sammel-Includedatei'' $image.h$ ist nur für die vereinfachte
Programmierung einfacher Anwendungen gedacht und sollte in 
ICE-Funktionsmodulen nicht verwendet werden. Eine Verwendung in 
ICE-Funktionen selbst hätte zur Folge, dass jede Änderung einer 
Header-Datei die Neuübersetzung aller dieser Module auslöst.

Konstante Werte aller Art sollten in den Header-Dateien als Konstanten 
definiert werden. Konstanten von generellem Interesse sind in ``defs.h''
konzentriert. Konstanten der einzelnen Module oder einzelner Funktionen
sollten in der jeweiligen Header-Datei definiert werden und sollten einen
den Modul oder die Funktion charakterisierenden Prefix erhalten, um
Kollisionen auszuschließen (zum Beispiel: TRM\_SHIFT, TRM\_AFFINE als 
Konstanten für Transformationen). 

Konstanten in Form von Präprozessor-Makros müssen groß geschrieben werden.
Generell sind Konstanten der Form \verb+const int vier = 4;+ Präprozessor-Makros
vorzuziehen. Insbesondere sollten in Klassen statische Klassen-Konstanten
verwendet werden.

Alle ICE-Funktionen sind im Namensraum {\bf ice} zu definieren.

Alle Namen sollten so gewählt werden, dass der Sinn der Funktion, Methode oder
Klasse möglichst gut erkennbar ist. Klassennamen sollten groß geschrieben 
werden, Funktions- und Methoden-Namen klein.
Der Name von Funktionen und Methoden sollte mit einem Verb beginnen.
Schlecht ist zum Beispiel der Name der Funktion {\bf Line()} zum 
Zeichnen von Geraden und Strecken, weil dies keine Aktion beinhaltet. 
Besser wäre also {\bf drawLine()}. 
Einzelne selbständige Namensbestandteile werden groß geschrieben, 
unabhängig davon ob es sich um Substantive handelt oder nicht, 
z.B. fitPolygonContur (CamelCase-Schreibweise).

Neue Datentypen werden vorzugsweise als Klassen angelegt. 
Basisfunktionalität zur Arbeit mit der Klasse sollte als Methode 
implementiert werden. Dazu gehören besonders Konstruktoren, 
Typwandlungskonstruktoren, Zugriffsmethoden.

Bei Typwandlungskonstruktoren ist zu prüfen, wieweit eine versehentliche,
ungewollte Typwandlung auftreten könnte: Ist dies der Fall, sollte der 
Konstruktor durch das Schlüsselwort {\bf explicit} gekennzeichnet werden.

Im Normalfall sollten alle Methoden als {\bf virtual} deklariert
werden. Ausnahmen sind nur bei {\bf zeitkritischen Methoden}
sinnvoll. Dies sollte auch klar dokumentiert werden, da sonst in 
Zusammenhang mit abgeleiteten Klassen ein unerwartetes Verhalten 
auftreten kann.

Umfangreichere Funktionalität sollte nicht als Methode, sondern als Funktion
implementiert werden, da bei Verwendung einer Klasse in einem Programm alle
Methoden dazugelinkt werden.

\subsection{Fehlerbehandlung}
Ziel der Fehlerbehandlung ist, dass Fehlermeldungen möglichst nur durch die
vom Anwender aufgerufenen Funktionen erfolgen. Deshalb muss jede ICE-Funktion,
die andere Funktionen aufruft, bei denen Fehler auftreten können, diese
auswerten und ihrerseits geeignet reagieren.

Das bedeutet, dass vor Aufruf einer Funktion die Fehlermeldungen auszuschalten
sind (\see{OffMessage}) und danach wieder einzuschalten. Die Funktionen zum
Ein- und Ausschalten müssen unbedingt symmetrisch aufgerufen werden, da bei 
Verschachtelung durch Aufruf anderer ICE-Funktionen jedes Ausschalten 
wieder aufgehoben werden muss. 
Nach dem Aufruf ist auf Fehler zu testen und bei Fehler geeignet zu 
reagieren. Das kann auch das einfache Weiterreichen des Fehlers sein 
(mit eventuell angepaßter Fehlermeldung).

Details zur Fehlerbehandlung und geeignete Werkzeuge finden sich im Abschnitt
\see{Fehlerbehandlung}.

\subsection{Dokumentation}
Jede Funktion ist in die Dokumentation aufzunehmen. Die Dokumenation erfolgt
im TEX-Format, woraus sowohl eine druckbare Version als auch eine HTML-Version
erzeugt wird.

Die Funktionen müssen in eine geeignete Sektion eingeordnet werden. Ist keine
entsprechende Sektion vorhanden, so muss diese angelegt werden. Neue Sektionen
sollten neben den Funktionsbeschreibungen auch eine kurze allgemeine Einführung
beinhalten. Diese soll insbesondere die zugrundeliegende Philosophie der
Funktionen verdeutlichen.

Die Beschreibung der Funktionen erfolgt mit folgenden Makros:
\begin{itemize}
\item
\begin{verbatim}
\proch{returntype}{funktionsname}{parameterliste}{includefile}
\end{verbatim}
Kopf der Funktionsdeklaration.
\item
\begin{verbatim}
\procf{returntype}{funktionsname}{parameterliste}
\end{verbatim}
Weitere mögliche Form des Aufrufs, eine Deklaration mit {\bf proch} muss
vorangegangen sein.
\item
\begin{verbatim}
\descr{beschreibung}
\end{verbatim}
Inhaltliche Erklärung der Funktion.
\end{itemize}

Für Klassen erleichtern folgende Latex-Makros die Arbeit:
\begin{itemize}
\item
\begin{verbatim}
\ctor{klassenname}{parameterliste}{includefile}
\end{verbatim}
Der erste Konstruktor einer Klasse.
\item
\begin{verbatim}
\ctorf{klassenname}{parameterliste}
\end{verbatim}
Folgende Konstruktoren einer Klasse. Die Verwendung von \verb+\ctor+ muss
vorausgegangen sein.
\item
\begin{verbatim}
\method{returntype}{methodenname}{parameterliste}
\end{verbatim}
Methoden einer Klasse. Die Verwendung von \verb+\ctor+ muss vorausgegangen 
sein. Die Beschreibung der Methode wird mit \verb+\descr+ angehängt.
\end{itemize}

In der Funktionsbeschreibung werden die Namen von Parametern im 
mathematischen Modus geschrieben, also mit \$ eingeklammert.\\

Beispiel:
\begin{verbatim}
\proch{void}{OpenAlpha}{unsigned char *windowname}{visual.h}
\descr{
Es wird ein Textfenster mit dem Namen $windowname$ angelegt und 
geöffnet. Auch ohne expliziten Aufruf dieser Funktion wird ein 
Textfenster geöffnet, wenn eine Funktion zur Ausgabe bzw. Eingabe 
aufgerufen wird.
}
\end{verbatim}

%%% Local Variables: 
%%% mode: latex
%%% TeX-master: "icedoc_base"
%%% End: 
