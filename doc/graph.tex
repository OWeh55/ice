\nsection{Graphen}{graph}
\subsection{Bäume}
\subsubsection{KDTree}
\hypertarget{KDTree}
Ein KDTree ist ein Binärbaum, dessen Knoten k-dimensionale Werte (Vektoren) 
enthalten. In jeder Verzweigung wird eines der k Merkmale ausgewählt und die Vektoren 
werden anhand dieses Merkmales auf den linken oder rechten Teilbaum verteilt. 
Links wird gewählt, wenn das gewählte Merkmal kleiner als das Merkmal in 
diesem Blatt ist, sonst rechts.

Die Template-Klasse {\bf KDTree} realisiert einen KDTree. Der Template-Typ {\bf T}
der zu speichernden Elemente muss die Eigenschaften von {\bf vector} besitzen, das 
heißt die Operatoren \verb+operator[](int)+ und die Methode \verb+size()+ besitzen.
Für die Verwendung im Kontext von Klassifizierungsaufgaben ist als Typ die Klasse
{\bf ClassSample} geeignet.

\ctor{KDTree}{}{KDTree.h}
\descr{Legt einen noch leeren KDTree an.}

\method{void}{create}{\cvector{T} s}
\descr{Baut den Baum mit den Elementen des Vektors auf.}

\method{T*}{findNearest}{\cvector{double} v}
\methodf{T*}{findNearest}{const T \&s}
\descr{Findet im Baum den Knoten mit geringstem Abstand zum gegebenen Vektor.}

\method{void}{findKNearest}{\cvector{double} v, int k,\vector{const T*} \&neighbors, \vector{double} \&distances}
\methodf{void}{findKNearest}{const T \&v,int k,\vector{const T*} \&neighbors,\vector{double} \& distances}
\descr{Findet die k nächsten Nachbarn zum Vektor $v$ und gibt diese und die Distanzen auf $neighbors$ 
und $distances$ zurück. 
}

\method{void}{write}{ostream \&os}
\descr{Schreibt den Baum in den übergebenen Stream.}
\method{void}{read}{istream \&is}
\descr{liest den Baum aus dem übergebenen Stream.}

Das Format ist von {\bf read} und {\bf write} ist nicht für die Darstellung für den Menschen geeignet. Es dient
dazu, einen Baum in einer Datei zu speichern und wieder zu lesen. Es ist nicht garantiert, dass die mit 
dieser Version erzeugten Dateien auch in späteren Versionen wieder lesbar sind, da sich das Format ändern kann.

\subsubsection{ObliqueTree}
\subsection{Forest}
\hypertarget{Forest}{}
Die Klasse {\bf Forest} ist eine Repräsentation von Bäumen.
Jedes der {\bf n} Elemente einer zugrunde liegenden Menge sind in je genau 
einem Baum enthalten. Der Wald kann somit aus bis zu {\bf n} Bäumen 
bestehen, die dann jeder aus genau einen Knoten (Wurzel) bestehen. 
Das andere Extrem ist {\bf ein} Baum, der alle Knoten enthält. Die 
Elemente der zugrunde liegenden Menge werden über ihren Index 
(ganze Zahl von {\bf 0} bis {\bf n-1}) verwaltet. Jeder Knoten 
ausser der Wurzel hat einen Verweis auf den Vater.

\subsubsection{Konstruktoren}
\proch{}{Forest::Forest}{int n=0}{forest.h}
\descr{Legt einen Wald aus einer Menge mit $n$ Elementen an. Jedes Element 
ist Wurzel seines eigenen Baumes.}

\proch{int}{Forest::Reset}{int n}{forest.h}
\descr{Setzt einen Wald in den Ursprungszustand zurück wie im Konstruktor 
beschrieben.}

\subsubsection{Methoden}
\proch{int}{Forest::Size}{}{forest.h}
\descr{
Gibt die Anzahl der Elemente der zugrundeliegenden Menge des Waldes zurück.
}

\proch{int}{Forest::nTree}{bool ignoreZeroLengthTrees=false}{forest.h}
\descr{Gibt die Anzahl der Bäume des Waldes zurück. Ist $ignoreZeroLengthTrees$ wahr,
so werden Bäume ignoriert, die nur aus der Wurzel bestehen.
}

\proch{int}{Forest::getRefCount}{int x}{forest.h}
\descr{Zählt die zum Knoten $x$ gehörigen Kinder. Ist für Blätter von 
Bäumen gleich Null.}

\proch{int}{Forest::getNeighborCount}{int x}{forest.h}
\descr{Zählt die Zahl der Nachbarn des Knotens $x$.}

\proch{void}{Forest::setFather}{int x,int f=Forest::rootval}{forest.h}
\descr{
Setzt den Knoten $f$ als Vater des Knotens $x$ ein. Damit wird $x$ dem 
entsprechenden Baum zugeordnet. Der Defaultwert $Forest::rootval$ bewirkt, 
dass der Knoten keinen Vater besitzt und selbst zur Wurzel wird.
}

\proch{bool}{Forest::isRoot}{int x}{forest.h}
\descr{Gibt zurück, ob der Knoten $x$ Wurzel eines Baumes ist.}

\proch{bool}{Forest::isLeaf}{int x}{forest.h}
\descr{Gibt zurück, ob der Knoten $x$ ein Blatt ist. Eine Wurzel ist 
kein Blatt, auch wenn sie allein steht und somit keine Kinder hat.}

\proch{void}{Forest::makeRoot}{int x}{forest.h}
\descr{
Organisiert den zu $x$ gehörenden Baum derart um, dass $x$ die Wurzel wird.
}

\proch{int}{Forest::Father}{int x}{forest.h}
\descr{
Gibt den Index des Vaters des Knotens $x$ zurück. Wenn $x$ die Wurzel 
des Baumes ist, wird der Wert Forest::rootval zurückgegeben.
}

\proch{int}{Forest::Root}{int x}{forest.h}
\descr{
Ermittelt die Wurzel des Baumes, zu dem der Knoten $x$ gehört.
}

\proch{int}{Forest::Depth}{int x}{forest.h}
\descr{Ermittelt die Tiefe des Knotens $x$, das heisst die Zahl der Schritte
bis zur Wurzel.}

\subsection{Minimal aufspannender Baum}
\hyperref{MinTree}{}
Der minimal aufspannende Baum zu einer Punktmenge ist ein Baum, der alle Punkte 
als Knoten enthält und für den die Summe der Kantenlängen minimal ist. Um 
einen minimal aufspannenden Baum zu erzeugen, wird zunächst eine Instanz der
Klasse \class{MinTree} angelegt. Der Konstruktor erzeugt eine interne
Repräsentation des minimal auspannenden Baumes. Mit den Methoden 
von \class{MinTree} kann der Baum als \class{Forest} abgefragt werden.

\ctor{MinTree}{const \vector{Point} \&pointlist}{mintree.h}
\descr{Erzeugt einen minimal aufspannenden Baum zu den Punkten der 
Punktliste pointlist anhand der euklidischen Distanzen.}

\ctor{MinTree}{const \vector{\vector{double} } \&nodelist,
const VectorDistance \&vd = EuclideanVectorDistance()}{mintree.h}
\descr{Erzeugt einen minimal aufspannenden Baum zu den Knoten von
nodelist. Jeder Knoten ist durch einen Vektor von Merkmalen 
charakterisiert. Durch Vorgabe einer \class{VectorDistance} kann
eine beliebige Metrik verwendet werden.}
 
\ctor{MinTree}{const Matrix \&distances}{mintree.h}
\descr{Erzeugt einen minimal aufspannenden Baum zu Knoten, deren
Abstände in der Matrix distances vorgegeben sind (\see{DistanceMatrix}).}
 
\method{double}{getTree}{Forest \&f}
\descr{Speichert den ermittelten minimal aufspannenden Baum 
im \class{Forest} f.}

\method{double}{getForest}{Forest \&f, double maxlen}
\descr{Erzeugt aus dem minimal aufspannenden Baum einen Graphen,
der um alle Kanten länger als maxlen reduziert ist. Das Ergebnis wird 
in \class{Forest} f gespeichert. Dieser Graph besteht aus mehreren Bäumen.}

\method{double}{getCluster}{Forest \&f, int nCluster}
\descr{Erzeugt aus dem minimal aufspannenden Baum einen Graphen,
der durch Entfernen der längsten Kanten in nCluster Teilbäume
zerlegt ist. Das Ergebnis wird in \class{Forest} f gespeichert. 
Sind mehrere Kanten gleich lang, so können auch mehr als nCluster
Teilbäume entstehen, da bei Kanten gleicher Länge immer alle diese 
Kanten entfernt werden.}

\proch{double}{computeMinTree}{const vector$<$Point$>$ \&pointlist,Forest \&
tree,double maxlen=0.0}{mintree.h}
\procf{double}{computeMinTree}{const Matrix \&pointlist,Forest \& tree,double maxlen=0.0}
\descr{Für die Punkte der übergebene Liste $pointlist$ wird der minimal 
aufspannende Baum berechnet und auf $tree$ zurückgegeben. 
Mit dem Parameter $maxlen$ kann eine Grenze für die Länge der
Kanten angegeben werden. Das Ergebnis ist dann ein Wald, also mehrere Bäume. 
Ist $maxlen$ gleich 0 wird keine Schranke angenommen. 
Rückgabewert ist die Summe der Kantenlängen im Baum bzw. in den Bäumen.}

\proch{vector$<$vector$<$int$>$ $>$}{SplitToBranches}{Forest f}{mintree.h}
\descr{Zerlegt einen Baum in Äste. Äste sind Pfade zwischen 
Verzweigungspunkten. Ein Ast wird durch 
die Indizes der Knoten als vector$<$int$>$ beschrieben. Alle Äste werden 
dementsprechend als vector$<$vector$<$int$>$ $>$ zurückgegeben. Bäume in $f$, 
die nur aus einer Wurzel bestehen, werden ignoriert.}

\proch{int}{cutShortBranches}{Forest \&f,const vector$<$Point$>$ \&pointlist,double minlen}{mintree.h}
\procf{int}{cutShortBranches}{Forest \&f,const Matrix \&pointlist,double minlen}
\descr{
Beschneidet Zweige eines Baumes, die eine Länge unter $minlen$ aufweisen. Die 
abgeschnittenen Knoten werden in $f$ als Bäume ``der Länge Null'' geführt: 
Sie sind Wurzel eines Baumes ohne weiteren Punkt. 
}

\begprogr
\begin{verbatim}
// Erzeugung und Darstellung eines minimal aufspannenden Baumes 

   vector<Point> points; // Punktliste anlegen

   for (int i=0;i<n;i++)
       points.push_back(Point(x,y)); // Punkte an Punktliste anhängen

   Forest mintree;  
   MinTree(points,mintree); // minimal aufspannenden Baum ermitteln
   
   for (int i=0;i<mintree.Size();i++)
    {
      int father = mintree.Father(i); // Vater des aktuellen Knotens
      if (father != Forest::rootval) // Knoten ist nicht die Wurzel
      {
        Line(points[i],points[father],2,pic); // Verbindung einzeichnen
      }
    }
// Analyse der Äste des Baues
   vector<vector<int> > branches=SplitToBranches(mintree);

   for (int i=0;i<branches.size();i++)  // für alle Äste
     {
       for (int j=1;j<branches[i].size();j++) // alle Kanten des Astes        
         {
            Line(points[branches[i][j-1]],points[branches[i][j]],i,pic);
         }
     }

\end{verbatim}
\endprogr

\subsection{Dijkstra}
Der Algorithmus von Dijkstra sucht einen Pfad mit minimalen Kosten in 
einem Graphen. In bezug auf Bilder wird der Nachbarschaftsgraph für die
8er-Nachbarschaft verwendet und die Kosten entsprechen der Grauwertsumme 
auf dem Pfad.

\proch{\bsee{Contur}}{Dijkstra}{const Image \&img,\bsee{IPoint} s,IPoint e}{dijkstra.h}
\descr{
Bestimmt einen Pfad mit minimalen Kosten im Bild $img$ zwischen Startpunkt s 
und Zielpunkt $e$ und gibt diesen als \bsee{Contur} zurück.
}

\proch{Contur}{Dijkstra}{const Image \&img,IPoint s,Image \&e}{dijkstra.h}
\descr{
Bestimmt einen Pfad mit minimalen Kosten im Bild $img$ zwischen Startpunkt s 
und einem in $e$ markierten Punkt und gibt diesen als \bsee{Contur} zurück. Das 
Bild $e$ muss die gleiche Größe wie $img$ haben und die möglichen Endpunkte müssen 
einen Wert ungleich Null besitzen. Nach Ablauf von Dijkstra stehen in dem Bild $e$ 
die Richtungskodes der untersuchten Pfade.
}

\begprogr
\begin{verbatim}
// Ermittlung und Darstellung eines optimalen Pfades mit Dijkstra

   Image img;
   Image mrk;
   ...
   Contur res=Dijkstra(img,IPoint(5,5),IPoint(225,225));
   MarkContur(c,1,mrk);
   ...
   ClearImg(mrk);
   Line(555,4,666,500,1,mrk); // "ausgedehntes" Ziel
   res=Dijkstra(img,IPoint(5,5),mrk);
   ClearImg(mrk);
   MarkContur(c,1,mrk);

\end{verbatim}
\endprogr

\subsection{Delaunay-Triangulation}
Die Delaunay-Triangulation einer Punkt-Menge erfolgt mittels der Klasse
\class{Delaunay}. Die Konstruktion der Klasse mittels einer Punktmenge als 
\verb+vector<Point>+ führt die Triangulation aus. Danach kann mittels der 
weiteren Methoden das Ergebnis in verschiedenen Formen abgefragt werden 
und mittels Delaunay::draw gezeichnet werden.

\subsubsection{Konstruktoren}
\proch{}{Delaunay::Delaunay}{\vector{Point} \&nodesp}{delaunay.h}
\descr{Konstruiert einen Delaunay-Graphen mit den übergebenen Punkten 
als Ecken.}

\proch{}{Delaunay::Delaunay}{const PolygonalCurve \&curv}{delaunay.h}
\descr{Konstruiert einen Delaunay-Graphen aus den Eckpunkten einer
geschlossenen polygonalen Kurve (\class{PolygonalCurve}) oder einem 
Polygon (\class{Polygon}).
}

\subsubsection{Ergebnis-Abfrage}
Die Ergebnisse werden als Liste von Kanten beziehungsweise Dreiecken 
bereitgestellt. Dabei gibt es jeweils Varianten, die dies als 
geometrisches Objekt oder über die Indizes der Knoten in der 
originalen Punktliste bereitstellen. Optional kann die Ausgabe dabei
auf Kanten und Dreiecke beschränkt werden, deren Kantenlänge ein übergebenes
Limit nicht übersteigt. Wird das Limit nicht angegeben oder ist dessen
Wert kleiner gleich Null, so erfolgt keine Bschränkung.

\proch{void}{Delaunay::getTriangles}{\vector{Triangle} \&output, double limit = -1}{delaunay.h}
\descr{Die Dreiecke werden als Vektor von \class{Triangle} zurückgegeben.}

\proc{void}{Delaunay::getTrianglesI}{\vector{\vector{int}} \&output, double limit = -1}
\descr{Die Indizes der Eckpunkte jedes Dreiecks in der Originalpunktliste
werden als \vector{int} mit 3 Elementen zurückgegeben, alle Dreiecke bilden 
demnach einen \vector{\vector{int}}.
}

\proch{void}{Delaunay::getEdges}{\vector{LineSeg} \&output, double limit = -1}{delaunay.h}
\descr{Die Kanten werden als Vektor von \class{LineSeg} zurückgegeben.}

\proch{void}{Delaunay::getEdgesI}{\vector{\vector{int}} \&output, double limit = -1}{delaunay.h}
\descr{Die Kanten werden durch die Indizes ihrer Endpunkte in der 
Originalpunktliste in einem \vector{int} der Länge 2 beschrieben. Alle Kanten
bilden demnach einen \vector{\vector{int}}.}

\proch{void}{Delaunay::getRegion}{Region \&region, double limit = -1}{delaunay.h}
\descr{Die Fläche der Triangulation = die Fläche aller Dreiecke wird
als Region bereitgestellt. Ohne Beschränkung durch limit ist dies
die konvexe Hülle.}

\proch{void}{Delaunay::draw}{const Image \&img, int edgeValue = 1, int fillValue = -1, double limit = -1}{delaunay.h}
\descr{Stellt die Triangulation im Bild img dar. Die Kanten werden mit dem Wert
edgeValue, die Fläche der Dreiecke mit fillValue gezeichnet. Werden diese
Werte mit negativen Werten angegeben, erfolgt keine Darstellung.}

 
