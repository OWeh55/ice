\nsection{Klassifikation}{classification}

Alle Klassifikatoren bauen auf der abstrakten Basisklasse 
\class{Classifier} auf. Diese bestimmt das normale Interface
für die Verwendung von Klassifikatoren. 

Die Klassifikatoren ordnen den Klassen die Zahlen von 0 ... n-1 zu,
wenn der Klassifikator n Klassen klassifiziert. Negative Klassennummern
als Klassifikationsergebnis kennzeichnen Sonderfälle:
\begin{itemize}
\item[-1] Rückweisung, Objekt gehört zu keiner Klasse
\item[-2] Ergebnis ist nicht eindeutig (Rückweisung 2. Art)
\end{itemize}

Der typische Lebenszyklus eines Klassifikators besteht aus:
\begin{itemize}
\item Initalisierung und Konfiguration\\
Die Parameter eines Konstruktors werden im Konstruktor
oder in der Methode {\bf Init} festgelegt. Die 
Methode {\bf Init} erlaubt eine spätere Festlegung der 
Parameter, wenn zum Anlegen der Standard-Konstruktor 
verwendet werden soll.
\item Anlernen\\
Die Methoden {\bf Train} erlauben das Anlernen des Klassifikators
durch Vorgaben einzelner Stichproben oder einer Liste.
\item Abschluss des Lernvorganges\\
Methode {\bf Finish} verarbeitet die angelernten Stichproben. Bei 
den meisten Klassifikatoren erfolgt hier der Anlern-Vorgang, der 
die angelernten Stichproben verarbeitet.
\item Klassifizieren\\
Die Methoden {\bf Classify} können nach dem Abschluss des Lernvorganges
genutzt werden, um anhand eines Merkmalsvektors zu klassifizieren. 
\end{itemize}

\subsection{Klasse ClassSample}
Die Klasse \class{ClassSample} enthält die Merkmale einer Probe als 
\verb+vector<double>+ und die Klassen-Nummer. Die Klasse enthält die 
öffentlichen (public) Instanz-Variablen \verb+int classnr+ und 
\verb+vector<double> features+.

\ctor{ClassSample}{}{ClassSample.h}
\descr{Standard-Konstruktor.}

\ctor{ClassSample}{int c, \cvector{double} f}{classc.h}
\descr{Legt eine Instanz von \class{ClassSample} mit den gegebenen
Werten an.}

\method{double \&}{operator[]}{int i}
\descr{Gibt das Merkmal $i$ zurück.}

\method{int}{size}{}
\descr{Gibt die Zahl der Merkmale zurück.}

\subsection{Abstrakte Klasse Classifier}
\hypertarget{Classifier}{}
Die abstrakte Klasse \class{Classifier} ist die Oberklasse zu allen 
Klassifikator-Klassen.

\ctor{Classifier}{}{Classifier.h}
\descr{
Standard-Konstruktor. Wird ein Klassifikator hiermit angelegt,
so muss die Methode Init verwendet werden, um die grundlegenden
Parameter vorzugeben.
}
\ctor{Classifier}{int classes,int dimension}{Classifier.h}
\descr{
Konstruktor, der den Klassifikator mit den notwendigen
Parametern initialisiert. Bei einzelnen Klassifikatoren sind 
mehr Parameter als die Zahl der Klassen $classes$ und die
Zahl der Merkmale $dimension$ möglich oder erforderlich.
}
\ctor{Classifier}{const Classifier \&c}{Classifier.h}
\descr{
Kopier-Konstruktor. Erzeugt ein Duplikat des übergebenen 
Klassifikators $c$. Das Kopieren sollte nur bei Notwendigkeit 
erfolgen, da dies bei einzelnen Klassifikatoren die Kopie großer
Datenmengen erfordert.
}

\method{int}{Init}{}
\descr{Rücksetzen des Klasifikators. 
Verwirft alle angelernten Werte und erlaubt das neue Anlernen 
mit den gleichen Parametern.}

\method{int}{Init}{int classes,int dimension}
\descr{
Initialisieren des Klassifikators mit den gegebenen Parametern.
Bei einzelnen Klassifikatoren sind mehr Parameter als die Zahl 
der Klassen $classes$ und die Zahl der Merkmale $dimension$ 
erforderlich.
}

\method{int}{Train}{const ClassSample \&sample}
\methodf{int}{Train}{int cl,const \vector{double} \&feat}
\methodf{int}{Train}{int cl,const Vector \&feat}
\descr{
Belehrt den Klassifikator mit einer Probe $sample$ oder einem einzelnen 
Merkmalsvektor $feat$, der der Klasse $cl$ zu zuordnen ist
}

\method{int}{Train}{const \vector{ClassSamples} \&samples}
\descr{
Belehrt den Klassifikator mit einer Liste von Stichproben $samples$.}

\method{int}{Train}{const Matrix \&m}
\descr{
Belehrt den Klassifikator mit einer klassifizierten Stichprobe.
Die Matrix $m$ ist eine (zeilenweise) Liste von Vektoren, die
die Merkmale und in der letzten Spalte die Klassennummer 
enthalten}

\method{int}{Train}{const Matrix \&features,const IVector \&classnr}
\descr{
Belehrt den Klassifikator mit einer Liste von Stichproben.
Die Matrix $features$ ist eine Liste von (zeilenweisen) Merkmalsvektoren,
der Integer-Vektor $classnr$ enthält die jeweilige Klassennummer.}

\method{int}{Finish}{}
\descr{Das Anlernen des Klassifikators wird beendet. Hier finden
Klassifikator-spezifische Berechnungen statt. 
Finish {\bf muss} vor der Verwendung des Klassifikators aufgerufen werden.}

\method{double}{Test}{const \vector{ClassSample} \&samples}
\method{double}{Test}{const Matrix \&m}
\method{double}{Test}{const Matrix \&features,const IVector \&classnr}
\descr{
Verifiziert den Klassifikator anhand der Stichprobe.
Der Rückgabe-Wert enthält den relativen Anteil korrekt
klassifizierter Proben.
}

\method{double}{Test}{const Matrix \&m,Matrix \&f}
\descr{
Verifiziert den Klassifikator anhand der klassifizierten 
Stichprobe in der Matrix $m$. Der Rückgabe-Wert enthält den
relativen Anteil richtig klassifizierter Proben. In der 
zurückgegebenen Matrix f werden für jede Objektklasse
(Zeilen) angegeben, welcher Anteil von Objekten dieser
Klasse in die jeweiligen Klassen (Spalten) klassifiziert 
wurden.
}

\method{int}{Classify}{const \vector{double} \&features}
\methodf{int}{Classify}{const Vector \&features}
\descr{
Klassifiziert den Merkmalsvektor $feat$ und liefert die Klassennummer 
zurück. Rückweisungen erster Art (das Muster gehört zu keiner Klasse) 
werden durch den Wert Classifier::rejected == -1 gekennzeichnet. 
Bei Rückweisungen zweiter Art (das Muster gehört zu mehreren Klassen) 
ist der Rückgabewert Classifier::ambiguous == -2.
}

\method{int}{Classify}{const \vector{double} \&features, 
                             \vector{double} \&probabilities}
\descr{
Klassifiziert den Merkmalsvektor $features$ und liefert die Klassennummer 
zurück. Auf dem Vektor $probabilities$ werden für alle Klassen 
Wahrscheinlichkeiten bereitgestellt. Für Klassifikatoren, bei denen
keine differenzierte Angabe der Wahrscheinlichkeiten erfolgen kann, ist 
die Wahrscheinlichkeit der erkannten Klasse auf 1.0, die der anderen 
Klassen auf 0.0 gesetzt.
}

\proc{int}{Classifier::Classify}{Matrix \&m}
\descr{
Klassifiziert die Liste von Merkmalsvektoren (Zeilen) in der
Matrix $m$. Das Ergebnis wird als zusätzliche Spalte an die
Matrix angehängt.
}

\proc{int}{Classifier::Classify}{const Matrix \&m,IVector \&cl}
\descr{
Klassifiziert die Liste von Merkmalsvektoren (Zeilen) in der
Matrix $m$. Das Ergebnis wird in Integer-Vektor $cl$ 
eingetragen.
}

\proc{int}{Classifier::Write}{const string \&fn}
\descr{
Speichert den Klassifikator in der Datei $fn$. Der Klassifikator muss
angelernt und mit \bsee{Classifier::Finish} verarbeitet worden sein.
}

\proc{int}{Classifier::Read}{const string \&fn}
\descr{
Liest den Klassifikator aus der Datei $fn$ ein.
}

\subsection{Abstrakte Klasse ClassifierNormal}
Die abstrakte Klasse \class{ClassifierNormal} dient der Implementierung
von Klassifikatoren, die
\begin{itemize}
\item alle Stichproben während des Anlernen aufsammeln und
innerhalb von {\bf Finish} verarbeiten,
\item (optional) eine Normalisierung der Merkmalswerte vornehmen wollen.
\end{itemize}

\ctor{ClassifierNormal}{}{ClassifierNormal.h}
\descr{
Standard-Konstruktor. Wird ein Klassifikator hiermit angelegt,
so muss die Methode Init verwendet werden, um die grundlegenden
Parameter vorzugeben.
}

\ctor{ClassifierNormal}{int classes,int dimension}{ClassifierNormal.h}
\descr{
Konstruktor, der den Klassifikator mit $classes$ Klassen und
jeweils $dimension$ Merkmalen initialisiert.
}

\method{void}{doNormalization}{int mode = normalizeCenter $|$ normalizeScaling $|$ normalizeIsotropic}
\descr{Die Methode wird von abgeleiteten Klassen in {\bf Finish} 
aufgerufen, um eine Normalisierung zu ermitteln und die Stichprobe 
zu normalisieren. Der Parameter $mode$ steuert die vorzunehmenden
Normalisierung:}
\begin{tabular}{|l|l|} \hline
Konstante: & Normalisiert ... \\ \hline
ClassifierNormal::normalizeCenter & ... die Mittelwerte auf 0.0 \\ \hline
ClassifierNormal::normalizeScaling & ... die Standardabweichung der
einzelnen Merkmale auf 1.0 \\ \hline
ClassifierNormal::normalizeIsotropic & ... die Standardabweichung 
aller Merkmale auf 1.0 \\ \hline
\end{tabular}

\method{void}{normalize}{\vector{double} \&feat}
\descr{Die Methode wird in der Klassifiaktion in abgeleiteten Klassen
aufgerufen, um den Merkmalsvektor mit der ermitteleten Klassifikation 
zu normalisieren.}

\subsection{Minimum-Distanz-Klassifikator}
\hypertarget{ClassifierMD}{}
Der Minimum-Distanz-Klassifikator ermittelt für jede Klasse einen
Repräsentanten, der dem Mittelwert der angelernten Stichprobe 
entspricht. Zur Klassifikation einer Probe wird die Klasse ermittelt, 
deren Repräsentant den geringsten Abstand zur Probe hat.

\ctor{ClassifierMD}{}{ClassifierMinimumDistance.h}
\descr{
Standard-Konstruktor. Wird ein Klassifikator hiermit angelegt,
so muss die Methode Init verwendet werden, um die grundlegenden
Parameter vorzugeben.
}

\ctor{ClassifierMD}{int classes,int dimension}{ClassifierMinimumDistance.h}
\descr{
Konstruktor, der den Klassifikator bereits mit den notwendigen
Parametern initialisiert. Parameter sind die Zahl der 
Klassen $classes$ und die Zahl der Merkmale $dimension$.
}
\ctor{ClassifierMD}{const ClassifierMD \&c}{ClassifierMinimumDistance.h}
\descr{
Kopier-Konstruktor. Erzeugt ein Duplikat des übergebenen 
Klassifikators $c$.
}

Alle Methoden dieser Klasse entsprechen in Ihrem Verhalten dem
im Abschnitt für die Klasse $Classifier$ beschriebenen.

\subsection{Nearest-Neighbour-Klassifikator}
\hypertarget{ClassifierNearestNeighbor}{}

Der Nearest-Neighbour-Klassifikator klassifiziert anhand aller 
beim Anlernen gesehenen Proben. Dabei wird die Probe gesucht, die
vom aktuellen Merkmalsvektor den geringsten Abstand hat, Resultat 
ist dann dessen Klassenzugehörigkeit. Durch Festlegen einer Zahl 
von Nachbarn wird der Klassifikator zum K-Nearest-Neighbor-Klassifikator,
der unter den $k$ nächsten Nachbarn die Klasse durch Abstimmung 
ermittelt.

\ctor{ClassifierNearestNeighbor}{}{ClassifierNearestNeighbor.h}
\descr{
Standard-Konstruktor. Wird ein Klassifikator hiermit angelegt,
so muss die Methode Init verwendet werden, um die grundlegenden
Parameter vorzugeben.
}

\ctor{ClassifierNearestNeighbor}{int classes,int dimension,int nNeighbors = 1,bool norm = false}ClassifierNearestNeighbor.h}
\descr{
Konstruktor, der einen Klassifikator mit $classes$ Klassen und
$dimension$ Merkmalen initialisiert. Durch Angabe ein Zahl von Nachbarn 
größer 1 wird der Klassifikator zum K-Nearest-Neighbor-Klassifikator.
Ist $norm$ gleich $true$ werden die Merkmale bezüglich anisotroper 
Skalierung normalisiert.
}

\ctor{ClassifierNearestNeighbor}{const ClassifierNearestNeighbor \&c}{ClassifierNearestNeighbor.h}
\descr{
Kopier-Konstruktor. Erzeugt ein Duplikat des übergebenen 
Klassifikators $c$.
}

\method{void}{Init}{int classes,int dimension,int neighbors = 1, bool norm = false}
\descr{[Re-]Initialisiert den Klassifikator mit den gegebenen Werten
für die Zahl der Klassen $classes$ und die Zahl der Merkmale $dimension$.
Durch Angabe ein Zahl von Nachbarn größer 1 wird der Klassifikator zum
K-Nearest-Neighbor-Klassifikator.
Ist $norm$ gleich $true$ wird eine Normalisierung bezüglich anisotroper 
Skalierung aktviert.}

\method{void}{setNNeighbors}{int k}
\methodf{int}{getNNeighbors}{}
\descr{Setzt die Zahl der zu suchenden nächsten Nachbarn fest. Der 
Klassifikator wird zum k-Nearest-Neighbor-Klassifikator. Dies kann 
auch noch nach dem Aufruf von {\bf Finish()} erfolgen.}

\method{void}{setConsiderDistances}{bool considerDistances}
\methodf{bool}{getConsiderDistances}{}
\descr{Setzt beziehungsweise liest ein Flag, welches die Berücksichtigung
der Distanzen bei der Entscheidung des K-Nearest-Neighbor-Klassifikators
ermöglicht.}

\method{void}{setRejectionThreshold}{double theshold}
\methodf{double}{getRejectionThreshold}{}
\descr{Diese Schwelle gibt an, welcher Teil der k nächsten Nachbarn zu
einer Klasse gehören muss. Überschreitet die häufigtse Klasse diese
Schwelle nicht, erfolgt eine Zuordnung zur Rückweisungsklasse (-1).
}

Alle weiteren Methoden dieser Klasse entsprechen in Ihrem Verhalten dem
im Abschnitt für die Klasse $Classifier$ beschriebenen.

\subsection{Bayes-Klassifikator für zwei Klassen}
\hypertarget{ClassifierBayes2}{}
Der Bayes-Klassifikator für 2 Klassen ohne Rückweisung verlangt die Angabe der
Verwechslungskosten c01 von Klasse 0 mit Klasse 1 und c10 von Klasse 1 mit
Klasse 0. Weiterhin müssen die A priori-Klassenwahrscheinlichkeiten p0 für
Klasse 0 und p1 für Klasse 1 angegeben werden. Dieser Bayes-Klassifikator setzt
Normalverteilung voraus und arbeitet mit den Mittelwertvektoren und
den Kovarianzmatrizen aller Merkmale für jede Klasse. 
Die Trennfunktion für die beiden Klassen stellt eine Hyper-Fläche 2. Ordnung dar.

\ctor{ClassifierBayes2}{}{ClassifierBayes2.h}
\descr{
Standard-Konstruktor. Wird ein Klassifikator hiermit angelegt,
so muss die Methode Init verwendet werden, um die grundlegenden
Parameter vorzugeben.
}
\ctor{ClassifierBayes2}{int classes,int dimension,
double p0,double p1,double c01,double c10}{ClassifierBayes2.h}
\descr{
Konstruktor, der den Klassifikator bereits mit den notwendigen
Parametern initialisiert. Parameter sind die Zahl der 
Klassen $classes$ und die Zahl der Merkmale $dimension$,
die a priori-Wahrscheinlichkeiten $p0$,$p1$ für die Klassen 
0 und 1, sowie die Kosten für eine Fehlklassifikation eines
Objekts der Klasse 0 in die Klasse1 bzw. der Klasse 1 in die
Klasse 0.
}
\ctor{ClassifierBayes2}{const ClassifierBayes2 \&c}{ClassifierBayes2.h}
\descr{
Kopier-Konstruktor. Erzeugt ein Duplikat des übergebenen 
Klassifikators $c$.
}

\proc{int}{ClassifierBayes2::Init}{int classes,int dimension,double p0,double p0,double c01, double c10}
\descr{
Initialisieren des Klassifikators mit den gegebenen Parametern.
Parameter sind die Zahl der Klassen $classes$ und die 
Zahl der Merkmale $dimension$,
die a priori-Wahrscheinlichkeiten $p0$,$p1$ für die Klassen 
0 und 1, sowie die Kosten für eine Fehlklassifikation eines
Objekts der Klasse 0 in die Klasse1 bzw. der Klasse 1 in die
Klasse 0.
}

Alle weiteren Methoden dieser Klasse entsprechen in Ihrem 
Verhalten dem im Abschnitt für die Klasse $Classifier$ beschriebenen.

\subsection{Bayes-Klassifikator}
\hypertarget{ClassifierBayes}{}
Der Bayes-Klassifikator setzt Normalverteilung voraus und arbeitet 
mit den Mittelwertvektoren und den Kovarianzmatrizen aller Merkmale 
für jede Klasse. Der Klassifikator ist oft auch bei Abweichung von der
Normalverteilung noch einsetzbar. Die Wahrscheinlichkeiten der Zugehörigkeit 
zu der jeweiligen Klasse, die von Classify bereitgestellt werden,
sind dann jedoch nicht zutreffend.

\ctor{ClassifierBayes}{}{ClassifierBayes.h}
\descr{
Standard-Konstruktor. Wird ein Klassifikator hiermit angelegt,
so muss die Methode Init verwendet werden, um die grundlegenden
Parameter vorzugeben.
}

\ctor{ClassifierBayes}{int classes,int dimension,
bool rejection=false, int apm=APM\_EQUAL}{ClassifierBayes.h}
\descr{
Konstruktor, der den Klassifikator bereits mit den notwendigen
Parametern initialisiert. Parameter sind die Zahl der 
Klassen $classes$ und die Zahl der Merkmale $dimension$,
ein boolscher Wert $rejection$ für die anzulegende Rückweisungsklasse und ein
Modus $apm$ für die Behandlung der a priori-Wahrscheinlichkeiten.
Der Modus APM\_EQUAL setzt gleiche a priori-Wahrscheinlichkeiten für alle
Klassen an. APM\_CONSTRUCTOR erlaubt die explicite Festlegung der a
priori-Wahrscheinlichkeiten durch Benutzung eines speziellen
Konstruktors. Bei Angabe von APM\_TEACH wird die Statistik der Lernstichprobe
verwendet.
}

\ctor{ClassifierBayes}{int classes, \cvector{double} app, 
                       bool rejection}{ClassifierBayes.h}
\descr{
Konstruktor, der den Klassifikator bereits mit den notwendigen
Parametern initialisiert. Parameter sind die Zahl der 
Klassen $classes$, ein Vektor mit den a priori-Wahrscheinlichkeiten
und ein boolscher Wert $rejection$ für die anzulegende Rückweisungsklasse.
}

\ctor{ClassifierBayes}{const ClassifierBayes \&c}{ClassifierBayes.h}
\descr{
Kopier-Konstruktor. Erzeugt ein Duplikat des übergebenen 
Klassifikators $c$.
}

\method{int}{Init}{int classes,int dimension}
\descr{
Initialisieren des Klassifikators mit den gegebenen Parametern.
Parameter sind die Zahl der Klassen $classes$ und die 
Zahl der Merkmale $dimension$.
}

Alle weiteren Methoden dieser Klasse entsprechen in Ihrem 
Verhalten dem im Abschnitt für die Klasse $Classifier$ beschriebenen.

\subsection{Baum-Klassifikator}
\subsection{Random Forest-Klassifikator}

\subsection{Ein Anwendungs-Beispiel}
Gegeben sei ein Farbbild $rgb$, als Merkmal der Pixel werden die 
drei Farbwerte rot, grün und blau verwendet.
Im Bild \verb+label+ sei mit den Werten 1 und 2 eine Klassifizierung 
in zwei Klassen vorgenommen worden. Es ist ein Farb-Klassifikator 
anzulernen, der diese Klassen trennt.

\begprogr\begin{verbatim}
void train(Classifier &clf, const ColorImage &rgb, const Image &label)
{
 clf.Init(2, 3); // Klassifikator mit 2 Klassen und 3 Merkmalen
 vector<double> features(3); // Merkmals-Vektor
 for (int y = 0; y < rgb.ysize; y++)
   for (int x = 0; x < rgb.xsize; x++)
     {
       int classNr = label.getPixel(x, y);
       if (classNr != 0) // wenn Punkt klassifiziert ist, ...
       {
         classNr--;    // Klassennummern im Bereich 0..1
         ColorValue cv = rgb.getPixel(x,y);
         features[0] = cv.red;
         features[1] = cv.green;
         features[2] = cv.blue;
         clf.Train(classNr, features);  // Anlernen des Klassifikators
    }
  }
 clf.Finish(); // Lernvorgang beenden
}
\end{verbatim}\endprogr

Der Klassifikator soll nun angewendet werden:
\begprogr\begin{verbatim}
void classify(const Classifier &clf, const ColorImage &rgb, const Image &classimg)
{
  vector<double> features(3); // Merkmals-Vektor
  for (int y = 0; y < rgb.ysize; y++)
    for (int x = 0; x < rgb.xsize; x++)
     {
       ColorValue cv = rgb.getPixel(x, y);
       features[0] = cv.red;
       features[1] = cv.green;
       features[2] = cv.blue;
       int classNr = clf.Classify(features); // Klassifizieren
       classimg.setPixel(x, y, classNr+1);
     }
}
\end{verbatim}\endprogr

\subsection{Implementierungshinweise für weitere Klassifikator-Klassen}

Alle Klassifikatorklassen werden von der Klasse \class{Classifier} 
beziehungweise
\class{ClassifierNormal} abgeleitet.
Diese gemeinsame Basisklasse \class{Classifier} erlaubt einen alternativen 
Einsatz verschiedenster Klassifikatoren (Polymorphismus) und stellt 
außerdem verschiedene Grundfunktionen allgemein zur Verfügung.

\begin{itemize}
\item Die Konstruktoren der abgeleiteten Klasse sollten die passenden Konstruktoren der
Basisklasse aufrufen. Diese organisieren die Speicherung der Zahl der Klassen (Attribut \verb+nClasses+)
und der Zahl der Merkmale (Attribut \verb+nFeatures+).
\item Die Klasse kann im allgemeinen die Methode \verb+Init()+ der Basisklasse verwenden, die ihrerseits
die bekannten Parameter einsetzt und \verb+Init(int nClasses,int nFeatures)+ aufruft.
\item Wenn in der Initialisierung weitere Variablen gesetzt werden müssen, muss die Methode 
\verb+Init(int nClasses,int nFeatures)+ überschrieben werden. Dabei sollte das entsprechende \verb+Init+ der 
Basisklassse aufgerufen werden (chaining). Um das parameterlose \verb+Init()+ der Basisklasse weiter 
nutzen zu können, ist die Angabe von \verb+using Classifier::Init;+ notwendig.
\item Die abgeleitete Klasse sollte nicht die Methoden \verb+Train+ überschreiben, sondern die 
\verb+protected+-Methode \verb+_train(int classNr,const std::vector<double> &features)+. Wenn die Methode 
\verb+Train+ der Basisklasse aufgerufen wird, testet sie zunächst die Parameter \verb+classNr+ und die Größe des 
\verb+vector<double>+ und zählt die Lernvorgänge pro Klasse (Attribut \verb+vector<int> classTrained+). 
Dann wird die \verb+_train+ aufgerufen.
\item Die abgeleitete Klasse sollte nicht die Methode \verb+Finish+ überschreiben, sondern die 
\verb+protected+-Methode \verb+_finish()+. Wenn die Methode \verb+Finish+ der Basisklasse aufgerufen wird, 
prüft sie zunächst, ob alle Klassen angelernt wurden und leitet dann den Aufruf an \verb+_finish+ weiter.
\item Die abgeleitete Klasse sollte nicht die Methoden \verb+Classify+ überschreiben, sondern die 
\verb+protected+-Methode \verb+int _classify(const vector<double> &features,vector<double> &probabilities)+. 
Wenn eine Methode \verb+Classify+ der Basisklasse aufgerufen wird, prüft diese ob das Anlernen korrekt
beendet wurde und ob die Zahl der Merkmale stimmt und reicht dann den Aufruf weiter an  
\verb+int _classify(const vector<double> &features,vector<double> &probabilities)+. Diese Methode
bestimmt vor allem die zu $features$ passende Klasse und gibt diese als Rückgabewert zurück. Zusätzlich
kann ein Klassifikator über den Vektor $probabilities$ Wahrscheinlichkeiten für alle Klassen zurückgeben, 
die Wahrscheinlichkeit der Klasse $n$ steht in \verb+probabilities[n]+. Es gelten folgende Regeln:
\begin{itemize}
\item Klassifikatoren, die keine Wahrscheinlichkeiten berechnen können, lassen den Vektor
\verb+probabilities+ unangetastet. Eine triviale Angabe der Wahrscheinlichkeiten 0.0/1.0 wird
dann in \verb+Classifier+ bereitgestellt.
\item Hat der übergebene Vektor \verb+probabilities+ die Länge Null, ist keine Berechnung 
der Wahrscheinlichkeiten erforderlich. Klassifikatoren, die die Wahrscheinlichkeiten in jedem Fall 
berechnen, können dies tun, die Werte in \verb+probabilities+ werden aber ignoriert.
\item Sonst berechnet der Klassifikator die Wahrscheinlichkeiten für alle Klassen. Der Vektor hat 
beim Aufruf bereits die richtige Länge \verb+nClasses+, die Werte sind als undefiniert zu betrachten.
Besitzt der Klassifikator keine Rückweisung, muss die Summe der Wahrscheinlichkeiten 1.0 ergeben, 
andernfalls beschreibt die Differenz zu 1.0 die Wahrscheinlichkeit der Rückweisung.
\end{itemize}
\end{itemize}
