\nsection{Bildbearbeitung}{processing}
\label{Bildbearbeitung}

In diesem Abschnitt werden Operationen über Bilder beschrieben, deren Ergebnis
in der Regel wieder Bilder sind. \bsee{Filter} werden in einem eigenen 
Abschnitt beschrieben (\see{Filter}).
Alle Typen von Bildrepräsentationen werden hier, soweit es geht, gemeinsam 
abgehandelt (\class{Image}, \class{ImageD}, \class{ColorImage}). 
Auf die Besonderheiten bei bestimmten Typen wird speziell hingewiesen. 

\subsection{Grauwerttransformationen}

\proch{int}{BinImg}{const Image \&src,Image \&dest,int thr=1,int val=-1}{arith.h}
\procf{int}{BinImg}{Image \&src,int thr=1}
\descr{
Binarisierung des Bildes $src$. Im Zielbild $dest$ werden alle Bildpunkte, 
deren Grauwerte im Bild $src$ größer oder gleich $thr$ sind, auf $val$ 
gesetzt, alle anderen Bildpunkte auf Null. Wird $val$ auf -1 gesetzt (default),
wird der maximale Grauwert des Zielbildes verwendet.
Bei Aufruf ohne $dest$ wird das Ergebnis in $src$ geschrieben.
Die Bestimmung der Schwelle $thr$ kann durch \see{CalcThreshold} erfolgen.
}

\proch{int}{FindMin}{const Image \&img, int \&x, int \&y}{arith.h}
\procf{int}{FindMin}{const Image \&img, IPoint \&p}
\procf{int}{FindMin}{const Image \&img}
\procf{int}{FindMax}{const Image \&img, int \&x, int \&y}
\procf{int}{FindMax}{const Image \&img, IPoint \&p}
\procf{int}{FindMax}{const Image \&img}
\descr{Die Funktionen finden den minimalen/maximalen Wert im Bild $img$ und geben diesen
als Funktionwert zurück. Bei Angabe der Parameter $x$ und $y$ bzw. $p$ wird die Position,
an der dieser Wert gefunden wurde, darauf zurückgegeben.
\seealso{PeakValuation}
}

\proch{int}{GetGrayLimits}{const Image \&img,int \&min,int \&max, 
int mode=GV\_QUANTILE, double quantile = 0.01}{lintrans.h}
\procf{int}{GetGrayLimits}{const Hist \& h, int\& min, int\& max, 
int mode=GV\_QUANTILE, double quantile = 0.01 }
\descr{
Es werden aus dem Bild $img$ bzw. Histogramm $h$ eine untere Grenze
$min$ und eine obere Grenze $max$ bestimmt. Dieses Grauwertintervall kann zum
Beispiel für die lineare Grauwert-Transformation \see{GrayTransformLimits}
genutzt werden, um den vollen Grauwertbereich auszunutzen. Diese Grenzen sind
nicht das exakte Minimum oder Maxmimum wie in \see{FindMin} und \see{FindMax},
sondern ignorieren extreme Wert, die als Ausreißer betrachtet werden können.
Für $mode$ kann GV\_QUANTILE (default) oder GV\_MEAN gewählt werden.
Bei $mode$=GV\_QUANTILE werden die Quantile $minq$ und $maxq$
für die Ermittlung der unteren und die oberen Grenze verwendet.
}

\proch{int}{GrayTransform}{const Image \&src, const Image \&dest, double a1, double a0}{lintrans.h}
\procf{int}{GrayTransform}{const Image \&src, double a1, double a0}
\descr{
Die Grauwerte des Bildes $src$ werden nach der Formel 
$g \prime  = a1 \cdot g + a0$
transformiert. Das Ergebnis steht in Bild $dest$ oder wird in $src$ eingetragen.
Bei Grauwert über- oder Unterlauf werden der maximale oder minimale Grauwert verwendet.
}

\proch{int}{GrayTransformLimits}{const Image \&src, const Image \& dest, int min, int max}{lintrans.h}
\procf{int}{GrayTransformLimits}{const Image \&img, int min, int max}
\descr{
Die Grauwerte des Bildes $src$ werden linear transformiert, so dass $min$ im
Zielbild den Wert 0 ergibt, $max$ den Wert $dst->maxval$. Das Ergebnis steht
in Bild $dst$ oder wird in $img$ eingetragen.
}

\proch{int}{GrayNormalize}{const Image \&src,const Image \& dest, int mode=GV\_QUANTILE, double quantile = 0.01}{lintrans.h}
\procf{int}{GrayNormalize}{const Image \&src, int mode=GV\_QUANTILE, double quantile = 0.01}
\descr{Die Grauwerte des Bildes $src$ werden normalisiert, indem mittels
\see{GetGrayLimits} der Grauwertbereich ermittelt und mittels
\see{GrayTransformLimits} auf den Grauwertbereich 0..$dst->maxval$
transformiert wird.}

\proch{int}{HistogramEqualization}{const Image \&img1,const Image \&img2}{icefunc.h}
\descr{
Es wird im Bild {\bf img1} ein Histogrammausgleich durchgeführt und 
das Ergebnis im Bild {\bf img2} abgespeichert.
}

\proch{int}{correctShading}{const Image \&img1,const Image \&img2, int ord=2}{shading.h}
\descr{
Es wird eine Shading-Korrektur des Bildes $img1$ durchgeführt und im Bild 
$img2$ abgespeichert. Dazu wird an das Bild eine polynomiale Funktion
angepasst. Die Ordnung des Polynoms wird durch den Parameter $ord$
bestimmt.
}

\proch{int}{correctShading}{const Image \&img1,const Image \&img2, Function2dWithFitting \&fn}{shading.h}
\descr{
Es wird eine Shading-Korrektur des Bildes $img1$ durchgeführt und im Bild 
$img2$ abgespeichert. Dazu wird die Funktion {\bf fn} an das Bild
angepasst.
}

\subsection{Bildarithmetik}
\subsubsection{Integer-Bilder}

Die Funktionen zur Bildarithmetik benötigen im allgemeinen zwei Quellbilder
und ein Zielbild. Bei fehlerfreier Abarbeitung wird der Wert $OK$
zurückgegeben, sonst ein Fehlercode. Die beteiligten Bilder müssen
die gleiche Größe haben.

\label{range}
\hypertarget{range}{}
Überschreitet der Wertebereich des Ergebnisses den Umfang des Zielbildes,
so sind mittels Parameter $mode$ folgende Strategien wählbar:

\begin{itemize}
\item MD\_NORMALIZE \\
Grauwerte werden auf den Bereich des Zielbildes linear transformiert.
\item MD\_LIMIT \\
Grauwerte werden auf den Bereich des Zielbildes begrenzt.
\item MD\_MODULO \\
Grauwerte werden durch Verwendung des Modulo-Operators auf den Bereich 
des Zielbildes begrenzt.
\end{itemize}

\proch{int}{ClearImg}{const Image \&img}{arith.h}
\descr{Alle Bildpunkte werden auf Null gesetzt.
}
\proch{int}{setImg}{const Image \&img,int val}{base.h}
\descr{Alle Bildpunkte werden auf den konstanten Wert $val$ gesetzt.
}
\proch{int}{setImg}{const Image \&img,const Function2d \& fn}{fitfn.h}
\descr{Alle Bildpunkte (x,y) werden auf den Wert der Funktion $fn(x,y)$ 
  gesetzt (\see{Func2D}). Überschreitet der Funktionswert den Bereich 
  [0..img->maxval], so wird der Wert begrenzt.
}
\proc{int}{CopyImg}{const Image \&src,const Image \&dest}
\descr{Das Bild $src$ wird nach $dest$ kopiert.}

\proch{int}{AddImg}{const Image \&img1,const Image \&img2,
  const Image \&dest,int mode=MD\_NORMALIZE}{arith.h}
\descr{
Punktweise Addition der Bilder $img1$ und $img2$:
$$vd=v1+v2$$
\hyperref{range}{}{}{{\bf mode}} steuert die Behandlung von 
Wertebereichsüberschreitungen.
}
\proc{int}{SubImg}{const Image \&img1,const Image \&img2,int smode,const Image \&dest,int mode=MD\_NORMALIZE}
\procf{int}{SubImg}{const Image \&img1,const Image \&img2,const Image \&dest,int smode=SMD\_ABSOLUTE,int mode=MD\_NORMALIZE}
\descr{
Punktweise Subtraktion der Bilder $img1$ und $img2$.

\begin{tabular}{|c|c|} \hline
smode & Wert \\ \hline
SMD\_ABSOLUTE & $vd=\vert v1-v2 \vert$ \\
SMD\_POSITIVE & $vd=max(v1-v2,0)$ \\
SMD\_SHIFT & $vd=v1-v2+max/2$ \\ \hline
\end{tabular}

\hyperref{range}{}{}{{\bf mode}} steuert die Behandlung von 
Wertebereichsüberschreitungen.
}

\proc{int}{MaxImg}{const Image \&img1,const Image \&img2,const Image \&dest,int mode=MD\_NORMALIZE}
\descr{Es wird punktweise das Maximum der Bilder $img1$ und $img2$ gebildet:
$$vd=max(v1,v2)$$
\hyperref{range}{}{}{{\bf mode}} steuert die Behandlung von 
Wertebereichsüberschreitungen.
}
\proc{int}{MinImg}{const Image \&img1,const Image \&img2,const Image \&dest,int mode=MD\_NORMALIZE}
\descr{Es wird punktweise das Maximum der Bilder $img1$ und $img2$ gebildet
$$vd=min(v1,v2)$$
\hyperref{range}{}{}{{\bf mode}} steuert die Behandlung von 
Wertebereichsüberschreitungen.
}
\proc{int}{InvertImg}{const Image \&img, const Image \&dest}
\procf{int}{InvertImg}{const Image \&img}
\descr{Bildinvertierung:
$$vd=(max1-vs)\cdot{maxd \over max1}$$
Bei Aufruf ohne $dest$ wird das Ergebnis in $img$ geschrieben.
}
\proch{int}{RenormImg}{const Image \&img,const Image \&dest}{arith.h}
\descr{
Das Bild $src$ wird auf die Größe und den Grauwertumfang von $dest$
umgerechnet und in $dest$ abgelegt.
}

\subsubsection{Gleitkommabilder}

\proch{int}{ClearImgD}{ImageD img}{darith.h}
\descr{Die Werte des Gleitkommabildes $img$ werden auf 0.0 gesetzt.}
\proc{int}{SetImgD}{ImageD img,double val}
\descr{Alle Bildpunkte werden auf den konstanten Wert $val$ gesetzt.}
\proc{int}{MoveImgD}{ImageD src,ImageD dest}
\descr{Das Bild $src$ wird nach $dest$ transportiert.}
\proc{int}{AddImgD}{ImageD img1,ImageD img2,ImageD dest}
\descr{Punktweise Addition der Bilder $img1$ und $img2$}
\proc{int}{MulImgD}{ImageD img1,ImageD img2,ImageD dest}
\descr{Punktweise Multiplikation der Bilder $img1$ und $img2$}
\proc{int}{LogImgD}{ImageD src,ImageD dest}
\descr{Punktweise Logarithmierung des Bildes $src$}

\subsection{Polar-Darstellung}

\proc{int}{PolarImgD}{ImageD src,ImageD dest,double r1=1,double r2=0,int sym=2}
\procf{int}{PolarImg}{const Image \&src,const Image \&dest,double r1=1,double r2=0,int sym=2}
\descr{
Transformiert das Bild $src$ in die Polar-Darstellung. Dazu werden die
Koordinaten jedes Punktes des Bildes $src$ bezüglich des Mittelpunktes
in die Betrag- und Phasenwinkel-Darstellung (r,phi) umgerechnet und der
Grauwert des jeweiligen Bildpunktes unter den Koordinaten (x=r,y=phi) 
im Bild $dest$ gespeichert.
Die Parameter $r1$ und $r2$ bestimmen den minimalen und den maximalen zu
verwendenden Radius. $r2==0$ veranlasst die Verwendung des maximal
möglichen Radius. 
Der Parameter $sym$ charakterisiert die $sym$-zählige Symmetrie des
Originalbildes (Standard sym=2 z.B. bei Amplituden-Spektren). 
Liegt mit sym>1 eine mehrzählige Rotations-Symmetrie vor, so wird 
in X-Richtung nur ein Winkelbereich von 0..360/sym erfasst.
}

\proc{int}{PolarC}{ImageD src,ImageD dest,double x,double y,double \&r,double \&phi,double r1=1,double r2=0,int sm=2}
\descr{
Berechnet Winkel $phi$ und Radius $r$, der einem bestimmten Koordinatenpaar
($x,y$) im Polarbild $dest$ entspricht. Alle anderen Parameter sind
genauso anzugeben, wie sie zur Erzeugung des Polarbildes verwendet
wurden. 
\seealso{PolarImg}
}
 
\proc{int}{LogPolarImgD}{ImageD src,ImageD dest,double r1=1,double r2=0,int sym=2}
\procf{int}{LogPolarImg}{const Image \&src,
  const Image \&dest,double r1=1,double r2=0,int sym=2}
\descr{
Transformiert das Bild src in die Polar-Darstellung. Dazu werden die
Koordinaten jedes Punktes des Bildes img1 bezüglich des Mittelpunktes
in die Betrag- und Phasenwinkel-Darstellung (r,phi) umgerechnet und der
Grauwert des jeweiligen Bildpunktes unter den Koordinaten (x=r,y=phi) 
im Bild img2 gespeichert.
Die Parameter r1 und r2 bestimmen den minimalen und den maximalen zu
verwendenden Radius. $r2==0$ veranlasst die Verwendung des maximal
möglichen Radius. 
Der Parameter sym charakterisiert die sym-zählige Symmetrie des
Originalbildes (Standard sym=2 z.B. bei Amplituden-Spektren). 
Liegt mit sym>1 eine mehrzählige Rotations-Symmetrie vor, so wird 
in X-Richtung nur ein Winkelbereich von 0..360/sym erfasst.
}

\proc{int}{LogPolarC}{ImageD src,ImageD dest,double x,double y,double \&r,double \&phi,double r1=1,double r2=0,int sym=2}
\descr{
Berechnet Winkel $phi$ und Radius $r$, der einem bestimmten Koordinatenpaar
($x,y$) im Log-Polarbild $dest$ entspricht. Alle anderen Parameter sind
genauso anzugeben, wie sie zur Erzeugung des Log-Polarbildes verwendet
wurden. 
\seealso{LogPolarImg}
}

\subsection{Farbraum-Transformationen}
Bilder der Klasse \class{ColorImage} speichern Farbwerte im RGB-Format.
Neben Funktionen zur Farbraum-Transformation einzelner 
Farbwerte (\see{RgbToHsi}) gibt es Funktionen zur Transformation von 
Bildern. Im Farbraum RGB werden Bilder der Klasse \class{ColorImage} 
verwendet, in anderen Farbräumen werden die einzelnen Komponenten in 
jeweils eigenen Bildern der Klasse \class{Image} abgelegt. Der 
Wertebereich der jeweiligen Farbkomponente wird dabei auf den 
Wertebereich des Bildes gestreckt.

\proch{int}{ColorImageToHsi}{const ColorImage \&src, const Image \&hue, const Image \&saturation, const Image \&intensity}{ColorSpace.h}
\procf{int}{HsiToColorImage}{const Image \&hue, const Image \&saturation, const Image \&intensity, const ColorImage \&dst}
\procf{int}{ColorImageToLab}{const ColorImage \&src,
  const Image \&lImg, const Image \&aImg, const Image \&bImg}
\procf{int}{LabToColorImage}{const Image \&lImg, const Image \&aImg, const Image \&bImg, const ColorImage \&src}
\procf{int}{ColorImageToYuv}{const ColorImage \&src, const Image \&y, const Image \&u, const Image \&v}
\procf{int}{YuvToColorImage}{const Image \&y, const Image \&u, const Image \&v, const ColorImage \&dst}
\descr{Farbraum-Transformation von RGB als \class{ColorImage} in die Farbräume YUV, HSI und Lab und zurück.}

%%% Local Variables: 
%%% mode: latex
%%% TeX-master: "icedoc_base"
%%% End: 
