\nsection{Funktionen}{functions}
\hypertarget{Funktionen}{}

\class{Function} und \class{Function2d} sind Klassen, die mittels 
des \verb+operator()+ skalare Funktionen von einem oder zwei Argumenten 
berechnen (''Funktoren''). Die in der Klassen-Hierarchie definierten 
Funktionen umfassen Polynome, geometrische Transformationen (analog 
\class{Trafo}), Verzeichnungen (analog \class{Distortion}) und erlauben 
die Verwendung von Bildern (\class{Image}, \class{ImageD}) als
zweidimensionale Funktionen.

Funktionshierarchie:
\begin{itemize}
\item \class{Function}
\begin{itemize}
\item \class{FunctionWithFitting}
\begin{itemize}
\item \class{Constant}
\item \class{Polynom1o}
\item \class{Polynom2o}
\item \class{Polynom}
\end{itemize}
\end{itemize}
\item \class{Function2d}
\begin{itemize}
\item \class{Function2dWithFitting}
\begin{itemize}
\item \class{Constant2d}
\item \class{Polynom2d1o}
\item \class{Polynom2d2o}
\item \class{Polynom2d}
\end{itemize}
\item \class{Function2dModifier}
\begin{itemize}
\item \class{Function2dParameterShift}
\item \class{Function2dValueTransform}
\item \class{Function2dParameterTransform}
\item \class{Function2dDistortion}
\end{itemize}
\item \class{ImageFunction}
\item \class{ImageDFunction}
\end{itemize}
\end{itemize}

\subsection{Fitting mit Funktionen}
\hypertarget{FittingWithFunctions}{}

Die von den Klassenhierarchien
\class{FunctionWithFitting} und \class{Function2dWithFitting} abgeleiteten 
Klassen erlauben die Bestimmung der Parameter dieser Funktionen mittels Fitting.

Fitting kann grundsätzlich über die Vorgabe einzelner Werte-Tupel erfolgen
oder durch Vorgabe einer Liste.
\begin{itemize}
\item Im Falle einzelner Werte-Tupel ist zunächst
das Fitting zu initialisieren \see{FunctionWithFitting::FitInit}, 
dann sind die Werte-Tupel mit \see{FunctionWithFitting::FitVal} einzutragen. 
Nach dem Abschluss mit \see{FunctionWithFitting::FitFinish} stehen die
gefitteten Parameter zur Verfügung.
\item Im Falle einer Liste erfolgt das komplette Fitting durch Aufruf der
 Methode \see{FunctionWithFitting::Fit}. Neben einer Wertetupel-Liste 
in Form einer \see{Matrix} können bei zweidimensionalen Funktionen auch 
Bilder zum Fitten übergeben werden.
\end{itemize}

\seealso{GaussFit}

\subsection{Eindimensionale Funktionen}
\label{Function}
\hypertarget{Function}{}
Die abstrakte Basisklasse \class{Function} beschreibt skalare 
Funktionen mit einem Parameter. Wesentliche Eigenschaft dieser
Klasse ist der \verb+operator()+, der mit einem Parameter vom Typ
\verb+double+ aufgerufen werden kann. 

\proch{double}{Function::operator()}{double x}{function.h}
\descr{Berechnet den Funktionswert der Funktion an der Stelle {\bf x}.}

\proch{void}{Function::getCoefficient}{\vector{double} \&c}{function.h}
\descr{Ermittelt die Parameter (Coefficienten) der Funktion. Diese werden
an den \vector{double} {\bf c} angehängt. Dies erlaubt das Aufsammeln 
der Parameter mehrerer Funktionen in einem \vector{double}.}

\proch{int}{Function::setCoefficient}{const \vector{double} \&para,int idx=0}{function.h}
\descr{Setzt die Parameter (Coefficienten) der Funktion. Diese werden
dem \vector{double} ab dem Index {\bf idx} entnommen. Die Funktion
gibt den Index nach dem letzten gelesenen Element zurück.
Dies erlaubt die Initialisierung mehrerer Funktionen aus einem 
\vector{double}.}

\subsection{Eindimensionale Funktionen mit Fitting}
\label{FunctionWithFitting}
\hypertarget{FunctionWithFitting}{}

Die von der Klasse \class{Function} abgeleitete abstrakte Klasse 
\class{FuntionWithFitting} beschreibt 
eindimensionale Funktionen, deren Parameter durch Fitting ermittelt werden
können. 

\proch{void}{FunctionWithFitting::FitInit}{void}{fitfn.h}
\descr{Die abstrakte Methode initialisiert das Fitten von Wertpaaren.}

\proch{void}{FunctionWithFitting::FitValue}{double x, double g, double w=1.0}{fitfn.h}
\descr{Die (abstrakte) Methode übergibt ein Wertepaar für das Fitting. Die
optionale Angabe des Gewichts $w$ erlaubt die Wichtung des Wertepaars,
negative Gewichte tragen quasi das Wertepaar wieder aus.}
\proch{int}{FunctionWithFitting::FitFinish}{}{fitfn.h}
\descr{Schließt den Prozess des Fitting ab. Danach sind die gefitteten 
Parameter der Funktion gültig.}

\proch{int}{FunctionWithFitting::Fit}{const Matrix \&m}{fitfn.h}
\descr{Die Funktion wird an die Wertepaare in der Matrix $m$ gefittet.
Dabei stellt jede Zeile der Matrix ein Wertepaar bereit: Die Matrix muss
 mindestens zwei Spalten haben. Diese Funktion umfasst alle Schritte von 
der Initialisierung (\see{FunctionWithFitting::FitInit}) über das 
Eintragen der Werte (\see{FunctionWithFitting::FitValue}) bis zum 
Abschluss des Fittens (\see{FunctionWithFitting::FitFinish}).
Vorher mit \bsee{FunctionWithFitting::FitValue} eingetragene Werte
gehen verloren.}

\proch{double}{FunctionWithFitting::operator()}{double x}{fitfn.h}
\descr{Gibt den Funktionswert der Funktion an der Stelle $x$ zurück}

\seebaseclass{Function}

\subsubsection{Konstante eindimensionale Funktionen}
\label{Constant}
\hypertarget{Constant}{}
Die Klasse Constant beschreibt konstante eindimensionale Funktionen der Form
$f(x) = a$ . Sie ist von der Klasse \class{FunctionWithFitting} abgeleitet. 

\proch{}{Constant::Constant}{void}{fitfn.h}
\descr{Standard-Konstruktor. Der Parameter $a$ wird mit $0$ initialisiert.}
\proch{}{Constant::Constant}{double a}{fitfn.h}
\descr{Konstruktor, der die Koeffizienten mit den übergebenen Werten initialisiert.}
\proch{}{Constant::Constant}{const \vector{double} \&v}{fitfn.h}
\descr{Konstruktor, der den Parameter a mit mit dem Wert $v[0]$ des Vektors $v$ initialisiert.}

\proch{}{Constant::Constant}{const Constant \&f}{fitfn.h}
\descr{Kopier-Konstruktor}

\proch{int}{Constant::setCoefficient}{double pa}{fitfn.h}
\descr{Setzt den Wert des Parameters a auf $pa$.}

\proch{int}{Constant::getCoefficient}{double \&pa}{fitfn.h}
\descr{Schreibt den Wert des Parameters auf $pa$.}

\seebaseclass{FunctionWithFitting}

\subsubsection{Lineare eindimensionale Funktionen}
\label{Polynom1o}
\hypertarget{Polynom1o}{}
Die Klasse Polynom1o beschreibt lineare eindimensionale Funktionen der Form
$ f(x) = a_1 \cdot x + a_0$ . Sie ist von der Basisklasse 
\class{FunctionWithFitting} abgeleitet.

\proch{}{Polynom1o::Polynom1o}{void}{fitfn.h}
\descr{Standard-Konstruktor. Die Koeffizienten werden mit $a_0=0$ und $a_1=0$ 
initialisiert.}
\proch{}{Polynom1o::Polynom1o}{double a0, double a1}{fitfn.h}
\descr{Konstruktor, der die Koeffizienten mit den übergebenen Werten initialisiert.}
\proch{}{Polynom1o::Polynom1o}{const \vector{double} \&v}{fitfn.h}
\descr{Konstruktor, der die Koeffizienten werden mit den Werten des 
\vector{double} $v[0]$ und $v[1]$ initialisiert.}
\proch{}{Polynom1o::Polynom1o}{const Polynom1o \&f}{fitfn.h}
\descr{Kopier-Konstruktor}

\proch{int}{Polynom1o::setCoefficient}{double a0, double a1}{fitfn.h}
\descr{Setzt die Werte der Parameter $a_0$ und $a_1$.}

\proch{int}{Polynom1o::getCoefficient}{double \&a0, double \&a1}{fitfn.h}
\descr{Schreibt die Koeffizienten-Werte auf $a0$ und $a1$.}

\seebaseclass{FunctionWithFitting}

\subsubsection{Quadratische eindimensionale Funktionen}
\label{Polynom2o}
\hypertarget{Polynom2o}{}

Die Klasse Polynom2o beschreibt quadratische eindimensionale Funktionen der 
Form $ f(x) = a_2 \cdot x^2 + a_1 \cdot x + a_0$ . Sie ist von der Basisklasse 
\class{FunctionWithFitting} abgeleitet.

\proch{}{Polynom2o::Polynom2o}{void}{fitfn.h}
\descr{Standard-Konstruktor. Die Koeffizienten werden mit $a_0=0$, $a_1=0$ und $a_2=0$ 
initialisiert.}
\proch{}{Polynom2o::Polynom2o}{double a0, double a1, double a2}{fitfn.h}
\descr{Konstruktor, der die Koeffizienten mit den übergebenen Werten initialisiert.}
\proch{}{Polynom2o::Polynom2o}{const Vector \&v}{fitfn.h}
\descr{Konstruktor, der die Koeffizienten werden mit den Werten des Vektors
$v[0]$, $v[1]$ und $v[2]$ initialisiert.}
\proch{}{Polynom2o::Polynom2o}{const Polynom2o \&f}{fitfn.h}
\descr{Kopier-Konstruktor}

\proch{int}{Polynom2o::setCoefficient}{double a0, double a1, double a2}{fitfn.h}
\descr{Setzt die Werte der Parameter $a_0$, $a_1$ und $a_2$.}

\proch{int}{Polynom2o::getCoefficient}{double \&a0, double \&a1, double \&a2}{fitfn.h}
\descr{Schreibt die Koeffizienten-Werte auf $a0$, $a1$ und $a2$.}

\seebaseclass{FunctionWithFitting}

\subsubsection{Allgemeine polynomiale Funktionen}
\label{Polynom}
\hypertarget{Polynom}{}

Die Klasse Polynom beschreibt allgemeine polynomiale Funktionen der Form
$ f(x) = \sum_{i=0}^n a_i \cdot x^i$ . Sie ist von der Basisklasse 
\class{FunctionWithFitting} abgeleitet.

\proch{}{Polynom::Polynom}{void}{fitfn.h}
\descr{Standard-Konstruktor. Polynomiale Funktion der Ordnung 0 - Konstante.}
\proch{}{Polynom::Polynom}{int ord}{fitfn.h}
\descr{Erzeugt eine Polynomiale Funktion der Ordnung $ord$. Die
Koeffizienten werden mit 0 initialisiert. }
\proch{}{Polynom::Polynom}{const \vector{double} \&v}{fitfn.h}
\descr{Konstruktor, der die Koeffizienten werden mit den Werten des Vektors
$v[i]$ initialisiert.}
\proch{}{Polynom::Polynom}{const Polynom \&f}{fitfn.h}
\descr{Kopier-Konstruktor}

\seebaseclass{FunctionWithFitting}

\subsection{Zweidimensionale Funktionen}
\label{Function2d}
\hypertarget{Function2d}{}

Die abstrakte Basisklasse \class{Function2d} beschreibt skalare 
Funktionen mit zwei Parametern. Wesentliche Eigenschaft dieser
Klasse ist der \verb+operator()+, der mit zwei Parameter vom Typ
\verb+double+ aufgerufen werden kann oder mit einem Parameter 
vom Typ \class{Point}.

\proch{double}{Function2d::operator()}{double x,doubel y}{function.h}
\procf{double}{Function2d::operator()}{Point p}
\descr{Berechnet den Funktionswert der Funktion an der Stelle {\bf x, y}
oder an der Stelle des Punkte {\bf p}.}

\proch{void}{Function2d::getCoefficient}{\vector{double} \&c}{function.h}
\descr{Ermittelt die Parameter (Koeffizienten) der Funktion. Diese werden
an den \vector{double} angehängt. 
Dies erlaubt das Aufsammeln der Parameter mehrerer Funktionen in 
einem \vector{double}.}

\proch{int}{Function2d::setCoefficient}{const \vector{double} \&para,int idx=0}{function.h}
\descr{Setzt die Parameter (Koeffizienten) der Funktion. Diese werden
dem \vector{double} ab dem Index {\bf idx} entnommen. 
Die Funktion gibt die Position nach dem letzten gelesenen Wert zurück.
Dies erlaubt die Initialisierung mehrerer Funktionen mit 
einem \vector{double}.}

\subsection{Zweidimensionale Funktionen mit Fitting}
\label{Function2dWithFitting}
\hypertarget{Function2dWithFitting}{}

Die abstrakte Klasse \class{Function2dWithFitting} ist von der Klasse
\class{Function2d} abgeleitet und beschreibt zweidimensionale Funktionen, 
deren Parameter durch Fitting ermittelt werden können.

\proch{int}{Function2dWithFitting::FitInit}{}{fitfn.h}
\descr{Die abstrakte Methode initialisiert das Fitten von Werttupeln.}
\proch{int}{Function2dWithFitting::FitVal}{double x, double y, double g, double w=1.0}{fitfn.h}
\descr{Die abstrakte Methode übergibt ein Wertetupel für das Fitten. Optional
ist es möglich, ein Gewicht $w$ des Typels anzugeben.}
\proch{int}{Function2dWithFitting::FitVal}{const Point \&p, double g, double w=1.0}{fitfn.h}
\descr{Die abstrakte Methode übergibt ein Wertetupel für das Fitten, wobei die
Koordinaten x und y von \see{Point} $p$ als Argument verwendet werden. Optional
ist es möglich, ein Gewicht $w$ des Typels anzugeben.}
\proch{int}{Function2dWithFitting::FitFinish}{}{fitfn.h}
\descr{Schließt den Prozess des Fitting ab. Danach sind die gefitteten 
Parameter der Funktion gültig.}

\proch{int}{Function2dWithFitting::Fit}{const Matrix \&m}{fitfn.h}
\descr{Die Funktion wird an die Wertetupel in der Matrix $m$ gefittet.
Dabei stellt jede Zeile der Matrix ein Wertetupel bereit. Die Matrix muß also
  mindestens drei Spalten haben.
Diese Funktion umfasst alle Schritte von 
der Initialisierung (\see{Function2dWithFitting::FitInit}) über das 
Eintragen der Werte (\see{Function2dWithFitting::FitValue}) bis zum 
Abschluss des Fittens (\see{Function2dWithFitting::FitFinish}).
Vorher mit \bsee{Function2dWithFitting::FitValue} eingetragene Werte
gehen verloren.}

\proch{int}{Function2dWithFitting::Fit}{Image img}{fitfn.h}
\proch{int}{Function2dWithFitting::Fit}{ImageD img}{fitfn.h}
\descr{Die Funktion wird an die Grauwerte des Bildes $img$ gefittet.}

\proch{double}{Func2d::operator()}{double x, double y}{fitfn.h}
\procf{double}{Func2d::operator()}{Point p}
\descr{Gibt den Funktionswert der Funktion an der Stelle $x,y$ zurück.}

\seebaseclass{Function2d}

\subsubsection{Konstante zweidimensionale Funktionen}
\label{Constant2d}
\hypertarget{Constant2d}{}

Die Klasse \class{Constant2d} beschreibt konstante 
zweidimensionale Funktionen der Form $ f(x,y) = a$ . 
Sie ist von der Basisklasse 
\class{Function2dWithFitting} abgeleitet.

\proch{}{Constant2d::Constant2d}{void}{fitfn.h}
\descr{Standard-Konstruktor. Der Parameter $a$ wird mit $0$ initialisiert.}
\proch{}{Constant2d::Constant2d}{double a}{fitfn.h}
\descr{Konstruktor, der den Parameter $a$ mit dem übergebenem Wert initialisiert.}
\proch{}{Constant2d::Constant2d}{const \vector{double} \&v}{fitfn.h}
\descr{Konstruktor, der Parameter $a$ wird mit dem Wert $v[0]$ des Vektors $v$
initialisiert.}
\proch{}{Constant2d::Constant2d}{const Constant2d \&f}{fitfn.h}
\descr{Kopier-Konstruktor}

\proch{void}{Constant2d::setCoefficient}{double a}{fitfn.h}
\descr{Setzt den Wert des Parameters $a$.}

\proch{void}{Constant2d::getCoefficient}{double \&a}{fitfn.h}
\descr{Schreibt den Wert des Parameters $a$ auf die übergebene 
Variable $a$.}

\seebaseclass{Function2dWithFitting}

\subsubsection{Lineare zweidimensionale Funktionen}
\label{Polynom2d1o}
\hypertarget{Polynom2d1o}{}

Die Klasse \class{Polynom2d1o} beschreibt lineare zweidimensionale 
Funktionen der Form
$ f(x,y) = a_{10} \cdot x + a_{01} \cdot y + a_{00}$ . Sie ist von der Basisklasse 
\class{Function2dWithFitting} abgeleitet.

\proch{}{Polynom2d1o::Polynom2d1o}{void}{fitfn.h}
\descr{Standard-Konstruktor. Die Koeffizienten werden mit $a_{00}=0$,
$a_{10}=0$ und $a_{01}=0$ initialisiert.}
\proch{}{Polynom2d1o::Polynom2d1o}{double a00, double a10, double a01}{fitfn.h}
\descr{Konstruktor, der die Koeffizienten mit den übergebenen Werten initialisiert.}
\proch{}{Polynom2d1o::Polynom2d1o}{const \vector{double} \&v}{fitfn.h}
\descr{Konstruktor, der die Koeffizienten werden mit den Werten des Vektors
$v[0]$,$v[1]$ und $v[2]$ initialisiert.}
\proch{}{Polynom2d1o::Polynom2d1o}{const Polynom2d1o \&f}{fitfn.h}
\descr{Kopier-Konstruktor}

\proch{void}{Polynom2d1o::setCoefficient}{double a00, double a10, double a01}{fitfn.h}
\descr{Setzt den Wert der Parameter $a_{00}, a_{10}$ und $a_{01} $.}

\proch{void}{Polynom2d1o::getCoefficient}{double \&a00, double \&a10, double \&a01}{fitfn.h}
\descr{Schreibt den Wert der Parameter auf die übergebene Variablen.}

\seebaseclass{Function2dWithFitting}

\subsubsection{Zweidimensionale Funktionen zweiter Ordnung}
\label{Polynom2d2o}
\hypertarget{Polynom2d2o}{}

Die Klasse \class{Polynom2d2o} beschreibt zweidimensionale 
Polynome zweiter Ordnung der Form 
$ f(x,y) = a_{20} \cdot x^2 + a_{11} \cdot x \cdot y + a_{02} \cdot y^2 + a_{10}
\cdot x + a_{01} \cdot y + a_{00}$ . 
Sie ist von der Basisklasse \class{Function2dWithFitting} abgeleitet.

\proch{}{Polynom2d2o::Polynom2d2o}{void}{fitfn.h}
\descr{Standard-Konstruktor. Alle Koeffizienten werden 0 initialisiert.}
\proch{}{Polynom2d2o::Polynom2d2o}{double a00, double a10, double a01, double a20, double
a11, double a02}{fitfn.h}
\descr{Konstruktor, der die Koeffizienten mit den übergebenen Werten initialisiert.}
\proch{}{Polynom2d2o::Polynom2d2o}{const \vector{double} \&v}{fitfn.h}
\descr{Konstruktor, der die Koeffizienten mit den Werten $v[0]$ bis $v[5]$ 
initialisiert.}
\proch{}{Polynom2d2o::Polynom2d2o}{const Polynom2d2o \&f}{fitfn.h}
\descr{Kopier-Konstruktor}

\proch{void}{Polynom2d2o::setCoefficient}{double a00, double a10, double a01, double a20, double a11, double a02}{fitfn.h}
\descr{Setzt den Wert der Parameter 
$a_{00}, a_{10}, a_{01}, a_{20}, a_{11}$ und $a_{20} $.}

\proch{void}{Polynom2d2o::getCoefficient}{double \&a00, double \&a10, double \&a01, double \&a20, double \&a11, double \&a02}{fitfn.h}
\descr{Schreibt den Wert der Parameter auf die übergebene Variablen.}

\seebaseclass{Function2dWithFitting}

\subsubsection{Allgemeine zweidimensionale polynomiale Funktionen}
\label{Polynom2d}
\hypertarget{Polynom2d}{}

Die Klasse \class{Polynom2d} beschreibt allgemeine polynomiale 
Funktionen der Form
$ f(x,y) = \sum_{i,j} a_{i,j} \cdot x^i y^j$ . 
Sie ist von der Basisklasse \see{Function2dWithFitting} abgeleitet. 
Die Ordnung $ord$ des Polynoms begrenzt die Potenzen auf
$i+j \le ord$.

\proch{}{Polynom2d::Polynom2d}{void}{fitfn.h}
\descr{Standard-Konstruktor, erzeugt eine konstante Funktion (Ordnung=0).}
\proch{}{Polynom2d::Polynom2d}{int ord}{fitfn.h}
\descr{Erzeugt eine Polynomiale Funktion der Ordnung $ord$. Die
Koeffizienten werden mit 0 initialisiert. }
\proch{}{Polynom2d::Polynom2d}{const \vector{double} \&v}{fitfn.h}
\descr{Konstruktor, der die Koeffizienten werden mit den Werten des Vektors
$v[i]$ initialisiert.}
\proch{}{Polynom2d::Polynom2d}{const Polynom2d \&f}{fitfn.h}
\descr{Kopier-Konstruktor}

\seebaseclass{Function2dWithFitting}

\subsection{Funktions-Modifizierer}
\label{Function2dModifier}
\hypertarget{Function2dModifier}{}

Die (abstrakte) Klasse \class{Function2dModifier} ist von der Basisklasse
\class{Function2d} abgeleitet und beschreibt Funktionen, die andere 
Funktionen modifizieren. Die Klassen \class{Function2dParameterShift},
\class{Function2dParameterTransform} und \class{Function2dDistortion}
transformieren die Parameter einer Funktion vor dem Aufruf.
Die Klasse \class{Function2dValueTransform} wendet auf den von einer
Funktion zurückgelieferten Wert eine lineare Funktion an.

\proch{}{Function2dModifier::Function2dModifier}{Function2d \&funcp}{trfunc.h}
\descr{Konstruktor, der als Parameter die zu modifizierende
Funktion erhält.}

\seebaseclass{Function2d}

\subsubsection{Verschiebung einer Funktion}
\label{Function2dParameterShift}
\hypertarget{Function2dParameterShift}{}
Die von der Klasse \class{Function2dModifier} abgeleitete 
Klasse \class{Function2dParameterShift} modifiziert eine
Funktion dadurch, dass auf von den Aufrufparametern x und y konstante
Werte dx und dy subtrahiert werden: $ g(x,y) = f ( x - dx, y - dy ) $.
Dies entspricht einer Verschiebung der Funktion um (dx,dy).

\proch{}{Function2dParameterShift::Function2dParameterShift}{Function2d \&funcp, double sx, double sy}{trfunc.h}
\descr{Konstruktor, der die zugrundeliegende Funktion und die Offset-Werte
für die Parameter übernimmt.}

\seebaseclass{Function2dModifier}

\subsubsection{Geometrische Transformation einer Funktion}
\label{Function2dParameterTransform}
\hypertarget{Function2dParameterTransform}{}
Die von der Klasse \class{Function2dModifier} abgeleitete 
Klasse \class{Function2dParameterTransform} unterwirft eine Funktion
einer Transformation $T$ der Klasse \class{Trafo}. Dazu werden die
Parameter x und y der entsprechenden inversen Transformation
unterworfen: $ g(x,y)=f(T^{-1}(x,y))$ .

\proch{}{Function2dParameterTransform::Function2dParameterTransform}{Function2d \&funcp, const Trafo \&tr}{trfunc.h}
\descr{Konstruktor, der die zugrundeliegende Funktion und die anzuwendende
Transformation übernimmt.}

\proch{Trafo \&}{Function2dParameterTransform::Tr}{}{trfunc.h}
\procf{const Trafo \&}{Function2dParameterTransform::Tr}{}
\descr{Zugriff auf die anzuwendende Transformation.}

\seebaseclass{Function2dModifier}

\subsubsection{Verzeichnung einer Funktion}
\label{Function2dDistortion}
\hypertarget{Function2dDistortion}{}
Die von der Klasse \class{Function2dModifier} abgeleitete 
Klasse \class{Function2dDistortion} unterwirft eine Funktion
einer Verzeichnung $D$ der Klasse \class{Distortion}. Dazu werden die
Parameter x und y der entsprechenden inversen Transformation
unterworfen: $ g(x,y)=f(D^{-1}(x,y))$ .

\proch{}{Function2dDistortion::Function2dDistortion}{Function2d \&funcp, Distortion \&tr}{trfunc.h}
\descr{Konstruktor, der die zugrundeliegende Funktion und die anzuwendende
Verzeichnung übernimmt. Die Verzeichnung wird als Referenz gespeichert. 
Die übergebene Instanz der Klasse \class{Distortion} muss deshalb
solange gültig bleiben, wie die Funktion genutzt wird. 
}

\seebaseclass{Function2dModifier}

\subsubsection{Transformation des Funktionswertes}
\label{Function2dValueTransform}
\hypertarget{Function2dValueTransform}{}
Die von der Klasse \class{Function2dModifier} abgeleitete 
Klasse \class{Function2dValueTransform} wendet auf den Funktionswert 
einer Funktion eine lineare Funktion an: $ g(x,y) = a_1 \cdot f(x,y) + a_0 $ .

\proch{}{Function2dValueTransform::Function2dValueTransform}{Function2d \&funcp, double a1, double a0}{trfunc.h}
\descr{Konstruktor, der die zugrundeliegende Funktion und die Parameter
der linearen Funktion übernimmt.}

\seebaseclass{Function2dModifier}

\subsection{Bilder als Funktionen}
\label{ImageFunction}
\label{ImageDFunction}
\hypertarget{ImageFunction}{}
\hypertarget{ImageDFunction}{}

Die von der Basisklasse \class{Function2d} abgeleiteten Klassen
\class{ImageFunction} und \class{ImageDFunction} stellen Bilder
der Klassen \class{Image} und \class{ImageD} als Funktion zur 
Verfügung. Dabei kann der Typ der Interpolation mittels
Parameter {\bf mode} gewählt werden:

\begin{tabular}{ll}
mode & Interpolation \\
DEFAULT & Nearest Neighbor \\
INTERPOL & Bilineare Interpolation \\
\end{tabular}

\proch{}{ImageFunction::ImageFunction}{const Image \&imgp, int mode = INTERPOL}{function.h}
\procf{}{ImageDFunction::ImageDFunction}{const ImageD \&imgp, int mode = INTERPOL}
\descr{Konstruktor für Bildfunktionen mit Übergabe des Bildes und Auswahl des
Interpolationstypes.}
