\nsection{Filter}{filter}

In diesem Abschnitt werden Filter-Operationen über Bilder beschrieben.
Alle Typen von Bildrepräsentationen werden hier soweit es geht gemeinsam 
abgehandelt (\class{Image}, \class{ImageD}, \class{ColorImage}). 
Auf die Besonderheiten bei bestimmten Typen wird speziell hingewiesen. 
Die Funktionen zur Bildfilterung benötigen im allgemeinen ein Quellbild
und ein Zielbild. Bei fehlerfreier Abarbeitung wird OK zurückgegeben, sonst
ein Fehlerkode.

\subsection{Lineare lokale Filter}
\hypertarget{LSIFilter}{}
\subsubsection{Klasse LSIFilter}
Die Klasse \class{LSIFilter} repräsentiert allgemeine LSI-Filter. 
LSI-Filter können durch die Vorgabe von Filtermasken 
(als \class{Matrix} oder \class{IMatrix}) oder mittels spezieller 
Funktionen (z.B. \see{mkDirSmearFilter}) erzeugt werden.

Ein LSI-Filter beschreibt immer eine LSI-Filterung der Form 
\[ g \prime (x,y) = \sum_{i,j} f(i,j) \cdot g(x+i,y+j) ,\]
wobei je nach Filtertyp unterschiedliche interne Repräsentationen 
zur Anwendung kommen. Ganzzahlige Filter sind typischerweise 
schneller als Gleitkomma-Filter (Initialisierung mit \class{IMatrix} 
statt \class{Matrix}).
 
\subtitle{Konstruktoren}
\proch{}{LSIFilter::LSIFilter}{}{lsifilter.h}
\descr{Standardkonstruktor. Ein durch den Standardkonstruktor erzeugtes 
Filter ist nicht benutzbar, sondern es muss durch Zuweisung ein 
richtiges Filter zugewiesen werden.}

\proch{}{LSIFilter::LSIFilter}{const LSIFilter \&f}{lsifilter.h}
\descr{Kopierkonstruktor.}

\proch{}{LSIFilter::LSIFilter}{const Matrix \&m}{lsifilter.h}
\descr{Konstruiert ein (double-)Filter mit der als \class{Matrix} $m$ 
übergebenen Maske. 
\[ g(x,y) = \sum_{i,j} m_{i,j} \cdot g(x+i,y+j) \]
}

\proch{}{LSIFilter::LSIFilter}{const IMatrix \&m,int norm}{lsifilter.h}
\procf{}{LSIFilter::LSIFilter}{int *m,int norm}
\descr{Konstruiert ein (int-)Filter mit der als \class{IMatrix} oder 
C-Array $m$ übergebenen Maske.
\[ g(x,y) = {1 \over norm} \cdot \sum_{i,j} m_{i,j} \cdot g(x+i,y+j) \]
}

\subtitle{Abfrage und Manipulation}

\proch{double}{LSIFilter::getMask}{int x,int y}{lsifilter.h}
\descr{Gibt den Wert der Filtermaske an der Stelle $x,y$ zurück.}

\proch{void}{LSIFilter::NegateMask}{}{lsifilter.h}
\descr{Negiert die Koeffizienten der Filtermaske.}

\proch{double}{LSIFilter::proposeOffset}{const Image \&img}{lsifilter.h}
\descr{Gibt einen Offset-Wert zurück, der bei der Filterung eines Integerbildes
angewendet werden sollte, damit die Ergebniswerte im positiven ganzzahligen 
Bereich der Grauwerte des Ergebnisbildes $img$ liegen.}


\subsubsection{Filtererzeugung}
Neben den Konstruktoren, die aus Matrizen mit den Filterkoeffizienten LSIFilter 
erzeugen, gibt es Funktionen zur Erzeugung spezieller Filter.

\proch{LSIFilter}{mkPolynomFilter}{int n,int grad,int i,int j}{lsifilter.h}
\descr{Erzeugt einen Filter, der als Filter-Ergebnis den Koeffizienten $a_{i,j}$ eines 
an der Stelle (x0,y0) an die Grauwertfunktion angepaßten Polynoms des Grades $grad$ 
ermittelt.
\[
g(x,y) = \sum_{i,j} a_{i,j} \cdot (x-x_0)^i \cdot (y-y_0)^j 
\]
Folgende Aufrufe erzeugen Filter, die äquivalent zu anderen Filterfunktionen sind:\\
\begin{tabular}{cc}
  mkPolynomFilter(3,1,0,0) & \bsee{SmearImg} \\
  mkPolynomFilter(3,1,1,0) & \bsee{GradXImg} \\
  mkPolynomFilter(3,1,0,1) & \bsee{GradYImg} \\
  mkPolynomFilter(3,2,0,0) & \bsee{MeanImg} \\
  mkPolynomFilter(3,2,2,0) & \bsee{LaplaceXImg} \\
  mkPolynomFilter(3,2,0,2) & \bsee{LaplaceYImg}
\end{tabular}
}

\setlength{\unitlength}{1cm}
\begin{picture}(6,6)
  \epsfig{figure=dirsmear.eps,width=6cm}
\end{picture}
\proch{LSIFilter}{mkDirSmearFilter}{int n,double dir,double len,double width}{lsifilter.h}
\descr{Erzeugt ein Mittelwertfilter, dessen Maske eine diskrete Approximation 
der im Bild dargestellten roten Fläche darstellt.}

\begin{picture}(6,6)
  \epsfig{figure=dirdob.eps,width=6cm}
\end{picture}
\proch{LSIFilter}{mkDirDoBFilter}{int n,double dir,double len,double width}{lsifilter.h}
\descr{Erzeugt ein dem DoB-Filter (\see{DoBImg}) ähnliches Filter, 
dessen Maske eine diskrete Approximation des Bildes dargestellt. Dabei
kennzeichnet die roten Fläche  positive Maskenwerte, die blaue Fläche negative
Maskenwerte. Die Werte sind so gewählt, dass in einem homogenen Bild das Filter
den Wert Null liefert.}

\begin{picture}(6,6)
  \epsfig{figure=diredge.eps,width=6cm}
\end{picture}
\proch{LSIFilter}{mkDirEdgeFilter}{int n,double dir,double rad}{lsifilter.h}
\descr{Erzeugt ein Kanten-Filter, dessen Maske eine diskrete Approximation 
des Bildes dargestellt. Dabei
kennzeichet die roten Fläche  positive Maskenwerte, die blaue Fläche negative
Maskenwerte. Die positiven und negativen Werte sind betragsmäßig gleich.}

\subsubsection{Anwendung von LSIFilter}

\proch{int}{LSIFilter::Filter}{const Image \&src,Image \&dst,int offset}{lsifilter.h}
\procf{int}{LSIFilter::Filter}{const Image \&src,ImageD dst}
\procf{int}{LSIFilter::Filter}{ImageD src,ImageD dst}
\descr{Wendet das Filter auf das Bild $src$ an und speichert das Ergebnis
im Bild $dest$. Im Falle der Integer-Bilder verschieben sich die 
Ergebniswerte um $offset$. Damit kann erreichtwerden, dass negative 
Filterergebnisse auf den Bereich des Zielbildes abgebildet werden 
(\see{LSIFilter::proposeOffset}). }

\proch{int}{LSIImg}{const Image \&src,Image \&dst,const LSIFilter \&f,int offset}{lsifilter.h}
\descr{Wendet das Filter $f$ auf das Bild $src$ an und speichert das Ergebnis
im Bild $dest$. Dies ist äquivalent zu LSIFilter::Filter(src,dst,offset).}

\seealso{GetVal}

\subtitle{Spezielle Aufrufformen}

In den speziellen Aufrufformen wird das LSIFilter implizit generiert 
und angewendet. Deshalb ergeben sich keine inhaltlichen Unterschiede 
zur Erzeugung eines Filters mit nachfolgender Filterung.

\proc{int}{LSIImg}{Image src,int nx,int ny,int *mask,int norm,int offset,Image dest}
\descr{
Allgemeines LSI-Filter. Mit mit den ungeraden Werten $nx$ und $ny$ wird 
die Größe der Filtermaske angegeben (3,5,7,\dots). Die Filtermaske $mask$ 
ist ein int-Array der Größe $ny \cdot nx$. Die Bildpunkte des 
Zielbildes werden nach der Formel
$$vd(x,y)={maxd \over norm \cdot maxs}\cdot \sum vs(x-i,y-j) \cdot mask(i,j) +
offset$$
berechnet.
}
\proc{int}{LSIImg}{Image src,int nx,int ny,double *mask,int offset,Image dest}
\descr{
Allgemeines LSI-Filter. Mit mit den ungeraden Werten $nx$ und $ny$ wird 
die Größe der Filtermaske angegeben (3,5,7,\dots). Die Filtermaske $mask$ 
ist ein double-Array der Größe $ny \cdot nx$. Die Bildpunkte des 
Zielbildes werden nach der Formel
$$vd(x,y)={maxd \over maxs}\cdot \sum vs(x-i,y-j) \cdot mask(i,j) + offset$$
berechnet.
}
\proc{int}{LSIImg}{Image src,const Matrix \&mask,int offset,Image dest}
\descr{
Allgemeines LSI-Filter. Die Filtermaske $mask$ ist eine $Matrix$. 
Die Bildpunkte des Zielbildes werden nach der Formel
$$vd(x,y)={maxd \over maxs}\cdot \sum vs(x-i,y-j) \cdot mask(i,j) + offset$$
berechnet.
}
\proc{int}{LSIImg}{Image src,const IMatrix \&mask,int norm,int offset,Image dest}
\descr{
Allgemeines LSI-Filter. Die Filtermaske $mask$ ist eine $IMatrix$. 
Die Bildpunkte des Zielbildes werden nach der Formel
$$vd(x,y)={maxd \over { maxs \cdot norm } }\cdot \sum vs(x-i,y-j) \cdot mask(i,j) + offset$$
berechnet.
}

\subsubsection{Spezielle lineare lokale Filter}
Die speziellen linearen lokalen Filter sind in der Laufzeit gegenüber 
den allgemeinen LSI-Filtern (\class{LSIFilter} optimiert. Viele 
beschränken sich auf eine 3*3-Umgebung.

\proch{int}{SmearImg}{const Image \&src,Image \&dest,int n=3}{filter.h}
\procf{int}{SmearImg}{const Image \&src,Image \&dest,int nx,int ny}
\procf{int}{SmearImg}{const Image \&img,int n}
\descr{
Fenster-Mittelwertfilter. Das Fenster wird quadratisch (Größe $n$*$n$) 
oder rechteckig (Größe $nx$*$ny$) gewählt. Die Laufzeit dieser (optimierten)
Variante ist nicht von der Fenstergröße abhängig.
}

\proch{int}{SmearImgD}{ImageD src,ImageD dest,int nx,int ny}{darith.h}
\procf{int}{SmearImgD}{ImageD src,ImageD dest,int n}
\descr{
Fenster-Mittelwertfilter. Das Fenster wird quadratisch (Größe $n$*$n$) 
oder rechteckig (Größe $nx$*$ny$) gewählt. Hier wird jedoch nicht der 
Mittelwert, sondern die Grauwert-Summe im Fenster als Ergebniswert 
abgelegt. Die Laufzeit dieser (optimierten) Variante ist nicht von der 
Fenstergröße abhängig.
}

\proch{int}{BoxImg}{const Image \&src,Image \&dest,int n=3}{filter.h}
\procf{int}{BoxImg}{const Image \&src,Image \&dest,int nx,int ny}
\descr{
Boxfilter, der die Werte im Fenster summiert. Der Unterschied zu 
\bsee{SmearImg} besteht in der fehlenden Division durch die Box-Größe. 
Die Ergebniswerte können deshalb sehr groß werden und werden auf den 
Grauwertumfang des Zielbildes beschränkt.
Das Fenster wird quadratisch (Größe $n$*$n$) 
oder rechteckig (Größe $nx$*$ny$) gewählt. Die Laufzeit dieser (optimierten)
Variante ist nicht von der Fenstergröße abhängig.
}

\proch{int}{DoBImg}{Image src,Image dest,int n1,int n2,int mode=SMD\_SHIFT}{filter.h}
\descr{
''Difference of Boxes'' - Differenz zweier Fenster-Mittelwertfilter mit den
Größen $n1$ und $n2$. Dieses Filter kann als schnelle Approximation des
MexicanHat-Filters verwendet werden. Der Parameter $modus$ steuert, wie die
Differenz (\see{SubImg}) gebildet wird:\\
\begin{tabular}{|c|c|} \hline
mode & Wert \\ \hline
SMD\_ABSOLUTE & $vd=\vert v1-v2 \vert$ \\
SMD\_POSITIVE & $vd=max(v1-v2,0)$ \\
SMD\_SHIFT & $vd=v1-v2+max/2$ \\ \hline
\end{tabular}\\
}

\proch{int}{GradXImg}{Image src,int norm,Image dest}{filter.h}
\descr{
Gradient in x-Richtung:
$${norm\cdot maxd \over 6 \cdot maxs}\left[{ \matrix{-1&0&1\cr-1&0&1\cr-1&0&1}}
\right] +{maxd \over 2}$$
}
\seealso{GradImg}
\seealsonext{GradDirImg}

\proc{int}{GradYImg}{Image src,int norm,Image dest}
\descr{
Gradient in y-Richtung:
$${norm\cdot maxd \over 6 \cdot maxs}\left[{ \matrix{-1&-1&-1\cr0&0&0\cr1&1&1}}
\right] +{maxd \over 2}$$
}
\seealso{GradImg}
\seealsonext{GradDirImg}

\proc{int}{MeanImg}{Image src,Image dest}
\descr{gewichtetes Mittelwertfilter:
$${maxd \over 9 \cdot maxs}\left[{ \matrix{-1&2&-1\cr2&5&2\cr-1&2&-1}}
\right]$$
}

\proc{int}{LaplaceXImg}{Image src,int norm,Image dest}
\descr{
Laplacefilter in x-Richtung:
$${norm\cdot maxd \over 12 \cdot maxs}\left[{ \matrix{1&-2&1\cr1&-2&1\cr1&-2&1}}
\right] +{maxd \over 2}$$
}
\proc{int}{LaplaceYImg}{Image src,int norm,Image dest}
\descr{
Laplacefilter in y-Richtung:
$${norm\cdot maxd \over 12 \cdot maxs}\left[{ \matrix{1&1&1\cr-2&-2&-2\cr1&1&1}}
\right] +{maxd \over 2}$$
}
\proc{int}{LaplaceImg}{Image src,int norm,Image dest}
\descr{
Laplacefilter:
$${norm\cdot maxd \over 16 \cdot maxs}\left[{ \matrix{2&-1&2\cr-1&-4&-1\cr2&-1&2}}
\right] +{maxd \over 2}$$
}

\proc{int}{GaussImg}{const Image \&src,Image \&dest,int neighb,double sigma}
\procf{int}{GaussImg}{const Image \&src,ImageD dest,int neighb,double sigma}
\procf{int}{GaussImg}{ImageD src,ImageD dest,int neighb,double sigma}
\descr{
Gaussfilterung eines Bildes. Es wird eine Filtermaske der Größe
$neighb*neighb$ nach der folgenden Formel berechnet:\\
$$mask(i,j)={1\over norm}e^{i^2+j^2 \over 2 sigma^2}\quad ,\quad
i,j=-{neighb-1 \over 2},\dots,{neighb-1 \over 2}$$
$norm$ wird so festgelegt, daß die Summe der Filterkoeffizienten 1 ergibt.
}

\proch{int}{MexicanHatImg}{const Image \&src,Image \&dest,double sigma,int neighb=0}{filter.h}
\procf{int}{MexicanHatImg}{const Image \&src,ImageD dest,double sigma,int neighb=0}
\procf{int}{MexicanHatImg}{ImageD src,ImageD dest,double sigma,int neighb=0}
\descr{
Es wird das Mexican Hat-Filter (auch Laplace of Gaussian-Operator oder Marr-Hildret-Operator) 
auf das Bild $src$ angewendet und das Ergebnis in Bild $dest$ gespeichert. 
Dabei ist $sigma$ der Parameter der Gauss-Funktion, $neighb$ die Größe der Filtermaske.
$neighb$ muss hinreichend groß gewählt werden, um das ideale Filter zu approximieren.
Ist $neighb$ gleich Null, so wird die Filtermaske als nächstgrößte ungerade Zahl zu 
$ 5 \cdot sigma$ gewählt.
}

\subsection{Nichtlineare lokale Filter}

\proc{int}{GradImg}{const Image \&src,int norm,Image \&dest}
\descr{Aus den Gradienten in x- und y-Richtung wird der Betrag berechnet:\\
$$val={norm\cdot maxd \over maxs}\cdot \sqrt{gradx^2+grady^2}$$
}

\proc{int}{GradDirImg}{const Image \&src,Image \&dest}
\descr{
Es wird die Richtung des Gradienten bestimmt und auf die acht
Freemanrichtungen (0 \dots 7) quantisiert.
}

\proch{int}{CalcDirectionImg}{const Image \&src,Image \&dir,int dsize=11}{filter.h}
\proc{int}{CalcDirectionImg}{const Image \&src,Image \&dir,int dsize=11,ImageD eval}
\descr{Im Bild $src$ wird aus dem Grauwertgradienten für jeden Punkt eine 
Vorzugsrichtung ermittelt. Der Gradient wird dabei über eine Umgebung der 
Größe $dsize$*$dsize$ gemittelt und der Winkel als Ergebnis in das Bild $dir$ eingetragen.
Der Winkelbereich $0 \le \varphi < 2 \pi$ wird dabei auf den Grauwertumfang 
des Zielbildes $dir$ abgebildet. 
Das Bild $dir$ ist direkt geeignet zur Verwendung mit dem gerichteten 
Filter \bsee{OrientedEdgeImg}.
Wenn das double-Bild $eval$ als Parameter angegeben werden, wird auf diesem der
Betrag des gemittelten Gradienten abgelegt, der als Maß für die Gerichtetheit im 
jeweiligen Punkt verwendet werden kann.}

\proch{int}{CalcDirectionStructImg}{const Image \&src,Image \&dir,int dsize=11}{filter.h}
\proc{int}{CalcDirectionStructImg}{const Image \&src,Image \&dir,int dsize=11,ImageD lambda1,ImageD lambda2}
\descr{Im Bild $src$ wird aus dem Grauwertgradienten mittels des 
Struktur-Tensors für jeden Punkt eine Vorzugsrichtung ermittelt. 
Der Winkel wird dabei über eine Umgebung der Größe $dsize$*$dsize$ 
gemittelt und das Ergebnis in das Bild $dir$ eingetragen.
Der Winkelbereich $0 \le \varphi < \pi$ wird dabei auf den 
Grauwertumfang des Zielbildes $dir$ abgebildet.
Das Bild $dir$ ist direkt geeignet zur Verwendung 
mit den gerichteten Filtern \bsee{OrientedSmearImg} und \bsee{OrientedDoBImg}.
Wenn die double-Bilder $lambda1$ und $lambda2$ 
als Parameter angegeben werden, werden auf diesen die Eigenwerte 
des Struktur-Tensors abgelegt, die als Maß für die Gerichtetheit im 
jeweiligen Punkt verwendet werden können.}

\proch{int}{OrientedSmearImg}{const Image \&src,const Image \&dir,
  Image \&dest,int fsize=11,int flength=10,int fwidth=1}{filter.h}
\proc{int}{OrientedDoBImg}{const Image \&src,const Image \&dir,
  Image \&dest,int fsize=11,int flength=10,int fwidth=1}
\descr{Gerichtete Mittelwertfilter (\see{mkDirSmearFilter}) bzw. 
gerichtete DoB-Filter (\see{mkDirDoBFilter}) werden 
punktweise angewendet, wobei die Orientierung des Filters aus dem 
Bild $dir$ (\see{CalcDirectionStructImg}) entnommen wird. Zur Bedeutung der 
Filterparameter siehe \bsee{mkDirSmearFilter} und \bsee{mkDirDoBFilter}.
}

\proch{int}{OrientedEdgeImg}{const Image \&src,const Image \&dir,
  Image \&dest,int fsize=11,int rad=10}{filter.h}
\descr{Der gerichtete Kantenfilter (\see{mkDirEdgeFilter}) wird 
punktweise angewendet, wobei die Orientierung des Filters aus dem 
Bild $dir$ (\see{CalcDirectionImg}) entnommen wird. Zur Bedeutung der 
Filterparameter siehe \bsee{mkDirEdgeFilter}.
}

\subsection{Morphologische und Rangordnungs-Filter}
\proch{int}{ErodeImg}{const Image \&src,int nx,int ny,Image \&dest}{filter.h}
\procf{int}{ErodeImg}{const Image \&src,Image \&dest,int nx=3, int ny=-1}
\descr{
Es wird eine Erosion in der $nx$*$ny$-Umgebung durchgeführt.
$nx$ und $ny$ müssen ungerade sein. Dies entspricht dem Minimum-Filter mit 
dem gegebenen Fenster. Beim Aufruf in der zweiten Variante steht Weglassen von
$ny$ für ein quadratisches Fenster. Der Vorzugswert für $nx$ ist 3.}
\seealso{MinMaxImg}

\proc{int}{ErodeImg}{const Image \&src,int neighb,int *mask,Image \&dest}
\procf{int}{ErodeImg}{const Image \&src,int nx,int ny,int *mask,Image \&dest}
\procf{int}{ErodeImg}{const Image \&src,const Imatrix \&mask,Image \&dest}
\procf{int}{ErodeImg}{const Image \&src,Image \&dest,const IMatrix \&mask}
\descr{
Es wird eine verallgemeinerte Erosion mit dem in der Maske {\bf mask} 
definierten strukturierenden Element durchgeführt. Die 
Bildpunkte in der Umgebung werden für die Minimum-Bildung berücksichtigt, 
wenn der enstsprechende Wert in der Maske verschieden von Null ist.\\
Die Maske kann als C-Array gegeben sein und muss $neighb * neighb$ 
beziehungsweise $nx * ny$ Elemente enthalten, wobei $neighb$, 
$nx$ und $ny$ ungerade Zahlen (3, 5, 7,\dots) sein müssen.\\
Wird {\bf mask} als \class{IMatrix} gegeben, so entspricht die 
Nachbarschaftsgröße der Matrix-Größe. Diese Größen müssen ebenfalls 
ungeradzahlig sein.
}

\proch{int}{DilateImg}{const Image \&src,int nx,int ny,Image \&dest}{filter.h}
\procf{int}{DilateImg}{const Image \&src,Image \&dest,int nx=3, int ny=-1}
\descr{
Es wird eine Dilatation in der $nx$*$ny$-Umgebung durchgeführt.
$nx$ und $ny$ müssen ungerade sein. Dies entspricht dem Maximum-Filter mit 
dem gegebenen Fenster. Beim Aufruf in der zweiten Variante steht Weglassen von
$ny$ für ein quadratisches Fenster. Der Vorzugswert für $nx$ ist 3.}
\seealso{MinMaxImg}

\proc{int}{DilateImg}{const Image \&src,int neighb,int *mask,Image \&dest}
\procf{int}{DilateImg}{const Image \&src,int nx,int ny,int *mask,Image \&dest}
\procf{int}{DilateImg}{const Image \&src,const Imatrix \&mask,Image \&dest}
\procf{int}{DilateImg}{const Image \&src,Image \&dest,const IMatrix \&mask}
\descr{
Es wird eine verallgemeinerte Dilatation mit dem in der Maske {\bf mask} 
definierten strukturierenden Element durchgeführt. Die 
Bildpunkte in der Umgebung werden für die Maximum-Bildung berücksichtigt, 
wenn der entsprechende Wert in der Maske verschieden von Null ist.\\
Die Maske kann als C-Array gegeben sein und muss $neighb * neighb$ 
beziehungsweise $nx * ny$ Elemente enthalten, wobei $neighb$, 
$nx$ und $ny$ ungerade Zahlen (3, 5, 7,\dots) sein müssen.\\
Wird {\bf mask} als \class{IMatrix} gegeben, so entspricht die 
Nachbarschaftsgröße der Matrix-Größe. Diese Größen müssen ebenfalls 
ungeradzahlig sein.
}

\proch{int}{OpeningImg}{const Image \&src,Image \&dest,int nx=3, int ny=-1}{filter.h}
\procf{int}{OpeningImg}{const Image \&src,Image \&dest,const IMatrix \&mask}
\descr{
Es wird ein Opening in der $nx$*$ny$-Umgebung durchgeführt. Dies ist die 
Hintereinanderausführung einer Erosion gefolgt von einer Dilatation. 
$nx$ und $ny$ müssen ungerade sein. Weglassen von $ny$ steht für ein 
quadratisches Fenster. Der Vorzugswert für $nx$ ist 3.\\
Bei Angabe einer \class{IMatrix} als Maske, werden Erosion und Dilatation 
mit dieser Maske ausgeführt.
}

\proch{int}{ClosingImg}{const Image \&src,Image \&dest,int nx=3, int ny=-1}{filter.h}
\procf{int}{ClosingImg}{const Image \&src,Image \&dest,const IMatrix \&mask}
\descr{
Es wird ein Closing in der $nx$*$ny$-Umgebung durchgeführt. Dies ist die 
Hintereinanderausführung einer Dilatation gefolgt von einer Erosion. 
$nx$ und $ny$ müssen ungerade sein.
Weglassen von $ny$ steht für ein quadratisches Fenster. 
Der Vorzugswert für $nx$ ist 3.\\
Bei Angabe einer \class{IMatrix} als Maske, werden Dilatation und Erosion
mit dieser Maske ausgeführt.}

\proc{int}{MinMaxImg}{const Image \&src,int nx, int ny,Image \&minimg, Image \&maximg}
\descr{Bestimmt gleichzeitig das Resultat eines Minimum-Filters 
(Erosion) und eines Maximum-Filters (Dilatation) mit gegebenem Fenster. 
Wenn {\bf beide} Filter benötigt werden, ist die Kombination beider 
Filter effektiver als die getrennte Ausführung von \bsee{ErodeImg} 
und \bsee{DilateImg}.
}

Das folgende Beispiel soll die Verwendung von ErodeImg() und DilateImg() 
demonstrieren.

\begprogr
\begin{verbatim}
/*Erosion und Dilatation*/
#include <image.h>

void main(int argc,char *argv[])
{
  Image img;
  int mask[3][3]={{1,0,0},{0,1,0},{0,0,1}};       // strukturierendes Element
  img=ReadImg(argv[1],NULL);                      // Bild einlesen
  Show(ON,img,"Original-Bild");                   // Bild darstellen
  Image erode_img = NewImg(img);
  Show(ON, erode_img, "Mit Maske erodiertes Bild");
  ErodeImg(img,3,mask,erode_img);                 // Erosion
  Image dilate_img = NewImg(img);
  Show(ON, dilate_img, "Mit Maske dilatiertes Bild");
  DilateImg(img,3,mask,dilate_img);                // Dilatation
}
\end{verbatim}
\endprogr

\proch{int}{RankImg}{const Image \&src,int neighb,int rank,Image \&dest}{rank.h}
\procf{int}{RankImg}{const Image \&src,int nx,int ny,int rank,Image \&dest}
\descr{
Für jeden Bildpunkt wird in einer Umgebung der Größe $neighb*neighb$
beziehungsweise $nx*ny$ der Wert mit dem Rang $rank$ bestimmt und 
ins Zielbild eingetragen. $neighb$, $nx$ und $ny$ müssen ungerade sein.
}

\proch{int}{MedianImg}{const Image \&img,int size,Image \&dest}{rank.h}
\descr{Bestimmt als Spezialfall von \bsee{RankImg} ein Bild $dest$, dessen
  Pixel den Wert des Median der Werte in der Umgebung $neighb*neighb$ des
  Originalbildes haben.}

\proc{int}{SubRankImg}{const Image \&src,int neighb,int rank,Image \&dest}
\descr{
Für jeden Bildpunkt wird in einer Umgebung der Größe $neighb*neighb$ der Wert
mit dem Rang $rank$ bestimmt und vom Grauwert dieses Punktes im Originalbild
subtrahiert. Falls das Ergebnis kleiner als Null ist, wird Null im Zielbild
eingetragen. $neighb$ muß eine ungerade Zahl sein.
}

\proch{int}{RelaxImg}{const Image \&src,Image \&dest,int neighb=3}{icefunc.h}
\descr{
Es wird ein Relaxationsfilter zur Schärfung auf Bild $img1$ angewendet und 
das Ergebnis im Bild $img2$ abgespeichert. Dabei wird im Quellbild $src$ in
einer $neighb*neighb$ Umgebung das Maximum und das Minimum der Grauwerte 
bestimmt und im Ergebnisbild wird derjenige dieser beiden Werte eingetragen, 
der näher am Originalgrauwert liegt. 
}

\proch{int}{SkelettImg}{const Image \&src,const Image \&dest,int lvl=1}{filter.h}
\descr{Erzeugt nach dem Algorithmus von Zhang und Suen ein binäres Skelettbild. 
Dabei wird das Quellbild als Binärbild betrachtet, in dem Grauwerte größer 
gleich $lvl$ als Objekt-, die anderen als Untergrundpunkte betrachtet werden. 
Das Ergebnis ist ein Binärbild, in dem die Skelett-Punkte mit dem maximalen 
Grauwert eingetragen sind. Untergrundpunkte haben den Wert 0.}

\subsection{Globale lineare Filterung}

Als globale lineare Filterungs-Funktionen stellt ICE die
Faltung zweier Bilder und deren Umkehrung bereit. Die linearen 
globalen Filterungs-Funktionen basieren auf der Fourier- und der
Hartleytransformation und sind deshalb dort dokumentiert.
\newline
\seealso{FourierImgD} \\
\seealso{HartleyImgD} \\
\seealso{ConvolutionImg} \\
\seealso{InvConvolutionImg} 

%%% Local Variables: 
%%% mode: latex
%%% TeX-master: "icedoc_base"
%%% End: 
