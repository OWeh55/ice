\nsection{Momente und Standardlagen}{moments}

\subsection{Momente}
\label{Moments}
\hypertarget{Moments}{}

Die Klasse \class{Moments} dient in ICE der Behandlung der Momente bis zur
4. Ordnung. Die Klasse ersetzt die veraltete Speicherung der Momente in
Feldern vom Typ $double$. 

Normalerweise werden Momente durch Aufruf eines Konstruktors angelegt und die
Werte der Momente eines übergebenen Objektes berechnet.

\subsubsection{Konstruktoren}

Konstruktoren legen eine Instanz der Klasse \class{Moments} an und 
berechnen in der Regel Momente eines übergebenen Objektes.

\proch{}{Moments::Moments}{}{momente.h}
\descr{Legt eine Datenstruktur an und belegt alle Momente mit 0.0.}
      
\proch{}{Moments::Moments}{double m[15]}{momente.h}
\descr{Legt eine Datenstruktur an und belegt alle Momente mit den Werten aus
dem Feld $m$ von $double$-Werten. Die Momente müssen in $m$ 
in der Reihenfolge $m_{00}$ ,$m_{10}$, $m_{01}$, $m_{20}$, $m_{11}$, $m_{02}$,
$m_{30}$, $m_{21}$, $m_{12}$, $m_{03}$, $m_{40}$, $m_{31}$, $m_{22}$,
$m_{13}$, $m_{04}$ abgelegt sein.}

\proch{}{Moments::Moments}{const Region \&r}{momente.h}
\procf{}{Moments::Moments}{const Contur \&c}
\descr{Die Momente des übergebenen Objektes werden berechnet. 
Diese Konstruktoren müssen explizit aufgerufen werden.}

\proch{}{Moments::Moments}{const Matrix \&p}{momente.h}
\descr{Die als Matrix angelegte Punktliste (mit einem Punkt pro Zeile) 
wird als Polygon getrachtet und die Flächen-Momente berechnet. Dieser Konstruktor
muß explizit aufgerufen werden.}

\proch{}{Moments::Moments}{PointList p}{momente.h}
\procf{}{Moments::Moments}{PointList p,int a1,int a2}
\descr{Es werden die Flächen-Momente eines in der Punktliste abgelegten Polygons
berechnet. In der zweiten Version des Aufrufes werden nur die Punkte mit den
Indizes von $a_1$..$a_2$ betrachtet. {\bf Achtung: Die Datenstruktur 
$PointList$ ist veraltet.}}

\subsubsection{Elementzugriff und einfache Werte}
Auf die Elemente der Datenstruktur $Moments$ kann über Methoden zugegriffen
werden.

\proch{double *}{Moments::Mom}{}{moments.h}
\descr{Liefert einen Zeiger auf ein Feld von $double$-Werten, das die Momente
in der Reihenfolge $m_{00}$,
$m_{10}$, $m_{01}$, $m_{20}$, $m_{11}$, $m_{02}$, $m_{30}$, $m_{21}$,
$m_{12}$, $m_{03}$, $m_{40}$, $m_{31}$, $m_{22}$, $m_{13}$, $m_{04}$
enthält.}

\proch{double \&}{operator ()}{int xi,int yi}{moments.h}
\descr{Liefert eine Referenz auf das Moment mit den entsprechenden 
Indizes/Potenzen. Dadurch können einzelne Momente gelesen oder 
geschrieben werden.}

Schwerpunkt

\proch{void}{Moments::getCentroid}{double \&xc,double \&yc}{moments.h}
\descr{
Berechnet den Schwerpunkt aus den Momenten und gibt ihn auf $xc,yc$ zurück.
}

\subsubsection{Transformationen}

Die Anwendung von geometrischen Transformationen auf Momente bedeutet, dass die
Momente bestimmt werden, wie sie das entsprechende transformierte Objekt
aufweisen würde. Die neuberechneten Momente werden zurückgegeben.

\proc{Moments}{Moments::CentralMoments}{}
\descr{Berechnet die zentrale Momente (auf den Schwerpunkt normiert).}

\proc{Moments}{Moments::Translate}{double dx,double dy}
\descr{Translation in der Richtung (dx,dy)}

\proc{Moments}{Moments::Rotate}{double phi}
\descr{Anwendung einer Rotation um den Winkel $phi$ um den Ursprung.}

\proc{Moments}{Moments::XShear}{double s}
\procf{Moments}{Moments::YShear}{double s}
\descr{Anwendung einer Scherung in X- bzw. Y-Richtung mit dem Parameter $s$.}

\proc{Moments}{Moments::Scale}{double a}
\procf{Moments}{Moments::Scale}{double a,double b}
\descr{Anwendung einer Skalierung um den Ursprung. Im ersten Fall ist die
Skalierung isotrop, im zweiten Fall bestimmen die zwei Parameter $a$ und $b$
die Skalierungen in X- bzw. Y-Richtung.}

\proc{Moments}{Moments::AffineTransform}{const Trafo \&tr}
\descr{Wendet eine {\bf affine} Transformation $tr$ an.}

\subsubsection{Invarianten}

\proc{Vector}{Moments::AffineAlgebraicInvariants}{} 
\descr{Aus den Momenten werden algebraische Invarianten nach Flusser
  berechnet und als Vector mit 4 Elementen zurückgegeben.}  
\proc{Vector}{Moments::HuInvariants}{}
\descr{Es werden aus den Momenten 7 Invarianten nach Hu berechnet
und als Vector mit 7 Elementen zurückgegeben.}
\proc{Vector}{Moments::AffineHuInvariants}{}
\descr{Es werden aus den Momenten 7 affine Invarianten nach Hu berechnet
und als Vector mit 7 Elementen zurückgegeben.
Dazu erfolgt eine Normalisierung von Translation, Scherung und Skalierung,
wonach die Invarianten nach Hu berechnet werden.}
Zu invarianten Momenten, die durch Normalisierung gewonnen werden, siehe auch
\bsee{Invariante Momente}.

\subsubsection{Affine Standardlagen und invariante Momente}
\label{Invariante Momente}

Normalisierung bestimmt eine Transformation, die ein Objekt in
eine Standardlage überführt. Die Standardlage wird durch festgelegte Werte
bestimmter Momente definiert. Das transformierte Objekt in Standardlage 
ist dann ein ausgezeichneter Repräsentant für die Objekte der Klasse 
(bezüglich affiner Transformationen). 
Ergebnisse dieser Methoden sind die Transformation (\see{Trafo}) und 
die Momente des Objekts in Standardlage. Diese Informationen können zum
Beispiel genutzt werden, um Objekte zu vergleichen. Die Momente des Objektes
in Standardlage stellen affininvariante Merkmale des Objektes dar 
(Invariante Momente).

\proch{Moments}{Moments::NormalizeSign}{}{moments.h}
\descr{
Alle Momente werden negiert, wenn das Moment $m_{0,0}$ negativ ist.
}

\proch{Moments}{Moments::NormalizeTranslation}{}{moments.h}
\procf{Moments}{Moments::NormalizeTranslation}{double \&x,double \&y}
\descr{
Normalisiert die Momente bezüglich der Translation. Die normalisierten 
Momente (zentrale Momente) werden zurückgegeben. Die Parameter der 
Translation werden den optionalen Parametern zugewiesen.
}

\proch{Moments}{Moments::NormalizeXShearing}{}{moments.h}
\procf{Moments}{Moments::NormalizeXShearing}{double \&beta}
\descr{Normalisiert die Momente bezüglich der X-Scherung 
$x' = x + \beta \cdot y$. Die normalisierten 
Momente werden zurückgegeben. Der Parameter der Scherung
 wird dem optionalen Parameter zugewiesen.
}

\proch{Moments}{Moments::NormalizeYShearing}{}{moments.h}
\procf{Moments}{Moments::NormalizeYShearing}{double \&beta}
\descr{
Normalisiert die Momente bezüglich der Y-Scherung 
$y' = y + \beta \cdot x$. Die normalisierten 
Momente werden zurückgegeben. Der Parameter der Scherung
 wird dem optionalen Parameter zugewiesen.
}

%    enum scalemode { isotropic, anisotropic };
\proch{Moments}{Moments::NormalizeScaling}{scalemode mode=anisotropic}{moments.h}
\procf{Moments}{Moments::NormalizeScaling}{double \&alpha}
\procf{Moments}{Moments::NormalizeScaling}{double \&alpha,double \&beta}
\descr{
Normalisiert die Momente bezüglich der Skalierung. Die normalisierten 
Momente werden zurückgegeben. Der Parameter $mode$ kann die Werte  
$isotropic$ oder $anisotropic$ annehmen und entscheidet über die Verwendung 
einer isotropen oder anisotropen Skalierung. Alternativ wird einem Parameter 
$alpha$ der Skalierungsfaktor zugewiesen (isotrope Skalierung) bzw. den 
Parametern $alpha$ und $beta$ die Skalierungsfaktoren in X- und Y-Richtung 
(anisotrope Skalierung).
}

\proch{Moments}{Moments::NormalizeRotation}{}{moments.h}
\procf{Moments}{Moments::NormalizeRotation}{double \&phi}
\procf{Moments}{Moments::NormalizeRotation}{double \&c,double \&s}
\descr{
Normalisiert die Momente bezüglich der Rotation. Die normalisierten 
Momente werden zurückgegeben. Bei Verwendung des Parameters $phi$ wird 
darauf der Winkel der Rotation zurückgegeben. Die Parameter $c$ und $s$ 
stehen für den Cosinus beziehungsweise den Sinus des Rotationswinkels.
}

\proch{Moments}{Moments::Normalize}{Trafo \&tr,nmode mode=Moments::standard}{moments.h}
\procf{Moments}{Moments::Normalize}{nmode mode=Moments::standard}
\descr{Führt eine Normalisierung bezüglich affiner Transformationen aus. 
Je nach dem Parameter $mode$ wird 
eine Normalisierung nach Standard-Methode (mode=Moments::standard),  
eine Normalisierung nach der Polynommethode (mode=Moments::polynom) oder
eine Normalisierung nach der Iterationsmethode (mode=Moments::iteration)
durchgeführt. Die Momente nach der Normalisierung werden als Rückgabewert
zurückgegeben, die Transformation wird im Parameter $tr$ abgelegt.}

\subsubsection{Fitting mit Momenten}
Fitting mit Momenten beruht darauf, die Momente des zu fittenden Objektes 
und des Referenzobjektes durch Normalisierung in eine Standardlage zu 
überführen. Die ermittelten Transformationen erlauben eine Transformation 
des Referenzobjektes in die Lage des gegebenen Objektes und liefern somit 
ein angepasstes Objekt der gewählten Klasse. Viele dieser Funktionen liefern
optional als Ergebnis ein Gütemass, welches die Abweichung von der idealen Form 
beschreibt. Der Wert von 0 stellt also eine ideale Anpassung dar.

\proc{Trafo}{AffineFit}{const Moments \&m1,const Moments \&m2}
\descr{Bestimmt eine affine Transformation (\see{Trafo}), die die Momente 
in die übergebenen Momente $m2$ überführt.}

\proc{Matrix}{FitTriangle}{const Moments \&m}
\descr{Passt ein Dreieck an, so dass dessen Momente den gegegebenen Momenten
entsprechen. Die Rückgabe der Eckpunktkoordinaten erfolgt als Zeilen einer
Matrix (``Punktliste'').} 

\proc{Matrix}{FitEquilateralTriangle}{const Moments \&m}
\procf{Matrix}{FitEquilateralTriangle}{const Moments \&m,double \&guetemass}
\descr{Passt ein gleichseitiges Dreieck an, so dass dessen Momente den 
gegegebenen Momenten entsprechen. Die Rückgabe der Eckpunktkoordinaten 
erfolgt als Zeilen einer Matrix (``Punktliste'').} 

\proc{Matrix}{FitIsoscelesTriangle}{const Moments \&m}
\procf{Matrix}{FitIsoscelesTriangle}{const Moments \&m,double \&guetemass}
\descr{Passt ein gleichschenkliges Dreieck an, so dass dessen Momente den 
gegegebenen Momenten entsprechen. Die Rückgabe der Eckpunktkoordinaten 
erfolgt als Zeilen einer Matrix (``Punktliste'').} 

\proc{Matrix}{FitSquare}{const Moments \&m}
\procf{Matrix}{FitSquare}{const Moments \&m,double \& guetemass}
\descr{Passt ein Quadrat an, so dass dessen Momente den gegegebenen 
Momenten entsprechen. Die Rückgabe der Eckpunktkoordinaten erfolgt als 
Zeilen einer Matrix (``Punktliste'').} 

\proc{Matrix}{FitRectangle}{const Moments \&m}
\procf{Matrix}{FitRectangle}{const Moments \&m,double \&guetemass}
\descr{Passt ein Rechteck an, so dass dessen Momente den gegegebenen 
Momenten entsprechen. Die Rückgabe der Eckpunktkoordinaten 
erfolgt als Zeilen einer Matrix (``Punktliste'').}

\proc{Matrix}{FitParallelogram}{const Moments \&m}
\procf{Matrix}{FitParallelogram}{const Moments \&m,double \& guetemass}
\descr{Passt ein Parallelogramm an, so dass dessen Momente den gegegebenen 
Momenten entsprechen. Die Rückgabe der Eckpunktkoordinaten erfolgt als 
Zeilen einer Matrix (``Punktliste'').} 

\proc{Matrix}{FitQuadrangle}{const Moments \&m}
\descr{Passt ein allgemeines Viereck an, so dass dessen Momente den 
gegegebenen Momenten entsprechen. Die Rückgabe der Eckpunktkoordinaten 
erfolgt als Zeilen einer Matrix (``Punktliste'').} 

\proc{Matrix}{FitPolygon}{const Moments \&m,const Matrix \&pl}
\procf{Matrix}{FitPolygon}{const Moments \&m,const Matrix \&pl,double \&guetemass}
\descr{Passt ein Polygon an, so dass dessen Momente den gegebenen Momenten 
entsprechen. Als Startlösung ist das Polygon $pl$ zu übergeben, welches dadurch 
auch die Anzahl der Ecken festlegt.}

\proc{int}{FitCircle}{const Moments \&m,double \&x0,double \&y0,double \&radius}
\proc{Circle}{FitCircle}{const Moments \&m}
\descr{Passt einen Kreis an, so dass dessen Momente den gegegebenen 
Momenten entsprechen. Die Rückgabe der Parameter des Kreises erfolgt über 
die Variablen-Parameter $x0$, $y0$, $radius$. Alternativ wird eine
Datenstruktur \see{Circle} zurückgegeben.}

\proc{Ellipse}{FitEllipse}{const Moments \&m}
\descr{Passt eine Ellipse an, so dass deren Momente den gegegebenen 
Momenten entsprechen. Es wird wird eine Datenstruktur \see{Ellipse} 
zurückgegeben.}

\proc{double}{Orientation}{const Moments \&m}
\descr{Liefert einen Winkel (im Bogenmaß), der die Orientierung
des Objekts beschreibt. Dieser Winkel kann zur Beschreibung der 
Orientierung im Vergleich mit gleichartigen Objekten dienen.}
