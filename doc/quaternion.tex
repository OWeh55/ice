\nsection{Quaternionen}{quaternions}
\hypertarget{Trafo}{}

\subsection{Elementare Quaternionen}
\hypertarget{Quaternion}{}

\subsubsection{Konstruktoren und Destruktoren}
\proch{}{Quaternion::Quaternion}{}{quaternion.h}
\descr
{
Es wird eine mit 0 initialisierte Quaternion angelegt.
}
\proc{}{Quaternion::Quaternion}{double real, double i, double j, double k}
\descr
{
Es wird eine Quaternion mit Realteil $real$ und den
Imaginäranteilen $i$, $j$ und $k$ angelegt.
}
\proc{}{Quaternion::Quaternion}{double real, Vector3d\& imaginary}
\descr
{
Es wird eine Quaternion mit Realteil $real$ und
Imaginäranteilen aus dem Vektor $imaginary$ (i,j,k) angelegt.
}
\proc{}{Quaternion::Quaternion}{const Quaternion\& b}
{
Es wird eine Kopie der Quaternion $b$ erzeugt. (Kopier-Konstruktor) 
}
\subsubsection{Operatoren}
\begin{tabular}{|c|c|}
\hline
Operator & Funktion\\ 
\hline
Quaternion = Quaternion & Zuweisungsoperator\\
\hline
Quaternion + Quaternion & Quaternionenaddition\\
\hline
Quaternion - Quaternion & Quaternionensubtraktion\\
\hline
-Quaternion & Negation der Quaternion\\
\hline
Quaternion * double & Multiplikation mit Skalar\\
\hline
double * Quaternion & Multiplikation mit Skalar\\
\hline
Quaternion * Quaternion & Quaternionenmultiplikation\\
\hline
\end{tabular}

\subsubsection{Elementfunktionen}
\proc{double}{Quaternion::getReal}{}
\descr
{
Liefert den Realteil der Quaternion.
}
\proc{double}{Quaternion::getI}{}
\descr
{
Liefert den i-Imaginärteil der Quaternion.
}

\proc{double}{Quaternion::getJ}{}
\descr
{
Liefert den j-Imaginärteil der Quaternion.
}

\proc{double}{Quaternion::getK}{}
\descr
{
Liefert den k-Imaginärteil der Quaternion.
}

\proc{vector3d}{Quaternion::getImaginary}{}
\descr
{
Liefert die Imaginärteile der Quaternion als Vektor (i,j,k)
}

\proc{void}{Quaternion::setReal}{double real}
\descr
{
Setzt den Realteil der Quaternion auf $real$.
}

\proc{void}{Quaternion::setI}{double i}
\descr
{
Setzt den i-Imaginärteil der Quaternion auf $i$.
}

\proc{void}{Quaternion::setJ}{double j}
\descr
{
Setzt den j-Imaginärteil der Quaternion auf $j$.
}

\proc{void}{Quaternion::setK}{double k}
\descr
{
Setzt den k-Imaginärteil der Quaternion auf $k$.
}

\proc{void}{Quaternion::setImaginary}{vector3d\& imaginary}
\descr
{
Setzt die Imaginärwerte der Quaternion auf die Werte des Vektors $imaginary$ (i,j,k).
}

\proc{Quaternion}{Quaternion::getNegate}{}
\descr
{
Gibt die negierte Quaternion zurück.
}

\proc{Quaternion}{Quaternion::getConjugate}{}
\descr
{
Gibt die conjugierte Quaternion zurück.
}

\proc{Quaternion}{Quaternion::getInverse}{}
\descr
{
Gibt die inverse Quaternion zurück, falls diese Vorhanden ist, sonst Fehlermeldung.
}

\proc{Quaternion}{Quaternion::getNormalize}{}
\descr
{
Gibt die normalisierte Quaternion zurück, falls diese Vorhanden ist, sonst Fehlermeldung.
}

\proc{Quaternion}{Quaternion::getSquare}{}
\descr
{
Gibt das Quadrat der Quaternion zurück.
}

\proc{double}{Quaternion::getNorm}{}
\descr
{
Gibt die Norm der Quaternion zurück.
}

\proc{double}{Quaternion::getSquareNorm}{}
\descr
{
Gibt das Quadrat der Norm der Quaternion zurück.
}

\proc{double}{Quaternion::getEigenwinkel}{}
\descr
{
Gibt den Eigenwinkel der Quaternion zurück, falls dieser Vorhanden, sonst Fehlermeldung.
}

\proc{Quaternion}{Quaternion::getEigenachse}{}
\descr
{
Gibt die Eigenachse der Quaternion zurück, falls diese vorhanden, sonst Fehlermeldung.
}

\proc{vector3d}{Quaternion::getPhases}{}
\descr
{
Gibt die Phasen der Quaternion als Vektor $(\alpha,\beta,\delta) $ zurück,falls
vorhanden, sonst Fehlermeldung.
}

\proc{bool}{Quaternion::isUnitQuaternion}{}
\descr
{
Liefert TRUE, falls die Quaternion eine Einheitsquaternion ist.
}

\subsubsection{Funktionen mit Quaternionen}
\proc{Quaternion}{qexp}{Quaternion\& x}
\descr
{
Gibt $e^x$ zurück.
}
\subsection{RotQuaternionen}
\hypertarget{RotQuaternion}{}
\subsubsection{Konstruktoren und Destruktoren}
\proch{}{RotQuaternion::RotQuaternion}{}{rotquaternion.h}
\descr
{
Es wird eine mit 1 initialisierte Rotationsquaternion angelegt.
}

\proc{}{RotQuaternion::RotQuaternion}{double angle, double x, double y, double z}
\descr
{
Es wird eine Rotationsquaternion mit Rotationswinkel $angle$ im Bogenmaß und
Rotationsachse ($x$,$y$,$z$) angelegt.
}

\procf{}{RotQuaternion::RotQuaternion}{double angle, vector3d\& axis}
\descr
{
Es wird eine Rotationsquaternion mit Rotationswinkel $angle$ im Bogenmaß und
Rotationsachse $axis$ angelegt.
}

\procf{}{RotQuaternion::RotQuaternion}{const RotQuaternion\& b)
{
Es wird eine Kopie der Rotationsquaternion b erzeugt. (Kopier-Konstruktor}
}

\subsubsection{Operatoren}
\begin{tabular}{|c|c|}
\hline
Operator & Funktion\\ 
\hline
RotQuaternion = RotQuaternion & Zuweisungsoperator\\
\hline
-RotQuaternion & Negation der Rotationsquaternion\\
\hline
RotQuaternion * RotQuaternion & Quaternionenmultiplikation\\
\hline
\end{tabular}

Operatoren, die den Bereich der Rotationsquaternionen verlassen und eine Quaternion als
Rückgabewert liefern.

\begin{tabular}{|c|c|}
\hline
Operator & Funktion\\ 
\hline
RotQuaternion + Rotquaternion & Quaternionenaddition\\
\hline
RotQuaternion - RotQuaternion & Quaternionensubtraktion\\
\hline
double * RotQuaternion & Multiplikation einer Quaternion mit einem Skalar\\
\hline
RotQuaternion * double & Multiplikation einer Quaternion mit einem Skalar\\
\hline
RotQuaternion*Quaternion & Quaternionenmultiplikation\\
\hline
Quaternion*RotQuaternion & Quaternionenmultiplikation\\
\hline
(Quaternion)* & Castingoperator
\end{tabular}

\subsubsection{Elementfunktionen}
\proc{double}{RotQuaternion::getReal}{}
\descr
{
Liefert den Realteil der Rotationsquaternion.
}

\proc{double}{RotQuaternion::getI}{}
\descr
{
Liefert den i-Imaginärteil der Rotationsquaternion.
}

\proc{double}{RotQuaternion::getJ}{}
\descr
{
Liefert den j-Imaginärteil der Rotationsquaternion.
}

\proc{double}{RotQuaternion::getK}{}
\descr
{
Liefert den k-Imaginärteil der Rotationsquaternion.
}

\proc{vector3d}{RotQuaternion::getImaginary}{}
\descr
{
Liefert die Imaginärteile der Rotationsquaternion als Vektor (i,j,k).
}

\proc{double}{RotQuaternion::getRotationAngle}{}
\descr
{
Liefert den Rotationswinkel der Rotationsquaternion im Bogenmaß.
}

\proc{vector3d}{RotQuaternion::getRotationAxis}{}
\descr
{
Liefert die Rotationsachse der Rotationsquaternion als Vektor (i,j,k).
}

\proc{void}{RotQuaternion::setRotationAngle}{double angle}
\descr
{
Setzt Rotationswinkel der Rotationsquaternion auf angle im Bogenmaß.
}

\proc{void}{RotQuaternion::setRotationAxis}{vector3d\& axis}
\descr
{
Setzt die Rotationsachse der Rotationsquaternion.
}

\proc{RotQuaternion}{RotQuaternion::getNegate}{}
\descr
{
Gibt die negierte Rotationsquaternion zurück.
}

\proc{RotQuaternion}{RotQuaternion::getConjugate}{}
\descr
{
Gibt die conjugierte Rotationsquaternion zurück.
}

\proc{RotQuaternion}{RotQuaternion::getInverse}{}
\descr
{
Gibt die inverse Rotationsquaternion zurück.
}

\proc{RotQuaternion}{RotQuaternion::getSquare}{}
\descr
{
Gibt das Quadrat der Quaternion zurück.
}

\proc{double}{RotQuaternion::getNorm}{}
\descr
{
Gibt die Norm (1.0) der Rotationsquaternion zurück.
}

\proc{double}{RotQuaternion::getSquareNorm}{}
\descr
{
Gibt das Quadrat (1.0) der Norm der Rotationsqquaternion zurück.
}

\proc{double}{RotQuaternion::getEigenwinkel}{}
\descr
{
Gibt den Eigenwinkel der Rotationsquaternion zurück, falls dieser Vorhanden, sonst
Fehlermeldung.
}

\proc{RotQuaternion}{RotQuaternion::getEigenachse}{}
\descr
{
Gibt die Eigenachse der Rotationsquaternion zurück, falls diese vorhanden, sonst
Fehlermeldung.
}

\proc{vector3d}{RotQuaternion::getPhases}{}
\descr
{
Gibt die Phasen der Rotationsquaternion als Vektor $(\alpha,\beta,\delta)$ zurück,
falls vorhanden, sonst Fehlermeldung.
}

\proc{Matrix}{RotQuaternion::getRotationMatrix}{}
\descr
{
Konvertiert die Quaternion in eine $3\times 3$ Rotationsmatrix.
}


\subsubsection{Funktionen mit Rotationsquaternionen}
\proc{Quaternion}{qexp}{RotQuaternion\& x}
\descr
{
Gibt $e^x$ zurück.
}

\proc{RotQuaternion}{convertToRotQuaternion}{Quaternion\& in}
\descr
{
Konvertiert die Quaternion $in$ in eine Rotationsquaternion, falls dies
möglich, sonst Fehlermeldung.
}

\proc{RotQuaternion}{convertToRotQuaternion}{Matrix\& in}
\descr
{
Konvertiert die $3\times 3$ Rotationsmatrix $in$ in eine Rotationsquaternion,
falls dies möglich, sonst Fehlermeldung.
}
\subsection{QuatMatrix - Quaternionenwertige Matrizen}
\hypertarget{QuatMatrix}{}
\subsubsection{Konstruktoren und Destruktoren}
\proch{}{QuatMatrix::QuatMatrix}{}{quatmatrix.h}
\descr
{
Es wird eine quaternionenwertige Matrix unbestimmter Größe angelegt.
}

\procf{}{QuatMatrix::QuatMatrix}{unsigned int rows, unsigned int columns}
\descr
{
Es wird eine quaternionenwertige Matrix mit $rows$ Zeilen und $columns$
Zeilen angelegt.
}

\procf{}{QuatMatrix::QuatMatrix}{const QuatMatrix\& qm}
\descr
{
Es wird eine Kopie der quaternionenwertigen Matrix $qm$ erzeugt.
(Kopier-Konstruktor)
}

\subsubsection{Operatoren}
\begin{tabular}{|c|c|}
\hline
Operator & Funktion\\ 
\hline
QuatMatrix = QuatMatrix & Zuweisungsoperator\\
\hline
QuatMatrix + QuatMatrix & Matrixaddition\\
\hline
QuatMatrix - QuatMatrix & Matrixsubtraktion\\
\hline
QuatMatrix * double & Multiplikation mit Skalar\\
\hline
double * QuatMatrix & Multiplikation mit Skalar\\
\hline
QuatMatrix * Quaternion & Multiplikation mit Quaternion\\
\hline
Quaternion * QuatMatrix & Multiplikation mit Quaternion\\
\hline
Quaternion [int] & Zugriff auf eine Zeile der quaternionenwertigen Matrix\\
\hline
Quaternion [int i][int j] & Zugriff auf Element der Matrix
\end{tabular}

\subsubsection{Elementfunktionen}
\proc{unsigned int}{QuatMatrix::getRows}{}
\descr
{
Liefert die Anzahl der Zeilen der quaternionenwertigen Matrix.
}

\proc{unsigned int}{QuatMatrix::getColumns}{}
\descr
{
Liefert die Anzahl der Spalten der quaternionenwertigen Matrix.
}
\subsection{Quaternionenwertige Vektoren}
\hypertarget{QuatVector}{}
\subsubsection{Konstruktoren und Destruktoren}
\proch{}{QuatVector::QuatVector}{}{quatvektor.h}
\descr
{
Es wird ein quaternionenwertiger Vektor unbestimmter Größe angelegt.
}

\procf{}{QuatVector::QuatVector}{unsigned int dimension}
\descr
{
Es wird ein quaternionenwertiger Vektor der Größe $dimension$ angelegt.
}

\procf{}{QuatVector::QuatVector}{const QuatVector\& qv}
\descr
{
Es wird eine Kopie des quaternionenwertigen Vektors $qv$ erzeugt.
(Kopier-Konstruktor)
}

\subsubsection{Operatoren}
\begin{tabular}{|c|c|}
\hline
Operator & Funktion\\ 
\hline
QuatVector = QuatVector & Zuweisungsoperator\\
\hline
QuatVector + QuatVector & Vektorenaddition\\
\hline
QuatVector - QuatVector & Vektorensubtraktion\\
\hline
QuatVector * double & Multiplikation mit Skalar\\
\hline
double * QuatVector & Multiplikation mit Skalar\\
\hline
QuatVector * Quaternion & Multiplikation mit Quaternion\\
\hline
Quaternion * QuatVector & Multiplikation mit Quaternion\\
\hline
Quaternion [int] & Elementzugriff
\end{tabular}

\subsubsection{Elementfunktionen}
\proc{unsigned int}{QuatVector::getDimension}{}
\descr
{
Liefert die Dimension des quaternionenwertigen Vektors.
}
\subsection{Duale Quaternionen}
\hypertarget{DualQuaternion}{}
\subsubsection{Konstruktoren und Destruktoren}
\proch{}{DualQuaternion::DualQuaternion}{}{dualquaternion.h}
\descr
{
Es wird eine mit $\hat{q}=(0,\vec{0})+(0,\vec{0}\varepsilon)$ initialisierte duale
Quaternion angelegt.
}
\proc{}{DualQuaternion::DualQuaternion}{Quaternion\& real, Quaternion\& dual}
\descr
{
Es wird eine duale Quaternion mit Realteil $real$ und denm dualen Anteil 
$dual$ angelegt.
}
\proc{}{DualQuaternion::DualQuaternion}{const DualQuaternion\& b}
\descr
{
Es wird eine Kopie der dualen Quaternion $b$ erzeugt. (Kopier-Konstruktor)
}
\subsubsection{Operatoren}
\begin{tabular}{|c|c|}
\hline
Operator & Funktion\\ 
\hline
DualQuaternion = DualQuaternion & Zuweisungsoperator\\
\hline
DualQuaternion + DualQuaternion & Addition dualer Quaternionen\\
\hline
DualQuaternion - DualQuaternion & Subtraktion dualer Quaternionen\\
\hline
-DualQuaternion & Negation der dualen Quaternion\\
\hline
DualQuaternion * double & Multiplikation mit Skalar\\
\hline
double * DualQuaternion & Multiplikation mit Skalar\\
\hline
DualQuaternion * DualQuaternion & Multiplikation der dualen Quaternionen\\
\hline
\end{tabular}

\subsubsection{Elementfunktionen}
\proc{Quaternion}{DualQuaternion::getReal}{}
\descr
{
Liefert den Realteil der dualen Quaternion.
}

\proc{Quaternion}{DualQuaternion::getDual}{}
\descr
{
Liefert den Dualteil der dualen Quaternion.
}

\proc{void}{DualQuaternion::setReal}{Quaternion\& real}
\descr
{
Setzt den Realteil der dualen Quaternion auf $real$.
}

\proc{void}{DualQuaternion::setDual}{Quaternion\& dual}
\descr
{
Setzt den Dualteil der dualen Quaternion auf $dual$.
}

\proc{DualQuaternion}{DualQuaternion::getNegate}{}
\descr
{
Gibt die negierte duale Quaternion zurück.
}

\proc{DualQuaternion}{DualQuaternion::getConjugate}{}
\descr
{
Gibt die konjugierte duale Quaternion zurück.
}

\proc{DualQuaternion}{DualQuaternion::getTilde}{}
\descr
{
Gibt das Ergebnis der Anwendung des Tildeoperators zurück.
}

\proc{DualQuaternion}{DualQuaternion::getInverse}{}
\descr
{
Gibt die inverse duale Quaternion zurück, falls vorhanden, sonst Fehlermeldung.
}

\proc{DualQuaternion}{DualQuaternion::getQuasiNorm}{}
\descr
{
Gibt die Quasi-Norm der dualen Quaternion zurück.
}

\proc{bool}{DualQuaternion::isUnitDualQuaternion}{}
\descr
{
Liefert true, falls die duale Einheitsquaternion.
}
\subsection{Duale Transformationsquaternionen}
\hypertarget{TrafoDualQuaternion}{}
\subsubsection{Konstruktoren und Destruktoren}
\proch{}{TrafoDualQuaternion::TrafoDualQuaternion}{}{trafodualquaternion.h}
\descr
{
Es wird eine mit $\hat{q}=(1,\vec{0})+(0,\vec{0}\varepsilon)$ initialisierte duale
Transformationsquaternion angelegt.
}
\proc{}{TrafoDualQuaternion::TrafoDualQuaternion}{RotQuaternion\& rot, vector3d\& trans}
\descr
{
Es wird eine duale Transformationsquaternion aus dem Rotationsanteil $rot$ und dem
Translatiosvektor \textit{trans} angelegt.
}
\proc{}{TrafoDualQuaternion::TrafoDualQuaternion}{const TrafoDualQuaternion\& b}
\descr
{
Es wird eine Kopie der dualen Transformationsquaternion $b$ erzeugt.
(Kopier-Konstruktor)
}

\subsubsection{Operatoren}
\begin{tabular}{|c|c|}
\hline
Operator & Funktion\\ 
\hline
TrafoDualQuaternion = TrafoDualQuaternion & Zuweisungsoperator\\
-TrafoDualQuaternion & Negation einer dualen
Transformationsquaternion\\
\hline
TrafoDualQuaternion * TrafoDualQuaternion & Multiplikation dualer
Transformationsquaternionen\\
\hline
\end{tabular}

Operatoren, die den Bereich der dualen Transformationsquaternionen verlassen und eine
duale Quaternion als Rückgabewert liefern.

\subsubsection{Operatoren}
\begin{tabular}{|c|c|}
\hline
Operator & Funktion\\ 
\hline
TrafoDualQuaternion + TrafoDualQuaternion & Addition zweier dualer
Transformationsquaternioen\\
\hline
TrafoDualQuaternion + DualQuaternion & Addition duale Quaternion und duale
Transformationsquaternion\\
\hline
DualQuaternion + TrafoDualQuaternion & Addition duale Quaternion und duale
Transformationsquaternion\\
\hline
TrafoDualQuaternion - TrafoDualQuaternion & Subtraktion zweier dualer
Transformationsquaternioen\\
\hline
TrafoDualQuaternion - DualQuaternion & Subtraktion duale Transformationsquaternion und
duale duale Quaternion\\
\hline
DualQuaternion - TrafoDualQuaternion & Subtraktion duale Quaternion und duale
Transformationsquaternion\\
\hline
double * TrafoDualQuaternion & Multiplikation mit Skalar\\
\hline
TrafoDualQuaternion * double & Multiplikation mit Skalar\\
\hline
DualQuaternion * TrafoDualQuaternion & Multiplikation mit dualer Quaternion\\
\hline
TrafoDualQuaternion * DualQuaternion & Multiplikation mit dualer Quaternion\\
\hline
(TrafoDualQuaternion)* & Castingoperator\\
\hline
\end{tabular}

\subsubsection{Elementfunktionen}
\proc{Quaternion}{TrafoDualQuaternion::getReal}{}
\descr
{
Liefert den Realteil der dualen Quaternion.
}

\proc{Quaternion}{TrafoDualQuaternion::getDual}{}
\descr
{
Liefert den Dualteil der dualen Quaternion.
}

\proc{void}{TrafoDualQuaternion::setRotation}{RotQuaternion\& rot}
\descr
{
Setzt den Rotationsanteil der dualen Transformationsquaternion auf $rot$.
}

\proc{void}{TrafoDualQuaternion::setDual}{vector3d\& trans}
\descr
{
Setzt die Translation der der dualen Transformationsquaternion.
}

\proc{TrafoDualQuaternion}{TrafoDualQuaternion::getNegate}{}
\descr
{
Gibt die negierte duale Transformationsquaternion zurück.
}

\proc{TrafoDualQuaternion}{TrafoDualQuaternion::getConjugate}{}
\descr
{
Gibt die konjugierte duale Transformationsquaternion zurück.
}

\proc{TrafoDualQuaternion}{TrafoDualQuaternion::getTilde}{}
\descr
{
Gibt das Ergebnis der Anwendung des Tildeoperators zurück.
}

\proc{TrafoDualQuaternion}{TrafoDualQuaternion::getInverse}{}
\descr
{
Gibt die inverse duale Quaternion zurück.
}
\proc{Matrix}{TrafoDualQuaternion::getTransformationMatrix}{}
\descr
{
Gibt die zur dualen Transformationsquaternion äquivalente homogene Transformation zurück.
}
\subsubsection{Funktionen mit dualen Transformatiosquaternionen}
\proc{TrafoDualQuaternion}{convertToTrafoDualQuaternion}{const Matrix\& in}
\descr
{
Konvertiert die homogene Transformationsmatrix $in$ 
in eine duale Transformationsquaternion.
}

\proch{TrafoDualQuaternion}{estimateTransformation}{const
Matrix\& orig,const Matrix\& trans}{quatmatch.h}
\descr
{
Berechnet die Transformation (Rotation und Translation), durch die die in $orig$ gegebenen 
dreidimensionalen Originalpunkte in die in $trans$ gegeben dreidimensionalen Punkte
transformiert wurden.
} 
\subsection{Quaternionen-Fourier-Transformation}
\proch{int}{QFourier}{QuatMatrix\& input, QuatMatrix\& output,int option=NORMAL}{qft.h}
\descr
{
Quaternionenwertige Fourier-Transformation der quaternionenwertigen Matrix \textit{input} 
($mode$=NORMAL Hintransformation, $mode$=INVERS Tücktransformation). Das
Ergebnis der Transformation wird in $output$ abgelegt.
}
\subsubsection{Abgeleitete Spektren}
Für die unten aufgeführten abgeleiteten Spektren gilt:
\begin{itemize}
\item mode = CENTER: zentriertes Spektrum
\item mode = NOCENTER: unzentriertes Spektrum
\end{itemize}

\proch{int}{PowerSpektrumQFT}{QuatMatrix\& input, Image\& output,int type=POWER, int
mode=CENTER}{qft.h}
\descr
{
Berechnet das Leistungsspektrum der quaternionenwertigen Matrix \textit{input} und
legt dieses als Bild $output$ ab. Die Matrix $input$ sollte dabei ein
fouriertransformiertes Signal enthalten.\\
Dabei gilt:
\begin{itemize}
\item type=POWER: Spektrum der Quadratnorm
\item type=NORM: Spektrum der Norm
\item type=LOG: Spektrum in logarithmischer Skalierung
\end{itemize}
}

\proch{int}{EigenwinkelSpektrumQFT}{QuatMatrix\& input, Image\& r, Image\& g,
Image\& b, int mode=CENTER}{qft.h}
\descr{ 
Berechnet das Eigenwinkelspektrum der quaternionenwertigen Matrix $input$ und
legt dieses in den Graustufenbilder $r$, $g$ und $b$ ab.
Diese als RGB-Bild dargestellt, ergibt das Eigenwinkelspektrum. Die Funktion 
kann auch zur Visualisierung von quaternionenwertigen Korrelationen verwendet 
werden. }

\proch{int}{EigenachsenSpektrumQFT}{QuatMatrix\& input, Image\& r, Image\& g,
Image\& b, int mode=CENTER}{qft.h}
\descr
{Berechnet das Eigenachsenspektrum er quaternionenwertigen Matrix $input$ und
legt dieses in den Graustufenbilder $r$, $g$ und $b$ ab.
Diese als RGB-Bild dargestellt, ergibt das Eigenachsenspektrum. Die Funktion 
kann auch zur Visualisierung von quaternionenwertigen Korrelationen verwendet werden.}

\proch{int}{PhasenSpektrumQFT}{QuatMatrix\& input, Image\& alpha, Image\& beta,
Image\& delta, int mode=CENTER}{qft.h}
\descr
{ 
Berechnet das Phasenspektren der Winkel für die $\alpha,\beta,\delta)$ Exponentialform
einer Quaternion und legt diese als Graustufenbil $alpha$, $beta$ und
$delta$ ab.
}
