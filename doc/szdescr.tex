\lsection{Bildbeschreibung mit Nachbarschaftsstrukturen}
\subsection{Datenstrukturen}

Zur Bildbeschreibung und das spätere Matching mit Modelldaten sind relativ
komlexe Datenstrukturen erforderlich, die aus doppelt verketteten Listen 
für Knoten, Kanten und Maschen bestehen, die zusätzlich untereinander durch 
Zeiger
verknüpft sind. Die übergeordnete Datenstruktur ist $Description$, die jeweils
das erste und letzte Element der Knoten-, Kanten- und Maschenlisten enthält
sowie für das Matching mit den Modelldaten einen Verweis auf einen
Klassifikator.

Jedes Element der Bildbeschreibung ist in den Listen genau einmal enthalten.
\begprogr\begin{verbatim}
typedef struct Description_
{
  struct Mesh_          *FirstMesh;     /*Flaechenliste*/
  struct Mesh_          *LastMesh;
  struct Edge_          *FirstEdge;     /*Kantenliste*/
  struct Edge_          *LastEdge;
  struct Vertex_        *FirstVertex;   /*Knotenliste*/
  struct Vertex_        *LastVertex;
  void                  *C;             /*Klassifikator*/
} *Description;
\end{verbatim}\endprogr
Knoten enthalten neben den Bildebenenkoordinaten ein Feld von Pointern auf die
von dem Knoten ausgehenden Kanten und den Winkel dieser Kanten bezüglich einer
Referenzrichtung. Außerdem kann ein Verweis auf ein zugehöriges 3D-Element
eingetragen werden.
\begprogr\begin{verbatim}
typedef struct Vertex_
  {
    struct Vertex_ *prev,*next;         /*Listenverkettung*/
    int           state;                /*Status 0 - gelöscht*/
    int           edges;                /*Anzahl der ausgehenden Kanten*/
    double        p[2];                 /*Punktkoordinaten*/
    struct Edge_  **edg;                /*Kantenliste*/
    double        *angle;               /*Winkel d. Kante im Knoten*/
    struct Description_ *D;             /*zugeh. Beschr.-strukt.*/
    void             *Corr3d;           /*korrespondierendes 3D-Element*/
  } *Vertex;
\end{verbatim}\endprogr

Kanten enthalten Verweise auf die Anfangs- und Endknoten und auf die beiden
Maschen, die durch die Kante begrenzt werden. Zusätzlich sind Felder für die
nähere Charakterisierung der Kante (Hessesche Parameter, Länge, Kantenstärke)
vorgesehen.
\begprogr\begin{verbatim}
typedef struct Edge_
{ 
  struct Edge_   *prev,*next;    /*Listenverkettung*/
  double         par[5];         /*Geraden: Hessesche Parameter p und phi*/
                                 /*Ellipsen: xm,ym,a,b,phi*/
  struct Vertex_ *vx[2];         /*Knoten am Linienanfang u. -ende*/
  struct Mesh_   *msh[2];        /*Zeiger auf angr. Flaechen*/
  double        lng;             /*Segmentlaenge*/
  double          w;             /*Kantenstaerke*/
  int           type;            /*Kurventyp: 1 - Liniensegment*/
                                 /*           2 - Ellipsenbogensegment*/
  int           state;           /*Bearbeitungsstatus*/
                                 /*Bit 1: nicht geloescht*/
                                 /*Bit 2: Bildkante*/
                                 /*Bit 3: "künstliche" Kante*/
  struct Description_ *D;        /*zugeh. Beschr.-strukt.*/
  void       *Corr3d;            /*korrespondierendes 3D-Element*/
} *Edge;
\end{verbatim}\endprogr

Maschen enthalten jeweils ein Feld von Pointern auf die sie begrenzenden Kanten
und auf deren Anfangsknoten. Weiterhin sind Verweise auf Lochmaschen, die
vollständig in der Masche liegen, und Merkmale zur Klassifikation der Masche
vorgesehen. Weitere Felder werden für das Matching mit Modelldaten benötigt.
\begprogr\begin{verbatim}
typedef struct Mesh_
{
  struct Mesh_    *prev,*next;   /*Listenverkettung*/
  struct Edge_    **edg;         /*begrenzende Kanten*/
  struct Vertex_  **vx;          /*Anfangsknoten der Kanten*/
  int             edges;         /*Anzahl der Flaechen*/
  int             border;        /*Flag f. Randmasche*/
  int             ishole;        /*Lochmasche? (TRUE/FALSE)*/
  int             holes;         /*Anz. der umschlossenen Maschen*/
  struct Mesh_    **hol;         /*Liste der Lochmaschen*/
  int             matched;       /*Flag f. erfolgte Zuordnung*/
  int             matches;       /*Anz. d. zugeordn. Modellmaschen*/
  int             *matchmod;     /*Modellindizes*/
  int             *matchmsh;     /*Modellmaschenindizes*/
  Model           *models;       /*Verweis auf zugeordn. Modelle*/
  double          area;          /*Flaeche (vorzeichenbehaftet)*/
  double          length;        /*Umfang*/
  int             ellipse;       /*Masche ellipsenähnlich (TRUE/FALSE)*/
  int             class;         /*Klassifikationsergebnis*/
  double          *iv1;          /*Fuenf-Punkte-Invarianten*/
  double          *iv2;
  struct Description_ *D;        /*zugeh. Beschr.-strukt.*/
  void      *Corr3d;             /*korrespondierendes 3D-Element*/
} *Mesh;
\end{verbatim}\endprogr

\subsection{Verwaltung der Datenstrukturen}

\subsubsection{Knoten}

\proc{Vertex}{NewVertex}{Description D,double p[2]}
\descr{
neuen Knoten bereitstellen
}
\proc{int}{DeleteVertex}{Vertex vx}
\descr{
Knoten aus Liste entfernen
}
\proc{int}{ActualizeVertex}{Vertex vx}
\descr{
Knotendaten aktualisieren
}
\proc{double}{AngleEdge}{Vertex vx,Vertex vx1}
\descr{
Winkel einer Kante im Knoten
}
\proc{int}{InsertEdge}{Edge edg,Vertex vx}
\descr{
Kante mit Knoten verbinden
}
\proc{int}{ReleaseEdge}{Edge edg,Vertex vx}
\descr{
Kante aus Knoten entfernen
}

\subsubsection{Kanten}

\proc{Edge}{NewEdge}{Description D,Vertex vx1,Vertex vx2,int state}
\descr{
neue Kante bereitstellen
}
\proc{int}{DeleteEdge}{Edge edg}
\descr{
Kante aus Liste entfernen
}
\proc{Edge}{CopyEdge}{Edge edg,Description D}
\descr{
Die Kante $edg$ wird kopiert und die kopierte Kante in die Bildbeschreibung
$D$ aufgenommen. Die beiden Endknoten werden neu angelegt und ebenfalls in $D$
aufgenommen. Verweise auf Maschen und korrespondierende Kanten werden nicht
übernommen. 
}

\subsubsection{Maschen}

\proc{Mesh}{NewMesh}{Description D}
\descr{
Masche anfordern
}
\proc{int}{DeleteMesh}{Mesh msh}
\descr{
Masche löschen.
}


\subsubsection{Bildbeschreibung}

\proc{Description}{NewDescription}{void}
\descr{
Beschreibung anlegen
}
Description NewDescription(void);      /*Beschreibung anlegen*/
\proc{void}{DeleteDescription}{Description D}
\descr{
Beschreibung löschen und Speicher freigeben
}


\subsubsection{Konsistenzprüfung}

\proc{int}{IsVertex}{Description D,Vertex vx}
\descr{
ist der Knoten in der Beschreibung
}
\proc{int}{IsEdge}{Description D,Edge edg}
\descr{
ist die Kante in der Beschreibung?
}
\proc{int}{IsMesh}{Description D,Mesh msh}
\descr{
ist Masche in Liste?
}
\proc{int}{IsConsistent}{Description D}
\descr{
Konsistenz der Beschreibung pruefen
}


\subsection{Zugriff auf Bildbeschreibungen}

\proc{int}{EdgeOfVertex}{Edge edg,Vertex vx}
\descr{
Nummer Kante in Knoten
}
\proc{Vertex}{NeighborVertex}{Vertex vx,int enr}
\descr{
Über Kante benachbarter Knoten
}
\proc{int}{HasCommonEdge}{Vertex vx1,Vertex vx2}
\descr{
Liefert TRUE, wenn die Knoten $vx1$ und $vx2$ über eine Kante miteinander
verbunden sind.
}
\proc{int}{VertexOfEdge}{Vertex vx,Edge edg}
\descr{
Nummer Knoten in Kante
}
\proc{int}{MeshOfEdge}{Mesh msh,Edge edg}
\descr{
Nummer Masche in Kante
}
\proc{int}{EdgeOfMesh}{Edge edg,Mesh msh}
\descr{
Nummer Kante in Masche
}
\proc{Mesh}{NeighborMesh}{Edge edg,Mesh msh}
\descr{
Nachbarmasche bestimmen
}
\proc{Vertex}{MeshVertex}{Mesh msh,int i}
\descr{
Anf.-ktn. i-te Kante einer Masche
}

\proc{Vertex}{CommonVertex}{Edge edg1, Edge edg2}
\descr{
Knoten bestimmen, der beiden Kanten gemeinsam ist.
}
\proc{Edge}{ConnectingEdge}{Vertex vx1,Vertex vx2}
\descr{
Es wird die Kante zurückgegeben, die die beiden Knoten $vx1$ und $vx2$
miteinander verbindet.
}

\subsection{Zeichnen von Beschreibungen}

\proc{void}{DrawVertex}{Vertex vx,Image img,int val}
\descr{
Knoten zeichnen
}
\proc{void}{DrawEdge}{Edge edg,Image img, int val}
\descr{
Kante zeichnen
}
\proc{void}{DrawMesh}{Mesh msh,Image img, int val}
\descr{
Masche zeichnen
}
\proc{void}{DrawDescription}{Description D,int val,Image img}
\descr{
Beschreibung zeichnen
}

\subsection{Bearbeitung von Beschreibungen}

\proc{int}{MergeVertices}{Vertex vx1,Vertex vx2}
\descr{
Die Knoten $vx1$ und $vx2$ werden zu einem Knoten zusammengefaßt. Es wird der
durch Ausgleichsrechnung bestimmte ``Schnittpunkt'' aller Kanten zugeordnet.
}
\proc{int}{MergeVertexList}{Description D,double maxdist}
\descr{
Die Knotenliste wird nach Paaren von Knoten durchsucht, deren Abstand kleiner
als $maxdist$ ist. Solche Knotenpaare werden zu einem Knoten zusammengefaßt.
}
\proc{Vertex}{SplitEdge}{Edge edg,double p[2]}
\descr{
Die Kante $edg$ wird im Punkt $p$ geteilt und ein neuer Knoten eingefügt.
}
\proc{int}{CloseGaps}{Description D,double maxdist}
\descr{
Alle Knoten, deren Abstand nicht größer als $maxdist$ ist, werden durch eine
Kante verbunden.
}
\proc{int}{RemoveShortEdges}{Description D,double minlng}
\descr{
Alle Kanten, die kürzer als $minlng$ sind, werden entfernt und ihre Endknoten
zu einem Knoten zusammengefaßt.
}
\proc{double}{DistVertexEdge}{Vertex vx,Edge edg,double p[2],int *mode}
\descr{
Der Abstand des Knotens $vx$ zur Kante $edg$ wird zurückgegeben. Der dem
Knoten am nächsten liegende Punkt der Kante wird auf $p$ bereitgestellt und
$mode$ wird auf 0 gesetzt, wenn $p$ der Anfangspunkt der Kante ist, auf 1,
wenn $p$ der Endpunkt der Kante ist und auf 2, wenn $p$ zwischen Anfangs- und
Endpunkt liegt.
}
\proc{double}{MeshAngle}{Mesh msh,Vertex vx}
\descr{
Der Winkel zwischen den beiden zur Masche $msh$ gehörenden Kanten im Knoten
$vx$ wird berechnet.
}
\proc{int}{HasCrossPoint}{double p11[2],double p12[2],double p21[2],double p22[2]}
\descr{
Liefert TRUE, wenn sich die Geradensegmente $p11,p21$ und $p21,p22$ kreuzen.
}
\proc{int}{InvarMeshList}{Description D}
\descr{
Für alle Maschen mit mehr als vier Punkten werden die Fünf-Punkte-Invarianten
berechnet.
}
\proc{int}{PointInMesh}{double p[2],Mesh msh}
\descr{
Es wird TRUE zurückgegeben, wenn der Punkt $p$ innerhalb der Masche $msh$ liegt.
}





\proc{int}{VertexMergeList}{Description D,double rad,double maxd}
\descr{
Zusammenfassen von Knoten
}
\proc{int}{MergeEdges}{Vertex vx,double maxdist}
\descr{
Knoten evtl. entfernen
}
\proc{int}{MergeEdgeList}{Description D,double maxdist}
\descr{
Kanten vereinigen
}
\proc{int}{RemoveDoubleEdges}{Description D}
\descr{
Entfernen von doppelten Kanten
}
\proc{Mesh}{CreateMesh}{Edge edg,int side}
\descr{
Masche zu Kante erzeugen
}
\proc{int}{ListMeshes}{Description D}
\descr{
Maschenliste erzeugen
}
\proc{Mesh}{DeleteVertexMeshes}{Vertex vx,Mesh msh}
\descr{
alle Maschen, die den Knoten $vx$ berühren, werden gelöscht.
}
\proc{int}{CreateVertexMeshes}{Vertex vx}
\descr{
Es werden alle Maschen erzeugt, die den Knoten beinhalten.
}
\proc{int}{ClassifyMesh}{Mesh msh}
\descr{
Die Masche $msh$ wird klassifiziert.
}
\proc{int}{ClassifyMeshList}{Description D}
\descr{
Alle Maschen der Liste werden klassifiziert.
}
\proc{int}{IsConcaveVertex}{Mesh msh,int j}
\descr{
Liefert TRUE, wenn die Masche im $j$-ten Knoten konkav ist.
}
\proc{int}{OppositeVertex}{Mesh msh,int i,double dmin}
\descr{
Es wird ein ``gegenüberliegender'' Knoten gesucht.
}
\proc{int}{IsInMesh}{Mesh msh,int i,int j}
\descr{
Es wird getestet, ob die Strecke vom $i$-ten zum $j$-ten Knoten der Masche
$msh$ vollständig innerhalb der Masche verläuft.
}
\proc{int}{SplitMesh}{Mesh msh,int i,int j,MomentClassifyer *mc}
\descr{
Die Masche $msh$ wird geteilt, indem eine Kante vom $i$-ten zum $j$-ten Knoten
eingefügt wird. Die entstehenden Teilmaschen werden klassifiziert.
}
\proc{int}{ClassifySubMesh}{Mesh msh,int i,int j,MomentClassifyer *mc,double *d}
\descr{
Die durch Trennung der Masche $msh$ zwischen dem $i$-ten zum $j$-ten Knoten
entstehende Masche wird klassifiziert.
}
\proc{int}{CutMesh}{Mesh msh,MomentClassifyer *mc}
\descr{
Es wird versucht, durch einfügen von Kanten die Masche $msh$ so zu
unterteilen, daß die entstehenden Maschen einer Klasse zugeordnet werden können.
}
\proc{int}{SearchSubMesh}{Mesh msh,MomentClassifyer *mc}
\descr{
Es wird versucht, durch einfügen einer Kante von der Masche $msh$ eine Masche
abzutrennen, die einer Klasse zugeordnet werden kann.
}
\proc{int}{CutMeshList}{Description D}
\descr{
Die Maschenliste wird nach unklassifizierten Maschen durchsucht, aus denen
durch Unterteilung klassifizierbare Maschen erzeugt werden können.
}
\proc{int}{ClearInnerEdges}{Mesh msh}
\descr{
In die Masche $msh$ hineinragende, doppelt durchlaufene Kanten werden entfernt.
}
\proc{int}{ClearInnerEdgeList}{Description D}
\descr{
In allen Maschen der Maschenliste werden in die Masche hineinragende, doppelt
durchlaufene Kanten entfernt.
}
\proc{Mesh}{RemoveTVertex}{Mesh msh,double dist}
\descr{
In der unklassifizierten Masche $msh$ werden T-Knoten aufgelöst.
}
\proc{int}{RemoveTVertexList}{Description D}
\descr{
In allen unklassifizierten Maschen der Maschenliste werden T-Knoten aufgelöst.
}
\proc{int}{SearchHoles}{Description D}
\descr{
Lochmaschen bestimmen und markieren
}
\proc{char}{IsHole}{Mesh msh}
\descr{
Liefert TRUE, wenn die Masche eine Lochmasche ist.
}
\proc{int}{CutBorderMeshList}{Description D}
\descr{
Es wird versucht, aus den Randmaschen durch Hinzufügen oder Löschen von Kanten
sinnvolle Maschen abzuspalten.
}
\proc{int}{CutBorderMesh}{Mesh msh}
\descr{
Es wird versucht, aus der Randmasche $msh$ durch Hinzufügen oder Löschen von Kanten
sinnvolle Maschen abzuspalten.
}
\proc{char}{SegmentInMesh}{double p1[2],double p2[2],Mesh msh}
\descr{
Liefert TRUE, wenn die Strecke $p1,p2$ vollständig innerhalb der Masche $msh$ verläuft.
}
\proc{char}{CrossMesh}{double p1[2],double p2[2],Mesh msh}
\descr{
Liefert TRUE, wenn die Strecke $p1,p2$ eine Kante der Masche $msh$ schneidet.
}
\proc{int}{SearchConcavity}{Mesh msh,int i}
\descr{
Beginnend beim $i$-ten Knoten der Masche $msh$ wird nach einer Konkavität
gesucht.
}

\proc{int}{AddPolygon}{Contur c, int lmin, double dmax, Description D}
\descr{
Die Kontur $c$ wird mit der minimalen Segmentlänge $lmin$ und dem maximalen
Abstand $dmax$ in Geradensegmente segmentiert und als zusammenhängender
Polygonzug in die Beschreibung $D$ aufgenommen.
}
\proc{int}{SearchVertexEdge}{Image imgv,Image imgo,Vertex vx}
\descr{
Es wird nach Kanten gesucht, die vom Knoten $vx$ ausgehen,
}
\proc{int}{SearchMeshEdge}{Image imgv,Image imgo,Mesh msh,int n}
\descr{
Es wird innerhalb der Masche $msh$ nach Kanten gesucht, die von dem $n$-ten Knoten
ausgehen.  
}
\proc{int}{MeshEdges}{Description D,Image imgv,Image imgo}
\descr{
Es werden innerhab der in der Beschreibung $D$ enthaltenen Maschen Kanten gesucht.
}
\proc{int}{SearchLowEdges}{Image imgv,Image imgo,Description D}
\descr{
Es werden Kanten gesucht, die von Knoten der Beschreibung $D$ ausgehen.
}
\proc{int}{SearchStrongEdges}{Image imgv,Image imgo,int maxgrd,Description D}
\descr{
Im Bild $imgv$ werden mit der Schwelle $maxgrd$ für die Startpunktsuche
gradientenorientiert Kanten gesucht.
}
\proc{Description}{CreateEdgeList}{Image imgv,Image imgo}
\descr{
Im Bild $imgv$ werden gradientenorientiert Kanten gesucht und in eine neu
angelegte Beschreibung eingetragen.
}
