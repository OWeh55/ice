\nsection{Allgemeine Werkzeuge}{genericTools}

\subsection{String-Funktionen}

\proc{string}{NumberString}{int i,unsigned int width=0}
\procf{string}{NumberString}{unsigned int i,unsigned int width=0}
\procf{string}{NumberString}{long int i,unsigned int width=0}
\procf{string}{NumberString}{long long int i,unsigned int width=0}
\procf{string}{NumberString}{double f,unsigned int width=0,unsigned int
  prec=0}
\descr{Gibt einen String zurück, der den übergebenen Wert in Textform
  darstellt. Sind die Werte von $width$ oder $prec$ ungleich Null, so wird die
  Ausgabe entsprechend dieser Werte formatiert.}

\proc{int}{Integer}{const string\& s}
\descr{Wandelt den übergebenen String $s$ nach Integer (int).}

\proc{double}{Double}{const string \&s}
\descr{Wandelt den übergebenen String $s$ nach double.}

\proc{string}{lowercase}{const string \&s}
\proc{string}{uppercase}{const string \&s}
\descr{Wandelt die Buchstaben des übergebenen Strings $s$ in Klein-
  bzw. Großbuchstaben um.}

\proch{string}{Split}{string \&text,const string \&del,int erase=true,
int need\_del=false}{strtool.h}
\descr{Die Funktion spaltet vom gegebenen String $text$ von vorne einen
 Teilstring ab und gibt diesen zurück, der durch den Begrenzer $del$ 
abgegrenzt wird. Ist $erase = true$ (Standard), so wird $text$ um den 
abgespalteten Text und den Begrenzer gekürzt. Kommt der Begrenzer $del$ 
im String $text$ nicht vor, so wird bei $need\_del=false$ (Standard) der 
gesamte restliche String zurückgegeben, bei $need\_del=true$ wird in diesem
Fall nichts abgespalten.}

\proch{string}{SplitFilename}{const string \&fullname,string \&path,string \&basename,string \&ext}{strtool.h}
\procf{string}{SplitFilename}{const string \&fullname,string \&path,string \&filename}
\descr{Die Funktion spaltet den gegebenen Namen einer Datei in Pfad (Verzeichnis) 
und Dateinamen auf. Wird der Parameter $ext$ angegeben wird der Dateinamen in
den Basisnamen (ohne Erweiterung) und die Dateierweiterung aufgespalten. 
Die Trennzeichen '/' am Ende des Pfades und '.' zwischen Dateinamen und Erweiterung
gehören nicht zu den Ergebnis-Zeichenketten.}

\subsection{Alphanumerische Bildschirmsteuerung}

Für ICE-Implementierungen unter X11 und MS-Windows steht ein
terminalähnliches Textfenster zur Verfügung, an das auch dann Eingaben
weitergeleitet werden, wenn das ICE-Grafikdisplayfenster den Focus hat. Damit
sind Anwendungsprogramme mit einem einfachen Nutzerinterface, das sich auf
einem Textbildschirm realisieren läßt, zwischen den verschiedenen
Implementierungen portabel.

\proc{void}{OpenAlpha}{unsigned char *windowname}
\descr{
Es wird ein Textfenster mit dem Namen $windowname$ angelegt geöffnet.\\
Auch ohne expliziten Aufruf dieser Funktion wird ein Textfenster
geöffnet, wenn eine Funktion zur Ausgabe bzw. Eingabe aufgerufen wird.
}
\proc{void}{CloseAlpha}{void}
\descr{
Das Textfenster wird abgemeldet.
}
\proc{void}{Alpha}{int mode}
\descr{
Das Textfenster wird mit $mode$=OFF minimiert und mit $mode$=ON wieder eingeschaltet.
}
\proc{void}{PushAlpha}{void}
\descr{
Der aktuelle Zustand des Textfensters wird auf einem Stack gesichert.
}
\proc{void}{PopAlpha}{void}
\descr{
Das zuletzt auf dem Stack gesicherte Textfenster wird wiederhergestellt.
}
\proc{void}{ClearAlpha}{void}
\descr{
Der Inhalt des Textfensters wird gelöscht.
}
\proc{void}{SetAlphaCursor}{int x,int y}
\descr{
Ein Textkursor wird im Textfenster auf die Position $(x,y)$ gesetzt.
}
\proc{void}{GetAlphaCursor}{int *x,int *y}
\descr{
Die aktuelle Position des Textkursors wird auf den 
Variablen $x$ und $y$ bereitgestellt.
}
\proc{void}{Bell}{void}
\descr{
Es wird ein Ton ausgegeben.
}
\proc{int}{AlphaSizeX}{void}
\descr{
Die Anzahl der Spalten des Textfensters wird zurückgegeben.
}
\proc{int}{AlphaSizeY}{void}
\descr{
Die Anzahl der Zeilen des Textfensters wird zurückgegeben.
}
\proc{void}{SetAttribute}{int fg,int bg,int inv,int high}
\descr{
Es werden Attribute für den Textbildschirm gesetzt. Die Vordergrundfarbe wird
durch die niederwertigen drei Bits von $fg$, die Hintergrundfarbe duch die
niederwertigen drei Bits von $bg$ bestimmt. Für $inv \ne 0$ wird der Text
invertiert dargestellt und mit $high \ne 0$ hervorgehoben.\\
Für die Farbangaben können die Konstanten C\_BLACK, C\_RED, C\_GREEN,
C\_BLUE, C\_YELLOW, C\_CYAN, C\_MAGENTA und C\_WHITE verwendet werden.
}


\subsection{Ein- und Ausgabefunktionen}

\proc {int}{PutChar}{int c}
\descr{
Es wird ein Zeichen $c$ an das Textfenster ausgegeben.
}
\proc {int}{GetChar}{void}
\descr{
Es wird die Eingabe eines Zeichens erwartet und der Zeichencode der
gedrücketen Taste zurückgegeben.
}
\proc {int}{GetKey}{void}
\descr{
Es wird getestet, ob eine Taste gedrückt wurde. Ist das der Fall, wird der
zugehörige Zeichencode zurückgegeben, wenn keine Taste gedrückt wurde, ist der
Rückgabewert 0.
}
\proc{int}{Printf}{const char *format,...}
\descr{
Ausgabe in das Textfenster. Der Formatstring und die restlichen Parameter sind
aufgebaut wie bei der Standardfunktion printf().
}
\proc{int}{Print}{const string \&s}
\descr{
Ausgabe des string s in das Textfenster.
}
\proc{int}{Input}{const char *prompt}
\procf{int}{Input}{const char *prompt,int defvalue}
\procf{int}{Input}{const string \&prompt}
\procf{int}{Input}{const string \&prompt,int defvalue}
\descr{
Eingabe einer Integerzahl. Der String $prompt$ wird als Prompt ausgegeben. Bei
fehlerhafter Eingabe wird die Anforderung wiederholt. Ist $defvalue$ gegeben, 
so wird dieser Wert als Default-Wert angeboten.
}

\proc{double}{InputD}{const char *prompt}
\procf{double}{InputD}{const char *prompt,double defvalue}
\procf{double}{InputD}{const string \&prompt}
\procf{double}{InputD}{const string \&prompt,double defvalue}
\descr{
Eingabe einer Gleitkommazahl. Der String $prompt$ wird als Prompt ausgegeben. 
Bei fehlerhafter Eingabe wird die Anforderung wiederholt. 
Ist $defvalue$ gegeben, so wird dieser Wert als Default-Wert angeboten.
}
\proc{void}{InputS}{const char *p,char *s}
\procf{string}{InputS}{const string \&p}
\descr{
Eingabe einer Zeichenkette. Der String $p$ wird als Prompt ausgegeben.
}
\proc{int}{InputString}{char *Str,int Control,int *Ptr,int *scroll}
\descr{
Eingabe einer Zeichenkette. Falls $Control$ ungleich Null ist, wird die
Eingabe auch durch Kursortasten abgebrochen und als Rückgabewert der
entsprechende Tastendruck geliefert.
}

\subsection{Zeitmessung}

\proc{double}{TimeD}{int mode=TM\_WORLD}
\descr{
Es wird die Zeit in Sekunden zurückgegeben. Abhängig von $mode$ können 
verschiedene Zeiten abgefragt werden:
\begin{itemize}
\item TM\_WORLD Die Systemzeit. Widerspiegelt die real abgelaufene Zeit.
\item TM\_PROCESS Die Prozesszeit. Widerspiegelt die vom Prozess genutzte
  Rechenzeit inklusive Systemaufrufen.
\item TM\_USER Die ``user time''. Widerspiegelt die vom Prozess genutzte
  Rechenzeit ohne Systemaufrufe.
\item TM\_CHILD Die Kindzeit. Widerspiegelt die von Kindern des aktuellen Prozesses genutzte
  Rechenzeit inklusive Systemaufrufen.
\item TM\_CHILDUSER Die ``user''-Kindzeit. Widerspiegelt die von Kindern des
  aktuellen Prozesses genutzte Rechenzeit ohne Systemaufrufe.
\end{itemize}
}
\proc{void}{Delay}{double time}
\descr{
Verzögerung von $time$ Sekunden.
}

\subsection{Menüs und Verzeichnisse}

\proch{void}{Directory}{vector$<$string$>$ \&t,const string \&mask,int mode=DIR\_FILE $|$ DIR\_DIR}{dirfunc.h}
\descr{Im vector$<$string$>$ $t$ werden die Namen der gefundenen Dateien abgelegt, 
die der Auswahlmaske $mask$ entsprechen. Die Maske ist den Konventionen des jeweiligen 
Betriebssystems entsprechend aufzubauen. Enthält die Maske keinen Pfad,
wird das aktuelle Verzeichnis benutzt.
Enthält die Liste bereits Einträge, so bleiben diese erhalten. Die neuen Einträge
der Liste werden alphabetisch sortiert angehängt.\\
\begin{tabular}{ll}
Modus & Bedeutung\\
DIR\_FILE & suche reguläre Dateien \\
DIR\_DIR & suche Verzeichnisse \\
DIR\_WITHPATH & Ergänze die Eintrage um den Pfad \\
\end{tabular}
}
\seealso{SplitFilename}

\proch{int}{Menu}{const vector$<$string$>$ \&t,int x1=-1,int y1=-1,int x2=-1,int y2=-1, bool restore=false,const string \&title=''''}{visual/menu.h}
\procf{int}{Menu}{const vector$<$string$>$ \&t,const vector$<$int$>$ \&menuId, int x1=2,int y1=2,int x2=30,int y2=22, bool restore=false,const string \&title=''''}
\descr{
  Aus der Liste von Menüpunkten $t$ wird in dem durch die Koordinaten $x1,y1$ und $x2,y2$
  festgelegten Ausschnitt des Textfensters ein Menü aufgebaut.
  Die Auswahl erfolgt mit den Kursortasten. Ist im Menütext ein Buchstabe durch ein 
  vorangestelltes '~' markiert, so kann dieser Menüpunkt durch die entsprechende 
  Taste ausgewählt werden, sonst wird der Anfangsbuchstabe verwendet.
  Rückgabewert der Funktion ist die Nummer $nr$ des ausgewählten Texteintrages, 
  auf den dann mit $t[nr]$ zugegriffen werden kann. Wird der Parameter $menuId$
  angegeben, so wird der entsprechende Wert aus $menuId$ zurückgegeben.
  Bei Abbruch mit $<$ESC$>$ ist der Rückgabewert negativ (=ERROR).
  Ist $restore=true$ wird nach der Menüauswahl der vorherige Bildschirminhalt 
  wiederhergestellt.
}

\proch{bool}{SelFile}{const string \&mask,string \&filename,string \&dirname,int mode=DIR\_FILE,const string \&title='''', int x1=-1,int y1=-1,int x2=-1,int y2=-1}{visual/menu.h}
\procf{bool}{SelFile}{const string \&mask, string \&filename}
\descr{Eine einfache interaktive Dateiauswahl. Mittels \see{Directory} wird eine
Dateiliste aufgebaut, aus der mittels \see{Menu} eine Datei ausgewählt werden
kann. Je nach $mode$ kann eine Datei und/oder ein Verzeichnis gewählt werden. Ist bei
$mode$ der Wert $DIR\_SELPATH$ angegeben, kann man damit eine Datei in einem beliebigen
erreichbaren Verzeichnis ausgewählt werden. Die Funktion gibt nach erfolgreicher Auswahl 
den Wert $true$ zurück. Die ausgewählte Datei steht in $filename$, das Verzeichnis in 
$dirname$. Bei Aufruf in der zweiten Form wird ist $filename$ aus Verzeichnis und 
Dateinamen zusammengesetzt. Der Verzeichnisname ist relativ oder absolut, je nachdem wie
das Verzeichnis in $mask$ angegeben wurde.}

%\proc{string}{FileSel}{const string \&mask,int x1=2,int y1=2,int x2=30,int y2=22,bool restore=true}
%\descr{
%Eine einfache interaktive Dateiauswahl. Mittels \see{Directory} wird eine
%Dateiliste aufgebaut, aus der mittels \see{Menu} eine Datei ausgewählt werden
%kann. Im Fehlerfalle (keine Datei vorhanden) oder bei Abbruch durch den Nutzer
%wird der Leerstring zurückgegeben.}

%%% Local Variables: 
%%% mode: latex
%%% TeX-master: "icedoc_base"
%%% End: 
