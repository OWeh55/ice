\nsection{Grundlegende Datenstrukturen}{datastructures}
\subsection{Punkte - Datenstrukturen Point und IPoint}
\hypertarget{Point}{}
\hypertarget{IPoint}{}
Die Klassen \class{Point} und \class{IPoint} beschreiben Punkte der 
Ebene durch ihre Koordinaten. \class{Point} speichert die Koordinaten 
als {\bf double}-Werte, \class{IPoint} als {\bf int}-Werte. Oft werden 
Punkte auch als 2D-Vektoren aufgefasst.

Für einen sehr schnellen Zugriff sind die Instanz-Variablen $x$ und $y$ von 
aussen zugreifbar. Da \class{Point} und \class{IPoint} als 
Spezialisierungen einer Template-Klasse implementiert sind, gelten die 
folgenden Methoden für beide Klassen, auch wenn hier jeweils nur 
\class{Point} genannt ist.

Konstruktoren

\ctor{Point}{}{point.h}
\descr{Standard-Konstruktor. Die Koordinaten werden mit 0 initialisiert.}

\ctor{Point}{double x, double y}{point.h}
\descr{Die Koordinaten werden mit $(x,y)$ initialisiert.}

\ctor[explicit]{Point}{const Vector \&v}{point.h}
\descr{Legt eine Datenstruktur Point an und initialisiert $x$ und $y$ mit den
ersten beiden Werten des \see{Vector} v.}

\ctor[explicit]{Point}{const IVector \&v}{point.h}
\descr{Legt eine Datenstruktur Point an und initialisiert $x$ und $y$ mit den
ersten beiden Werten des \see{IVector} v.}

\ctor[explicit]{Point}{double d[]}{point.h}
\descr{Legt eine Datenstruktur Point an und initialisiert $x$ und $y$ mit den
ersten beiden Werten des Feldes d. }

Zugriffsmethoden

\method{double}{X}{}
\methodf{double}{Y}{}
\descr{Lesender Zugriff auf die Koordinatenwerte.}

Operatoren

\method{bool}{operator ==}{const Point \&p}
\methodf{bool}{operator !=}{const Point \&p}
\descr{Test zweier Punkte auf Gleichheit/Ungleichheit.}

\method{Point}{operator +}{const Point \&p}
\methodf{Point}{operator +=}{const Point \&p}
\descr{Komponentenweise Summe zweier Punkte.}

\method{Point}{operator -}{const Point \&p}
\methodf{Point}{operator -=}{const Point \&p}
\descr{Komponentenweise Differenz zweier Punkte.}

\method{Point}{operator *}{double f}
\methodf{Point}{operator *=}{double f}
\descr{Produkt mit Zahl $f$.}

\method{double}{operator *}{Point second}
\descr{Skalarprodukt zweier Punkte}

\method{Point}{operator /}{double d}
\methodf{Point}{operator /=}{double d}
\descr{Quotient mit Zahl $d$.}

Methoden

\method{int}{Shift}{double dx,double dy}
\descr{Verschiebt den Punkt um $(dx,dy)$.}

\method{double}{r}{}
\methodf{double}{phi}{}
\descr{Polarkoordinaten-Darstellung}

\method{Point}{normalized}{}
\descr{Liefert den normalisierten Vektor.}

\subsection{Walker Klassenhierarchie}
\hypertarget{Walker}{}

Walker verwenden eine ähnliche Idee wie die Iteratoren in der STL. 
Während Iteratoren das Fortschreiten über die Elemente eines 
Containers mittels der Operatoren ++ und -- organisieren, 
wird die Klasse \class{Walker} verwendet, um damit eine Sequenz 
von Punktkoordinaten zu erzeugen und damit Pfade beziehungsweise 
Flächen (im Bild) abzulaufen.

Die Klasse \class{Walker} ist von \class{IPoint} abgeleitet. Eine 
Instanz von \class{Walker} kann deshalb direkt anstelle 
eines \class{IPoint} verwendet werden, um zum Beispiel Koordinaten 
im Bild zu adressieren.

Die Klasse Walker verfügt über Methoden, um das ``Wandern durchs Bild''
zu organisieren. 
\see{Walker::init()} setzt den \class{Walker} auf den Ausgangspunkt zurück, 
\see{Walker::next()} setzt auf den nächsten Punkt.
Mittels \see{Walker::ready()} kann abgefragt werden, ob das Ende des 
Pfades erreicht ist.

Durch Rücksetzen mit \see{Walker::init()} kann ein Walker mehrfach 
verwendet werden. \see{Walker::moveTo()} verschiebt den 
Walker auf eine neue Position. Bezug ist dabei der Referenz-Punkt, 
dessen genaue Lage von der Art des Walkers abhängt. Zum Beispiel ist 
bei Konturen der Referenzpunkt der Startpunkt, bei 
Nachbarschafts-Walkern ist der zentrale Ausgangspunkt Referenzpunkt.

Die Klasse \class{Walker} ist eine abstrakte Basisklasse, von der 
sich eine Klassen-Hierarchie ableitet:

\begin{itemize}
\item \class{Walker} Abstrakte Basisklasse, abgeleitet von \class{IPoint}
\begin{itemize}
\item \class{WindowWalker} Walker über ein Fenster im Bild.
\item \class{PointListWalker} Walker über eine Punktliste, typisch genutzt für Pfade
\item \class{RegionWalker} Walker über eine Region, nutzbar für Flächen
\end{itemize}
\end{itemize}

Eine typische Anwendung eines Walkers zeigt das Beispiel, das eine
Region in ein Bild einzeichnet (analog \see{Region::draw()}):

\begin{verbatim}
Region region = ... ;
RegionWalker rw(region);
for (rw.init(); ! rw.ready(); rw.next())
    img.setPixel(rw, img.maxval);
\end{verbatim}

\subsubsection{Walker}
Die Klasse \class{Walker} ist eine abstrakte Basisklasse für alle in
ICE implementierten Walker. Sie ist ihrerseits abgeleitet von 
\class{IPoint}.

\ctor{Walker}{}{Walker.h}
\ctorf{Walker}{IPoint p}
\ctorf{Walker}{int x, int y}
\descr{Konstruiert einen Walker, dessen Referenzpunkt im Ursprung (0,0)
beziehungsweise im gegebenen Punkt liegt.}

\method{void}{moveTo}{IPoint p}
\descr{Verschiebt den Walker. 
Dabei wird der Referenzpunkt in den Punkt $p$ verschoben. 
Schließt den Aufruf von \see{Walker::init()} ein.}

\method{void}{init}{}
\descr{Setzt Walker auf Anfangspunkt.}

\method{void}{next}{}
\descr{Setzt Walker auf nächsten Punkt. Wird beim Weitersetzen 
das Ende des Pfades {\bf überschritten}, wird ready 
auf true gesetzt und der aktuelle Punkt gehört nicht mehr zum Pfad.}

\method{void}{next}{int steps}
\descr{Setzt Walker um $steps$ Schritte weiter. 
Wird beim Weitersetzen das Ende des Pfades {\bf überschritten}, 
wird ready auf true gesetzt und der aktuelle Punkt 
gehört nicht mehr zum Pfad.}

\method{bool}{ready}{}
\descr{Gibt zurück, ob der Pfad abgelaufen ist.}

\subsubsection{WindowWalker}
\hypertarget{WindowWalker}{}
Die Klasse \class{WindowWalker} ist ein \class{Walker}, der alle
Punkte eines Fensters (\see{Window}) durchläuft.

\ctor{WindowWalker}{const Window \&w}{Walker.h}
\descr{Legt einen Walker für das Fenster w an. Referenzpunkt des 
Walkers ist die linke obere Ecke des Fensters.}

\ctor{WindowWalker}{const Image \&img center}{Walker.h}
\descr{Legt einen Walker für das Bild img an. Referenzpunkt des 
Walkers ist der Punkt (0,0).}

\ctor{WindowWalker}{IPoint center, int sizex, int sizey = -1}{Walker.h}
\descr{Legt einen Walker für ein Fenster um den Punkt center an. Das Fenster
hat die Breite sizex und die Höhe sizey. Wird sizey weggelassen oder 
ist sizey gleich -1, ist das Fenster quadratisch mit der Größe sizex.
Der Referenzpunkt des Walkers ist das Zentrum des Fensters.}

\seebaseclass{Walker}

\subsubsection{PointListWalker}
\hypertarget{PointlistWalker}{}
Die Klasse \class{WindowWalker} ist ein \class{Walker}, der alle in
einer Punktliste enthaltenen Punkte durchläuft. 
%
%Die von dieser 
%Klasse abgeleiteten ähnlichen Walker, die sich vor allem in der 
%Initialisierung unterscheiden werden hier mitbehandelt.
\ctor{PointListWalker}{}{PointListWalker.h}
\descr{Standard-Konstruktor. Bevor dieser Walker nutzbar ist, muss er mit
\see{PointListWalker::setPointList} initialisiert werden.}

\ctor{PointListWalker}{IPoint p}{PointListWalker.h}
\descr{Konstruktor, der nur den Referenzpunkt setzt. 
Bevor dieser Walker nutzbar ist, muss er mit 
\see{PointListWalker::setPointList} initialisiert werden.}

\ctor{PointListWalker}{const \vector{IPoint} \&pl,int idx = 0}{PointListWalker.h}
\descr{Der Walker wird mit der gegebenen Punktliste initialisiert. Mittels
des Parameters idx kann ein Start-Index gewählt werden. Ist idx ungleich Null
arbeitet der Walker zyklisch vom Index idx bis zum Index idx-1.
Der Referenzpunkt des Walkers wird auf den Punkt pl[idx] gesetzt.}

\ctor{PointListWalker}{const Contur \&c}{PointListWalker.h}
\descr{Ein Walker wird mit den Konturpunkten der Contur c erzeugt.}

\method{void}{setPointList}{const \vector{IPoint} \&v,int idx = 0}
\descr{Der Walker wird mit der gegebenen Punktliste initialisiert. Mittels
des Parameters idx kann ein Start-Index gewählt werden. Ist idx ungleich Null
arbeitet der Walker zyklisch vom Index idx bis zum Index idx-1.
Der Referenzpunkt des Walkers wird auf den Punkt pl[idx] gesetzt.
Der Aufruf dieser Methode impliziert eine Neuinitialisierung.}

\method{void}{setStartIndex}{int idx = 0}
\descr{Setzt den als Start zu verwendenden Index. Ist idx ungleich Null
arbeitet der Walker zyklisch vom Index idx bis zum Index idx-1.
 Der Aufruf dieser Methode impliziert eine Neuinitialisierung.}

\seebaseclass{Walker}

\subsubsection{Nachbarschafts-Walker}
\hypertarget{Neighbor8Walker}{}
\hypertarget{Neighbor4Walker}{}

Die Klassen \class{Neighbor4Walker} und \class{Neighbor8Walker} 
erzeugen Walker, die die Vierer- bzw Achter-Nachbarschaft eines 
gegebenen Punktes durchlaufen.

\ctor{Neighbor4Walker}{IPoint p}{PointListWalker.h}
\descr{Erzeugt einen Walker, der die Punkte in der Vierer-Nachbarschaft
des Punktes p durchläuft. Referenzpunkt ist der Punkt p.}

\ctor{Neighbor8Walker}{IPoint p}{PointListWalker.h}
\descr{Erzeugt einen Walker, der die Punkte in der Achter-Nachbarschaft
des Punktes p durchläuft. Referenzpunkt ist der Punkt p.}

\seebaseclass{Walker}\\
\seebaseclass{PointListWalker}

\subsubsection{RegionWalker}
\hypertarget{RegionWalker}{}

Ein \class{RegionWalker} durchläuft eine Region, die zum Beispiel durch
eine Instanz der Klasse \class{Region} oder eine Kontur (\class{Contur})
bestimmt wird. 

\ctor{RegionWalker}{const Region \&region}{RegionWalker.h}
\descr{Legt einen RegionWalker für die Region region an. Dieser 
RegionWalker durchläuft alle Punkte der Region.}

\ctor{RegionWalker}{const Contur \&contur}{RegionWalker.h}
\descr{Legt einen RegionWalker an, der das durch die Punkte des durch die
\class{Contur} bestimmten Objektes durchläuft.}

\seebaseclass{Walker}

\subsection{Point3d, Vector3d, IPoint3d und IVector3d}
\hypertarget{Point3d}{}
\hypertarget{Vector3d}{}
\hypertarget{IPoint3d}{}
\hypertarget{IVector3d}{}
Die Klassen \class{Point3d},\class{Vector3d}, \class{IPoint3d} 
und \class{IVector3d} dienen der Repräsentation von Raumpunkten. 
Die Begriffe {\bf Vector} und {\bf Point} sind hier gleichwertig 
und können synonym verwendet werden, im folgenden verwenden wir nur 
\class{Vector3d}. 
\class{IPoint3d} und \class{IVector3d} repräsentieren Raumpunkte durch 
ganzzahlige Koordinaten,  \class{Point3d} und \class{Vector3d} 
dagegen Koordinaten vom Typ {\bf double}. Die folgende Beschreibung 
verwendet \class{Vector3d} als Typangabe, \class{IVector3d} verhält 
sich entsprechend.

Für einen sehr schnellen Zugriff sind die Instanz-Variablen 
$x$, $y$ und $z$ von außen zugreifbar. 

Operatoren und Konstruktoren ermöglichen eine problemlose Umwandlung 
von \class{Vector3d} in \class{Vector} und umgekehrt.

Konstruktoren:

\proch{}{Vector3d::Vector3d}{}{point3d.h}
\descr{Legt einen Punkt im Ursprung an.}

\proc{}{Vector3d::Vector3d}{const Vector3d \&p}
\procf{}{Vector3d::Vector3d}{double xp,double yp,double zp}
\descr{Legt einen Punkt mit der gegebenen Position an.}

\proc{explicit}{Vector3d::Vector3d}{const Vector \&v}
\procf{explicit}{Vector3d::Vector3d}{const IVector \&v}
\procf{explicit}{Vector3d::Vector3d}{double p[]}
\descr{Legt einen Punkt mit der gegebenen Position an. Zur Vermeidung 
ungewollter Konvertierungen muss der Aufruf explizit erfolgen.}

Operatoren:
\begin{center}
\begin{tabular}{ccc@{=}c}
%%Operand & Operator &  Operand & Resultat \\
Vector3d & + & Vector3d & Vector3d\\
Vector3d & - & Vector3d & Vector3d\\
 & - & Vector3d & Vector3d\\
Vector3d & * & Vector3d & double \\
double & * & Vector3d & Vector3d \\
Vector3d & * & double & Vector3d \\
\end{tabular}
\end{center}

Methoden:

\proc{Vector}{Vector3d::operator Vector}{}
\descr{Typwandlung in $Vector$.}

\proc{Vector3d}{Vector3d::Normalized}{}
\descr{Gibt einen normalisierten Vektor (der Länge 1) zurück.}

\proc{void}{Vector3d::Normalize}{}
\descr{Normalisiert den Vektor.}

\subsection{Window}
\hypertarget{Window}{}

Die Klasse \class{Window} beschreibt ein rechteckiges achsparalleles 
Fenster anhand der linken oberen und der rechten unteren Ecke bzw. 
der minimalen und maximalen Koordinaten-Werte in X- und Y-Richtung.

Konstruktoren

\proch{}{Window::Window}{}{window.h}
\descr{Standard-Konstruktor. Die Koordinaten werden mit 0 initialisiert.}

\proch{}{Window::Window}{int x1,int y1,int x2,int y2}{window.h}
\descr{Die linke obere Ecke wird mit $(x1,y1)$, die rechte untere Ecke mit
  $(x2,y2)$ initialisiert.}

Zugriffs-Methoden

\proch{int}{Window::Width}{}{window.h}
\procf{int}{Window::Height}{}
\descr{Die Methoden geben die Breite bzw. die Höhe des Fensters zurück.}

\proch{int}{Window::GetLeft}{}{window.h}
\procf{int}{Window::XI}{}
\procf{int}{Window::GetRight}{}
\procf{int}{Window::XA}{}
\procf{int}{Window::GetTop}{}
\procf{int}{Window::Yi}{}
\procf{int}{Window::GetBottom}{}
\procf{int}{Window::YA}{}
\descr{Zugriff auf die X-Koordinatenwerte des linken oder rechten Randes
  bzw. auf die Y-Koordinatenwerte des oberen oder unteren Randes.}

\proch{const IPoint \&}{Window::P1}{}{window.h}
\procf{const IPoint \&}{Window::P2}{}
\descr{Zugriff auf die linke obere Ecke bzw. die rechte untere Ecke als
  \see{IPoint}.}

\proch{IPoint}{Window::getSize}{}{window.h}
\descr{Zugriff auf die Größe des Fensters als Datenstruktur \class{IPoint}.}

Methoden

\proch{bool}{Window::Inside}{const IPoint \&p}{window.h}
\procf{bool}{Window::Inside}{int x,int y}
\descr{Testet, ob der Punkt $p$ bzw. $(x,y)$ im Fenster liegt}

\proch{bool}{Window::operator ==}{const Window \&w}{window.h}
\procf{bool}{Window::operator !=}{const Window \&w}
\descr{Vergleich zweier Fenster auf Gleichheit bzw. Ungleichheit.}

\proch{int}{Window::Shift}{int dx,int dy}{window.h}
\descr{Verschiebt das Fenster um $(dx,dy)$.}

\subsection{Vektoren}
\hypertarget{Vector}{}
\hypertarget{IVector}{}

Der Datentyp (class) \class{Vector} repräsentiert ein Feld 
von $double$-Werten. 
Die Dimension $n$ kann ein beliebiger Wert größer Null sein. Die Index-
Numerierung der Elemente läuft von $0$ bis $n-1$. 
Gegenüber der STL-Klasse \verb+vector<double>+ sind Vektoren der Klasse
\class{Vector} um viele mathematische Operationen erweitert.

Der Datentyp (class) \class{IVector} repräsentiert ein Feld von 
\verb+int+-Werten.
Die Beschreibung des \class{Vector} trifft sinngemäß weitgehend auch für die 
Klasse \class{IVector} zu, wenn der Typ $double$ durch $int$ ersetzt wird.\\
Die Klasse \class{IVector} kann auch als Wert überall dort eingesetzt werden, 
wo auch ein \class{Vector} stehen könnte, da die Klasse \class{Vector} über 
einen entsprechenden Tywandlungskonstruktor verfügt.
\\
Eine spezielle Implementierung dreidimensionaler Vektoren \class{Vector3d} 
kann für eine effiziente Rechnung mit dreidimensionalen Vektoren verwendet 
werden. Eine Konvertierung von und nach der Klasse \class{Vector} kann 
erfolgen:
\begin{itemize}
\item Konstruktor der Klasse \class{Vector3d} mit Parameter vom 
Typ \class{Vector}  (Typwandlungs-Konstruktor)
\item $operator$ \class{Vector} der Klasse \class{Vector3d}
\end{itemize}
Diese ermöglichen eine Verwendung von \class{Vector3d} als 
\class{Vector} ohne explizite Konvertierung.
\\
\subsubsection{Konstruktoren}
\proc{}{Vector::Vector}{}
\descr{Ein Vektor der Länge 0 wird angelegt.}

\proc{}{Vector::Vector}{int n}
\descr{Ein Vektor der Dimension $n$ wird angelegt.}

\proc{}{Vector::Vector}{int n,double *data}
\descr{Ein Vektor der Dimension $n$ wird angelegt und mit den Werten des
  C-array $data$ belegt.}

\proc{}{Vector::Vector}{const Vector \&v}
\descr{Ein Vektor wird als Kopie des Vektor $v$ angelegt.}

\proc{explicit}{Vector::Vector}{const vector<double> \&v}
\descr{Der vector<double> $v$ wird in einen Vektor konvertiert.}

\proc{}{Vector::Vector}{double d1,double d2}
\proc{}{Vector::Vector}{double d1,double d2,double d3}
\proc{}{Vector::Vector}{double d1,double d2,double d3,double d4}
\proc{}{Vector::Vector}{double d1,double d2,double d3,double d4,double d5}
\descr{
Ein Vektor der entsprechenden Länge wird angelegt und die Elemente werden mit
$d1$, $d2$ ... initialisiert.
}

\subsubsection{Operatoren}

\begin{tabular}{|c|l|}
\hline
Operator & Funktion \\
\hline
Vector = Vector & Zuweisungs-Operator \\
Vector + Vector & Vektor-Addition \\
Vector - Vector & Vektor-Subtraktion \\
- Vector        & Negation des Vektors \\
Vector * Vector & Skalar-Produkt \\
Vector * double & Multiplikation mit Skalar \\
double * Vector & Multiplikation mit Skalar \\
Vector [ int ]  & Zugriff auf ein beliebiges Element \\
Vector (int,int)& Teil-Vektor (Vektor-Wert) \\
\hline
\end{tabular}

\proch{double \&}{Vector::at}{int idx}{VectorO.h}
\descr{Zugriff auf das Element mit dem Index $idx$ des Vektors analog zum 
Operator[], aber ohne Bereichstestung.}

\subsubsection{Element-Funktionen}

\proch{int}{Vector::Size}{}{VectorO.h}
\procf{unsigned int}{Vector::size}{}
\descr{Gibt die Dimension des Vektors zurück.}

\proch{bool}{Vector::empty}{}{VectorO.h}
\descr{Gibt $true$ zurück, wenn der Vektor leer ist, also 0 Elemente hat.}

\proch{void}{Clear}{}{VectorO.h}
\descr{Löscht den Vektor. Der Vektor hat danach 0 Elemente. }

\proch{void}{Resize}{int n}{VectorO.h}
\descr{Reinitialisiert den Vektor und legt $n$ Elemente an. Der vorhandene
  Inhalt wird zerstört!}

\proch{void}{Vector::Set}{double val}{VectorO.h}
\descr{Setzt alle Elemente auf den Wert $val$.}

\proch{void}{Vector::SetV}{double d1}{VectorO.h}
\procf{void}{Vector::SetV}{double d1,double d2}
\procf{void}{Vector::SetV}{double d1,double d2,double d3}
\procf{void}{Vector::SetV}{double d1,double d2,double d3,double d4}
\procf{void}{Vector::SetV}{double d1,double d2,double d3,double d4,double d5}
\descr{Setzt die ersten Elemente des Vektors auf die übergebenen Werte.}

\proch{void}{Vector::Append}{double val}{VectorO.h}
\proch{void}{Vector::Append}{const Vector \&v}{VectorO.h}
\descr{Verlängert den Vektor indem der Wert $val$ beziehungsweise die Werte
  des Vektors $v$ angehängt werden.}

\proch{void}{Vector::Exchange}{int i1,int i2}{VectorO.h}
\descr{Tauscht die Elemente mit dem Index $i1$ und $i2$ aus.}

\proch{void}{Vector::Delete}{int i1,int i2}{VectorO.h}
\procf{void}{Vector::Delete}{int i}
\descr{Löscht die Elemente mit von Index $i1$ bis $i2$ oder das
Element mit Index $i$.}

\proch{double}{Vector::Length}{}{VectorO.h}
\descr{Gibt die Länge (den Betrag) des Vektors zurück}

\proch{void}{Vector::Normalize}{}{VectorO.h}
\descr{Normiert den Vektor auf eine Länge (Betrag) von Eins.\\
\seealso{Normalize}
}

\proc{void}{Vector::Sort}{int order=0}
\descr{
Sortiert die Elemente des Vektors in aufsteigender (order=0) oder absteigender
(order=1) Ordnung.
}

\subsubsection{Funktionen mit Vektoren}
\label{Vector-Funktionen}

\proc{Vector}{Normalize}{const Vector \&v}
\descr{
Gibt den normalisierten Vector $v$ zurück.\\
\seealso{Vector::Normalize}
}

\proc{Vector}{Cross}{const Vector \&v1,const Vector \&v2}
\descr{
Gibt das Kreuzprodukt der Vektoren $v1$ und $v2$ zurück.
}

\subsection{Matrizen}
\hypertarget{Matrix}{}

Der Datentyp \class{Matrix} repräsentiert ein zwei-dimensionales Feld von 
{\bf double}-Werten bzw. ein Feld von \hyperlink{Vector}{Vektoren}.
Die Dimensionen $m$ und $n$ können beliebige Werte größer gleich Null sein. 
Die Index-Numerierung der Elemente läuft von $0$ bis $n-1$ bzw. von $0$ bis
$m-1$.

\hypertarget{IMatrix}{}
Der Datentyp \class{IMatrix} verwendet statt des Datentypes $double$ für die
Elemente den Typ Integer ($int$). Durch einen Typwandlungskonstruktor kann
eine $Matrix$ aus einer $IMatrix$ erzeugt werden, was an vielen Stellen den
Einsatz von $IMatrix$ äquivalent zu $Matrix$ erlaubt.

Diese Klassendefinitionen ersetzen den alten Datentyp \see{MatrixStruct}.
Zur {\bf Anwendung} von Matrizen siehe auch Abschnitt \bsee{Matrix-Algebra}.

\subsubsection{Konstruktoren}
\proc{}{Matrix::Matrix}{}
\descr{Legt eine Matrix der Groesse (0,0) an. Diese Matrix kann so 
nicht verwendet werden.}

\proc{}{Matrix::Matrix}{const int rows,const int cols,int initmode=0}
\descr{
Legt eine Matrix mit $rows$ Zeilen und $cols$ Spalten an. Alle Werte sind mit
Null belegt. Wird $initmode=1$ gesetzt, so wird die Matrix als Einheitmatrix
initialisiert.
}

\proc{explicit}{Matrix::Matrix}{const int rows,const int cols,double *init}
\descr{
Legt eine Matrix mit $rows$ Zeilen und $cols$ Spalten an. Die Werte werden mit
den Werten im Feld $init$ belegt. Die rows*cols Werte müssen zeilenweise
abgelegt sein.
}

\proc{}{Matrix::Matrix}{const Matrix\& m}
\descr{Legt eine neue Matrix als Kopie der gegebenen Matrix an 
(Kopierkonstruktor).}

\proc{}{Matrix::Matrix}{const IMatrix\& im}
\descr{
Legt eine neue Matrix mit den konvertierten Werten der gegebenen 
Integer-Matrix an.
}

\proc{explicit}{Matrix::Matrix}{const Image\& m,int mode=RAW,int sign=UNSIGNED}
\descr{
Legt eine neue Matrix mit der Größe des Bildes $m$ an und trägt die Werte 
in die Matrix ein. Es erfolgt eine Transformation der
Werte entsprechend des durch $mode$ und $sign$ bestimmten Modus äquivalent 
zu \see{ConvImgImgD}.
}

\proc{explicit}{Matrix::Matrix}{const ImageD \& m}
\descr{
Legt eine neue Matrix mit der Größe des Gleitkomma-Bildes $m$ 
an und trägt die Werte in die Matrix ein.
}

\subsubsection{Operatoren}
\begin{tabular}{|c|l|}
\hline
Operator & Funktion \\
\hline
Matrix = Matrix & Zuweisungs-Operator \\
Matrix + Matrix & Matrix-Addition (elementweise)\\
Matrix - Matrix & Matrix-Subtraktion (elementweise)\\
- Matrix & Negierte Matrix\\
Matrix * Matrix & Matrix-Multiplikation \\
Matrix * double & Multiplikation aller Elemente mit einer Konstanten \\
double * Matrix & Multiplikation aller Elemente mit einer Konstanten \\
Matrix [int] & Zugriff auf Zeile einer Matrix (-> Vector) \\
Matrix [int i][int j] & Zugriff auf Element der Matrix\\
Matrix (int i1,int j1,int i2,int j2) & Teil-Matrix \\
Matrix $||$ Matrix & Nebeneinander Anordnen und Verbinden \\
Matrix \&\& Matrix & Untereinander Anordnen und Verbinden \\
! Matrix & Transponierte Matrix \\
\hline
\end{tabular} \\
Bei den Operatoren $||$ und \&\& können sinngemäß statt Matrizen auch 
Vektoren angewendet werden. Diese werden als Spalten-Vektor ($||$) bzw.
als Zeilen-Vektor (\&\&) interpretiert.\\
\\
Bei der Multiplikation ist anwendbar:
\begin{tabular}{l}
Vector * Matrix = Vector\\
Matrix * Vector = Vector\\
\end{tabular}

Folgende Methoden unterstützen die Multiplikation mit der transponierten
Matrix:
\proch{Matrix}{Matrix::MulTrans}{const Matrix \&m2}{MatrixO.h}
\descr{Gibt das Produkt der transponierten Matrix mit der Matrix $m2$ zurück.}
\proch{Vector}{Matrix::MulTrans}{const Vector \&v}{MatrixO.h}
\descr{Gibt das Produkt der transponierten Matrix mit dem Vektor $v$ zurück.}

\subsubsection{Element-Funktionen}

\proc{int}{Matrix::rows}{}
\descr{Gibt Zeilenzahl der Matrix zurück.}

\proc{int}{Matrix::cols}{}
\descr{Gibt Spaltenzahl der Matrix zurück.}

\proc{int}{Append}{const Vector \&v}
\descr{Hängt den Inhalt des Vektors $v$ als neue Zeile an die bestehende
  Matrix an. Die Größe des Vektors muss gleich der Zahl der Spalten der Matrix
  sein.}

\proc{void}{Matrix::ExchangeRow}{int i1,int i2}
\descr{Austausch der Zeilen $i1$ und $i2$.}

\proc{void}{Matrix::ExchangeCol}{int i1,int i2}
\descr{Austausch der Spalten $i1$ und $i2$.}

\proc{int}{Matrix::DeleteRow}{int n}
\procf{int}{Matrix::DeleteRow}{int n1,int n2}
\descr{
Löscht die Zeile $n$ bzw. die Zeilen $n1$ bis $n2$ aus der
Matrix.
}

\proc{int}{Matrix::DeleteCol}{int n}
\procf{int}{Matrix::DeleteCol}{int n1,int n2}
\descr{
Löscht die Spalte $n$ bzw. die Spalten $n1$ bis $n2$ aus der
Matrix.
}

\proc{double}{Matrix::MaxVal}{}
\descr{Liefert den Wert des größten Elements der Matrix.}

\proc{void}{Matrix::SumRows}{Vector \&sum}
\proc{void}{Matrix::SumCols}{Vector \&sum}
\descr{Trägt die Werte der Zeilensummen/Spaltensummen in den Vektor $sum$ ein.}

\proc{int}{Matrix::AddDyadicProd}{const Vector \& v,const Vector \& w,double factor}
\proc{int}{Matrix::AddDyadicProd}{const Vector \& v, double factor}
\descr{Bildet das dyadische Produkt der Vektoren $v$ und $w$ beziehungsweise
  von $v$ mit sich selbst, multipliziert dieses mit $factor$ und addiert
  dieses zur Matrix.}

\proc{int}{Matrix::Sort}{int col=0,bool asc=true}
\descr{
Sortiert die Zeilen der Matrix aufsteigend ($asc=true$) oder absteigend
($asc=false$) nach dem Wert der Spalte $col$.
}

\subsection{Punktlisten und Polygone}
\label{Punktliste}
Listen von 2D-Punkten können als \bsee{Matrix} gespeichert werden, 
viele Funktionen benutzen aber auch \verb+vector<Point>+ als Punktliste.\\
In Matrizen wird jeder Punkt als eine Zeile der Matrix abgelegt.
Die Matrix hat mindestens 2 Spalten, die die Koordinaten x und y der Ebene
enthalten. Ist eine dritte Spalte vorhanden, wird diese in einigen Funktionen
als Gewicht interpretiert. Zusätzlich vorhandene Spalten werden ignoriert und
können zusätzlich punktbezogene Daten enthalten.

In {\bf einigen Funktionen} werden solche Punktlisten als Polygon 
interpretiert (siehe aber auch \bsee{Polygon}).
Dabei werden aufeinanderfolgende Punkte und 
der Anfangs- und Endpunkt der Punktliste logisch durch Kanten verbunden. 
Der Anfangspunkt darf nicht als Endpunkt wiederholt werden, 
sonst eine zusätzliche Kante der Länge Null eingeführt wird. 
Die Kanten eines Polygons dürfen sich nicht 
gegenseitig schneiden (Test durch IsPolygon()).

\proc{int}{isPolygon}{const Matrix \& pl}
\descr{Es wird geprüft, dass sich die Kanten eines Polygons nicht 
gegenseitig schneiden.}

\proch{bool}{InsidePolygon}{const Matrix \&pl,const Point \&p}{polygon.h}
\descr{Es wird untersucht, ob der Punkt $p$ innerhalb des Polygons $pl$ liegt.}

\proch{Matrix}{ReducePolygon}{const Matrix \&pl,int n}{polygon.h}
\procf{Matrix}{ReducePolygon}{const Contur \&c,int n}
\descr{Erzeugt ein reduziertes Polygon mit $n$ Ecken aus dem Polygon $pl$ und
gibt dieses zurück.
Dies geschieht durch suksessives Weglassen ``unwichtiger'' Ecken. Die
Signifikanz einer Ecke ergibt sich wie folgt: Seien $e_k$ und $e_{k+1}$ die
Vektoren der beiden anliegenden Kanten. Die Signifikanz ist dann
\[
s_k=\frac{|e_k x e_{k+1}|}{|e_k| + |e_{k+1}|}
\]
Beim alternativen Aufruf wird die Contur $c$ als Polygon interpretiert. 
}

\proch{Matrix}{FitPolygonContur}{const Matrix \&pl, const Contur \&c,int step=0}{polygon.h}
\descr{Die Matrix $pl$ muss ein Polygon enthalten, welches 
  näherungsweise die Contur $c$ beschreibt. Diese kann beispielsweise durch
  \see{ReducePolygon} entstanden sein. Diese Funktion passt iterativ das
  Polygon besser an die Kontur an. Der Parameter $step$ entspricht dem
  Parameter $step$ von \see{FitLine}, welches intern verwendet
  wird. Ein Speziallfall ist $n=-1$, dann wird Fitting mit linearer
  Optimierung verwendet. Dies ist im allgemeinen robuster gegen Ausreißer,
  dauert aber länger.}

\proch{vector$<$Point$>$}{ConvexHull}{const vector$<$Point$>$ \&pl}{convexhull.h}
\descr{
Zu der übergebenen Punktliste $pl$ wird die konvexe Hülle erzeugt 
und als vector$<$Point$>$ zurückgegeben. Die Punkte der zurückgegeben Liste 
sind geordnet, so das die Punktfolge ein Polygon definiert. Ein \bsee{Polygon} kann 
direkt mit Hilfe des Konstruktors erzeugt werden.
}

\proch{Matrix}{ConvexHull}{const Matrix \&pl}{convex.h}
\procf{Contur}{ConvexHull}{const Contur \&c}
\descr{
Erzeugt die konvexe Hülle der als Matrix übergebenen Punktmenge bzw. der übergebenen Kontur.
}

\seealso{ConturPointlist}
\seealso{PointlistContur}

\subsection{ColorValue}
\hypertarget{ColorValue}{}
Die Klasse \class{ColorValue} dient der Speicherung von Farbwerten in Form
von ganzzahligen Intensitätswerten für die Grundfarben Rot, Grün und 
Blau. Werte vom Typ \class{ColorValue} werden zum Setzen und Lesen von 
Werten in Farbbildern der Klasse \class{ColorImage} verwendet. Für 
einen schnellen Zugriff 
sind die int-Variablen \verb+red+, \verb+green+ und \verb+blue+ der 
Klasse \verb+public+. Die Klasse \class{ColorValue} ist nicht 
als Basisklasse einer Klassenhierarchie konzipiert und es sollten keine
Klassen davon abgeleitet werden: Die Methoden sind alle nicht virtuell.

\subsubsection{Konstruktoren}

\proch{}{ColorValue::ColorValue}{}{ColorValue.h}
\descr{Standard-Konstruktor. Die Farbwerte $red$,
$green$ und $blue$ werden mit 0 intialisiert.}

\proch{explicit}{ColorValue::ColorValue}{int val}{ColorValue.h}
\descr{Typwandlungs-Konstruktor. Alle Farbwerte $red$,
$green$ und $blue$ werden mit $val$ intialisiert.}

\proch{}{ColorValue::ColorValue}{int redval,int greenval,int blueval}{ColorValue.h}
\descr{Konstruktor, der die Farbwerte $red$,
$green$ und $blue$ werden mit den gegebenen Werten intialisiert.}

\subsubsection{Operatoren}

\begin{center}
\begin{tabular}{ccc@{=}c}
%%Operand & Operator &  Operand & Resultat \\
ColorVal & + & ColorVal & ColorVal\\
ColorVal\& & += & ColorVal & const ColorVal\&\\
ColorVal & - & ColorVal & ColorVal\\
ColorVal\& & -= & ColorVal & const ColorVal\&\\
ColorVal & * & int & ColorVal\\
ColorVal\& & *= & int & const ColorVal\&\\
ColorVal & / & int & ColorVal\\
ColorVal\& & /= & int & const ColorVal\&\\
ColorVal & * & double & ColorVal\\
ColorVal\& & *= & double & const ColorVal\&\\
ColorVal & / & double & ColorVal\\
ColorVal\& & /= & double & const ColorVal\&\\
\end{tabular}
\end{center}

\subsubsection{Methoden}
\proch{double}{ColorValue::abs}{}{ColorValue.h}
\descr{Berechnet den Betrag (Länge) des durch die Werte
 $red$, $green$ und $blue$ gebildeten Vektors.}

\proch{double}{ColorValue::abs2}{}{ColorValue.h}
\descr{Berechnet das Quadrat des Betrages (Länge) des durch die Werte
 $red$, $green$ und $blue$ gebildeten Vektors.}

\proch{int}{ColorValue::absL1}{}{ColorValue.h}
\descr{Summe der Beträge der Werte $red$, $green$ und $blue$.}

\proch{int}{ColorValue::getGray}{}{ColorValue.h}
\descr{Berechnet einen Grauwert als Mittelwert von 
$red$, $green$ und $blue$.}

\proch{ColorValue}{ColorValue::Limited}{int maxValue}{ColorValue.h}
\descr{Ermittelt einen Farbwert, dessen Werte $red$, 
$green$ und $blue$ im Interval $[0,maxValue]$ liegen.
Werte ausserhalb des Intervalls werden auf den jeweiligen 
Extremwert gesetzt. Dies kann benutzt werden, um beim Schreiben 
in ein Farbbild mittels \bsee{ColorImage::setPixel} Überläufe zu
vermeiden.}

\subsection{Farbraum-Transformationen}
Werte der Klasse \class{ColorValue} repräsentieren Farbwerte im
RGB-Farbraum. Die folgenden Funktionen konvertieren diese RGB-Werte
in andere Farbräume und zurück. Dabei werden Farbwerte in anderen 
Farbräumen durch drei Gleitkomma-Werte vom Typ \verb+double+ beschrieben.
Die Werte liegen im Intervall 0..1 bzw -0.5..0.5, nur bei Werten des
Lab-Farbraumes werden die dort üblichen Bereich 0..100, -150..100 und
-100..150 für L, a und b verwendet. Weiterhin gibt es Funktionen zur
Farbraum-Transformation von Bildern (\see{ColorImageToHSI}..) .
 
%Zu den Farbräumen siehe zum Beispiel 
%\href{http://www.brucelindbloom.com/index.html?Math.html}{hier}.

\proch{void}{RgbToHsi}{const ColorValue \&src, int maxval, double \&h, double \&s, double \&i}{ColorSpace.h}
\procf{void}{RgbToHsi}{const ColorValue \&src, double \&h, double \&s, double \&i}
\procf{void}{RgbToYuv}{const ColorValue \&src, int maxval, double \&h, double \&s, double \&i}
\procf{void}{RgbToYuv}{const ColorValue \&src, double \&h, double \&s, double \&i}
\procf{void}{RgbToXyz}{const ColorValue \&src, int maxval, double \&x, double \&y, double \&z}
\procf{void}{RgbToXyz}{const ColorValue \&src, double \&x, double \&y, double \&z}
\procf{void}{RgbToLab}{const ColorValue \&src, int maxval, double \&x, double \&y, double \&z}
\procf{void}{RgbToLab}{const ColorValue \&src, double \&x, double \&y, double \&z}
\descr{Transformation von Werten von \class{ColorValue} aus dem RGB-Farbraum in 
die Farbräume HSI, YUV, XYZ und Lab. Wird der Maximalwert maxval angegeben, 
so werden die RGB-Wert in \class{ColorValue} als Werte im Bereich 0..maxval 
angenommen, sonst 0..255.}

\proch{void}{HsiToRgb}{double h, double s, double i, ColorValue \&dst, int maxval = 255}{ColorSpace.h}
\procf{void}{YuvToRgb}{double y, double u, double v, ColorValue \&dst, int maxval = 255}
\procf{void}{XyzToRgb}{double x, double y, double z, ColorValue \&dst, int maxval = 255}
\procf{void}{LabToRgb}{double l, double a, double b, ColorValue \&dst, int maxval = 255}
\descr{Transformieren Farbwerte aus den Farbräumen HSI,YUV, XYZ und LAB in den
RGB-Farbraum. Die RGB-Werte werden in Variablen der Klasse \class{ColorValue} 
gespeichert. Der Umfang der RGB-Werte beträgt 0..maxval.}

\proch{void}{HsiToRgb}{double h, double s, double i, double \&r, double \&g, double \&b}{ColorSpace.h}
\procf{void}{RgbToHsi}{double r, double g, double b, double \&h, double \&s, double \&i}
\procf{void}{YuvToRgb}{double y, double u, double v, double \&r, double \&g, double \&b}
\procf{void}{RgbToYuv}{double r, double g, double b, double \&h, double \&s, double \&i}
\procf{void}{RgbToXyz}{double r, double g, double b, double \&x, double \&y, double \&z}
\procf{void}{XyzToRgb}{double x, double y, double z, double \&r, double \&g, double \&b}
\procf{void}{XyzToLab}{double x, double y, double z, double \&l, double \&a, double \&b}
\procf{void}{LabToXyz}{double l, double a, double b, double \&x, double \&y, double \&z}
\procf{void}{RgbToLab}{double r, double g, double b, double \&x, double \&y, double \&z}
\procf{void}{LabToRgb}{double l, double a, double b, double \&rr, double \&gg, double \&bb}
\descr{Allgemeine Farbraumtransformationen.}
