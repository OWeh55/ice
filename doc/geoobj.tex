\nsection{Klassen-Hierarchie Geometrisches Objekt}{geoObject}

Alle Klassen, die der Beschreibung von geometrischen Objekten der Ebene 
dienen, leiten sich von der abstrakten Basis-Klasse \class{GeoObject} ab.
In dieser Klasse wird das grundsätzliche Verhalten der Methoden 
beschrieben, die in den abgeleiteten Klassen reimplementiert werden. 

\vspace{1cm}
\noindent
Klassen-Hierarchie:

\begin{description}
\item[\see{GeoObject}] Abstrakte Basisklasse
\begin{description}
\item[\see{Circle}] Kreis
\begin{description}
\item[\see{CircleSeg}] Kreissegment
\item[\see{Ellipse}] Ellipse
\begin{description}
\item[\see{EllipseSeg}] Ellipsensegment
\end{description}
\end{description}
\item[\see{LineSeg}] Strecke, Strahl oder Gerade
\item[\see{Triangle}] Dreieck
\item[\see{PolygonalCurve}] PolygonalCurve
\begin{description}
\item[\see{Polygon}] Polygon
\end{description}
\end{description}
\end{description}

\subsection{GeoObject}
\hypertarget{GeoObject}{}
\label{GeoObject}

GeoObject ist eine abstrakte Basisklasse für 
\hyperlink{geometricObject}{geometrische Objekte} der Ebene.

\proc{}{GeoObject::GeoObject}{}
\descr{Legt ein Objekt der Klasse GeoObject an.}
\proc{}{GeoObject::GeoObject}{const GeoObject \&c}
\descr{Legt ein Objekt der Klasse GeoObject an und initialisiert 
die Parameter mit den Werten des übergebenen Objektes (Kopierkonstruktor).}
\proc{}{GeoObject::GeoObject}{double xm,double ym}
\procf{explicit}{GeoObject::GeoObject}{Point m}
\procf{explicit}{GeoObject::GeoObject}{IPoint m}
\procf{explicit}{GeoObject::GeoObject}{const Vector \&v}
\descr{Legt ein Objekt GeoObject an und initialisiert die Koordinaten mit den
übergebenen Werten.}

\proch{Point}{GeoObject::getPos}{}{geo\_ob.h}
\descr{Liefert die Koordinaten (des Referenzpunktes) als \bsee{Point}.}

\proch{void}{GeoObject::setPos}{Point x}{geo\_ob.h}
\descr{Setzt die Koordinaten (des Referenzpunktes) des geometrischen Objekts. 
Dies stellt eine Verschiebung des ganzen Objektes dar.}

\proch{void}{GeoObject::Shift}{double dx,double dy}{geo\_ob.h}
\procf{void}{GeoObject::Shift}{Point p}
\descr{Verschieben des geometrischen Objekts um $dx$ in X-Richtung und 
$dy$ in Y-Richtung.}

\proch{double}{GeoObject::Distance}{double x,double y}{geo\_ob.h}
\procf{double}{GeoObject::Distance}{\bsee{IPoint} p}
\procf{double}{GeoObject::Distance}{\bsee{Point} p}
\descr{Ermittelt den Abstand zwischen dem gegebenen Punkt $p$ 
und der Randlinie des Objektes.
}
\proch{double}{GeoObject::Distance}{const GeoObject \&obj}{geo\_ob.h}
\descr{Ermittelt den Abstand des Referenzpunktes des
geometrischen Objektes $obj$ von der Randlinie des Objektes.}

\proch{bool}{GeoObject::Inside}{double xp,double yp}{geo\_ob.h}
\procf{bool}{GeoObject::Inside}{IPoint p}
\procf{bool}{GeoObject::Inside}{Point p}
\descr{Ermittelt, ob der Punkt (x,y) bzw. $p$ innerhalb des Objektes liegt. 
Die Methode der (abstrakten) Basisklasse $GeoObject$ liefert hier immer 
false zurück. Dies gilt auch für geometrische Objekte ohne Fläche wie 
z.B. \bsee{LineSeg}.
}

\subsection{Circle}
\label{Circle}
\hypertarget{Circle}{}

\begin{figure}[h]
  \epsfig{figure=circle.eps,width=5cm}
  \label{fig:circle}
\end{figure}

Die Klasse Circle ist direkt von der Basis-Klasse \class{GeoObject} abgeleitet
und repräsentiert Kreise in der Ebene.

\noindent
Konstruktoren:

\proc{}{Circle::Circle}{}
\descr{Legt ein Objekt Circle an.}
\proc{}{Circle::Circle}{const Circle \&c}
\descr{Legt ein Objekt Circle an und initialisiert die Parameter mit den
Werten des übergebenen Kreises (Kopierkonstruktor).}
\proc{}{Circle::Circle}{double xm,double ym,double r}
\procf{}{Circle::Circle}{Point m,double r}
\descr{Legt ein Objekt Circle an und initialisiert die Parameter mit den
übergebenen Werten für Mittelpunkt und Radius.}

\proc{explicit}{Circle::Circle}{const Vector \&c}
\descr{Legt ein Objekt Circle an und initialisiert die Parameter
des Mittelpunktes (xm,ym) und den Radius mit Werten aus dem 
übergebenen \see{Vector}.}

\proc{explicit}{Circle::Circle}{const double d[]}
\descr{Legt ein Objekt Circle an und initialisiert die Parameter 
des Mittelpunktes (xm,ym) und den Radius mit Werten aus dem 
übergebenen Feld d.}

\noindent
Methoden:
\proc{double}{Circle::getR}{}
\descr{Zugriff auf den Radius des Kreises.}
\proc{void}{Circle::setR}{double val}
\descr{Setzt den Wert des Radius auf den übergebenen Wert.}

Zu den wesentlichen ererbten und überschriebenen Methoden siehe:
\begin{itemize}
\item \see{GeoObject::setPos}, ..
\item \see{GeoObject::getPos}, ..
\item \see{GeoObject::Shift}
\end{itemize}

\subsection{CircleSeg}
\label{CircleSeg}
\hypertarget{CircleSeg}{}

\begin{figure}[h]
  \epsfig{figure=circleseg.eps,width=5cm}
  \label{fig:circle}
\end{figure}

Die Klasse \class{CircleSeg} ist über die Klasse \class{Circle} von der 
Basis-Klasse \class{GeoObject} abgeleitet und repräsentiert 
Kreis-Segmente (Kreisausschnitte) in der Ebene. 

\noindent
Konstruktoren:

\proc{}{CircleSeg::CircleSeg}{}
\descr{Legt ein Objekt CircleSeg an.}
\proc{}{CircleSeg::CircleSeg}{const CircleSeg \&c}
\descr{Legt ein Objekt CircleSeg an und initialisiert die Parameter mit den
Werten des übergebenen Kreis-Segments (Kopierkonstruktor).}
\proc{}{CircleSeg::CircleSeg}{double xm,double ym,double r,
  double phi1,double phi2}
\proc{}{CircleSeg::CircleSeg}{Point p,double r,double phi1,double phi2}
\descr{Legt ein Objekt CircleSeg an und initialisiert die Parameter mit den
übergebenen Werten für Mittelpunkt und Radius. 
$phi1$ und $phi2$ sind Anfangs- und Endwinkel des Segments bezogen auf die X-Achse.}

\proc{explicit}{CircleSeg::CircleSeg}{const Vector \&c}
\descr{Legt ein Objekt CircleSeg an und initialisiert die Koordinaten
des Mittelpunktes (xm,ym), den Radius und die Segmentwinkel mit Werten 
aus dem übergebenen \see{Vector}.}

\proc{explicit}{CircleSeg::CircleSeg}{const double d[]}
\descr{Legt ein Objekt CircleSeg an und initialisiert die Koordinaten
des Mittelpunktes (xm,ym), den Radius und die Segmentwinkel mit Werten 
aus dem übergebenen Feld d.}

\noindent
Methoden:
\proc{double }{CircleSeg::getPhi1}{}
\descr{Zugriff auf den Wert des Anfangswinkels.}
\proc{void}{CircleSeg::setPhi1}{double val}
\descr{Setzt den Wert des Anfangswinkels auf den übergebenen Wert.}
\proc{double}{CircleSeg::getPhi2}{}
\descr{Zugriff auf den Wert des Endwinkels.}
\proc{void}{CircleSeg::setPhi2}{double val}
\descr{Setzt den Wert des Endwinkels auf den übergebenen Wert.}

Zu den wesentlichen ererbten und überschriebenen Methoden siehe:
\begin{itemize}
\item \see{GeoObject::setPos}, ..
\item \see{GeoObject::getPos}, ..
\item \see{GeoObject::Shift}
\end{itemize}

\subsection{Ellipse}
\label{Ellipse}
\hypertarget{Ellipse}{}

\begin{figure}[h]
  \epsfig{figure=ellipse.eps,width=5cm}
  \label{fig:ellipse}
\end{figure}

Die Klasse Ellipse ist über die Klasse \class{Circle} von der 
Basis-Klasse \class{GeoObject} abgeleitet und repräsentiert Ellipsen
in der Ebene.

\noindent
Konstruktoren:

\proc{}{Ellipse::Ellipse}{}
\descr{Legt ein Objekt Ellipse an.}
\proc{}{Ellipse::Ellipse}{const Ellipse \&c}
\descr{Legt ein Objekt Ellipse an und initialisiert die Parameter mit den
Werten der übergebenen Ellipse (Kopierkonstruktor).}
\proc{}{Ellipse::Ellipse}{double xm,double ym,double r1,double r2,double phi}
\descr{Legt ein Objekt Ellipse an und initialisiert die Parameter mit den
übergebenen Werten für Mittelpunkt, große und kleine Halbachse und 
den Winkel $\varphi$.}

\proc{explicit}{Ellipse::Ellipse}{const Vector \&c}
\descr{Legt ein Objekt Ellipse an und initialisiert die Koordinaten
des Mittelpunktes (xm,ym), die Halbachsen und den Winkel $\varphi$ mit 
Werten aus dem übergebenen \see{Vector}.}

\proc{explicit}{Ellipse::Ellipse}{const double d[]}
\descr{Legt ein Objekt Ellipse an und initialisiert die Koordinaten
des Mittelpunktes (xm,ym), die Halbachsen und den Winkel $\varphi$ mit 
Werten aus dem übergebenen Feld $d$.}

\noindent
Methoden:
\proc{double}{Ellipse::getR1}{}
\descr{Liefert den aktuellen Wert der großen Halbachse zurück.}
\proc{void}{Ellipse::setR1}{double val}
\descr{Setzt den Wert der großen Halbachse auf den übergebenen Wert.}
\proc{double}{Ellipse::getR2}{}
\descr{Liefert den aktuellen Wert der kleinen Halbachse zurück.}
\proc{void}{Ellipse::setR2}{double val}
\descr{Setzt den Wert der kleinen Halbachse auf den übergebenen Wert.}
\begin{itemize}
\item Statt getR1() und setR1() können genauso die Methode getR() und setR() 
verwendet werden, die von \see{Circle} ererbt wurden. 
\item Wird durch Setzen mit diesen Methoden der Wert der kleinen Halbachse 
größer als der Wert der großen Halbachse werden in einer (internen)
Normalisierung die Werte vertauscht und der Winkel $\varphi$ angepaßt.
\end{itemize}

\noindent
Weiterhin sind die ererbten und überschriebenen Methoden von \class{Circle} und
\class{GeoObject} nutzbar.

\seealso{Circle::setR}
\seealsonext{Circle::getR} \\
\seealso{GeoObject::setPos}
\seealsonext{GeoObject::getPos}\\
\seealso{Draw}

\subsection{EllipseSeg}
\label{EllipseSeg}
\hypertarget{EllipseSeg}{}

\begin{figure}[h]
  \epsfig{figure=ellipseseg.eps,width=7cm}
  \label{fig:circle}
\end{figure}

Die Klasse \class{EllipseSeg} ist über die Klassen \class{Ellipse} 
und \class{Circle} von der Basis-Klasse \class{GeoObject} abgeleitet 
und repräsentiert Ellipsen-Segmente in der Ebene. 

\noindent
Konstruktoren:

\proc{}{EllipseSeg::EllipseSeg}{void}
\descr{Legt ein Objekt EllipseSeg an.}
\proc{}{EllipseSeg::EllipseSeg}{const EllipseSeg \&c}
\descr{Legt ein Objekt EllipseSeg an und initialisiert die Parameter mit den
Werten des übergebenen Ellipsen-Segments (Kopierkonstruktor).}
\proc{}{EllipseSeg::EllipseSeg}{double xm,double ym,double r1,double r2,double phi,double phi1,double phi2}
\descr{Legt ein Objekt EllipseSeg an und initialisiert die Parameter mit den
übergebenen Werten für Mittelpunkt, große und kleine Halbachse und 
Neigungswinkel $\varphi$ der Ellipse. $phi1 und $phi2 sind Anfangs-
und Endwinkel des Segments bezogen auf die große Achse der Ellipse.}

\proc{explicit}{EllipseSeg::EllipseSeg}{const Vector \&c}
\descr{Legt ein Objekt EllipseSeg an und initialisiert die Koordinaten
des Mittelpunktes (xm,ym), die Halbachsen, den Neigungswinkel und die 
Segmentwinkel mit Werten aus dem übergebenen \see{Vector}.}

\proc{explicit}{EllipseSeg::EllipseSeg}{const double d[]}
\descr{Legt ein Objekt EllipseSeg an und initialisiert die Koordinaten
des Mittelpunktes (xm,ym), die Halbachsen, den Neigungswinkel und die 
Segmentwinkel mit Werten aus dem übergebenen Feld $d$.}

\noindent
Methoden:
\proc{double}{EllipseSeg::Phi1}{void}
\descr{Liefert den aktuellen Wert des Anfangswinkels zurück.}
\proc{void}{EllipseSeg::setPhi1}{double val}
\descr{Setzt den Wert des Anfangswinkels auf den übergebenen Wert.}
\proc{double}{EllipseSeg::Phi2}{void}
\descr{Liefert den aktuellen Wert des Endwinkels zurück.}
\proc{void}{EllipseSeg::setPhi2}{double val}
\descr{Setzt den Wert des Endwinkels auf den übergebenen Wert.}
\seealso{Ellipse:setR1}
\seealsonext{Ellipse:getR1}
\seealsonext{Ellipse:setR2}
\seealsonext{Ellipse:getR2}\\
\seealso{GeoObject::setPos}
\seealsonext{GeoObject::getPos}

\noindent
Weiterhin sind die ererbten und überschriebenen Methoden von 
\class{Ellipse}, \class{Circle} und \class{GeoObject} nutzbar.

\noindent
\seealso{GeoObject::Draw}
\seealsonext{GeoObject::Shift}
\pagebreak
\subsection{LineSeg}
\label{LineSeg}
\hypertarget{LineSeg}{}

\begin{figure}[h]
  \epsfig{figure=lineseg.eps,width=12cm}
  \label{fig:circle}
\end{figure}

Die Klasse LineSeg ist direkt abgeleitet von der Basisklasse \class{GeoObject} 
und beschreibt Strecken, Strahlen und Geraden in der Ebene. Die Varianten
werden über ein Attribut $type$ unterschieden:
\begin{tabular}{|l|l|} \hline
  Typ & Wert \\ \hline
  Gerade & LineSeg::line \\ \hline
  Strahl & LineSeg::ray \\ \hline
  Strahl (in Gegenrichtung)& LineSeg::inv\_ray \\ \hline
  Strecke & LineSeg::segment \\ \hline
\end{tabular}

Die Strecken, Strahlen und Geraden der Klasse LineSeg sind gerichtet vom 
Startpunkt P1 zum Endpunkt P2. Diese Orientierung spielt in einigen Methoden 
eine Rolle (\bsee{LineSeg::RightOf},\bsee{LineSeg::OrientationAngle},
\bsee{LineSeg::Angle}).

\noindent

\subtitle{Konstruktoren}

\proc{}{LineSeg::LineSeg}{}
\descr{Legt ein Objekt LineSeg an.}
\proc{}{LineSeg::LineSeg}{Point p1,Point p2,int type=LineSeg::segment}
\descr{Legt ein Objekt LineSeg an und initialisiert mit den übergebenen Werten
von Startpunkt \see{Point} p1 und Endpunkt \see{Point} p2 der Strecke.}
\proc{}{LineSeg::LineSeg}{IPoint p1,IPoint p2,int type=LineSeg::segment}
\descr{Legt ein Objekt LineSeg an und initialisiert mit den übergebenen Werten
von Startpunkt \see{IPoint} p1 und Endpunkt \see{IPoint} p2 der Strecke.}
\proc{}{LineSeg::LineSeg}{const LineSeg \&p}
\descr{Legt ein Objekt LineSeg an und initialisiert mit den Werten des
übergebenen Objekts $p$ (Kopierkonstruktor).}
\proc{}{LineSeg::LineSeg}{double x1,double y1,double x2,double y2,int type=LineSeg::segment}
\descr{Legt ein Objekt LineSeg an und initialisiert mit den übergebenen Werten
von Startpunkt (x1,y1) und Endpunkt (x2,y2) der Strecke.}
\proc{}{LineSeg::LineSeg}{double p,double phi}
\descr{Legt ein Objekt LineSeg an und initialisiert mit den übergebenen Werten
der Hesseschen Geradenparameter. Der Typ ist $LineSeg::line$}
\proc{explicit}{LineSeg::LineSeg}{const Vector \&v,int type=LineSeg::segment}
\descr{Legt ein Objekt LineSeg an und initialisiert mit Werten für die
Start- und Endpunktkoordinaten aus dem übergebenen \see{Vector} v.}
\proc{explicit}{LineSeg::LineSeg}{double d[],int type=LineSeg::segment}
\descr{Legt ein Objekt LineSeg an und initialisiert mit Werten für die
Start- und Endpunktkoordinaten aus dem übergebenen Feld d.}
\noindent

\subtitle{Zugriffs-Methoden}

\proc{Point}{LineSeg::getP1}{}
\descr{Liefert die Koordinaten des Startpunktes als \bsee{Point}.}

\proc{Point}{LineSeg::getP2}{}
\descr{Liefert die Koordinaten des Endpunktes als \bsee{Point}.}

Die folgenden set-Methoden setzen nur den jeweiligen Endpunkt, im Gegensatz 
zu \bsee{GeoObject::setPos}.

\proc{void}{LineSeg::setP1}{Point p}
\descr{Setzt den Startpunkt auf $p$.}

\proc{void}{LineSeg::setP2}{Point p}
\descr{Setzt den Endpunkt auf $p$.}

\proc{int}{LineSeg::Type}{}
\descr{Gibt den Typ von LineSeg zurück.}
\proc{void}{LineSeg::setType}{int type}
\descr{Setzt den Typ des LineSeg auf $type$}

\subtitle{Parameterdarstellung}

Betrachten wir die Gerade gegeben durch die zwei Punkte $p_1$ und $p_2$, 
so wird ein Punkt der Geraden durch die Parameterdarstellung 
$p = p_1 + \mu \cdot (p_2 - p_1)$
beschrieben.

\proc{Point}{LineSeg::RelPoint}{double my}
\descr{Liefert den Punkt der Geraden, der in der Parameterdarstellung 
dem Parameter $my$ zugeordnet ist.}

\proc{double}{LineSeg::LimitedMy}{double my}
\descr{
Liefert zu einem gegebenen $my$ einen auf den jeweiligen Wertebereich 
beschränkten Wert zurück. Der Bereich ergibt sich aus dem Typ:\\
\begin{tabular}{|l|l|} \hline
  Typ & Wertebereich $\mu$ \\ \hline
  Gerade & $-\infty < \mu < \infty$ \\ \hline
  Strahl & $ 0 \le \mu < \infty$ \\ \hline
  Strahl (in Gegenrichtung)& $ -\infty < \mu \le 1$ \\ \hline
  Strecke & $ 0 \le \mu \le 1$ \\ \hline
\end{tabular}
}

\subtitle{Hessesche Normalform}

\proc{void}{LineSeg::CalcHesse}{double \&p,double \&phi}
\descr{
  Berechnet die Parameter $p$ und $phi$ der Darstellung der Geraden in 
  Hessescher Normalform $x \cdot cos(\varphi)+y \cdot sin(\varphi)=p$ .
  In der hier gewählten Normalisierung gilt $p \ge 0$ und $0 \le \varphi < 2 \cdot \pi$.
}
\proc{double}{LineSeg::P}{}
\descr{Liefert den Parameter $p$ der hessischen Normalform 
$x \cdot cos(\varphi)+y \cdot sin(\varphi)=p$ der Geraden}
\proc{double}{LineSeg::Phi}{}
\descr{Liefert den Parameter $\varphi$ der hessischen Normalform 
$x \cdot cos(\varphi)+y \cdot sin(\varphi)=p$ der Geraden}

\subtitle{Analytische Geometrie}

\proc{double}{LineSeg::OrientationAngle}{}
\descr{Liefert den Winkel des Vektors von p1 nach p2 bezüglich der X-Achse.}

\proc{double}{LineSeg::Angle}{const LineSeg \&sec}
\descr{Liefert den Winkel zwischen dem $LineSeg$ (*this) und $sec$. Dabei sind 
die beiden Geraden gerichtet und der Winkel wird zwischen $*this$ und $sec$ 
berechnet. Der Winkel ist immer nichtnegativ im Intervall $0 \le \varphi < \pi$.}

\proc{bool}{LineSeg::RightOf}{Point p}
\descr{Die Methode RightOf ermittelt, ob der gegebene Punkt $p$ rechts der 
{\bf Geraden} liegt. Dabei wird die Gerade als von Punkt p1 nach p2 
orientiert betrachtet bzw. in positiver Richtung der 
Parameterdarstellung $my$.
}

\proc{bool}{LineSeg::LeftOf}{Point p}
\descr{Die Methode LeftOf ermittelt, ob der gegebene Punkt $p$ links der 
{\bf Geraden} liegt. Dabei wird die Gerade als von Punkt p1 nach p2 
orientiert betrachtet bzw. in positiver Richtung der 
Parameterdarstellung $my$.
}

\proc{Point}{LineSeg::ClosestPoint}{Point p}
\procf{Point}{LineSeg::ClosestPoint}{Point p,double \&my}
\descr{Berechnet den zum Punkt $p$ nächstgelegenen Punkt auf der Geraden, auf
  dem Strahl oder auf der Geraden. Die zweite Aufrufform liefert zusätzlich 
den Parameter des Punktes in der Parameterdarstellung der Geraden}

\proc{double}{LineSeg::Distance}{Point p}
\descr{Liefert die Distanz des Punktes $p$ zum Geradensegment. Dies ist 
identisch zum Abstand des nächsten Punktes (\bsee{LineSeg::ClosestPoint})
der Geraden zu $p$.}

\proch{bool}{LineSeg::Intersection}{const LineSeg \&second}{lineseg.h}
\procf{bool}{LineSeg::Intersection}{const LineSeg \&second,Point \&ip}
\procf{bool}{LineSeg::Intersection}{const LineSeg \&second,Point \&ip,double \&my1,double \&my2}
\descr{Ermittelt, ob sich die Strecke, Strahl oder Gerade mit der als 
Parameter gegebenen zweiten Strecke, Strahl oder Gerade schneiden. Wenn 
als Parameter vorhanden, wird der Schnittpunkt als $ip$ zurückgegeben. Die 
Parameter $my1$ und $my2$ liefern die Parameterdarstellung des Schnittpunktes
bezüglich sich selbst (*this) bzw. $second$ zurück.
}

\proc{Point}{LineSeg::Normal}{}
\descr{Liefert die Normale auf der Geraden als \bsee{Point}.}

Zu den wesentlichen ererbten Methode siehe:
\begin{itemize}
  \item \bsee{GeoObject::setPos}
  \item \bsee{GeoObject::Shift}
\end{itemize}

\seealso{Draw}

\subsection{Triangle}
\label{Triangle}
\hypertarget{Triangle}{}
Dreiecke werden durch ihre drei Eckpunkte definiert.

\subtitle{Konstruktoren}

\proch{}{Triangle::Triangle}{}{triangle.h}
\descr{Standard-Konstruktor. Erzeugt ein Dreieck mit den 
Eckpunkten (-1,0), (0,1), (1,0).}

\proch{}{Triangle::Triangle}{Point p1, Point p2, Point p3}{triangle.h}
\descr{Erzeugt ein Dreieck mit den gegebenen Eckpunkten.}

\subtitle{Zugriffs-Methoden}

\proch{bool}{Triangle::isValid}{}{triangle.h}
\descr{Ein Dreieck ist gültig, wenn die Eckpunkte nicht auf 
einer geraden liegen. Im Falle einer solchen Entartung sind die
Parameter des Umkreises ungültig.}

\proc{Point}{Triangle::getCorner}{int i}
\procf{const Point \&}{Triangle::P1}{}
\procf{const Point \&}{Triangle::P2}{}
\procf{const Point \&}{Triangle::P3}{}
\descr{Abfrage der Eckpunkte. Die Eckpunkte werden intern umgeordnet. Ihre
Zuordung entspricht nicht der im Konstruktor angegeben Reihenfolge.}

\proc{Point}{Triangle::getCCCenter}{}
\proc{double}{Triangle::getCCRadius}{}
\descr{Ermittelt den Mittelpunkt und den Radius des Umkreises (circumcircle).
Diese Werte sind nur gültig, wenn das Dreieck nicht entartet ist 
(\bsee{Triangle::isValid}). }

\proc{bool}{Triangle::isInsideCC}{Point point}
\descr{Ermittelt, ob der Punkt point innerhalb des Umkreise liegt.}

\proc{void}{Triangle2Region}{const Triangle \&t, Region \&r}
\descr{Die Funktion erzeugt zu dem Dreieck t eine \class{Region}, die 
die Fläche des Dreiecks beschreibt.}

\seebaseclass{GeoObject}

\subsection{PolygonalCurve}
\label{PolygonalCurve}
\hypertarget{PolygonalCurve}{}

Polygonale Kurven werden durch eine Reihe Punkte definiert, die durch 
geradlinige Kanten verbunden sind. Polygonale Kurven können geschlossen sein, 
das heißt, der Anfangspunkt stellt auch den Endpunkt dar. Polygonale 
Kurven sind geschlossen, wenn dies bei der Konstruktion der Instanz so
festgelegt wurde, die Lage der Punkte spielt keine Rolle. Eine spätere Änderung 
dieser Eigenschaft ist nicht möglich.
 
Der erste Eckpunkt ist mittels des Konstruktors bzw. der Methode 
\bsee{PolygonalCurve::Reset} festzulegen. Erfolgt dies nicht, wird dieser 
Punkt auf $(0,0)$ gesetzt. Die weiteren Eckpunkte 
werden mit der Methode \bsee{PolygonalCurve::Add} angehängt. Alternativ 
kann die ganze Kurve durch Angabe einer Liste von Punkten bzw. einer Kontur 
konstruiert werden.

\subtitle{Konstruktoren}
\proch{}{PolygonalCurve::PolygonalCurve}{bool closed=false}{polygonalcurve.h}
\descr{Legt eine polygonale Kurve mit einem Eckpunkt an, der Eckpunkt 
liegt im Ursprung. Im allgemeinen wird danach die eigentliche Konstruktion 
der polygonalen Kurve mit \bsee{PolygonalCurve::Reset} und 
\bsee{PolygonalCurve::Add} erfolgen.}

\proch{}{PolygonalCurve::PolygonalCurve}{const PolygonalCurve \&p}{polygonalcurve.h}
\descr{Kopierkonstruktor.}

\proch{}{PolygonalCurve::PolygonalCurve}{\bsee{Point} p,bool closed=false}{polygonalcurve.h}
\descr{Legt eine polygonale Kurve mit einer Ecke an, der Eckpunkt ist $p$. 
Im allgemeinen wird die weitere Konstruktion der polygonalen Kurve mit 
\bsee{PolygonalCurve::Add} erfolgen.}

\proch{explicit}{PolygonalCurve::PolygonalCurve}{const \bsee{Matrix} \&m,
                 bool closed=false}{polygonalcurve.h}
\procf{}{PolygonalCurve::PolygonalCurve}{const vector<Point> \&pl,bool closed=false}
\descr{Konstruiert eine polygonale Kurve aus einer Liste von Punkten.}

\proch{explicit}{PolygonalCurve::PolygonalCurve}{const \bsee{Contur} \&c,
                 bool closed=false}{polygonalcurve.h}
\descr{Konstruiert eine polygonale Kurve aus einer Kontur.
Als Eckpunkte der Kurve werden 
die Punkte der Kontur ( = Randpunkte des Objektes) verwendet. Die polygonale Kurve verläuft 
damit in dem Objekt, ist also nicht die Randlinie.
}

\subtitle{Zugriffs-Methoden}

\proch{int}{PolygonalCurve::size}{}{polygonalcurve.h}
\descr{Gibt die Zahl der Ecken der polygonalen Kurve zurück.}

\proch{bool}{PolygonalCurve::isClosed()}{}{polygonalcurve.h}
\descr{Gibt zurück, ob die polygonale Kurve gschlossen ist.}

\proch{const Point \&}{PolygonalCurve::operator[]}{int i}{polygonalcurve.h}
\descr{Gibt die $i$. Ecke der polygonalen Kurve zurück.}

\proch{\bsee{LineSeg}}{PolygonalCurve::edge}{int i}{polygonalcurve.h}
\procf{void}{edge}{int i,LineSeg \&l}
\descr{Gibt die $i$. Kante der polygonalen Kurve als \bsee{LineSeg} zurück.}

\proch{void}{PolygonalCurve::Reset}{}{polygonalcurve.h}
\procf{void}{PolygonalCurve::Reset}{Point p}
\descr{Setzt die polygonale Kurve zurück. Die polygonale Kurve besteht danach 
aus nur einem Eckpunkt. Dieser wird auf $(0,0)$ bzw. den übergebenen Punkt 
$p$ gesetzt.
}

\proch{Contur}{MakeContur}{}{polygonalcurve.h}
\descr{Erzeugt eine Contur, die die Kurve beschreibt. Aufgrund der 
Ganzzahligkeit der Konturpunktkoordinaten ist dies eine Näherung.
}

\proch{int}{getClosestCorner}{Point p}{polygonalcurve.h}
\descr{Bestimmt den zum gegebenen Punkt p am nächsten liegenden Eckpunkt 
der Kurve.}

\proch{int}{getClosestEdge}{Point p}{polygonalcurve.h}
\descr{Bestimmt die zum gegebenen Eckpunkt p am nächsten liegende Kante 
der Kurve.}

\proch{PolygonalCurve}{Reduced}{int nr, int mode=1}{polygonalcurve.h}
\procf{void}{Reduced}{int nr, PolygonalCurve \&p, int mode=1}
\descr{Erzeugt eine Kurve mit geringerer Eckenzahl $nr$, die die gegebene
Kurve approximiert.\\
\begin{tabular}{cl}
mode == 1 & merge \\
mode == 2 & split \\
\end{tabular}
}

\proch{PolygonalCurve}{PolygonalCurve::ReducedToPrecision}{double prec,int mode=1}
{polygonalcurve.h}
\procf{void}{PolygonalCurve::ReducedToPrecision}{double prec,PolygonalCurve \&res,int mode=1}
\descr{Erzeugt eine Kurve mit geringerer Eckenzahl, die die gegebene
Kurve so gut approximiert, dass kein Punkt weiter als $prec$ von der 
Approximation entfernt ist.\\
\begin{tabular}{cl}
mode == 1 & merge \\
mode == 2 & split \\
\end{tabular}
}

\seealso{Draw}

\subsection{Polygon}
\label{Polygon}
\hypertarget{Polygon}{}

Polygone werden durch ihre Eckpunkte definiert. Der erste Eckpunkt ist mittels 
des Konstruktors bzw. der Methode \bsee{Polygon::Reset} festzulegen. Erfolgt 
dies nicht, wird dieser Punkt auf $(0,0)$ gesetzt. Die weiteren Eckpunkte 
werden mit der Methode \bsee{Polygon::Add} angehängt. Alternativ kann das 
ganze Polygon durch Angabe einer Liste von Punkten bzw. einer Kontur 
konstruiert werden. Neben der Klasse Polygon werden in einigen Funktionen 
auch Punktlisten/Matrizen als Polygon interpretiert.

\subtitle{Konstruktoren}
\proch{}{Polygon::Polygon}{}{polygon.h}
\descr{Legt ein Polygon mit einem Eckpunkt an, der Eckpunkt liegt im Ursprung. 
Im allgemeinen wird danach die eigentliche Konstruktion des Polygon mit 
\bsee{Polygon::Reset} und \bsee{Polygon::Add} erfolgen.}

\proch{}{Polygon::Polygon}{const Polygon \&p}{polygon.h}
\descr{Kopierkonstruktor.}

\proch{}{Polygon::Polygon}{\bsee{Point} p}{polygon.h}
\descr{Legt ein Polygon mit einer Ecke an, der Eckpunkt ist $p$. 
Im allgemeinen wird die weitere Konstruktion des Polygon mit 
\bsee{Polygon::Add} erfolgen.}

\proch{explicit}{Polygon::Polygon}{const \bsee{Matrix} \&m}{polygon.h}
\procf{}{Polygon::Polygon}{const vector<Point> \&pl}
\descr{Konstruiert das Polygon aus einer Liste von Punkten.}

\proch{explicit}{Polygon::Polygon}{const \bsee{Contur} \&c}{polygon.h}
\descr{Konstruiert ein Polygon aus einer Kontur.
Als Eckpunkte des Polygons werden die Punkte der Kontur ( = Randpunkte des Objektes) verwendet. 
Das Polygon liegt damit in dem Objekt, ist also nicht die Randlinie.
}

\subtitle{Zugriffs-Methoden}

\proch{int}{Polygon::size}{}{polygon.h}
\descr{Gibt die Zahl der Ecken des Polygon zurück.}

\proch{const Point \&}{Polygon::operator[]}{int i}{polygon.h}
\descr{Gibt die $i$. Ecke der Polygons zurück.}

\proch{\bsee{LineSeg}}{Polygon::edge}{int i}{polygon.h}
\procf{void}{edge}{int i,LineSeg \&l}
\descr{Gibt die $i$. Kante des Polygons als \bsee{LineSeg} zurück.}

\proch{void}{Polygon::Reset}{}{polygon.h}
\procf{void}{Polygon::Reset}{Point p}
\descr{Setzt das Polygon zurück. Das Polygon besteht danach aus nur 
einem Eckpunkt. Dieser wird auf $(0,0)$ bzw. den übergebenen Punkt $p$ gesetzt.
}

\seealso{Draw}

\subsection{Zeichenfunktionen}
\proch{int}{Draw}{const \bsee{Circle} \&c,Image \&img,int val,int fval=-1}{draw.h}
\procf{int}{Draw}{const \bsee{CircleSeg} \&c,Image \&img,int val,int fval=-1}
\procf{int}{Draw}{const \bsee{Ellipse} \&c,Image \&img,int val,int fval=-1}
\procf{int}{Draw}{const \bsee{EllipseSeg} \&c,Image \&img,int val,int fval=-1}
\procf{int}{Draw}{const \bsee{LineSeg} \&c,Image \&img,int val}
\procf{int}{Draw}{const \bsee{Polygon} \&c,Image \&img,int val,int fval=-1}
\descr{Stellt das jeweilige geometrische Objekt im Bild $img$ dar. 
Dabei wird die Objektbegrenzung mit dem Wert $val$, das Objektinnere 
mit dem Wert $fval$ eingetragen. Wird $fval$ mit dem speziellen Wert -1 
vorgegeben, so wird das Objekt nicht gefüllt. }

\subsection{Implementierungshinweise für geometrische Objekte}
Klassen für weitere \hyperlink{SECTION:geoObject}{geometrische Objekte} können 
durch Ableitung von einer geeigneten Klasse der Klassenhierarchie 
\class{GeoObject} erfolgen. Es ist darauf zu achten, dass
alle abstrakten Methoden der Basisklasse implementiert werden. 
Insbesondere ist folgendes zu beachten:
\begin{itemize}
\item Die Methoden Distance(..) und Inside(..) mit unterschiedlicher Form 
der Parameterangaben werden intern in einen Aufruf von 
{\bf \verb+double distance(Point p) const;+} bzw.
{\bf \verb+bool inside(Point p) const;+} umgewandelt. Abgeleitete 
Klassen müssen diese (kleingeschriebenen) Methoden (re-)implementieren.
\item {\bf distance} bezieht sich immer auf den Abstand zur Begrenzungslinie 
des Objektes, nicht die Fläche.
\item {\bf inside} bezieht sich immer auf die Objekte-Fläche. Punkte der 
Begrenzungslinie sind nicht innerhalb. Geometrische Objekte ohne Fläche 
geben immer {\bf false} zurück, was bereits in der Basisklasse 
\class{GeoObject} so implementiert ist und somit nicht neu implementiert 
werden muss.
\end{itemize}

%===============================================================================
\nsection{Klassen-Hierarchie Geometrisches 3D-Objekt}{geoObject3d}
\label{Geometrisches3DObjekt}
\hypertarget{geometric3dObject}{}

Alle Klassen, die der Beschreibung von geometrischen Objekten des Raumes 
dienen, leiten sich von der abstrakten Basis-Klasse \class{GeoObject3d} ab.
\vspace{1cm}
\noindent
Klassen-Hierarchie:

\begin{description}
\item[\class{GeoObject3d}] Abstrakte Basisklasse
\begin{description}
\item[\class{Line3d}] Gerade, Strecke, Strahl
\item[\class{Sphere}] Kugel
\end{description}
\end{description}

\subsection{GeoObject3d}
\label{GeoObject3d}
\hypertarget{GeoObject3d}{}

\class{GeoObject3d} ist eine abstrakte Basisklasse für 
\hyperlink{geometric3dObject}{geometrische Objekte} des Raumes.
In dieser Klasse definierte Methoden sind in allen abgeleiteten Klassen
ebenfalls vorhanden.

Konstruktoren:

\proch{}{GeoObject3d::GeoObject3d}{void}{geo3d.h}
\descr{Legt ein 3D-Objekt im Ursprung an.}

\proc{}{GeoObject3d::GeoObject3d}{const GeoObject3d \&p}
\procf{}{GeoObject3d::GeoObject3d}{const Vector3d \&p}
\procf{}{GeoObject3d::GeoObject3d}{double xp,double yp,double zp}
\descr{Legt ein 3D-Objekt im angegebenen Punkt an.}

\proc{explicit}{GeoObject3d::GeoObject3d}{const Vector \&v}
\proc{explicit}{GeoObject3d::GeoObject3d}{double p[]}
\descr{Legt ein 3D-Objekt im angegebenen Punkt an.
 Zur Vermeidung ungewollter  Konvertierungen muß der Aufruf explizit erfolgen.}

\proc{double}{GeoObject3d::X}{void}
\procf{double}{GeoObject3d::Y}{void}
\procf{double}{GeoObject3d::Z}{void}
\descr{Lesender Zugriff auf die Koordinaten des 3D-Objekts}

\proc{void}{GeoObject3d::setX}{double x}
\procf{void}{GeoObject3d::setY}{double y}
\procf{void}{GeoObject3d::setZ}{double z}
\procf{void}{GeoObject3d::set}{const Vector \&v}
\descr{Setzen der Koordinatenwerte der Objektposition.}

\proc{Vector3d \&}{GeoObject3d::Pos}{}
\descr{Zugriff auf die Position des 3D-Objektes.}

\proc{void}{GeoObject3d::Shift}{double dx,double dy,double dz}
\proc{void}{GeoObject3d::Shift}{const Vector3d \&v}
\descr{Verschiebung des 3D-Objektes um die gegebene Verschiebung.}

\proc{double}{GeoObject3d::Distance}{double dx,double dy,double dz}
\procf{double}{GeoObject3d::Distance}{const Vector3d \&ob}
\descr{Euklidischer Abstand des gegebenen Raumpunktes vom 3D-Objekt}

\proc{double}{GeoObject3d::Volume}{}
\descr{Gibt das Volumen des 3D-Objektes zurück. Diese Methode ist in der
  Basis-Klasse abstrakt.}

\subsection{Sphere}
\label{Sphere}
\hypertarget{Sphere}{}

\class{Sphere} ist eine von \class{GeoObject3d} abgeleitetet Klasse 
zur Repräsentation von Kugeln.

Konstruktoren:
\proch{}{Sphere::Sphere}{}{sphere.h}
\descr{Legt eine Einheits-Kugel im Ursprung an.}

\proc{}{Sphere::Sphere}{double xp,double yp,double zp,double rp}
\procf{}{Sphere::Sphere}{const Vector3d \&p,double rp}
\procf{}{Sphere::Sphere}{const Point3d \&p,double rp}
\descr{Legt eine Kugel mit den gegebenen Parametern für Mittelpunkt und Radius
  an.}

\proc{}{Sphere::Sphere}{const Sphere \&p}
\descr{Legt eine Kopie der Kugel $p$ an (Kopierkonstruktor).}

\proc{explicit}{Sphere::Sphere}{const Vector \&v}
\procf{explicit}{Sphere::Sphere}{double d[]}
\descr{Legt eine Kugel mit den gegebenen Parametern für Mittelpunkt und Radius
  an. Der Aufruf muß explizit erfolgen.}

Methoden:

\proc{double}{Sphere::R}{}
\descr{Gibt den Radius der Kugel zurück.}

\proc{void}{Sphere::setR}{double vr}
\descr{Setzt den Radius der Kugel auf $vr$.}

\proc{double}{Sphere::Volume}{}
\descr{Berechnet das Volumen der Kugel.}

Siehe auch weitere Methoden der Basis-Klasse \see{GeoObject3d}.
\newcommand{\pfeil}[1]{\stackrel{\rightarrow}{{#1}}}

\subsection{Line3d}
\label{Line3d}
\hypertarget{Line3d}{}

\class{Line3d} ist eine von \class{GeoObject3d} abgeleitetet Klasse 
zur Repräsentation von Geraden, Strahlen und Strecken des Raumes. 
Die Gerade wird dabei in ihrer
Parameter-Form $\pfeil{x}=\pfeil{x_1}+p \cdot (\pfeil{x_2}-\pfeil{x_1})=\pfeil{x_1}+p \cdot \pfeil{dx}$ dargestellt. Der Parameter
$p$ ist bei Geraden unbeschränkt, ist bei Strahlen größer als Null und liegt
bei Strecken im Intervall $[0..1]$.

Konstruktoren:

\proch{}{Line3d::Line3d}{}{line3d.h}
\descr{Legt eine undefiniert Strecke an. Vor der Verwendung müssen die
  Endpunkte sinnvoll gesetzt werden.}

\proc{}{Line3d::Line3d}{double xp,double yp,double zp,double x2p,double y2p,double z2p}
\procf{}{Line3d::Line3d}{const Point3d \&p1,const Point3d \&p2}
\descr{Legt eine Strecke mit den gegebenen Endpunkten an.}

\proc{}{Line3d::Line3d}{const Line3d \&p}
\descr{Legt eine Kopie der gegebenen Gerade an (Kopierkonstruktor).}

\proc{explicit}{Line3d::Line3d}{const Vector \&v}
\procf{explicit}{Line3d::Line3d}{double d[]}
\descr{Legt eine Strecke mit den in Parametern angegebenen Endpunkten an. Der
  Aufruf muß explizit erfolgen.}

Elementzugriff:

\proc{double}{Line3d::X1}{}
\proc{double}{Line3d::Y1}{}
\proc{double}{Line3d::Z1}{}
\proc{Vector3d \&}{Line3d::P1}{}
\descr{Zugriff auf die Koordinatenwerte des Startpunktes. Diese Methoden sind
  identisch zu den gleichfalls nutzbaren Methoden X(),Y() und Z() der
  Basisklasse $GeoObject3d$.}

\proc{double}{Line3d::X2}{}
\proc{double}{Line3d::Y2}{}
\proc{double}{Line3d::Z2}{}
\proc{Vector3d}{Line3d::P2}{}
\descr{Zugriff auf die Koordinatenwerte des Endpunktes. Diese Werte
  werden berechnet, da intern nur ein Richtungsvektor gespeichert wird.}

\proc{double}{Line3d::DX}{}
\proc{double}{Line3d::DY}{}
\proc{double}{Line3d::DZ}{}
\proc{Vector3d \&}{Line3d::DP}{}
\descr{Zugriff auf die Werte des Richtungsvektors vom Start- zum Endpunkt.}

Methoden:

\proc{Vector3d}{Line3d::operator()}{double p}
\descr{Liefert die Koordinaten des Punktes $x$ der Geraden mit
  $\pfeil{x}=\pfeil{x_1}+p \cdot (\pfeil{x_2}-\pfeil{x_1})=\pfeil{x_1}+p \cdot
  \pfeil{dx}$}

\proc{double}{Line3d::Volume}{}
\descr{Gibt 0 zurück.}
