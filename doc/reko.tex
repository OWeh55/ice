\lsection{Monokulare Rekonstruktion}
\subsection{Planare Objekte}

\proc{int}{RecTriangle}{camera cam,PointCorr *pcorr,int n,Frame frame[2]}
\descr{
Aus der Punktkorrespondenzliste $pcorr$ wird eine Dreieck mit möglichst großer
Fläche ausgewählt und die Lage Rekonstruiert. Die beiden Lösungen werden auf
$frame$ zurückgegeben.
}
\proc{int}{RecPlanar}{camera cam,PointCorr *pcorr,int nbr,Frame *f}
\descr{
Die Lage der planaren Punktkonfiguration $pcorr$ mit mindestens vier Punkten
wird rekonstruiert und auf $f$ zurückgegeben.
}
\proc{int}{RecGeneralPlanar}{PointList plb,PointList plm,camera cam, Frame f[2]}
\descr{ 
  Mit Hilfe der Flächenmomente und Standardlagen des Modellobjektes
  $plm$ und des zugehörigen Bildobjektes $plb$ wird die Lage des
  Modellobjektes rekonstruiert. Die beiden Lösungen werden auf $f$
  zurückgegeben.  
}
\proc{int}{RecPlanarIterate}{PointList plb,PointList plm,int nbr,camera cam,Frame *f}
\descr{
  Ausgehend von der auf $f$ übergebenen Startlösung wird die Lage eines
  planaren Objektes iterativ bestimmt. Dazu werden iterativ an $nbr$
  äquidistanten Stützstellen aus $plm$ korrespondierende Punkte in $plb$
  gesucht und mit den gefundenen Paaren die Lage neu bestimmt.
}
\proc{int}{RecMoments}{double mm[15],double mb[15],camera cam, Frame f[2]}
\descr{
Aus den Flächenmomenten eines planaren Modellobjektes $mm$ und des zugehörigen
Bildobjektes $mb$ wird die Lage des Modellobjektes rekonstruiert. Die beiden
Lösungen werden auf $f$ zurückgegeben.
}
\proc{int}{RecSquare}{camera cam,PointCorr *pc,Frame *f}
\descr{
Es wird die Lage eins planaren Vierecks rekonstruiert, indem der Diagonalenschnittpunkt
als Linearkombination von jeweils zwei Seiten ausgedrückt wird. Aus dem daraus
folgenden Gleichungssystem können die Abstände der einzelnen Punkte zum
Projektionszentrum berechnet werden.
}


\subsection{Räumliche Objekte}

\proc{int}{LinRecNPoint}{camera cam,PointCorr *pcorr,int nbr,Frame *f,double *mse}
\descr{
Durch lineare Ausgleichsrechnung wird die Lage der räumlichen Punktekonfiguration
$pcorr$ rekonstruiert. $nbr$ gibt die Anzahl der Punktkorrespondenzen an. Das
Rekonstruktinsergebnis wird auf $f$ zurückgegeben, $mse$ gibt den Restfehler
der Ausgleichung an.
}
\proc{int}{NonLinRecNPoint}{camera cam,PointCorr *pcorr,int nbr,Frame *f}
\descr{
Durch nichtlineare Ausgleichsrechnung (Newton-Raphson) wird die Lage der
räumlichen Punktekonfiguration $pcorr$ rekonstruiert. $nbr$ gibt die Anzahl
der Punktkorrespondenzen an. $f$ muß eine geeignete Startlösung enthalten und
wird durch die Funktion verändert.
}
\proc{int}{IterateRec}{camera cam,PointCorr *pcorr,int nbr,Frame *f}
\descr{
Es wird ein Iterationsschritt zur Lagerekonstruktion der Punktekonfiguration
$pcorr$ ausgeführt (Newton-Raphson).
}
\proc{int}{NonLinRecNLine}{camera cam,LineCorr *lcorr,int nbr,Frame *f}
\descr{
Es wird die Lage der Geradenkonfiguration $lcorr$ durch nichtlineare
Ausgleichsrechnung (Newton-Raphson) rekonstruiert. $nbr$ gibt die Anzahl
der Linienkorrespondenzen an. $f$ muß eine geeignete Startlösung enthalten und
wird durch die Funktion verändert.
}
\proc{double}{RecError}{camera cam,PointCorr *pcorr,int nbr,Frame *f}
\descr{
Zurückgegeben wird der mittlere Abstand der Bildpunkte von den mit $f$ und
$cam$ in die Bildebene projizierten Raumpunkten.
}
