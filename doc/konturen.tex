\nsection{Konturen, Kanten und Regionen}{conturs}

\subsection{Konturfolge}
\label{Konturen}

\subsubsection{Objekte}
Die Konturfolge dient in der Bildverarbeitung dazu, Objekte zu finden. Das
Innere der gefundene Kontur repräsentiert im Bild das Objekt. Ausgangspunkt
ist die Möglichkeit, für jedes einzelne Pixel im Bild die Entscheidung fällen
zu können, ob es sich um einen Untergrund- oder einen Objektpunkt handelt. Im
einfachsten Falle geschieht dies durch die Festlegung eines Grauwert-Pegels:
Alle Punkte mit Werten größer gleich diesem Schwellwert sind Objektpunkte,
die anderen Untergrundpunkte.

\begin{figure}[h]
  \epsfig{figure=contur1.eps,width=8cm}
  \label{object}
  \caption{Ein Objekt und seine Kontur}
\end{figure}

Im Bild sind Objektpunkte blau markiert. Randpunkte eine Objekts
sind Objektpunkte, in deren 4-Nachbarschaft Untergrundpunkte
liegen. Randpunkte und benachbarte Untergrundpunkte bilden
Objekt-/Untergrund-Punktpaare. Umläuft man das Objekt, so entsteht eine Folge
von Objekt-/Untergrundpaaren, die die Kontur bilden. In ICE wird als Contur
die Folge der Randpunkte des Objekts abgelegt, die zugehörigen Untergrundpunkte
lassen sich eindeutig rekonstruieren.

Grundidee der Konturfolge ({\see{CalcContur}}) ist, daß man zunächst 
zeilenweise oder spaltenweise das Bild durchmustert und einen Startpunkt 
sucht. Ein geeigneter Startpunkt ist der erste Objektpunkt nach einem
Untergrund-/Objekt-Übergang. Ausgehend von diesem bekannten
Objekt-/Untergrund-Punktpaar wird dann in der Nachbarschaft nach weiteren
Paaren gesucht und so die Kontur verfolgt (gestrichelte Linie). Nach dem
Umlauf um das Objekt kehrt der Algorithmus immer zum Startpunkt zurück. Die
Konturen sind deshalb immer geschlossene Konturen, der Start- und der Endpunkt
sind dieselben. Geschlossene Konturen umschließen eine Fläche und besitzen
deshalb zusätzliche Merkmale, z.B. die Fläche und den Formfaktor.

\subsubsection{Löcher}
\begin{figure}[h]
  \epsfig{figure=contur2.eps,width=8cm}
  \label{fig:hole}
  \caption{Das Loch im Objekt und seine Kontur}
\end{figure}

In analoger Weise werden Konturen von Löchern gefunden. Trifft der Algorithmus
auf ein Loch, so findet in genau gleicher Weise eine Konturfolge statt. Im
Bild sind die Objekt-/Untergrund-Punktpaare und der Weg der
Konturfolge für das Loch in dem Objekt dargestellt. Der Pfeil an der
gestrichelten Konturlinie verdeutlicht, daß der Umlaufsinn einer Lochkontur
entgegengesetzt zur Kontur eines Objekts ist. Auch Lochkonturen sind immer
geschlossen. Wegen des negativen Umlaufsinns ist der Flächeninhalt einer
Lochkontur negativ.

\subsubsection{Undefinierte Punkte}
\label{UndefiniertePunkte}

\begin{figure}[h]
  \epsfig{figure=contur3.eps,width=8cm}
  \label{fig:hole}
  \caption{Konturfolge und undefinierte Punkte}
\end{figure}

Oft entsteht die Situation, daß die Entscheidung Objekt oder Untergrund nicht
für jedes Pixel eindeutig zu treffen ist. Dies gilt zum Beispiel auch für
Punkte außerhalb des Bildes. Oft werden diese der Einfachheit halber als
Untergrund-Punkte behandelt, genaugenommen ist eine solche Aussage nicht
möglich. Eine ähnliche Situation kann für Punkte im Bild entstehen. Diese
Punkte wollen wir im folgenden als ''undefiniert'' betrachten. Das Bild 
verdeutlicht diese Situation. Undefinierte Punkte sind rot markiert.

Die Kontur läßt sich jetzt immer noch genauso als Folge von
Objekt-/Untergrund-Punktpaaren definieren. Der Übergang von einer dualen 
Entscheidung zur ternären Entscheidung mit Objekt-Punkten, Untergrund-Punkten
und undefinierten Punkten hat zur Folge, daß die gefundenen Konturen nicht 
geschlossen sein müssen.

\subsection{Die Klasse Contur}
\hypertarget{Contur}{}

Für die Speicherung und Bearbeitung von Konturen wird die Klasse \class{Contur}
verwendet. Konturen entstehen als Ergebnis der Objektsuche im Bild 
mittels Konturfolge. Aber auch beliebige ebene geometrische 
Figuren als \class{Contur} beschrieben werden. 

Die Datenstruktur \class{Contur} repräsentiert eine Folge von Punkten, wobei 
in der Folge benachbarte Punkte auch im Bild benachbart sind. Sind Anfangs-
und Endpunkt der Punktfolge identisch, so ist die Kontur
geschlossen. Geschlossene Konturen umschließen eine Fläche und besitzen
deshalb zusätzliche Merkmale, z.B. die Fläche und den Formfaktor. Je nach
Umlaufsinn um die Fläche ist der Flächeninhalt positiv (positiver Umlaufsinn,
Objekt-Kontur) oder negativ (negativer Umlaufsinn, Loch-Kontur).

\subsubsection{Konstruktoren und Destruktoren}

\proch{}{Contur::Contur}{}{contur.h}
\descr{
Legt eine (noch ungültige) Kontur an. Für derartige, nicht initialisierte
Konturen hat $valid$ den Wert $false$. \\
\seealso{Contur::isValid}\\
Fehlgeschlagene Funktionen, die Konturen zurückliefern, geben ebenfalls
eine derartige Kontur zurück.
}
\proc{}{Contur::Contur}{const Contur \&c}
\descr{
Legt eine Kopie der Kontur $c$ an.
}

\procf{}{Contur::Contur}{IPoint p}
\proc{}{Contur::Contur}{int x,int y}
\descr{
Legt eine Kontur mit dem Startpunkt $p$ bzw. ($x,y$) an. Die Kontur hat die
Länge Null. 
}

\subsubsection{Erstellung und Bearbeitung einer Kontur}
Konturen werden nach der Definition eines Startpunktes (kann auch
beim Konstruktor schon geschehen) durch Anhängen weiterer Teile
aufgebaut. Angehängt werden Richtungscodes, Verbindungslinien zu 
weiteren Punkten oder ganze Konturen.

\proc{int}{Contur::Reset}{}
\descr{Die Kontur wird zurückgesetzt und ist danach wieder undefiniert 
(isValid()==false).\\
\seealso{Contur::isValid}
}

\proc{int}{Contur::Reset}{IPoint p}
\procf{int}{Contur::Reset}{int x,int y}
\descr{Die Kontur wird zurückgesetzt und mit dem Startpunkt $p$ bzw. $(x,y)$
neu initialisiert.
}

\proc{int}{Contur::SetStart}{IPoint p}
\procf{int}{Contur::SetStart}{int x,int y}
\descr{Der Startpunkt der Kontur wird auf ($x,y$) bzw. $p$ gesetzt. 
Dabei wird eine bis dahin ungültige Kontur gültig und hat die Länge 0. 
Bei einer bereits vorhandenen Kontur wird der Startpunkt versetzt, was 
einer Verschiebung der ganzen Kontur entspricht.}

\proc{int}{Contur::Add}{Freeman dir}
\descr{Hängt an eine gültige Kontur den Richtungskode $dir$
(Freeman-Richtung) an.\\
\seealso{Freeman}
}

\proc{int}{Contur::Add}{IPoint p}
\procf{int}{Contur::Add}{int x,int y}
\descr{Hängt an eine gültige Kontur die geradlinige Verbindung 
zu einem Punkt an.}

\proc{int}{Contur::Add}{const Contur \&c}
\descr{Hängt an eine gültige Kontur die Kontur $c$ an. 
Dazu wird zunächst die geradlinige Verbindung zwischen bisherigen 
Endpunkt der Kontur und dem Startpunkt der Kontur $c$ hergestellt
und dann die weiteren Punkte der Kontur $c$ angehängt.}

\proc{Contur}{operator+}{const Contur \&c1,const Contur \&c2}
\descr{ Die Verkettung der Konturen c1 und c2 wird zurückgegeben. 
Die Verkettung erfolgt genauso wie beim Anhängen einer Kontur 
an eine Kontur.}

\proc{void}{Contur::Close}{}
\descr{Schließt die Kontur, indem der Startpunkt angehängt wird.}

\proc{int}{Contur::InvDir}{}
\descr{Kehrt die Richtung der Kontur um. Die entstehende Kontur
besteht aus den gleichen Punkten, wird aber im entgegengesetzten
Umlaufsinn durchlaufen. \\
Achtung: Dadurch wird zwar eine Objekt-Kontur in eine Loch-Kontur
umgewandelt, es handelt sich aber nicht um das zum Objekt
komplementäre (flächengleiche) Loch.
}

\proc{Contur \&}{Contur::operator =}{const Trafo \&}
\descr{Zuweisungs-Operator}

\subsubsection{Abfrage elementarer Eigenschaften}

\proc{int}{Contur::isValid}{}   
\descr{Liefert den Wert {\bf true} genau dann, wenn es sich um eine gültige
(initialisierte) Kontur handelt.}
\proc{int}{Contur::Number}{}
\descr{Liefert die Anzahl der Richtungscodes innerhalb der Kontur.}
\proc{double}{Contur::Length}{}
\descr{Liefert die geometrische Länge der Kontur.}
\proc{int}{Contur::isClosed}{}
\descr{Liefert {\bf true}, falls es sich bei der Kontur um eine gültige
geschlossene Kontur handelt.}
\proc{IPoint}{Contur::Start}{}
\descr{Liefert den Startpunkt der Kontur.}
\proc{IPoint}{Contur::End}{}
\descr{Liefert den Endpunkt der Kontur.}
\proc{int}{Contur::StartX}{}
\procf{int}{Contur::StartY}{}
\descr{Liefert die Koordinaten des Startpunktes der Kontur.}
\proc{int}{Contur::EndX}{}
\procf{int}{Contur::EndY}{}
\descr{Liefert die Koordinaten des Endpunktes der Kontur.}
\proc{int}{Contur::isHole}{}
\descr{Liefert {\bf true }, falls es sich bei der Kontur um eine 
geschlossene, gültige Kontur handelt, die ein Loch beschreibt.}
\proc{int}{Contur::getRect}{int \&xi,int \&yi,int \&xa,int \&ya}
\descr{Schreibt die Koordinaten des umschreibenden
Rechtecks der Kontur auf die Variablen $xi$, $yi$, $xa$ und $ya$.}
\seealso{FeatureContur}

\subsubsection{Zugriff auf Kontur-Daten}
\proc{Freeman}{Contur::getDirection}{int i}
\descr{Liefert den Freemancode an der Stelle $i$ in der Kontur. Gültige Werte 
von i liegen im Interval 0...Anzahl der Richtungscodes-1.\\
\seealso{Freeman}
\seealsonext{Contur::Number}
}

\proc{const std::vector\textless{}Freeman\textgreater{}\&}{Contur::getDirections}{}
\descr{Liefert eine Referenz auf einen vector aller Freemancodes der Kontur.\\
\seealso{Freeman}
\seealsonext{Contur::Number}
}

\proc{int}{Contur::getDirections}{std::vector\textless{}Freeman\textgreater{}\& fc}
\descr{Belegt Vector $fc$ mit den Freemancodes der Kontur.
\\
\seealso{Freeman}
\seealsonext{Contur::Number}
}
\proc{int}{Contur::getPairs}{std::vector\textless{}IPoint\textgreater{} \&opl,std::vector\textless{}IPoint\textgreater{} \&upl}
\procf{int}{Contur::getPairs}{std::vector\textless{}IPoint\textgreater{} \&pl,std::vector\textless{}Freeman\textgreater{} \&fc}
\descr{Ermittelt zu einer Kontur die Objekt/Untergrund-Pixelpaare. Das Ergebnis 
wird bereitgestellt als 
Liste von Objektpunkten $opl$ und Liste von Untergrundpunkten $upl$ oder als
Liste von Objektpunkten $pl$ und Liste von Richtungen $fc$, die vom Objektpunkt 
auf den Untergrundpunkt verweisen.}

\proc{int}{Contur::DirCode}{int i}
\descr{Liefert den Richtungscode an der Stelle $i$ in der Kontur.
Gültige Werte von i liegen im Interval 0...Anzahl der Richtungscodes-1.\\
\seealso{Freeman}
\seealsonext{Contur::Number}
}

\proc{IPoint}{Contur::getPoint}{int nr}
\procf{int}{Contur::getPoint}{int nr,int \&x,int \&y}
\descr{Berechnet den Punkt an der Stelle $nr$ der Kontur. Gültige
Werte von $nr$ liegen im Interval 0...Anzahl der Richtungscodes 
(\see{Contur::Number}). Der spezielle Wert $nr=0$ liefert den 
Startpunkt der Kontur, $nr=Contur::Number()$ liefert den 
Endpunkt der Kontur.\\
\seealso{Contur::Start}
\seealsonext{Contur::End}
}

\proc{void}{Contur::getPoints}{vector<IPoint> \&pl,bool close=false}
\descr{
Erzeugt eine Liste von Punkten der Kontur. Da bei geschlossenen Konturen der
Endpunkt gleich dem Anfangspunkt ist, kann für diese mit dem Parameter $close$ 
entschieden werden, ob der Endpunkt in die Liste aufgenommen ($close = true$)
oder weggelassen werden soll ($close = false$).
}

\proc{\bsee{IMatrix}}{ConturPointlist}{const Contur \&c,int diff=1,bool close=true}
\descr{Erzeugt aus einer Contur eine Liste von Punkten als \see{IMatrix}.
  Durch Angabe von $diff$ kann festgelegt werden, dass nur jeder diff. Punkt in 
  die Liste eingetragen wird. Bei offenen Konturen werden Anfangs- und Endpunkt immer in 
  die Liste eingetragen. Bei geschlossenen Konturen entscheidet der Parameter $close$,
  ob der Endpunkt, der ja gleich dem Anfangspunkt ist, an die Liste angehängt wird.
  Punktlisten werden beispielsweise beim adaptiven Fitting
  benötigt (\see{Adaptives Fitting}). }

\subsubsection{Kontur-Merkmale}
\proc{int}{FeatureContur}{Contur c,double \&length,double \&area,double
\&form,double \&conv}
\descr{
Für eine geschlossene Kontur $c$ werden die Merkmale Konturlänge $length$,
Anzahl der zu dem von der Kontur berandeten Objekt gehörenden Bildpunkte
$area$, Formfaktor $form$ und Konvexitätsmaß $conv$ berechnet. 
\begin{itemize}
\item Die Konturlänge
ergibt sich als Summe der Freemancodes, wobei gerade Richtungen mit 1 und
ungerade Richtungen mit $\sqrt 2$ gewichtet werden. 
\item Der Formfaktor wird nach
der Formel $form={length^2 \over 4\pi\cdot area}$ berechnet, so daß sich für
einen idealen Kreis der Wert 1 ergeben würde. Für reale Objekte ist der 
Formfaktor immer größer als 1. 
\item Die Konvexität wird nach der Formel $conv={3 \cdot l_0 +
l_1 \over 2 \cdot (x_a-x_i+y_a-y_i+p_a-p_i+ma-mi)}$ berechnet, wobei $l_0$ die
Anzahl der Richtungscodes, $l_1$ die Anzahl der ungeraden Richtungscodes,
$x_i=min(x)$, $x_a=max(x)$, $y_i=min(y)$, $y_a=max(y)$, $p_i=min(x+y)$,
$p_a=max(x+y)$, $m_i=min(x-y)$und $m_a=max(x-y)$ ist.
\end{itemize}
}

\proc{double}{RoughnessContur}{Contur c, int diff}
\descr{
Es wird ein Maß für die ``Rauhigkeit'' der Kontur $c$ als Verhältnis der
Konturlänge zur Länge eines Polygonzuges bestimmt, der sich ergibt, wenn man
Konturpunkte mit dem Abstand $diff$ in der Richtungscodeliste durch
Geradensegmente verbindet.
}

Als weitere Konturmerkmale lassen sich die Momente (\see{Moments}) und
invariante Momente ({\see{Invariante Momente}) berechnen.

\subsection{Kontursuche}

Die Konturfolge beruht auf einer Klassifikation der
Bildpunkte in ``Objektpunkte'' und ``Untergrundpunkte''. 
Konturpunkte, die mittels der Konturfolge (\see{CalcContur}) gefunden 
werden, sind stets Randpunkte des Objektes, also Objektpunkte, in 
deren (Vierer-)Nachbarschaft sich
Untergrundpunkte befinden. Die Konturfolge umrundet ein Objekt, indem
jeweils Objekt-/Untergrund-Punktpaare gefunden werden. Der Kontur 
werden die Objektpunkte hinzugefügt. Möglich ist auch eine Beschreibung der
Kontur durch Richtungskodes, die die Bewegung von Konturpunkt zu Konturpunkt
beschreiben.

Bei der binären Klassifikation ``Objekt'' / ``Untergrund''
ist gesichert, daß immer geschlossene Konturen gefunden werden.
Bei der dreiwertigen Unterscheidung der Pixel nach Objekt/Untergrund/Undefiniert
können auch offene Konturen gefunden werden (\see{UndefiniertePunkte}). 

Neben dem reinen Grauwertbild wird bei der Konturfolge auch ein
``Markierungsbild'' betrachtet. Markierte Punkte, also Punkte für die
der Wert im Markierungsbild ungleich Null ist, und Punkte außerhalb 
des Bildes werden standardmäßig als Untergrund betrachtet. Damit
können bestimmte (markierte) Bildbereiche bei der Konturfolge 
ausgeschlossen werden. Dies wird normalerweise auch dazu benutzt, 
um bereits gefundene Objekte von der weiteren Suche auszuschließen.

Es besteht auch die Möglichkeit, markierte Punkte und Punkte außerhalb 
des Bildes als ``Objekt'' bzw. ``Undefiniert'' zu erklären.
``Undefinierte'' Punkte sind weder Objektpunkte noch Untergrundpunkte. Dies
kann für Punkte verwendet werden, an denen keine Entscheidung für Objekt oder
Untergrund möglich ist. Die Konturfolge bricht ab, wenn es kein 
Objekt-/Untergrund-Punktpaar zur Fortsetzung gibt. Das kann der Fall 
sein, wenn undefinierten Punkte erreicht werden. Es entstehen so
offenen Konturen.

Standardmäßig wird eine Schwellwert-Entscheidung bezüglich des Grauwertes
verwendet, um über Objekt und Untergrund zu entscheiden.

Um die Konturfolge möglichst flexibel zu gestalten, kann aber auch eine 
eigene Funktion zur Bildpunktklassifikation definiert werden, die
durch einen Funktionspointer übergeben wird. Die Funktion muß dem 
folgenden Prototyp genügen\\
\\
object\_rc {\bf cls}(const Image \&iv,int x,int y,int thr)\\
\\
wobei $x$ und $y$ die Koordinaten des zu klassifizierenden Bildpunktes sind
und $thr$ ein Parameter für die Klassifikation, der von den Funktionen zur
Startpunkt- und Kontursuche durchgereicht wird. Als Funktionswert muß der 
Wert $isobject$ für Objektpunkte und $isunderground$  für 
Untergrundpunkte zurückgegeben werden. Undefinierte Punkte 
geben $isunknown$ zurück.

Wenn für die Klassifikationsfunktion der Null-Pointer übergeben wird, 
wird die Funktion {\em ObjectThr} verwendet, die die übliche pegelbasierte
Konturfolge verwirklicht. Bildpunkte, deren Grauwerte größer oder gleich 
$thr$ sind, werden als Objektpunkte klassifiziert, sofern sie nicht 
markiert sind. Der Einsatz von {\em ObjectThrInv}
invertiert dieses Verhalten: Objekte sind Punkte mit niedrigem Grauwert, also
einem Grauwert kleiner als $trsh$.

Die Klassifikationsfunktion {\em LocalSegObj} kann zusammen mit den lokalen
Segmentierung mittels \see{LocalSeg} verwendet werden oder wenn andersweitig
bereits ein Bild vorliegt, dessen Pixelwerte zwischen Objekt (=1), Untergrund
(=0) und undefiniert (=2) unterscheiden. Der Parameter $thr$ hat hier keine
Bedeutung. 

\proch{double}{CalcThreshold}{const Hist \&h}{threshold.h}
\procf{double}{CalcThreshold}{const Hist \&h,double \&discmax}
\procf{int}{CalcThreshold}{const Histogram \&h}
\procf{int}{CalcThreshold}{const Histogram \&h,double \&discmax}
\descr{
Aus dem Histogramm $h$ wird nach der Methode von Otsu eine 
Schwelle zur Binarisierung und Konturdetektion bestimmt. Die Variable
$discmax$ wird mit dem maximierten Wert der Diskriminanzfunktion belegt.
}

\proc{int}{CalcThreshold}{Image img}
\procf{int}{CalcThreshold}{Image img,double \&discmax,int diff=1}
\descr{
Aus dem Bild $img$ wird mit Hilfe des Histogramms nach der Methode von Otsu
eine Schwelle zur Konturdetektion bestimmt. Die Variable
$discmax$ wird mit dem maximierten Wert der Diskriminanzfunktion belegt.
Für die Berechnung des Histogramms bei einem gegebenen Bild kann 
mit dem Parameter $diff$ festgelegt werden, dass nur jeder $diff$. Punkt 
verwendet wird.
}

\proch{int}{SearchStart}{Image iv,Image io,object\_rc (*cls)(),int thr,int
diff,int ps[2],int mode=HORZ}{contur.h}
\procf{int}{SearchStart}{Image iv,Image io,int (*cls)(),int thr,int
diff,IPoint \&ps,int mode=HORZ}
\descr{
Das Bild $iv$ wird auf Suchzeilen ($mode$=HORZ, Standard) bzw. 
Suchspalten ($mode$=VERT) im Abstand $diff$ nach einem Startpunkt für 
die Konturfolge durchsucht. Die Suche wird bei dem durch $ps$ gegebenen 
Punkt begonnen. Ein Startpunkt im Sinne dieser Funktion ist ein 
Objektpunkt, unter dessen 4-Nachbarn mindestens ein Untergrundpunkt ist. 
Der gefundene Startpunkt wird auf 
$ps$ zurückgegeben und dient als Ausgangspunkt für die Konturfolge 
mit CalcContur(). Wenn ein Startpunkt gefunden wurde, ist der 
Funktionswert OK, sonst NOT\_FOUND. Für die Startpunktsuche sind 
markierte Punkte und Punkte außerhalb des Bildes immer ``Undefiniert'' 
(=$isunknown$).
}

\proc{Contur}{CalcContur}{Image iv,Image io,object\_rc (*cls)(),int thr,int
ps[2],int lng=0,object\_rc marked=isunderground,object\_rc outside=isunderground}
\procf{Contur}{CalcContur}{Image iv,Image io,object\_rc (*cls)(),int thr,const IPoint \&ps,int lng=0,object\_rc marked=isunderground,object\_rc outside=isunderground}
\descr{
Ausgehend vom Startpunkt $ps$ wird die Kontur eines über die 4-Nachbarschaft
zusammenhängenden Objektes bestimmt. Für die Bildpunktklassifikation wird
normalerweise die gleiche Funktion und die gleiche Schwelle verwendet,
wie bei der Startpunktsuche (\see{SearchStart}). $lng$ gibt die maximal 
zulässige Konturlänge an, bei $lng=0$ (Standard) wird eine beliebig lange 
Kontur zugelassen.
Die wahlfreien Parameter $marked$ und $outside$ bestimmen, wie markierte
Punkte bzw. Punkte außerhalb des Bildes behandelt werden. 
Es gibt die Möglichkeiten:
\begin{itemize}
\item $isunderground$ - der Punkt wird als Untergrundpunkt betrachtet
\item $isobject$ - der Punkt wird als Objekt betrachtet
\item $isunknown$ - der Punkt wird als Undefiniert betrachtet
\end{itemize}
}

Am folgenden Beispiel soll die Verwendung der Funktionen demonstriert werden.

\begprogr\begin{verbatim}
/* Konturfolge */
#include <stdio.h>
#include <image.h>
#define PGL     10
#define IMG     "cont_t.tif"
void main(int argc,char *argv[])
{
  Image imgv = ReadImg(IMG);                     // Bild einlesen
  Image imgo = NewImg(imgv->xsize,imgv->ysize,3); // Markierungsbild anlegen
  ClearImg(imgo);                          // und löschen

  Show(OVERLAY,imgv,imgo);                 // Bilddarstellung

  IPoint start(0,0);                       // Startpunktsuche beginnt bei 0,0
  Contur c;

  while (SearchStart(imgv,imgo,NULL,PGL,1,start)==OK)
  {                                // solange weiterer Startpunkt gefunden
    c=CalcContur(imgv,imgo,NULL,PGL,start);   // Kontur verfolgen
    if (c.valid)
    {
      MarkContur(c,1,imgo);                // markieren 
    }
    else PutVal(imgo,start.x,start.y,1);   // als bearbeitet markieren
  }
  GetChar();
}
\end{verbatim}
\endprogr

\proc{Contur}{GetContur}{int ps[2],int (*cls)(int,int,void*),void *par,int lng}
\descr{
Ausgehend von dem Startpunkt $ps$ wird die Kontur der Funktion $cls()$
verfolgt. Für Objektpunkte muß der Funktionswert von $cls()$ Null sein
und für Punkte außerhalb des Objektes kleiner Null. 
Der Funktionswert -4 bewirkt einen Abbruch der Konturfolge, so daß 
eine offene Kontur erzeugt wird. Der Pointer $par$ dient zur 
Parameterübergabe an die Funktion $cls()$ und wird an diese durchgereicht.
}

\proch{int}{LocalSeg}{Image source,Image oimg,int neighb,int level}{lseg.h}
\descr{Die Funktion erzeugt aus dem Bild $source$ ein Bild $oimg$, in dem
  jedes Pixel einen Wert hat, der entscheidet ob das Pixel im
  Originalbild $source$ ein Objekt- (=LSobject) oder Untergrundp>ixel (=LSunderground) ist, 
  oder ob dies undefiniert (=LSunknown) ist. Die Entscheidung erfolgt, indem
  in der Umgebung $neighb$ das Maximum und das Minimum der Grauwerte gesucht
  werden. Überschreitet die Differenz vom Maximum und Minimum die Schwelle
  $level$, so wird anhand des Schwellwertes $(Maximum+Minimum)/2$ über Objekt
  und Untergrund entschieden. Sonst ist der Punkt undefiniert. Unter
  Verwendung der Funktion \see{LocalSegObj} kann das so erzeugte Bild für die
  Konturfolge verwendet werden.}

\proch{int}{LocalSeg}{Image source,Image oimg,int neighb,int level1,int level2}{lseg.h}
\descr{Analog zur Funktion LocalSeg mit einem Parameter $level$ erzeugt die
  Funktion aus dem Bild $source$ ein Bild $oimg$, in dem
  jedes Pixel eine einen Wert hat, der entscheidet ob das Pixel im
  Originalbild $source$ ein Objekt- (=LSobject) oder Untergrundpixel
  (=LSunderground) ist, oder ob dies undefiniert (=LSunknown) ist. Zusätzlich
  gibt es die Entscheidungsmöglichkeit schwaches Objekt (=LSweakobject) und
  schwacher Untergrund (=LSweakunderground).
  Die Entscheidung erfolgt, indem in der Umgebung $neighb$ das Maximum 
  und das Minimum der Grauwerte gesucht werden. 
  Überschreitet die Differenz vom Maximum und Minimum die Schwelle
  $level1$, so wird anhand des Schwellwertes $(Maximum+Minimum)/2$ über Objekt
  und Untergrund entschieden.
  Wird die nur (geringere) Schwelle $level2$ überschritten, so werden die
  Pixel als schwaches Objekt oder schwacher Untergrund klassifiziert.
  Sonst ist der Punkt undefiniert. 
  Unter Verwendung der Funktion \see{LocalSegObj} bzw. \see{LocalSegObjHigh} 
  können im so erzeugten Bild mittels Konturfolge schwache oder starke Objekte 
  gefunden werden.}


\subsection{Kantendetektion}

\proc{int}{GradThreshold}{Image img}
\descr{Aus dem Bild $img$ wird ein geeigneter Schwellwert für die
  Kantenverfolgung bestimmt. Für die Startpunktsuche kann dieser Wert direkt
verwendet werden, für die Kantenverfolgung kann er auf ca. ein Drittel
reduziert werden.
}
\proc{int}{SearchGradStart}{Image iv,Image io,int thr,int diff,int
ps[2],int mode}
\descr{
Das Bild $iv$ wird auf Suchzeilen ($mode$=HORZ) bzw. Suchspalten ($mode$=VERT)
im Abstand $diff$ nach einem lokalen Maximum des Gradientenbetrages
gesucht. Die Suche wird bei dem durch $ps$ gegebenen Punkt begonnen. Wenn
der Gradientenbetrag in einem lokalen Maximum den Wert $thr$ übersteigt, wird
der Punkt als Startpunkt für die Kantenverfolgung auf $ps$
zurückgegeben. Wenn ein Startpunkt gefunden wurde, ist der Funktionswert OK,
sonst NOT\_FOUND.
}
\proc{int}{SearchCircStart}{Image iv,Image io,int c[2],int r,int pgl,int
ps[2]}
\descr{
Auf einem Kreis um den Punkt $c$ mit dem Radius $r$ wird nach einem lokalen
Maximum des Gradientenbetrages gesucht. Wenn der Gradientenbetrag in einem
lokalen Maximum den Wert $thr$ übersteigt, wird der Punkt als Startpunkt für
die Kantenverfolgung auf $ps$ zurückgegeben. Wenn ein Startpunkt gefunden
wurde, ist der Funktionswert OK, sonst NOT\_FOUND.
}
\proc{int}{SearchCStart}{Image iv,Image io,Contur c,int *ptr,int pgl,int ps[2]}
\descr{
Auf der durch $c$ gegebenen Kurve wird nach einem lokalen Maximum des
Gradientenbetrages gesucht. Die Suche wird bei dem $*ptr$-ten Punkt
begonnen. Wenn der Gradientenbetrag in einem lokalen Maximum den Wert $thr$
übersteigt, wird der Punkt als Startpunkt für die Kantenverfolgung auf $ps$
zurückgegeben und die aktuelle Position wird für eine spätere Weitersuche auf
$*ptr$ bereitgestellt. Wenn ein Startpunkt gefunden wurde, ist der Funktionswert OK,
sonst NOT\_FOUND.
}
\proc{Contur}{CalcGradContur}{Image iv,Image io,int thr,int maxg,int ps[2],int lng}
\descr{
Ausgehend vom Startpunkt $ps$ wird eine Kante (Kurve mit maximalem
Gradientenbetrag) verfolgt (in beiden Richtungen), solange der
Gradientenbetrag größer als $thr$ ist. Die Bildpunkte mit dem größeren
Grauwert liegen jeweils rechts der Kante. Kleinere Lücken der 
Länge $maxg$ werden übersprungen, wenn noch ein lokales Maximum 
des Gradientenbetrages vorhanden
ist. Der Wert $thr$ kann kleiner sein, als bei der Startpunktsuche
(ca. 1/3). $lng$ gibt die maximal zulässige Konturlänge an, bei $lng=0$ wird
eine beliebig lange Kontur zugelassen.
}

\begprogr\begin{verbatim}
/* Kantendetektion */
#include <stdio.h>
#include <image.h>
#define PGL     10
#define IMG     "cont_t.tif"
void main(int argc,char *argv[])
{
  Image imgv,imgo;
  Contur c;
  int ps[2]={0,0};
  imgv=ReadImg(IMG,NULL);                    /* Bild einlesen */
  imgo=NewImg(imgv->xsize,imgv->ysize,63);   /* Overlaybild anlegen */
  ClearImg(imgo);
  Show(OVERLAY,imgv,imgo);                   /* Bilddarstellung */
  Display(ON);
  while(SearchGradStart(imgv,imgo,PGL,5,ps,HORZ)==OK)
  {                                          /* Bild nach Startpunkt durchsuchen */
    c=CalcGradContur(imgv,imgo,PGL,0,ps,0);  /* Kontur verfolgen */
    if (c.valid)
    {
      MarkContur(c,1,imgo);                  /* Kontur markieren */
    }
    ps[0]++;
  }
  getchar();
  Display(OFF);
}
\end{verbatim}\endprogr

\subsection{Liniendetektion}

\proc{int}{RidgeThreshold}{Image img}
\descr{Aus dem Bild $img$ wird ein geeigneter Schwellwert für die Linienverfolgung
bestimmt. Für die Startpunktsuche kann dieser Wert direkt verwendet werden,
für die Linienverfolgung kann er auf ca. ein Drittel reduziert werden.
}
\proc{int}{SearchRidgeStart}{Image imgv,Image imgo,int thr,int diff,int ps[2],int
mode}
\descr{Das Bild wird zeilenweise ($mode$=HORZ) bzw. spaltenweise ($mode$=VERT), beginnend
bei $ps$ nach einem Startpunkt für die Linienverfolgung durchsucht. Die
Suchzeilen bzw. -Suchspalten haben den Abstand $diff$. Die Startpunktsuche stopt
an Punkten, deren Laplacewert in x- bzw y-Richtung größer als $thr$ ist und in
deren direkter Nachbarschaft kein Punkt markiert ist. Der Schwellwert $thr$
sollte mit Hilfe der Funktion RidgeThreshold bestimmt werden. Wenn ein Startpunkt
gefunden wurde, ist der Funktionswert OK, sonst NOT\_FOUND.
}
\proc{Contur}{CalcRidgeContur}{Image imgv,Image imgo,int thr,int ps[2],int lng}
\descr{Ausgehend vom Startpunkt $ps$ wird (in beiden Richtungen) eine Linie
(``Kammlinie im Grauwertgebirge'') verfolgt. Die Linienverfolgung wird
abgebrochen, wenn sich eine Fortsetzung nicht mindestens um den Wert $thr$ vom
Untergrund abheben würde oder wenn ein markierter Punkt erreicht wird. Der
Schwellwert $thr$ sollte mit Hilfe der Funktion RidgeThreshold() bestimmt
werden.
}

\subsection{Konturlisten}
\hypertarget{ConturList}{}

Für die Analyse komplexerer Szenen wird die Verwaltung von 
Konturen in Listen unterstützt. Die Klasse \class{ConturList} liefert
die Methoden zur Verwaltung von Konturen in Listen.

\proc{}{ConturList::ConturList}{}
\descr{Anlegen einer leeren Konturliste.}
\proc{}{ConturList::ConturList}{const ConturList\& cl}
\descr{Anlegen einer neuen Konturliste als Kopie der 
übergebenen Konturliste $cl$.}

\proc{ConturList \&}{ConturList::operator=}{const ConturList\& cl}
\descr{Kopieren der Konturliste $cl$ in die Konturliste.}

\proc{int}{ConturList::Add}{const Contur \&c}
\descr{Die Kontur $c$ wird an die Liste angehängt. In der
Liste wird eine Kopie von $c$ angelegt, so daß $c$ freigegeben oder
andersweitig weiter verwendet werden kann.}
\proc{int}{ConturList::Del}{int i}
\descr{Die Kontur mit Index $i$ wird aus der Konturliste entfernt.}

\proc{Contur *}{ConturList::GetContur}{int i}
\descr{Es wird ein Zeiger auf die Kontur Nummer $i$ zurückgegeben.}

Die Anzahl der Konturen kann über die konstante Referenz 
ConturList::number abgefragt werden.

Das folgende Beispiel zeigt die Verwendung einer Konturliste:

\begprogr
\begin{verbatim}
/* Arbeit mit Konturlisten */
#include <stdio.h>
#include <image.h>

#define PGL     10
#define IMG     "cont_t.tif"

void main(int argc,char *argv[])
{
  Image imgv,imgo;
  Contur c;
  ConturList cl;
  int ps[2]={0,0};
  imgv=ReadImg(IMG,NULL);                  // Bild einlesen
  imgo=NewImg(imgv->xsize,imgv->ysize,63); // Overlaybild anlegen
  ClearImg(imgo);                          // und löschen

  while(SearchStart(imgv,imgo,NULL,PGL,5,ps,HORZ)==OK)
  {                                        // Bild nach Startpunkt durchsuchen
    c=CalcContur(imgv,imgo,NULL,PGL,ps,0); // Kontur verfolgen 
    if (c.valid)                           // Konturfolge erfolgreich ?
    {
        cl.Add(c);                         // Kontur in liste eintragen
        MarkContur(c,1,imgo);              // markieren 
    }
    else PutVal(imgo,x,y,1);               // Punkt als bearbeitet markieren
  }
                                           // nun alle Konturen füllen
  for (i=0;i<cl.number;i++)                // Schleife über alle Konturen
  {
    FillRegion(*(cl.GetContur(i)),2,imgo);
  }

// jetzt Konturliste mit einer leeren Konturliste überschreiben
// alle gespeicherten Konturen gehen verloren und der Speicher
// wird freigegeben

  cl=ConturList();
}
\end{verbatim}
\endprogr

\subsection{Hilfsfunktionen zur Arbeit mit Konturen}
\proc{bool}{pointInside}{double x,double y,const Contur \&c}
\procf{bool}{pointInside}{const Point \&p,const Contur \&c}
\proc{bool}{pointInside}{double x,double y,const Matrix \&pl}
\procf{bool}{pointInside}{const Point \&p,const Matrix \&pl}
\descr{
 Für einen gegebenen Punkt wird entschieden, ob dieser innerhalb einer 
gegebenen Kontur bzw. innerhalb eines gegebenen Polygons liegt. Das Gebiet 
innerhalb der Kontur/des Polygons ist durch die stetige Verbindungslinie 
benachbarter Punkte definiert (einschließlich dieser Linie).}

\proc{}{conturFromPolygon}{const Matrix \&pl,Contur \&c}
\descr{
Aus dem in der Punktliste $pl$ gegebenen Polygon wird eine Kontur
erstellt. Die Konturpunkte sind dabei die äußersten Pixel, deren Mittelpunkte
noch innerhalb des Polygons liegen (einschließlich Rand).}

\subsection{Arbeiten mit Richtungen}
\hypertarget{Freeman}{}

Diskrete Richtungen werden oft mit dem Freeman-Code angegeben. 

\begin{figure}
  \epsfig{figure=freeman.eps,width=5cm}
  \label{fig:freeman}
  \caption{Die Freeman-Richtungskodes}
\end{figure}

Die Klasse \class{Freeman} enthält einen entsprechenden Richtungswert und erlaubt Operationen auf diesem,
wobei der zyklische Character der Richtungscodes berücksichtigt wird. Richtungen vom Typ Freeman 
lassen sich durch einen Typwandlungskonstruktor aus int-Werten erzeugen und mittels der Methode 
Int() wieder in int-Werte umwandeln.

Die Operationen zur Verschiebung eines Punktes beziehungsweise zur Berechnung des Nachbarn in
einer Richtung verwenden die folgenden Offsets.

\begin{tabular}{|l|c|c|c|c|c|c|c|c|} \hline
$dir$     & 0 & 1 & 2 & 3 & 4 & 5 & 6 & 7 \\ \hline
$\Delta x$ & 1 & 1 & 0 & -1 & -1 & -1 & 0 & 1 \\ \hline
$\Delta y$ & 0 & 1 & 1 & 1 & 0 & -1 & -1 & -1 \\ \hline
\end{tabular}

\subsubsection{Konstruktoren und Typwandlung}

\proch{}{Freeman::Freeman}{}{freeman.h}
\descr{Standard-Konstruktor. Erzeugt eine Richtung mit dem Freemancode 0.}

\proch{}{Freeman::Freeman}{int dir}{freeman.h}
\descr{Typwandlungs-Konstruktor. Erzeugt eine Richtung mit dem Freemancode $dir$. Werte außerhalb
des Intervalls 0..7 werden Modulo 8 behandelt.}

\proch{}{Freeman::Freeman}{const Freeman \&sec}{freeman.h}
\descr{Kopierkonstruktor}

\proc{int}{Freeman::Int}{} 
\descr{Liefert den Richtungscode als int-Wert im Intervall 0..7 .}

\subsubsection{Anwendung auf Punkte}

\proch{void}{Freeman::move}{int x,int y,int \&xn,int \&yn}{freeman.h}
\descr{Berechnet das Ergebnis der Verschiebung des Punktes $(x,y)$ um 
einen Schritt und gibt das Ergebnis auf $xn$ und $yn$ zurück.}

\proch{void}{Freeman::move}{int \&x,int \&y}{freeman.h}
\descr{Verschiebt den Punkt (x,y) um einen Schritt.}

\proch{void}{Freeman::move}{IPoint p1,IPoint \&p2}{freeman.h}
\descr{Berechnet das Ergebnis der Verschiebung des Punktes $p1$ um 
einen Schritt und gibt das Ergebnis auf $p2$ zurück.}

\proch{void}{Freeman::move}{IPoint \&p}{freeman.h}
\descr{Verschiebt den Punkt $p$ um einen Schritt.}

\proch{IPoint}{Freeman::step}{IPoint \&p}{freeman.h}
\descr{Berechnet das Ziel eines Schrittes vom Punkt $p$.}

\subsubsection{Operatoren}
Die Operatoren rechnen analog zum Typ int mit den Richtungskodes, berücksichtigen 
aber, dass Richtungskodes zyklisch sind ($7+1 \rightarrow 1$).

\begin{tabular}{ll}
Operator      & Bedeutung \\
\verb#a + b#  & Summe \\
\verb#a += b# & \verb#a = a + b# \\
\verb#a - b#  & Differenz \\
\verb#a -= b# & \verb#a = a - b# \\
\verb#a++#    & \verb#a += 1# \\
\verb#++a#    & \\
\verb#a--#    & \verb#a -= 1# \\
\verb#--a#    & \\
\verb#a==b#   & Test auf Gleichheit \\
\end{tabular}

\proch{Freeman}{Inverse}{}{freeman.h}
\descr{Berechnet die entgegengesetzte Richtung.}


\subsection{Regionen}
\hypertarget{Region}{}

Regionen sind Mengen von Bildpunkten. Regionen beschreiben Gebiete in Bildern,
die nicht notwendigerweise zusammenhängen müssen. Genauso ist es möglich, dass
Regionen Löcher aufweisen. Mittels der Klasse \class{Region} können solche Gebiete
erfaßt, dargestellt und bearbeitet werden. Ausgangspunkt für die Berechnung von
Gebieten können \class{Konturen} sein. Es ist auch möglich, Regionen aufzubauen,
indem zu anfänglich leeren Regionen Punkte(Pixel), Fenster (\class{Window}) 
und andere Regionen hinzugefügt werden.\\
Zur Anwendung von Regionen siehe auch \see{RegionGrow}.

\subsubsection{Konstruktoren und Konvertierungen}

\ctor{Region}{}{region.h}
\descr{Legt eine leere Region an.}

\ctor{Region}{const Region \&r}{region.h}
\descr{Legt eine Kopie der Region $r$ an.}

\ctor{Region}{const Contur \&c}{region.h}
\descr{Legt eine Region an, die die inneren und Randpunkte der Kontur $c$ umfasst.}

\subsubsection{Regionen-Arithmetik}
Unter Regionen-Arithmetik werde die Konstruktion von Regionen durch Hinzufügen
und Entfernen von Punkten(Pixel), Fenstern (achsparallele Rechtecke) und
Regionen verstanden. 
Bei Verwendung der Ausdruck-Schreibweise (Operatoren +, - und \&) ist zu 
beachten, dass die dabei notwendige Erzeugung temporärer Objekte
Zeit- und Speicherplatz-aufwendig ist.

\method{void}{add}{int x,int y}
\methodf{void}{add}{const IPoint \&p}
\methodf{void}{add}{const Point \&p}
\descr{Fügt den durch die Koordinaten $x$ und $y$ bzw. durch Point $p$
  gegebenen Punkt der Region hinzu.}

\method{void}{add}{int x1,int y1,int x2,int y2}
\methodf{void}{add}{const Window \&w}
\descr{Fügt das durch die Eckpunkte $x1,y1$,$x2,y2$ gegebene achsparallele
  Rechteck bzw. das \class{Window} w der Region hinzu.}

\method{void}{add}{const Region \&r}
\descr{Fügt die Region $r$ der Region hinzu.}

\method{const Region \&}{operator +=}{const Region \&r2}
\descr{Vereinigung mit der Region r2.}

\method{Region}{operator +}{const Region \&r1,const Region \&r2}
\descr{Vereinigt die beiden Regionen und gibt die vereinigte Region als
  Funktionswert zurück.}

\method{void}{del}{int x,int y}
\methodf{void}{del}{const Point \&p}
\methodf{void}{del}{const IPoint \&p}
\descr{Entfernt den durch die Koordinaten $x$ und $y$ bzw. durch Point p
  gegebenen Punkt aus der Region.}

\method{void}{del}{int x1,int y1,int x2,int y2}
\descr{Entfernt das durch die Eckpunkte $x1,y1$,$x2,y2$ gegebene 
achsparallele Rechteck aus der Region.}

\method{void}{del}{const Region \&r}
\descr{Entfernt die Region $r$ aus der Region.}

\method{Region}{operator -}{const Region \&r1,const Region \&r2}
\descr{Bildet eine neue Region aus Punkten, die in $r1$ enthalten und in $r2$
  nicht enthalten sind und gibt diese Region als Funktionswert zurück.}

%\method{Region}{operator \&}{const Region \&r1,const Region \&r2}
%\descr{Führt die Und-Verknüpfung der Regionen $r1$ und $r2$ aus.}

\subsubsection{Abfrage von Eigenschaften}

\method{bool}{inside}{int x,int y}
\descr{Gibt zurück, ob der Punkt $x,y$ in der Region enthalten ist}

\method{bool}{isEmpty}{void}
\descr{Gibt zurück, ob die Region leer ist, also keine Punkte enthält}

\method{int}{getArea}{}
\descr{Gibt die Fläche der Region (=Zahl der Pixel) zurück.}

\method{int}{CalcMoments}{Moments \&m}
\descr{Berechnet die Momente der Region und legt das Ergebnis in der
  Datenstruktur m von Typ \class{Moments} ab.}

\method{void}{getPoints}{\vector{IPoint} \&points}
\descr{Gibt eine Punktliste aller Punkte der Region zurück.}
%%% Local Variables: 
%%% mode: latex
%%% TeX-master: "icedoc_base"
%%% End: 
