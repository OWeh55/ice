\nsection{Kameramodelle}{cameraModel}
\subsection{Datenstruktur Camera}
\label{Camera}
\hypertarget{Camera}{}
Die Datenstruktur \class{Camera} beschreibt die Eigenschaften einer Kamera. Dies
umfaßt die inneren und äußeren Parameter der Kamera bezüglich der projektiven
Abbildung sowie die Parameter der Verzeichnung (\see{Distortion}).

Die äußeren Kameraparameter beschreiben eine Transformation der
Weltkoordinaten eines Punktes in die Kamerakoordinaten durch eine Bewegung. 
Der Ursprung des Kamerakoordinatensystems sitzt im Projektionszentrum der 
projektiven Abbildung der Kamera, die Z-Achse liegt auf der optischen Achse
der Kamera von der Kamera weg, die X-Achse ist parallel zu horizontalen 
Achse der Bildebene, die Y-Achse ist parallel zur vertikalen Bildachse, aber
umgekehrt orientiert. Dies trägt der Tatsache Rechnung, daß in der
Bildverarbeitung üblicherweise die vertikale Achse nach unten gerichtet ist,
also entgegengesetzt zur üblichen Orientierung im Raum.
Die Bewegung (Transformation Weltkoordinaten in Kamerakoordinaten) wird
beschrieben durch die Eulerschen Winkel einer Drehung um die Z-Achse des
Weltkoordinatensystems ($\gamma$), Drehung um die Y-Achse des
Weltkoordinatensystems ($\beta$) und einer Drehung um die X-Achse des
Weltkoordinatensystems ($\alpha$), ausgeführt in dieser Reihenfolge und eine
Verschiebung. 

Die projektive 3D-2D-Transformation von Kamerakoordinaten in Bildkoordinaten
$(u,v)$ wird durch die inneren Kameraparameter charakterisiert. 
Dies sind die Kammerkonstante $f$, die Anisotropie $a$, die die abweichende
Skalierung bezüglich der Bildachsen beschreibt, die Scherung $s$ in u-Richtung
sowie der Hauptpunkt $(u_0,v_0)$. Die resultierende Transformationsmatrix ist
\centerline{
$\left( 
\begin{array}{cccc}
f & s \cdot f & u_0 & 0\\
0 & -a \cdot f & v_0 & 0\\
0 & 0 & 1 & 0
\end{array} 
 \right)$}

Für die Verzeichnung können alle vorhandenen Verzeichnungsmodelle
(\see{Distortion}) genutzt werden.

\subsubsection{Konstruktoren, Zuweisung, Destruktor}

\proch{}{Camera::Camera}{int dtyp=1}{camera.h}
\descr{Legt eine Kamera mit dem gegebenen Verzeichnungstyp $dtyp$ an (\see{Distortion}). }

\proch{}{Camera::Camera}{const Vector \&v,int dtyp=1}{camera.h}
\descr{Legt eine Kamera mit dem gegebenen Verzeichnungstyp $dtyp$ an
  (\see{Distortion}) und initialisiert diese mit den im Vektor $v$ gegebenen
  Parametern 
(\see{Camera::Set},\see{Camera::MakeVector})}

\proch{}{Camera::Camera}{const Camera \&c}{camera.h}
\descr{Kopierkonstruktor.}

\proch{Camera \&}{operator=}{const Camera \&c}{camera.h}
\descr{Zuweisungsoperator: Setzen einer Kamera (linke Seite)
auf die Werte der Kamera der rechten Seite.}

\subsubsection{Zugriff auf Komponenten und Parameter}

\proc{void}{Camera::Set}{double fp,double ap,double sp,
                   double u0p,double v0p,const Distortion \&d}
\procf{void}{Camera::Set}{double fp,double ap,double sp,double u0p,double v0p}
\descr{Setzt die internen Parameter einer Kamera entsprechend der gegebenen
Werte, mit bzw. ohne die Verzeichnungsparameter.}
\proc{void}{Camera::SetExt}{double dxp,double dyp,double dzp,double ap,double
bp,double cp}
\descr{Setzt die äußeren Kamera-Parameter.}

\proc{void}{Camera::Get}{double \&fp,double \&ap,double \&sp,double \&u0p,double \&v0p}
\descr{Ermittelt die internen Parameter der Kamera.}
\proc{void}{Camera::GetExt}{double \&dxp,double \&dyp,double \&dzp,
double \&ap,double \&bp,double \&cp}
\descr{Gibt die äußeren Kamera-Parameter zurück.}

\proc{const Distortion}{Camera::Dist}{}
\descr{Gibt eine konstante Referenz der Verzeichnung der Kamera zurück.}
\proc{int}{Camera::DistType}{}
\descr{Gibt den Type der Verzeichnung als Integer-Wert zurück.}

\proc{void}{Camera::SetDist}{const Distortion \&d}
\descr{Setzt die Verzeichnungsparameter der Kamera.}

Die speziellen Zugriffsmethoden, die die Parameter der Kamera in einem
\bsee{Vector} ablegen bzw. aus einem \bsee{Vector} laden dienen der
Nutzung mit Funktionen, die Vektoren als Parameter benötigen, wie zum Beispiel
die Funktion \bsee{LMDif}.

\proch{Vector}{Camera::MakeVector}{int what=Camera::all}{camera.h}
\descr{Erzeugt einen Vektor, in dem die Parameter der Kamera gespeichert
sind. Mittels Parameter $what$ kann festgelegt werden, welche Parameter im
Vektor abgelegt werden:
\begin{itemize}
\item[Camera::all] Es werden alle Parameter abgelegt.
\item[Camera::internal] Es werden die internen Parameter abgelegt
(einschließlich Verzeichnung). 
\item[Camera::external] Es werden die äußeren Parameter abgelegt.
\end{itemize}
}

\proch{void}{Camera::Set}{const Vector \&v,int what=Camera::all}{camera.h}
\descr{Setzt die Parameter etsprechend der Daten im Vektor $v$. Zur Bedeutung
des Parameters $what$ siehe \see{Camera::MakeVector}.}

\subsubsection{Transformationen}

\proch{Vector}{Camera::Transform}{const Vector \&v}{camera.h}
\descr{Abbildung eines als Vektor gegebenenen Raumpunktes in die
Bildebene und Rückgabe als Vektor.}

\proch{Point}{Camera::Transform}{const Vector3d \&v}{camera.h}
\descr{Abbildung eines als Vektor3d gegebenenen Raumpunktes in die
Bildebene und Rückgabe als Point.}

\proch{Point}{Camera::Transform}{const Point3d \&v}{camera.h}
\descr{Abbildung eines als Point3d gegebenenen Raumpunktes in die
Bildebene und Rückgabe als Point.}

\proc{void}{Camera::Transform}{double x,double y,double z,double \&u,double \&v}
\descr{Abbildung eines durch die Koordinaten x, y und z gegebenenen
Raumpunktes in die Bildebene. Die Rückgabe der Koordinaten erfolgt über die
Parameter u und v.}

\proc{Line3d}{Ray}{const Point \&bp}
\procf{Line3d}{Ray}{double u,double v}
\procf{Line3d}{Ray}{const Vector \&bp}
\descr{Berechnet zu einem gegebenen Bildpunkt den Sehstrahl.}

\subsubsection{Ein- und Ausgabe}

\proch{string}{Camera::toString}{const string \&del}{camera.h}
\descr{Erzeugt einen beschreibenden String. $del$ wird als Trennzeichen
zwischen den Parametern verwendet. Typisch ist die Verwendung des
Zeilenvorschubs für eine zeilenweise Ausgabe der Parameter (default) oder 
Leerzeichen, Komma oder ähnliches für die Ausgabe auf einer Zeile.}

\subsection{Planare Selbstkalibrierung mit einem Muster nach Zhang}

Mit der Funktion \see{DrawPattern} erstellt man das zu verwendende planare
Kalibriermuster, welches man dann auf z.B. ein weißes A4 Blatt druckt oder
gleich auf dem planaren Monitor darstellt. Wird das Muster auf Papier
gedruckt, muss man darauf achten, dass das Papier planar aufliegt, es darf sich
z.B. nicht wellen usw. Von diesem Muster macht man mit einer Kamera
mindestens drei Aufnahmen aus verschiedenen Blickrichtungen ohne die internen
Kameraparameter dabei zu verändern.\\
Wichtig:\\
-die internen Parameter dürfen sich nicht ändern, z.B. nicht zoomen von
einer Aufnahme zur anderen!\\
-mindestens 3 Aufnahmen aus 3 verschiedenen Blickrichtungen. Dabei müssen
die Blickrichtungen verschieden bez. 2 Rotationsachsen sein. Liegt z.B. das
Muster auf einem Tisch, so reicht es nicht nur Aufnahmen zu machen, indem man
um den Tisch geht und aus gleicher Höhe das Muster aufnimmt, sondern man muss
auch die Höhe der Kamera noch variieren, sonst entstehen numerische Instabilitäten.\\
 Weiterhin ist bei den Aufnahmen zu beachten, dass das Muster einen großen
Teil des Bildes einnimmt. Mit der Funktion \see{Calibrate} kann man dann
einfach die inneren und äußeren Kameraparameter bestimmen. Falls es
dabei oder bei den anderen Funktionen zu Fehlermeldungen kommt, kann man den
Prekompilerbefehle \#define selfcalib\_debug in der selfcalib.h setzten und
erhält somit eine genauere Textausgabe in der Konsole sowie eine
Visualisierung der kritischen Zwischenschritte mittels zweier Images, die als
Defaulteinstellung als ungültig gesetzt sind. Nach jedem relevanten
Zwischenschritt bestätigt man mit $<$Enter$>$. Dabei kann man vor dem
Aufruf der folgenden Funktionen ein Textfenster mit der Funktion
\see{OpenAlpha} und ein Visualisierungsfenster mit der Funktion \see{Display}
bereitstellen und die beiden Bilder $debug\_image$ und $debug\_mark$ mit den
gleichen Dimensionen wie die Aufnahmen initialisieren. Mittels dem Aufruf
Show(OVERLAY,debug\_image,debug\_mark) bekommt man die Bilder überlagert
angezeigt. Für Verzeichnungen gibt es verschiedene Modelle, es wird immer
das Modell verwendet, das im Camera object festgelegt wurde, entweder
explizit, oder der Standardkonstruktur setzt einen DEFAULT-Typ.
\\
Im Verzeichnis applications findet man ein Rahmenprogramm
calib\_zhang.cpp
\\

\proch{int}{Calibrate}{vector$<$Image$>$ \&images, Camera \&c, bool dist=true, Image debug\_image = Image(), Image debug\_mark = Image()}{selfcalib.h}
\descr{Kalibrierung aus mindestens 3 gleichgroßen Aufnahmen des Testmusters. Mit $images$ werden diese Bilder als ein Vektor übergeben. $c$ ist vom Typ Camera und es werden die inneren Kameraparameter zurückgegeben. Wenn $dist=true$ ist, dann werden die Verzeichnungen berücksichtigt und mit zurückgegeben.}

\proch{int}{Calibrate}{vector$<$Image$>$ \&images, vector$<$Camera$>$ \&cv, bool dist=true, Image debug\_image = Image(), Image debug\_mark = Image()}{selfcalib.h}
\descr{Kalibrierung aus mindestens 3 gleichgroßen Aufnahmen des Testmusters. Mit $images$ werden diese Bilder als ein Vektor übergeben. $cv$ ist ein Vektor aus Kameras und es werden innere und aüßere Kameraparameter zurückgegeben, wobei die Dimensionen der beiden Vektoren gleich sein muss. Wenn $dist=true$ ist, dann werden die Verzeichnungen berücksichtigt und mit zurückgegeben.}

\proch{int}{AssignCalibPattern}{Image \&image, Matrix \&imagepoints, Matrix \&worldpoints, Image debug\_image = Image(), Image debug\_mark = Image()}{selfcalib.h}
\descr{Bestimmung der Koordinaten des Musters in der Aufnahme und Rückgabe der theoretischen Koordinaten des Musters in Bezug auf die Bilddimensionen ohne Berücksichtigung einer Verzeichnung. Mit $image$ wird eine Aufnahme des Testmussters übergeben. Mit $imagepoints$ wird eine Matrix(50,2) der ermittelten Bildkoordinaten und mit $worldpoints$ wird eine Matrix(50,3) der Weltkoordinaten des Testmusters zurückgegeben.}

\proch{int}{AssignCalibPattern}{Image \&image, Matrix \&imagepoints, Matrix \&worldpoints, Camera \&c, Image debug\_image = Image(), Image debug\_mark = Image()}{selfcalib.h}
\descr{Bestimmung der Koordinaten des Musters in der Aufnahme und Rückgabe der theoretischen Koordinaten des Musters in Bezug auf die Bilddimensionen mit Berücksichtigung einer Verzeichnung gegeben durch $c$. Mit $image$ wird eine Aufnahme des Testmussters übergeben. Mit $imagepoints$ wird eine Matrix(50,2) der ermittelten Bildkoordinaten und mit $worldpoints$ wird eine Matrix(50,3) der Weltkoordinaten des Testmusters zurückgegeben.}

\proch{int}{ComputeHomography}{Matrix \&imagepoints, Matrix \&worldpoints, Trafo \&H}{selfcalib.h}
\descr{Bestimmung der Homographie aus den Koordinaten des Musters im Bild und der theoretischen Koordinaten des Musters. Mit $imagepoints$ wird eine Matrix(50,2) der ermittelten Bildkoordinaten und mit $worldpoints$ wird eine Matrix(50,3) der Weltkoordinaten des Testmusters übergeben. Mit $H$ bekommt man die Homographie zurück.}

\proch{int}{CalibrateWithPattern}{vector$<$Matrix$>$ \&imagepoints, vector$<$Matrix$>$ \&worldpoints, Camera \&c}{selfcalib.h}
\descr{Kalibrierung aus Koordinatenpaaren gegeben als jeweils zwei Felder von mindestens 3 Aufnahmen des Testmusters. Mit einem Vektor aus $imagepoints$ der Matrizen(50,2) werden die ermittelten Bildkoordinaten aus verschiedenen Bildern des Testmusters und mit einem Vektor aus $worldpoints$ der Matrizen(50,3) werden die Weltkoordinaten des Testmusters übergeben. $c$ ist vom Typ Camera und es werden die inneren Kameraparameter zurückgegeben.}

\proch{int}{CalibrateWithPattern}{vector$<$Matrix$>$ \&imagepoints, vector$<$Matrix$>$ \&worldpoints, vector$<$Camera$>$ \&cv}{selfcalib.h}
\descr{Kalibrierung aus Koordinatenpaaren gegeben als jeweils zwei Felder von mindestens 3 Aufnahmen des Testmusters. Mit einem Vektor aus $imagepoints$ der Matrizen(50,2) werden die ermittelten Bildkoordinaten aus verschiedenen Bildern des Testmusters und mit einem Vektor aus $worldpoints$ der Matrizen(50,3) werden die Weltkoordinaten des Testmusters übergeben. $cv$ ist ein Vektor aus Kameras, wobei die Dimensionen aller Vektoren gleich sein muss, und es werden innere und aüßere Kameraparameter zurückgegeben.}

\proch{int}{CalibrateWithHomographies}{vector$<$Trafo$>$ \&H, Camera \&c}{selfcalib.h}
\descr{Kalibrierung aus mindestens 3 Homographien die als Vektor $H$ übergeben werden. $c$ ist vom Typ Camera und es werden die inneren Kameraparameter zurückgegeben.}

\proch{int}{CalibrateWithHomographies}{vector$<$Trafo$>$ \&H, vector$<$Camera$>$ \&cv}{selfcalib.h}
\descr{Kalibrierung aus mindestens 3 Homographien die als Vektor $H$ übergeben werden. $cv$ ist ein Vektor aus Kameras, wobei die Dimensionen der beiden Vektoren gleich sein muss, und es werden innere und aüßere Kameraparameter zurückgegeben.}

\proch{Image}{DrawPattern}{int dimx, int dimy}{selfcalib.h}
\descr{Es wird ein Bild der Größe $dimx$ mal $dimy$ zurückgegeben, welches 
das Testmuster für die Kalibrierfunktionen enthält. Das Bild kann 
dann z.B. ausgedruckt werden. Eine geeignete Größe ist z.B. 1200x800.}

%%% Local Variables: 
%%% mode: latex
%%% TeX-master: t
%%% End: 
