\nsection{Datenstrukturen für Bilder}{Images}

Bilder werden mit Hilfe von Instanzen der Bild-Klassen \class{Image}, 
\class{ColorImage} und \class{Image3d} verwaltet. Sie speichern 
die Bildinformation
in Form von Grauwerten: Der Wert 0 entspricht hier ``weiß'' 
beziehungsweise hoher Intensität und der Maximalwert 
entspricht ``schwarz''.

Bilder können visualisiert werden. Dabei werden die Bilder einmal zur 
Visualisierung angemeldet (\see{Show}) und werden ab diesem Zeitpunkt 
auf dem Bildschirm dargestellt. Alle Änderungen im Bild sind sofort
sichtbar.

Die Bildklassen selbst stellen eine {\bf Referenz} auf die eigentlichen 
Bilddaten dar. Sie haben Eigenschaften ähnlich einem Zeiger (''Smart Pointer''). 
In den meisten Fällen kann man diese Referenz aber als ''das Bild'' ansehen, 
mit dem man am Ende arbeitet. Es gibt jedoch einige Besonderheiten zu beachten:
\begin{itemize}
\item Alle Konstruktoren erzeugen kein neues Bild, sondern zunächst nur eine 
Referenz. 
\item Der Standard-Konstruktor erzeugt zunächst ein ''ungültiges''
Bild (\see{Image::isValid}). Vor der Verwendung muss noch das 
eigentliche Bild initialisiert werden (\see{NewImg}, \see{Image::create} ).
\item Der Kopierkonstruktor und damit auch Parameterübergaben über den Wert 
erzeugen {\bf kein} neues Bild, sondern nur eine neue Referenz auf das 
Ursprungsbild. Es handelt sich also um das gleiche Bild. Das gilt auch für den 
Zuweisungsoperator.
\item Bilder, genauer gesagt die Referenzen,  werden als konstant 
betrachtet, solange sich nicht die Bildgröße oder der Grauwertumfang 
ändert. Bilder werden deshalb an Funktionen meist als konstante Referenz 
übergeben (\verb+const Image &image+), auch wenn sich der Inhalt des Bildes
ändert.
\item Zum Anlegen von Bildern gibt es Methoden (\see{Image::create}) 
und Funktionen (\see{NewImg}), ebenso zum Kopieren bestehender Bilder 
(\see{CopyImg}, \see{Image::copy}).
\item Es ist möglich, Bilder explizit freizugeben (mittels \see{FreeImg}), 
im Normalfall erfolgt eine Freigabe implizit mit dem Ungültigwerden 
der Variable. 
\end{itemize}

Die Klasse \class{Image} dient zur Behandlung von Grauwertbildern. Der 
Wert eines Pixels wird hier durch einen ganzzahligen Wert repräsentiert. 

Zur Behandlung von Farbbildern ist die Klasse \class{ColorImage}
zu verwenden, die in 3 Bildern der Klasse \class{Image} die Farbauszüge 
Rot, Grün und Blau verwaltet.

3D-Bilder können mit der Template-Klasse class{Image3d} angelegt werden.
Es handelt sich um eine Sequenz von Bildern.

\subsection{Die Klasse Image}
\label{Image}
\hypertarget{Image}{}

Die Klasse \class{Image} stellt ein zweidimensionales Grauwert-Bild dar. 
Sie speichert die Bildinformation in Form von Grauwerten: Der 
Wert 0 entspricht hier ''weiß'' und der Maximalwert 
entspricht ''schwarz''.

Die Grauwerte neu angelegter Bilder werden standardmäßig 
nicht initialisiert, so dass ihr Wert unbestimmt ist (\see{ClearImg}, \see{Image::set}).

\subsection{Die Klasse ColorImage}
\label{ColorImage}
\hypertarget{ColorImage}{}

Die Klasse \class{ColorImage} stellt ein zweidimensinales Farbbild dar. Die Farben
werden in den drei ''Kanälen'' rot, grün und blau gespeichert. Die ''Kanäle'' selbst 
sind Bilder der Klasse \class{Image}, auf die auch einzeln zugegriffen werden kann.
Der Zugriff auf Farbbildpunkte arbeitet mit Werten vom Typ \class{ColorValue}, dessen
Komponenten die {\bf Intensitäten} in der Kanälen rot, grün und blau darstellen.

\subsection{Organisation und Verwaltung}
\subsubsection{Image}
\proch{}{Image::Image}{}{ImageC.h}
\descr{Standard-Konstruktor. Das so erzeugte Bild ist zunächst ungültig
  \see{Image::isValid}.}

\proch{}{Image::Image}{const Image \&img}{ImageC.h}
\descr{Kopier-Konstruktor. Erzeugt eine weitere {\bf Referenz} auf das
  übergebene Bild.}

\proch{void}{Image::create}{int xsize,int ysize,int maxval,const string \&title=''''}{ImageC.h}
\procf{void}{Image::create}{const Image \&img,const string \&title=''''}
\descr{Falls das Bild bereits initialisiert ist, wird zunächst das bestehende 
Bild freigegeben.
Es wird dann ein neues Bild der Größe $xsize$ * $ysize$ angelegt. 
Der Maximalwert der zu speichernden Grauwerte beträgt $maxval$.
Das Bild erhält den durch $title$ gegebenen Namen, der zum Beispiel bei der Darstellung
verwendet wird.
In der zweiten Form werden die Daten des Bildes $img$ als 
Vorlage verwendet. Der Bildinhalt wird nicht kopiert.}

\proch{void}{Image::copy}{const Image \&src}{ImageC.h}
\descr{Falls das Bild bereits initialisiert ist, wird zunächst das bestehende 
Bild freigegeben. Ein neues Bild wird dann als Kopie des übergebenen Bildes erzeugt.}

\proch{static Image}{Image::createImage}{int sizeX, int sizeY, int maxValue, const std::string \&title = ''''}{ImageC.h}
\descr{Legt ein neues Bild der Größe $sizeX$ * $sizeY$ mit dem
maximalen Grauwert $maxValue$ an und gibt ein {\bf Image} zurück.
Das Bild erhält den durch $title$ gegebenen Namen, der zum Beispiel bei der Darstellung
verwendet wird.
}
Dies ist eine statische Methode und kann in der Form \verb+Image image=Image::createImage(1280,720,255);+
verwendet werden. (''named constructor'')

\proch{static Image}{createImage}{const Image \&src, bool copy, const std::string \&title = ''''}{ImageC.h}
\descr{Legt ein neues Bild der Größe mit den Parametern von $src$ an und gibt ein {\bf Image} zurück.
Das Bild erhält den durch $title$ gegebenen Namen, der zum Beispiel bei der Darstellung
verwendet wird.
}
Dies ist eine statische Methode und kann in der Form verb+Image image2=Image::createImage(image);
verwendet werden. (''named constructor'')

\proch{Image}{NewImg}{int xsize,int ysize,int maxval,const string \&title=''''}{ImageC.h}
\procf{Image}{NewImg}{const Image \&img,int copy=false,const string \&title=''''}
\descr{
Es wird ein neues Bild angelegt und als Funktionswert zurückgegeben.
Die Größe wird durch $xsize$ und $ysize$ bestimmt, der maximale Grauwert 
ist $maxval$. Das Bild erhält den durch $title$ gegebenen Namen, der zum Beispiel bei der Darstellung
verwendet wird.
Bei Aufruf mit einem vorhandenen Bild $img$ wird dessen Größe
als Vorlage gewählt. Ist $copy=true$, so wird auch der Bildinhalt ins neue
Bild kopiert.
}

\proch{int}{FreeImg}{Image \&img}{ImageC.h}
\descr{
Das Bild $img$ wird gelöscht und der dynamische Speicher freigegeben. Das Bild wird damit
ungültig (\see{Image::isValid}):
}

\proch{bool}{Image::isValid}{}{ImageC.h}
\descr{
Gibt $true$ zurück, wenn Image ein gültiges Bild ist.
}

\proc{bool}{IsImg}{const Image \&img}
\descr{
Die Funktion gibt $true$ zurück, wenn $img$ ein gültiges Bild ist (\see{Image::isValid}).
}
\subsubsection{ColorImage}
\proch{}{ColorImage::ColorImage}{}{ColorImage.h}
\descr{Standard-Konstruktor. Das so erzeugte Farbbild ist zunächst ungültig
  \see{ColorImage::isValid}.}

\proch{}{ColorImage::ColorImage}{const ColorImage \&img}{ColorImage.h}
\descr{Kopier-Konstruktor. Erzeugt eine weitere {\bf Referenz} auf das
  übergebene Farbbild.}

\proch{void}{ColorImage::create}{int xsize,int ysize,int maxval,const string \&title=''''}{ColorImage.h}
\procf{void}{ColorImage::create}{const ColorImage \&img,const string \&title=''''}
\descr{Falls das Farbbild bereits initialisiert ist, wird zunächst das bestehende 
Farbbild freigegeben.
Es wird dann ein neues Farbbild der Größe $xsize$ * $ysize$ angelegt. 
Der Maximalwert der zu speichernden Werte pro Farbkanal beträgt $maxval$.
Das Farbbild erhält den durch $title$ gegebenen Namen, der zum Beispiel bei der Darstellung
verwendet wird.
In der zweiten Form werden die Daten des Farbbildes $img$ als 
Vorlage verwendet. Der Bildinhalt wird nicht kopiert.}

\proch{void}{ColorImage::copy}{const ColorImage \&src}{ColorImage.h}
\descr{Falls das Farbbild bereits initialisiert ist, wird zunächst das bestehende 
Farbbild freigegeben. Ein neues Farbbild wird dann als Kopie des übergebenen Farbbildes erzeugt.}

\proch{static ColorImage}{ColorImage::createColorImage}{int sizeX, int sizeY, int maxValue, const std::string \&title = ''''}{ColorImage.h}
\descr{Legt ein neues Farbbild der Größe $sizeX$ * $sizeY$ mit einem maximalen Wert $maxValue$ 
in den Farbkanälen an und gibt ein {\bf ColorImage} zurück.
Das Farbbild erhält den durch $title$ gegebenen Namen, der zum Beispiel bei der Darstellung
verwendet wird.
}
Dies ist eine statische Methode und kann in der Form \verb+ColorImage image=ColorImage::createColorImage(1280,720,255);+
verwendet werden. (''named constructor'')

\proch{static ColorImage}{ColorImage::createColorImage}{const ColorImage \&src, bool copy, const std::string \&title = ''''}{ColorImage.h}
\descr{Legt ein neues Farbbild der Größe mit den Parametern von $src$ an und gibt ein {\bf ColorImage} zurück.
Das Farbbild erhält den durch $title$ gegebenen Namen, der zum Beispiel bei der Darstellung
verwendet wird.
}
Dies ist eine statische Methode und kann in der Form verb+ColorImage image2=ColorImage::createColorImage(image);
verwendet werden. (''named constructor'')

\proch{ColorImage}{NewImg}{int xsize,int ysize,int maxval,const string \&title=''''}{ColorImage.h}
\procf{ColorImage}{NewImg}{const ColorImage \&img,int copy=false,const string \&title=''''}
\descr{
Es wird ein neues Farbbild angelegt und als Funktionswert zurückgegeben.
Die Größe wird durch $xsize$ und $ysize$ bestimmt, der maximale Wert in den Farbkanälen
ist $maxval$. Das Farbbild erhält den durch $title$ gegebenen Namen, der zum Beispiel bei der Darstellung
verwendet wird.
Bei Aufruf mit einem vorhandenen Farbbild $img$ wird dessen Größe
als Vorlage gewählt. Ist $copy=true$, so wird auch der Bildinhalt ins neue
Farbbild kopiert.
}

\proch{int}{FreeImg}{ColorImage \&img}{ColorImage.h}
\descr{
Das Farbbild $img$ wird gelöscht und der dynamische Speicher freigegeben. Das Farbbild wird damit
ungültig (\see{ColorImage::isValid}):
}

\proch{bool}{ColorImage::isValid}{}{ColorImage.h}
\descr{
Gibt $true$ zurück, wenn ColorImage ein gültiges Farbbild ist.
}

\proc{bool}{IsImg}{const ColorImage \&img}
\descr{
Die Funktion gibt $true$ zurück, wenn $img$ ein gültiges Farbbild ist (\see{ColorImage::isValid}).
}

\proch{Image}{ColorImage::redImage}{void}{ColorImage.h}
\procf{Image}{ColorImage::greenImage}{void}
\procf{Image}{ColorImage::blueImage}{void}
\descr{
Liefert die Teilbilder vom Typ \see{Image} (Farbauszüge) zurück. 
Damit können die Funktionen für Bilder vom Typ \see{Image} auf die einzelnen 
Farbauszüge angewendet werden.
Bei der Bearbeitung der Teilbilder muß garantiert werden, dass diese 
immer die gleiche Größe behalten, sonst wird das Farbbild ungültig.
}

\subsection{Teilbilder - SubImages}
\hypertarget{SubImages}{}

In ICE wird die Idee einer ''Region of interest'' durch Teilbilder (SubImages)
realisiert. Durch den \hyperlink{SubImageConstructor}{SubImage-Konstruktor} wird
eine Referenz vom Typ $Image$ erzeugt, die auf ein rechteckiges Teilbild des
gegebenen Bildes weist. Teilbilder verhalten sich in fast jeder Hinsicht wie
andere Bilder, insbesondere kann auch von Teilbildern wieder ein Teilbild
angelegt werden. Werden jetzt irgendwelche Operationen auf dieses
(Teil-)Bild angewendet, so wirken sie sich auf diesen rechteckigen Teilbereich
des Ausgangsbildes aus. 

\hypertarget{SubImageConstructor}{}
\proch{}{Image::Image}{const Image \&img,const Window \&w,const string \&title=''''}{ImageC.h}
\descr{Legt eine neue Referenz auf die Bilddaten des Bildes $img$ an, welches
  dem durch $w$ gegebenem {\bf Teilbild} entspricht. Das Bild erhält den durch
  $title$ gegebenen Namen. Ein solches Teilbild verhält sich in (fast) 
  allen Belangen wie ein normales Bild, es wird jedoch immer ein Teilbild 
  des Bildes $img$ verarbeitet.}

\proch{Image}{Image::operator()}{const Window \&w}{ImageC.h}
\descr{Erzeugt ein Teilbild von Image}

\proch{}{ColorImage::ColorImage}{const ColorImage \&img,const Window \&w,const string \&title=''''}{ColorImage.h}
\descr{Legt eine neue Referenz auf die Bilddaten des Farbbildes $img$ an, welches
  dem durch $w$ gegebenem {\bf Teilbild} entspricht. Das Bild erhält den durch
  $title$ gegebenen Namen. Ein solches Teilbild verhält sich in (fast) 
  allen Belangen wie ein normales Bild, es wird jedoch immer ein Teilbild 
  des Bildes $img$ verarbeitet.}

\proch{ColorImage}{ColorImage::operator()}{const Window \&w}{ImageC.h}
\descr{Erzeugt ein Teilbild von Image}

Dadurch können beliebige mit Bildern arbeitende Funktionen auch auf Teilbilder 
angewendet werden:
\begin{verbatim} 
 Image img = ReadImg("test.jpg");
 Image subimg(img, Window(10,10,200,150));        // SubImage anlegen
 ClearImg(subimg);                                // Funktion auf Teilbereich anwenden
 SetImg(img(Window(200,100,300,200)),img.maxval); // SubImage mit operator() erzeugt
\end{verbatim}

\subsection{Bildpunktzugriff}

Der Zugriff auf einzelne Pixel von Bildern erfolgt mit Hilfe der Methoden
\see{Image::setPixel} und \see{Image::getPixel} oder mittels der
Funktionen \see{PutVal} und \see{GetVal}. \\
Die Unchecked-Varianten der Methoden erlauben einen 
beschleunigten Zugriff ohne Testung der übergebenen Parameter.
Diese sollten nur in gut getesteten Programmen zur Anwendung kommen.
Bei hohen Anforderungen an die Laufzeit des Programmes kann der 
\hyperref{direct_pixel_access}{}{}{low level-Bildpunktzugriff} verwendet werden.

\subsubsection{Methoden}

\proch{int}{Image::getPixel}{int x,int y}{ImageC.h}
\procf{int}{Image::getPixel}{IPoint p}
\descr{Abfrage des Grauwertes an der Stelle (x,y) oder p des Bildes. }

\proch{ColorValue}{ColorImage::getPixel}{int x,int y}{ColorImage.h}
\procf{ColorValue}{ColorImage::getPixel}{IPoint p}
\descr{Abfrage des Farbwertes an der Stelle (x,y) oder p des Bildes. }

\proch{int}{Image::getPixelUnchecked}{int x,int y}{ImageC.h}
\procf{int}{Image::getPixelUnchecked}{IPoint p}
\descr{Abfrage des Grauwertes an der Stelle (x,y) oder p des Bildes. 
Es erfolgt keine Testung, ob die übergebenen Koordinaten innerhalb des Bildes liegen.
}

\proch{ColorValue}{ColorImage::getPixelUnchecked}{int x,int y}{ColorImage.h}
\procf{ColorValue}{ColorImage::getPixelUnchecked}{IPoint p}
\descr{Abfrage des Farbwertes an der Stelle (x,y) oder p des Bildes.
Es erfolgt keine Testung, ob die übergebenen Koordinaten innerhalb des Bildes liegen.
}

\proch{int}{Image::getIntensity}{int x,int y}{ImageC.h}
\procf{int}{Image::getIntensity}{IPoint p}
\descr{Abfrage der Intensität an der Stelle (x,y) oder p des Bildes.}

\proch{int}{Image::getIntensityUnchecked}{int x,int y}{ImageC.h}
\procf{int}{Image::getIntensityUnchecked}{IPoint p}
\descr{Abfrage der Intensität an der Stelle (x,y) oder p des Bildes.
Es erfolgt keine Testung, ob die übergebenen Koordinaten innerhalb des Bildes liegen.
}

Das Ergebnis der Abfrage der Intensität ist äquivalent zu \verb+img.maxval - img.getPixel(x,y)+.

\proch{void}{Image::setPixel}{int x,int y,int value}{ImageC.h}
\procf{void}{Image::setPixel}{IPoint p,int value}
\descr{Setzt den Grauwert an der Stelle (x,y) oder p des Bildes auf den Wert $value$. }

\proch{void}{ColorImage::setPixel}{int x,int y,ColorValue cv}{ColorImage.h}
\procf{void}{ColorImage::setPixel}{IPoint p,ColorValue cv}
\descr{Setzt den Farbwert an der Stelle (x,y) oder p des Farbbildes auf den Wert $cv$. }

\proch{void}{Image::setPixelUnchecked}{int x,int y,int value}{ImageC.h}
\procf{void}{Image::setPixelUnchecked}{IPoint p,int value}
\descr{Setzt den Grauwert an der Stelle (x,y) oder p des Bildes auf den Wert $value$.
Es erfolgt keine Testung der übergebenen Koordinaten und des Wertes.
}

\proch{void}{ColorImage::setPixelUnchecked}{int x,int y,ColorValue cv}{ColorImage.h}
\procf{void}{ColorImage::setPixelUnchecked}{IPoint p,ColorValue cv}
\descr{Setzt den Farbwert an der Stelle (x,y) oder p des Farbbildes auf den Wert $cv$. 
Es erfolgt keine Testung der übergebenen Koordinaten und des Wertes.
}

\proch{int}{Image::setIntensity}{int x,int y,int value}{ImageC.h}
\procf{int}{Image::setIntensity}{IPoint p,int value}
\descr{Setzt die Intensität an der Stelle (x,y) oder p des Bildes auf 
den Wert $value$.}

\proch{int}{Image::setIntensityUnchecked}{int x,int y,int value}{ImageC.h}
\procf{int}{Image::setIntensityUnchecked}{IPoint p,int value}
\descr{Setzt die Intensität an der Stelle (x,y) oder p des Bildes auf den Wert
$value$. Es erfolgt keine Testung, ob die übergebenen Koordinaten und der 
Intensitäts-Wert zulässig sind.
}

Das Setzen der Intensität eines Pixels ist äquivalent zu zum Setzen des 
Grauwertes auf \verb+img.maxval - value+.

Die Methoden {\bf inside} sind nützlich, um festzustellen, ob die Kordinaten eines 
Punktes oder Fensters im Inneren eines Bildes liegen.

\proc{bool}{Image::inside}{const IPoint p}
\procf{bool}{Image::inside}{int x,int y}
\procf{bool}{Image::inside}{const Window \&w}
\procf{bool}{ColorImage::inside}{const IPoint p}
\procf{bool}{ColorImage::inside}{int x,int y}
\procf{bool}{ColorImage::inside}{const Window \&w}
\descr{Die Methoden testen ob ein gegebener Punkt beziehungsweise das gegebene
  Fenster (\class{Window}) im Bereich der Bildkoordinaten des Bildes liegt.
}

\subsubsection{Funktionen}

\proch{void}{PutVal}{\bsee{Image} \&img,int x,int y,int val}{base.h}
\descr{
Der Bildpunkt (x,y) im Bild $img$ wird mit dem Wert $val$ beschrieben. Es
muß $0\le val\le img \to maxval$ gelten.
}

\proc{int}{GetVal}{const \bsee{Image} \&img,int x,int y}
\descr{Der Wert des Bildpunktes $(x,y)$ des Bildes $img$ wird als Funktionswert
bereitgestellt.
}

\proc{int}{GetVal}{const \bsee{Image} \&img,double x,double y,mode=DEFAULT}
\descr{Es wird ein Wert der Grauwertfunktion des Bildes $img$ für den 
Punkt (x,y) bereitgestellt. Der Wert berechnet sich in Abhängigkeit von
$mode$:
\begin{itemize}
\item $mode=DEFAULT$: Der Wert des nächsten Pixels wird genutzt, 
\item $mode=INTERPOL$: Der Wert wird durch bilineare Interpolation aus den 4
  benachbarten Pixeln berechnet.
\end{itemize}
}

\proch{void}{PutVal}{\bsee{Image} \& img,const \bsee{IPoint} \&p,int val}{base.h}
\descr{Setzt das Pixel im Punkt $p$ auf den Wert $val$.}

\proch{int}{GetVal}{const \bsee{Image} img,const \bsee{IPoint} \&p}{base.h}
\descr{Liest den Wert des Pixels im Punkt $p$.}

\proch{double}{GetInterpolVal}{Image img,double x,double y}{base.h}
\procf{bool}{GetInterpolVal}{Image img,double x,double y,double \&val}
\descr{Für die Gleitkomma-Koordinaten $(x,y)$ wird durch bilineare Interpolation 
der Grauwerte der vier nächstliegenden Bildpunkte aus dem Bild $img$ ein
Grauwert berechnet.\\
In der zweiten Aufrufform gibt die Funktion zurück, ob die Koordinaten 
innerhalb des Bildes lagen. Der interpolierte Grauwert wird
hier auf $val$ zurückgegeben.}

\subsection{Schleifen in Bildern}

Sollen Operationen in einem Bild über alle Pixel erfolgen, so müssen 
in einer Schleife alle Koordinatenwerte durchlaufen werden. 
Wenn irgend möglich sollte dafür ein zeilenweises Durchlaufen gewählt 
werden, da dann die Caches des Computers effizienter arbeiten können 
und die Abarbeitung deutlich schneller erfolgt. 

Dies wird oft in der Form zweier Schleifen erfolgen:

\begprogr
\begin{verbatim}
Image img;
... 
for (int y=0; y < img.ysize; ++y) // äußere Schleife y
  for (int x=0; x < img.xsize; ++x) // innere Schleife x
    {
      PutVal(img,x,y,(x+y) % img->maxval);
    }
\end{verbatim}
\endprogr

Eine iterator-ähnliche Methode zum Durchlaufen aller Bildpunkte
verwendet die Klasse \class{Walker}.

\subsection{Bildpunktzugriff auf low level-Ebene}
\hypertarget{direct_pixel_access}{}
Um einen schnelleren Zugriff auf die Bilddaten in zeitkritischen Anwendungen
zu ermöglichen, kann ein Zeiger auf die internen Datenstrukturen abgefragt
und direkt auf die Daten im Speicher zugegriffen werden. Dabei gibt es
folgende Einschränkungen:
\begin{itemize}
\item Der Zugriff ist abhängig vom konkreten Typ des Bildes. Dieser Typ kann
  mit der Methode \verb+Image->ImageType()+ abgefragt werden. Als Typ-Wert wird 
  einer der Werte von 1 bis 3 zurückgegeben, woraus sich der zu verwendende
  Pixel-Datentyp ergibt.\\
\begin{tabular}{|l|l|l|} \hline
Bildtyp&Pixeltyp&C++-Datentyp\\
\verb+Image->ImageType()+ & & \\ \hline
1&PixelType1&unsigned char\\
2&PixelType2&unsigned short int\\
3&PixelType3&int\\ \hline
\end{tabular}\\
Die Angabe des zugrundeliegenden C++-Datentyps dient nur der Information 
und kann sich jederzeit ändern.
\item Es erfolgt keinerlei Testung der Parameter des Zugriffs.
\item Die Visualisierung der geänderten Daten muss durch
  Image-$>$needRefresh() explizit veranlasst werden.
\end{itemize}

\proch{int}{Image-$>$ImageType}{}{ImageC.h}
\descr{Gibt den Type des Bildes zurück.}

\proch{void*}{Image-$>$getDataPointer}{}{ImageC.h}
\descr{Gibt einen Zeiger T** auf die Bilddaten zurück. Dabei ist T der
  Pixeltyp, der in Abhängigkeit vom Bildtyp explizit zu casten ist.
  Der Zeiger zeigt auf ein Feld von Zeigern auf die Bildzeile. Die Bildzeile selbst
  ist ein C-array vom Pixel-Typ $T$.}

Ein Beispiel-Programmfragment:
\begprogr
\begin{verbatim}
Image img;
... 
if (img->ImageType()==1) // Spezialbehandlung für diesen Typ
{
  PixelType1** imgdata = (PixelType1**)img->getDataPtr();
  for (int y=0; y < img.ysize; ++y) // äußere Schleife y
    for (int x=0; x < img.xsize; ++x) // innere Schleife x
    {
      imgdata[y][x] = 1;
    }
  img->needRefresh();
}
else // alle anderen Bildtypen
{
  for (int y=0; y < img.ysize; ++y) // äußere Schleife y
    for (int x=0; x < img.xsize; ++x) // innere Schleife x
    {
      PutVal(img,x,y,1);
    }
}
\end{verbatim}
\endprogr

\subsection{Bildbearbeitung}
Operatoren über Bilder vom Typ \class{Image} verknüpfen ein oder zwei Bilder und
legen das Ergebnis im Zielbild ab. Alle Bilder, insbesondere auch das Zielbild
müssen gültige (mit \see{NewImg} angelegte) Bilder sein.
Die Bildgrößen aller Bilder müssen entsprechend übereinstimmen.

Die Beschreibung der Operatoren findet sich für alle Bildrepräsentationen im
Abschnitt \see{Bildbearbeitung}.

\subsection{3D-Bilder - die Klasse Image3D}
\hypertarget{Image3d}{}

Die Klasse \class{Image3d} repräsentiert 3D-Bildern. Das sind dreidimensionale 
Voxel-Bilder, wie sie beispielsweise bei der Computer-Tomografie oder 
Magnet-Resonanz-Tomografie entstehen.
{\bf Image3d}-Objekte stellen eine Sequenz von 2D-Bildern \class{Image} dar, 
die Scheiben des 3D-Bildes in XY-Richtung darstellen. Bei 3D-Bildern der Klasse
{\bf Image3d} kann die Auflösung in alle 3 Richtungen getrennt festgelegt 
werden, was beim Zugriff mittels {\bf GetValueD} berücksichtigt wird.
\subsubsection{Konstruktoren}
\proch{}{Image3d}{int xs, int ys, int zs, int maxval = 255,
  double xscale = 1.0, double yscale = 1.0, double zscale = 1.0}{image3d.h}
\descr{Erzeugt ein 3D-Bild der Klasse {\bf Image3d}. Die Ausdehnungen in 
X-, Y- und Z-Richtung in Voxeln wird durch $xs$, $ys$ und $zs$ festgelegt.
$maxval$ gibt den maximalen Grauwert an, der in Voxeln gespeichert werden kann.
$xscale$, $yscale$ und $zscale$ beschreiben die physische Größe des Voxels.}

\proch{}{Image3d}{const std::string \&filemask,
  double xscale = 1.0, double yscale = 1.0, double zscale = 1.0}{image3d.h}
\descr{Erzeugt ein 3D-Bild aus Bilddateien. Aus den Bilddateien wird die Größe ermittelt,
das 3D-Bild angelegt und die Bilder werden eingelesen. Die Datei-Maske muss die zu ladenden Bilder 
beschreiben. Die Reihenfolge der Bilder ist alphabetisch.
$xscale$, $yscale$ und $zscale$ beschreiben die physische Größe des Voxels.}

\proch{}{Image3d}{const Image3d \&src, bool copydata = true}{image3d.h}
\descr{Kopierkonstruktor. Bei Angabe von $copydate$ als $false$ wird nur das 3D-Bild
gleicher Größe angelegt, aber der Bildinhalt nicht kopiert.}

\proch{}{Image3d}{const Image3d \&src, Window3d w}{image3d.h}
\descr{Konstruktor eines 3D-Teilbildes, welches die Daten des Originalbildes $src$
referenziert. Änderungen des Teilbildes ändern den entsprechend Bildauschnitt des 
Origionalbildes $src$.}
 
\subsection{Bilder mit Gleitkommawerten}
\label{ImageD}
\hypertarget{Image3d}{}
Für einige Aufgaben, z.B. im Zusammenhang mit Fouriertransformationen, reicht
der Grauwertumfang der Integer-Bilder \class{Image} nicht aus. Analog zur 
Datenstruktur Image gibt es eine Datenstruktur \class{ImageD} mit Grauwerten 
vom Typ {\em double} eingeführt. Diese Bilder können nicht direkt visualisiert
werden, es stehen jedoch Funktionen zur Konvertierung in Integer-Bilder und
umgekehrt zur Verfügung (\see{ConvImgDImg}). Zusätzlich zum maximalen 
Wert $maxval$ wird in Gleitkomma-Bildern der minimale Wert $minval$ 
gespeichert. Abweichend von den Integer-Bildtypen haben diese Werte 
jedoch nur einen informativen Character. Sie stellen keine Begerenzung 
des Wertebereiches dar und es erfolgt bei Zuweisungen keine Testung auf eine 
Einhaltung des gegebenen Bereichs.

\proch{ImageD}{NewImgD}{int xsize,int ysize,double minval=0.0,double maxval=0.0}{based.h}
\procf{ImageD}{NewImgD}{ImageD img,bool copy=false}
\procf{ImageD}{NewImgD}{const Image \&img,bool copy=false}
\descr{
Es wird ein neues Gleitkommabild angelegt und als Funktionswert zurückgegeben.
Neben der Möglichkeit der expliziten Angabe der Größe kann auch ein anderes 
Bild (\class{Image}) oder Gleitkommabild als Muster vorgegeben werden. Wenn der 
zusätzliche Parameter $copy$ = true ist, wird der Inhalt der Vorlage ins neue 
Bild kopiert. Im Fehlerfall wird der NULL-Pointer zurückgegeben.
}

\proch{void}{PutValD}{ImageD img,int x,int y,double val}{macro.h}
\descr{
Der Bildpunkt $(x,y)$ im Bild $img$ wird mit dem Wert $val$ beschrieben. 
Es wird keine Kontrolle des Wertebereiches vorgenommen. 
}

\proc{double}{GetValD}{ImageD img,int x,int y}
\descr{
Es wird der Datenwert des Bildpunktes $(x,y)$ des Bildes $img$ als
Funktionswert bereitgestellt.
}

\proch{void}{PutValue}{ImageD img,IPoint p,double val}{based.h}
\descr{
Der Bildpunkt $p$ im Bild $img$ wird mit dem Wert $val$ beschrieben. 
Es wird keine Kontrolle des Wertebereiches vorgenommen. 
}

\proc{double}{GetValue}{ImageD img,IPoint p}
\descr{
  Es wird der Datenwert des Bildpunktes $p$ des Bildes $img$ als
Funktionswert bereitgestellt.
}

\proch{int}{UpdateLimitImgD}{ImageD img}{based.h}
\descr{
In dem Gleitkommabild $img$ werden die Werte für den minimalen und maximalen Grauwert
anhand der vorhandenen Werte des Bildes aktualisiert.
}

\proc{int}{ConvImgDImg}{ImageD imgs,Image imgd,int modus=ADAPTIVE,int sign=UNSIGNED}
\descr{
Das Gleitkommabild $imgs$ wird in ein Integer-Bild $imgd$ konvertiert. 
Die Parameter $modus$ und $sign$ steuern, welcher Zahlenbereich des 
Gleitkomma-Bildes dem Grauwert-Umfang des Integer-Bildes entspricht:
\begin{tabular}{|l||c|c|} \hline
$modus$ & $UNSIGNED$ & $SIGNED$ \\ \hline \hline
RAW & $[0,imgd->maxval]$ & $[-\frac{imgd->maxval}{2},\frac{imgd->maxval}{2}]$ \\ \hline
ADAPTIVE & \multicolumn{2}{c}{$[imgs->minval,imgs->maxval]$} \\ \hline
NORMALIZED & $[0,4]$ & $[-2,2]$ \\ \hline
\end{tabular}\\
Werte außerhalb des jeweiligen Wertebereichs werden auf den Maximal- bzw.
Minimal-Wert begrenzt. Ist $modus=ADAPTIVE$ wird der Wertebereich des
Gleitkommabildes neu ermittelt, der Parameter $sign$ wird hier ignoriert.
}

\proc{int}{ConvImgImgD}{const Image \&imgs,ImageD imgd,int modus=RAW,int sign=UNSIGNED}
\descr{
Das Integer-Bild $imgs$ wird in ein Gleitkommabild $imgd$ konvertiert.
Der Grauwert-Bereich des Quellbildes $[0,imgs->maxval]$ wird in Abhängigkeit
von $modus$ und $sign$ in die folgenden Gleitkomma-Zahlenbereiche 
transformiert:
\begin{tabular}{|l||c|c|} \hline
$modus$ & $UNSIGNED$ & $SIGNED$ \\ \hline \hline
RAW & $[0,imgs->maxval]$ & $[-\frac{imgs->maxval}{2},\frac{imgs->maxval}{2}-1]$ \\ \hline
NORMALIZED & $[0,4]$ & $[-2,2]$ \\ \hline
\end{tabular}
}

\proc{int}{FreeImgD}{ImageD img}
\descr{
Das Gleitkommabild $img$ wird gelöscht und der dynamische Speicher freigegeben.
}

%%% Local Variables: 
%%% mode: latex
%%% TeX-master: "icedoc_base"
%%% End: 
