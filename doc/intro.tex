\section{Einführung}

In dieser Dokumentation werden Werkzeuge zur 3D-Szenenanalyse und Robot Vision
beschrieben, die auf der ICE-Bibliothek aufbauen. Die Schwerpunkte liegen auf
der Kamerakalibrierung, der monokularen Lage- und Orientierungsschätzung, der
Beschreibung von Polyederszenen und Polyedermodellen mit orientierten
Nachbarschaftsstrukturen.

Die Funktionsprototypen und Datenstrukturen sind in dem Headerfile
``robvis.h'' zusammengafaßt, die Funktionen in der Bibliothek ``robvis.a''.

Zur Beschreibung von Kameras werden hauptsächlich die projektiven Parameter
$a_{ij}$ verwendet, so daß für die Abbildung eines Raumpunktes $pw=(x,y,z)^T$
in einen Bildpunkt $pc=(\xi,\eta)^T$ die Abbildungsgleichungen
\begin{eqnarray}
\xi ={a_{00}x+a_{01}y+a{02}z+a{03} \over
a_{20}x+a_{21}y+a{22}z+a{23}}\nonumber\cr \cr
\eta={a_{10}x+a_{11}y+a{12}z+a{13} \over a_{20}x+a_{21}y+a{22}z+a{23}}\nonumber
\end{eqnarray}
gelten. Zusätzlich können noch radialsymmetrische Verzeichnungen
berücksichtigt werden, die durch $\tilde r = r(1+k_1 r^2+k_2 r^4)$ modelliert
werden. $\tilde r$ ist der Abstand des verzeichneten Punktes vom Hauptpunkt und
$r$ der Abstand des unverzeichneten Punktes. Eine Trennung der inneren und
äußeren Kameraparameter und die Umrechnung zwischen physikalischen und
projektiven Kameraparametern ist möglich.

Für die Bestimmung der Lage und Orientierung von Objekten aus ihren
projektiven Abbildungen stehen hauptsächlich Funktionen zur monokularen
Rekonstruktion zur Verfügung, die die Lage und Orientierung eines körperfesten
Koordinatensystems aus gegebenen Korrespondenzen von Modell- und Bildpunkten
berechnen. Eine Ausnahme bilden die Funktionen zur Rekonstruktion anhand
geometrischer Momente, die keine expliziten Punktkorrespondenzen benötigen,
sondern die Flächenmomente korrespondierender planarer Objekte.

Zur Beschreibung von Bildern mit polyedrischen Objekten werden Datenstrukturen
verwendet, die aus mehrfach untereinander verketteten Listen für Eckpunkte,
Kanten und Flächen bestehen und den Bildinhalt als orientierte
Nachbarschaftsstruktur beschreiben. Die Beschreibung der dreidimensionalen
Polyedermodelle ist analog aufgebaut. Mit Hilfe solcher Beschreibungen ist das
Matching von Modell und Bildbeschreibung und die Lagerekonstruktion möglich.

Die Modellbeschreibung erlaubt durch ein den einzelnen Modellen zugeordnetes
Lageframe den Aufbau von Szenenbeschreibungen. Solche Szenenbeschreibungen
können wiederum für eine gegebene Kamera rechnerisch in eine
2D-Bildbeschreibung überführt werden, die nur die für die Kamera tatsächlich
sichtbaren Objektbestandteile enthält.
