\nsection{Veraltete Datenstrukturen und Funktionen}{obsolet}
Die in dieser Sektion gesammelten Datenstrukturen und Funktionen sind
veraltet. Ihre Dokumentation ist nur informativ, die Funktionen und Strukturen
sollten nicht mehr genutzt werden. 

\subsection{Vektoren als C-Array}
Die Klassen \class{Vector} bzw. \class{Vector3d} und ihre Methoden 
ersetzen die Funktionen dieses Abschnitts. 
Ein Einsatz sollte nicht mehr erfolgen.

\subsubsection{Dreidimensionale Vektoren}
\proc{double*}{MoveVec}{double v1[3],double v2[3]}
\descr{
Der Vektor $v1$ wird auf den Vektor $v2$ kopiert.
}
\proc{double}{LengthVec}{double v[3]}
\descr{
  Die Euklidische Länge des Vektors $v$ wird berechnet.
}
\proc{double*}{AddVec}{double v1[3],double v2[3],double v3[3]}
\descr{
Die Vektoren $v1$ und $v2$ werden addiert. Das Ergebnis wird auf $v3$
bereitgestellt.
}
\proc{double*}{SubVec}{double v1[3],double v2[3],double v3[3]}
\descr{
Die Vektoren $v1$ und $v2$ werden subtrahiert. Das Ergebnis wird auf $v3$
bereitgestellt.
}
\proc{double*}{NormVec}{double v1[3],double v2[3]}
\descr{
Der Vektor $v1$ wird auf die Länge 1 normiert. Das Ergebnis wird auf
$v2$ bereitgestellt.
}
\proc{double*}{ScaleVec}{double v1[3],double fac,double v2[3]}
\descr{
Die Komponenten des Vektors $v1$ werden mit $fac$ multipliziert. Das Ergebnis wird auf
$v2$ bereitgestellt.
}
\proc{double}{ScalProdVec}{double v1[3],double v2[3]}
\descr{
Rückgabewert ist das Skalarprodukt der Vektoren $v1$ und $v2$.
}
\proc{double*}{CrossProdVec}{double v1[3],double v2[3],double v3[3]}
\descr{
Auf $v3$ wird das Kreuzprodukt der Vektoren $v1$ und $v2$ bereitgestellt.
}
\proc{double}{SpatProdVec}{double v1[3],double v2[3],double v3[3]}
\descr{
Rückgabewert ist das Spatprodukt der Vektoren $v1$, $v2$ und $v3$.
}
\proc{double}{AngleVec}{double v1[3],double v2[3]}
\descr{
Rückgabewert ist der Winkel zwischen Vektoren $v1$ und $v2$.
}

\subsubsection{N-dimensionale Vektoren}

Der Parameter $dim$ in den folgenden Funktionen gibt jeweils die Dimension der
Vektoren an.

\proc{double*}{MoveVecRn}{double *v1,int dim,double *v2}
\descr{
Der Vektor $v1$ wird auf den Vektor $v2$ kopiert.
}
\proc{double}{LengthVecRn}{double *v,int dim}
\descr{
Die Euklidische Länge des Vektors $v$ wird berechnet
}
\proc{double*}{AddVecRn}{double *v1,double *v2,int dim,double *v3}
\descr{
Die Vektoren $v1$ und $v2$ werden addiert. Das Ergebnis wird auf $v3$
bereitgestellt.
}
\proc{double*}{SubVecRn}{double *v1,double *v2,int dim,double *v3}
\descr{
Die Vektoren $v1$ und $v2$ werden subtrahiert. Das Ergebnis wird auf $v3$
bereitgestellt.
}
\proc{double*}{NormVecRn}{double *v1,int dim,double *v2}
\descr{
Der Vektor $v1$ wird auf die Länge 1 normiert. Das Ergebnis wird auf
$v2$ bereitgestellt.
}
\proc{double*}{ScaleVecRn}{double *v1,int dim,double fac,double *v2}
\descr{
Die Komponenten des Vektors $v1$ werden mit $fac$ multipliziert. Das Ergebnis wird auf
$v2$ bereitgestellt.
}
\proc{double}{ScalProdVecRn}{double *v1,double *v2,int dim}
\descr{
Rückgabewert ist das Skalarprodukt der Vektoren $v1$ und $v2$.
}
\proc{double}{AngleVecRn}{double *v1,double *v2,int dim}
\descr{
Rückgabewert ist der Winkel zwischen Vektoren $v1$ und $v2$.
}
\proc{void}{PrintVecRn}{char *str,double *v,int dim}
\descr{Ausgabe von Vektoren auf Standardausgabegerät}

\subsection{Matrizen alten Typs}
\label{MatrixStruct}
Die Klasse \see{Matrix} und ihre Methoden ersetzen die Definitionen und
Funktionen dieses Abschnitts. 
Der Typ $MatrixStruct$ sollte nicht mehr eingesetzt werden.

Die Handhabung von Matrizen ist mit Hilfe von Pointern des Typs $MatrixStruct$
möglich. In der Datenstruktur ist die Größe der Matrix mit abgelegt. Die
Koeffizienten können vom Typ $double$, $int$ oder $char$ sein.

\begprogr\begin{verbatim}
typedef struct MatrixStruct_
{
  int type;
  int rsize,csize;
  double **data;
  int **datai;
  unsigned char **datac;
}  *MatrixStruct;
\end{verbatim}\endprogr

\proc{MatrixStruct}{NewMatrix}{int typ,int rows,int columns}
\descr{
  Es wird eine Matrix mit $rows$ Zeilen, $columns$ Spalten angelegt. Der Typ der
  Koeffizienten ist $double$ für $typ$=MAT\_DOUBLE, $int$ für $typ$=MAT\_INT und
  $char$ für $typ$=MAT\_CHAR.
  Für den Zugriff auf die Koeffizienten einer Matrix ist in Abhängigkeit vom
  vereinbarten Typ der Pointer data (MAT\_DOUBLE), datai (MAT\_INT) oder datac
  (MAT\_CHAR) zu verwenden.
}
\proc{int}{FreeMatrix}{MatrixStruct mat}
\descr{
Die Matrix $mat$ wird freigegeben.
}
\proc{int}{IsMatrix}{MatrixStruct mat}
\descr{
Es wird getestet, ob die Matrix $mat$ eine gültige Matrix ist, d.h ob sie in
der internen Verwaltung geführt ist. Wenn das der Fall ist, wird TRUE
zurückgegeben, sonst FALSE.
}
\proc{MatrixStruct}{MoveMat}{MatrixStruct m1,MatrixStruct m2}
\descr{
Die Matrix $m1$ wird auf die Matrix $m2$ kopiert. Wenn für $m2$ der
NULL-Pointer übergeben wird, wird die Zielmatrix intern angelegt. Rückgabewert
ist bei fehlerfreier Ausführung die Zielmatrix, sonst der NULL-Pointer.
}
\proc{MatrixStruct}{TranspMat}{MatrixStruct m1,MatrixStruct m2}
\descr{
Die Matrix $m1$ wird transponiert und auf die Matrix $m2$ kopiert. Wenn für
$m2$ der NULL-Pointer übergeben wird, wird die Zielmatrix intern
angelegt. Rückgabewert ist bei fehlerfreier Ausführung die Zielmatrix, sonst
der NULL-Pointer. $m1$ bleibt unverändert.
}
\proc{MatrixStruct}{MulMat}{MatrixStruct m1,MatrixStruct m2,MatrixStruct m3}
\descr{
Es wird das Matrizenprodukt $m3=m1*m2$ berechnet. Die Spaltenzahl von $m1$ muß
mit der Zeilenzahl von $m2$ übereinstimmen. Wenn für $m3$ der NULL-Pointer
übergeben wird, wird die Ergebnismatrix intern angelegt. Rückgabewert ist die
Ergebnismatrix bzw. im Fehlerfall der Nullpointer.
}
\proc{int}{MulMatVec}{MatrixStruct A,double *b,double *x}
\descr{
Es wird das Produkt der Matrix $A$ mit dem Vektor $b$ berechnet und auf $x$
bereitgestellt. Die Dimension der Vektoren $b$ bzw. $x$ muß mindestens so groß
wie die Spalten- bzw. Zeilenzahl der Matrix $A$ sein.
}
\proc{MatrixStruct}{InvertMat}{MatrixStruct A,MatrixStruct B}
\descr{
Es wird die Matrix $B$ als Inverse der quadratischen Matrix $A$
berechnet. Wenn für $B$ der NULL-Pointer übergeben wird, wird die Zielmatrix
intern angelegt. Rückgabewert ist bei fehlerfreier Ausführung die Zielmatrix,
sonst der NULL-Pointer. Die Matrix $A$ bleibt unverändert. 
}
\proc{MatrixStruct}{SortMatrix}{MatrixStruct A,int col,int mode}
\descr{
Es wird eine Matrix erzeugt und zurückgegeben, die die Zeilen der Matrix $A$
sortiert enthält. Durch $col$ wird die Spalte angegeben, nach deren Elementen
die Matrix sortiert wird. Mit $mode>=0$ wird in aufsteigender Reihenfolge und
mit $mode<0$ in absteigender Reihenfolge sortiert. Im Fehlerfall wird der
NULL-Pointer zurückgegeben.
}

Die Anwendung von Matrizen soll im folgenden Beispiel verdeutlicht werden:
\begprogr\begin{verbatim}
/*Anwendung von Matrizen*/
#include <stdio.h>
#include <image.h>
void main(void)
{
  MatrixStruct A,B,I;
  int i,j;
  double b[3],x[3];
  A=NewMatrix(MAT_DOUBLE,3,3);           /* Koeffizientenmatrix */
  A->data[0][0]= 3;  A->data[0][1]= 5;  A->data[0][2]=-1;
  A->data[1][0]= 7;  A->data[1][1]=-2;  A->data[1][2]= 3;
  A->data[2][0]=-2;  A->data[2][1]= 1;  A->data[2][2]= 8;
  b[0]=10;  b[1]=12;  b[2]=24;           /* Inhomogenitätsvektor */
  EquationSys(A,b,x);                    /* Gleichungssystem lösen */
  printf("%f  %f  %f\n",x[0],x[1],x[2]);
  NormalEquationSys(A,b,x);              /* Diese Lösung ist äquivalent */
  printf("%f  %f  %f\n",x[0],x[1],x[2]);
  B=InvertMat(A,NULL);                   /* Koeffizientenmatrix invertieren */
  MulMatVec(B,b,x);                      /* Multiplikation mit Inhomogenität */
  printf("%f  %f  %f\n",x[0],x[1],x[2]); /* wieder die gleiche Lösung */
  I=MulMat(A,B,NULL);                    /* Das Produkt ist die Einheitsmatrix */
  for(i=0;i<3;i++)
  { 
    for(j=0;j<3;j++) printf("%f  ",I->data[i][j]);
    printf("\n");
  }
  FreeMatrix(A);                         /* Matrizen freigeben */
  FreeMatrix(B);
  FreeMatrix(I);
} 
\end{verbatim}\endprogr

\subsection{Geometrische Transformationen als double[3][3]}
Projektive und affine 2D-2D-Transformationen können als 3x3-Matrix
(\verb+double m[3][3]+) beschrieben werden.
 
Für neue Anwendungen ist die Klasse \class{Trafo} vorzuziehen.

Es gibt Funktionen, die das Zusammensetzen einer Transformation
aus speziellen affinen Transformationen erlauben. Dabei wird die zu der
hinzuzufügenden Transformation gehörende Matrix jeweils von links mit der
bereits vorhandenen Transformation multipliziert.
\proc{int}{InitTrans}{double tr[3][3]}
\descr{Initialisieren einer Transformationsmatrix (Einheitsmatrix)}
\proc{int}{ShiftTrans}{double x0, double y0, double tr[3][3]}
\descr{Hinzufügen einer Translation mit $(x0,y0)$}
\proc{int}{RotTrans}{double x0, double y0, double phi, double tr[3][3]}
\descr{Hinzufügen einer Rotation mit dem Winkel $phi$ (Bogenmaß) um den 
Punkt $(x0,y0)$} 
\proc{int}{ScaleTrans}{double x0, double y0, double a, double b, double
tr[3][3]}
\descr{Hinzufügen einer anisotropen Skalierung mit dem Faktor $a$ in
x-Richtung und dem Faktor $b$ in y-Richtung}
\proc{int}{InvertTrans}{double tr[3][3]}
\descr{Die Transformation $tr$ wird invertiert.}
\proc{int}{FitAffineTrans}{PointList pl1,PointList pl2,double tr[3][3]}
\descr{Es wird mittels Ausgleichsrechnung eine affine Transformation
bestimmt, die die Punkte der Liste $pl1$ auf die Punkte der Liste $pl2$
transformiert. Punkte mit gleichem Index werden jeweils als Korrespondenzpaar
interpretiert. Die Punktepaare werden mit dem Gewicht aus der Liste $pl2$
gewichtet.
}
\seealso{MatchPointlists}
\seealsonext{AffinFitMoments}
\proc{int}{FitProjectiveTrans}{PointList pl1,PointList pl2,double tr[3][3]}
\descr{Es wird mittels Ausgleichsrechnung eine projektive Transformation
bestimmt, die die Punkte der Liste $pl1$ auf die Punkte der Liste $pl2$
transformiert. Punkte mit gleichem Index werden jeweils als Korrespondenzpaar
interpretiert. Die Punktepaare werden mit dem Gewicht aus der Liste $pl2$
gewichtet. 
}
\proc{double*}{TransPoint}{double p1[2],double tr[3][3],double p2[2]}
\descr{Der Punkt $p1$ wird mit der Transformation $tr$ in den Punkt $p2$
transformiert.
}
\proc{int}{TransImg}{Image imgs,double tr[3][3],int mode,Image imgd}
\descr{Das Bild $imgs$ wird mit der Transformation $tr$ in das Bild $imgd$
transformiert. Da man bei der Transformation der Bildpunkte im allgemeinen
keine ganzzahligen Koordinaten erhält, werden die Grauwerte für
$mode$=INTERPOL bilinear interpoliert, bei $mode$=DEFAULT werden die
Koordinaten gerundet.
}
\seealso{Transform}
\proc{Contur}{TransContur}{Contur c,double tr[3][3]}
\descr{
Es wird eine Kontur angelegt, die sich aus der Transformation der Kontur $c$
mit $tr$ ergibt.
}

\subsection{Statistische Maßzahlen mehrdimensionaler Zufallsvariablen}

Im Normalfall sollten die neuen Statistikfunktionen zur Klasse
\class{Statistics} verwendet werden.

Diese Funktionen zur Berechnung von statistischen Maßzahlen mehrdimensionaler
Zufallsvariablen (Mittelwertvektor, Kovarianz- und Korrelationsmatrix)
arbeiten mit einer Datenstruktur, auf die über einen Pointer vom Typ
$Statistic$ zugegriffen wird.

\proc{Statistic}{InitStatistic}{int dim}
\descr{Initialisierung eines Statistikbereiches für Zufallsvektoren der
Dimension $dim$
}
\proc{int}{PutStatistic}{Statistic st,double v[dim],double weight}
\descr{Gewichteter Eintrag eines Zufallsvektors in den Statistikbereich. Die
Dimension muß mit der bei der Initialisierung angegebenen Dimension
Übereinstimmen.
}
\proc{int}{GetStatisticDim}{Statistic st,int *dim}
\descr{Dimension $dim$ des Statistikbereiches abfragen.}
\proc{int}{GetStatisticWeight}{Statistic st,double *sweight}
\descr{Es wird die Summe der Gewichte bestimmt.}
\proc{int}{GetStatisticMean}{Statistic st,double *mean}
\descr{Es wird der Mittelwertvektor bestimmt.}
\proc{MatrixStruct}{GetStatisticCov}{Statistic st,MatrixStruct cov}
\descr{
Für die in $st$ akkumulierten Realisierungen wird die Kovarianzmatrix
$cov$ bestimmt. $cov$ muß eine Gleitkomma-Matrix sein, ihre Dimension muß mit
derjenigen der Zufallsvektoren übereinstimmen. Wenn für $cov$ der NULL-Pointer
übergeben wird, wird die Kovarianzmatrix intern angelegt. Rückgabewert ist die
Kovarianzmatrix oder bei Fehlern der NULL-Pointer
}
\proc{MatrixStruct}{GetStatisticCor}{Statistic st,MatrixStruct cor}
\descr{
Für die in $st$ akkumulierten Realisierungen wird die Korrelationsmatrix
$cor$ bestimmt. $cor$ muß eine Gleitkomma-Matrix sein, ihre Dimension muß mit
derjenigen der Zufallsvektoren übereinstimmen. Wenn für $cov$ der NULL-Pointer
übergeben wird, wird die Korrelationsmatrix intern angelegt. Rückgabewert ist die
Korrelationsmatrix oder bei Fehlern der NULL-Pointer
}
\proc{int}{WriteStatistic}{Statistic st,char *file}
\descr{Die Daten des Statistikbereiches werden für eine spätere
Weiterbearbeitung in ein Textfile $file$ geschrieben.}
\proc{Statistic}{ReadStatistic}{char *file}
\descr{Ein mit WriteStatistic() geschriebener Statistikbereich wird vom File
$file$ eingelesen.}

\subsection{Momente als Feld vom Typ double}

Einige veraltete Funktionen verwenden Felder vom Typ $double$ um Momente
abzulegen. Diese sollten nicht mehr verwendet werden. Ersatz dafür ist 
die Klasse \class{Moments}.

Momente werden in Feldern vom Typ $double$ in der Reihenfolge $m_{00}$,
$m_{10}$, $m_{01}$, $m_{20}$, $m_{11}$, $m_{02}$, $m_{30}$, $m_{21}$,
$m_{12}$, $m_{03}$, $m_{40}$, $m_{31}$, $m_{22}$, $m_{13}$, $m_{04}$
abgelegt. Bei der Berechnung der Momente erfolgt keine Normierung auf den
Schwerpunkt.

\proc{int}{MomentPolygon}{PointList p,double m[15],double s[2]}
\procf{int}{MomentPolygon}{const Matrix \&p,double m[15],double s[2]}
\descr{Es werden die Flächenmomente des durch die Punktliste $pl$
beschriebenen Polygons berechnet. Auf $s$ wird der Flächenschwerpunkt
bereitgestellt. Die Vorzeichen der Momente sind von der Orientierung
(Umlaufsinn) des Polygons abhängig.}
\proc{int}{PointListMoment}{PointList pl,int a1,int a2,double m[15],double s[2]}
\descr{Es werden die Flächenmomente eines Teilpolygons berechnet, dessen
Eckpunkte in der Punktliste $pl$ von Adresse $a1$ bis Adresse $a2$ stehen. Auf
$s$ wird der Flächenschwerpunkt bereitgestellt.}
\proc{int}{MomentRegion}{Contur c, double m[15], double s[2]}
\descr{Es werden die Flächenmomente der durch die Kontur $c$ umschlossenen
Fläche berechnet. Auf $s$ wird der Flächenschwerpunkt bereitgestellt.
Das Moment $m_{00}$ ist hier immer nichtnegativ.}

\subsubsection{Transformation von Momenten}

\proc{int}{TranslateMoments}{const double m1[15],double x,double y,double m2[15]}
\descr{
Transformation der Momente $m1$ entsprechend einer Translation mit $(x,y)$.}
\proc{int}{AffinTransMoments}{const double m1[15],double tr[3][3],double m2[15]}
\descr{Affine Transformation der Momente $m1$ entsprechend der
Transformationsmatrix $tr$.}
\proc{int}{XShearMoments}{const double m[15],double a,double ms[15]}
\descr{Transformation der Momente durch X-Scherung.}
\proc{int}{YShearMoments}{const double m1[15],double b,double m2[15]}
\descr{Transformation der Momente durch Y-Scherung.}
\proc{int}{ScaleMoments}{const double m1[15],double a,double b, double m2[15]}
\descr{Transformation der Momente durch anisotrope Skalierung.}
\proc{int}{RotateMoments}{const double m1[15],double c,double s, double m2[15]}
\descr{Transformation der Momente durch Rotation. $c$ und $s$ geben den
Cosinus bzw. den Sinus des Rotationswinkels an.}

\proc{int}{CalcCentralMoments}{const double m[15],double mc[15]}
\descr{Berechnet aus den Momenten in $m$ die zentralen Momente $mc$.}

\proc{int}{NormalizeMoments}{const double m1[15],double m2[15]}
\descr{Normierung der Momente $m1$ auf vergleichbare Größen, 
indem die n-ten Momente
durch $m_{00}^{(1+{n \over 2})}$ dividiert werden. Die normierten Momente
werden auf $m2$ bereitgestellt.}

\subsubsection{Algebraische Momentinvarianten}

\proc{int}{AffinAlgebraicInvar}{double m[15],double flu[4]} 
\descr{Aus den Momenten $m$ werden algebraische Invarianten nach Flusser
  berechnet und auf $flu$ bereitgestellt.}  
\proc{int}{AffinHuInvar}{double m[15],double hu[7]}
\descr{Nach einer Scherungsnormierung werden aus den Momenten $m$ 7
  Invarianten nach Hu berechnet und auf $hu$ bereitgestellt, die affin
  invariant sind.}

\subsubsection{Affine Standardlagen}

Durch die Normierung bestimmter Momente können affine Transformationen
bestimmt werden, die das zugehörige Objekt in eine Standardlage überführen.
\proc{int}{AffinNormMoments}{double m[21],double maf[21],double atr[3][3]}
\descr{ Berechnung einer Standardlage durch Normierung von Momenten
  auf $(m_{20}=1; m_{11}=0; m_{02}=1; m_{30}+m_{12}=0)$. Die Momente in der
  Standardlage werden auf $maf$ bereitgestellt, die affine Transformation zur
  Überführung in die Standardlage auf $atr$.}
\proc{int}{PolyNormMoments}{double m[21],double maf[21],double atr[3][3]}
\descr{ Berechnung einer Standardlage mit der Polynommethode durch Normierung
  der Momente auf $(m_{30}=0; m_{11}=0; m_{20}=1; m_{02}=1)$. Die Momente in
  der Standardlage werden auf $maf$ bereitgestellt, die affine Transformation
  zur Überführung in die Standardlage auf $atr$.}
\proc{int}{AffinIterateMoments}{double m[21],double maf[21],double atr[3][3]}
\descr{Berechnung einer Standardlage mit der Iterationsmethode durch
  Normierung der Momente auf $(m_{31}=0; m_{13}=0; m_{20}=1; m_{02}=1)$. Die
  Momente in der Standardlage werden auf $maf$ bereitgestellt, die affine
  Transformation zur Überführung in die Standardlage auf $atr$.  }

\subsubsection{Fitting mit Momenten}

\proc{double}{AffinFitMoments}{double m1[15],double m2[15],double tr[3][3]}
\descr{ Es wird eine affine Transformation bestimmt, die approximativ das
  Objekt mit den Momenten $m1$ in das Objekt mit den Momenten $m2$ überführt.
  Rückgabewert ist ein Fehlermaß, das die Abweichung der mit $tr$
  transformierten Momente $m1$ von $m2$ beschreibt. Der Algorithmus ist sehr
  schnell, versagt aber bei Objekten, die ausgezeichnete Symmetrien aufweisen,
  z.B. ähnlich einem Dreieck, einem Rechteck oder Quadrat. Bei Objekten, bei
  denen man a priori solche Symmetrien ausschliessen kann, arbeitet der
  Algorithmus sicher und ist sehr robust.}
\seealso{MatchObject}
\proc{double}{AffinFitPolygons}{PointList pl1,PointList pl2,double tr[3][3]}
\descr{ Es wird eine affine Transformation bestimmt, die approximativ das
  Objekt (Polygon) $pl1$ in das Objekt $pl22$ überführt. Rückgabewert ist ein
  Fehlermaß, das die Abweichung der mit $tr$ transformierten Punktliste $pl1$
  von $pl2$ beschreibt. Der Algorithmus ist sehr
  schnell, versagt aber bei Objekten, die ausgezeichnete Symmetrien aufweisen,
  z.B. ähnlich einem Dreieck, einem Rechteck oder Quadrat. Bei Objekten, bei
  denen man a priori solche Symmetrien ausschliessen kann, arbeitet der
  Algorithmus sicher und ist sehr robust.}
\proc{int}{FitTriangleMoments}{double moment[15],double corner[3][2]}
\descr{
Es werden die drei Eckpunkte $corner$ eines Dreiecks bestimmt, das mit der 
Polynommethode an das Objekt mit den Flächenmomenten $moment$ angepaßt ist.
}
\proc{int}{FitRectangleMoments}{double moment[15],double corner[4][2]}
\descr{
Es werden die vier Eckpunkte $corner$ eines Rechteckes bestimmt, das mit einer 
Momenten-Normierungsmethode an das Objekt mit den Flächenmomenten $moment$ angepaßt ist.
}
\proc{int}{FitParallelogramMoments}{double moment[15],double corner[4][2]}
\descr{
Es werden die vier Eckpunkte $corner$ eines Parallelogrammes bestimmt, das mit einer 
Momenten-Normierungsmethode an das Objekt mit den Flächenmomenten $moment$ angepaßt ist.
}
\proc{int}{FitParallelogramMoments}{double moment[15],double corner[4][2], double \& guetemass}
\descr{
Es werden die vier Eckpunkte $corner$ eines Parallelogrammes bestimmt, das mit einer 
Momenten-Normierungsmethode an das Objekt mit den Flächenmomenten $moment$ angepaßt ist.
Die Variable $guetemass$ erhält den quadratischen Abstand zwischen den Referenzmomenten
und den normierten Momenten des Objektes.
}

\proc{int}{FitQuadrangleMoments}{double moment[15],double corner[4][2]}
\descr{
Es werden die vier Eckpunkte $corner$ eines beliebigen Vierecks bestimmt, das mit einer 
Momenten-Normierungsmethode an das Objekt mit den Flächenmomenten $moment$ angepaßt ist.
}

\proc{int}{FitCircleMoments}{double moment[15],double \&x0,double \&y0,double \&radius}
\descr{
Es werden der Mittelpunkt $x0,y0$ und der Radius $radius$ eines Kreises bestimmt, der mit einer 
Momenten-Normierungsmethode an das Objekt mit den Flächenmomenten $moment$ angepaßt ist.
}
\proc{int}{FitCircularSegmentMoments}{double moment[15],double
circle\_par[3],double line\_start[2],double line\_end[2]}
\descr{
Eingabe sind die 15 Flächenmomente $moment$ des Objektes. Es wird
an das Objekt ein Kreissegment angepaßt. Dies ist kein Bogensegment eines
Kreises, sondern das gesamte Flächen-Objekt wird aufgefaßt als ein Kreis, von
dem ein Stück abgeschnitten wurde. Das Keissegment wird daher beschrieben
durch die eigentlichen Kreisparameter $circle\_par[3]$ in der Reihenfolge
Mittelpunkt und Radius (\see{DrawCircle}) und durch eine Strecke, die den Kreis
beschneidet. Dabei beschreibt $line\_start[2]$ den Anfangspunkt der Strecke und
$line\_end[2]$ den Endpunkt der Strecke. Als Algorithmus wird eine
Momenten-Normierungs-Methode verwendet.
}
\proc{int}{FitEllipseMoments}{double moment[15],double ell\_par[5]}
\descr{
Eingabe sind die 15 Flächenmomente $moment$ des Objektes. Es wird
an das Objekt eine Ellipse angepaßt. Diese Ellipse ist die bekannte
Trägheitsellipse eines Objektes, wobei die Größe der Trägkeitsellipse noch auf
die Größe des Objektes normiert wurde. Die 5 Ergebnisparameter in $ell\_par[5]$
der Ellipse sind genauso aufgebaut wie in allen anderen Funktionen, siehe (\see{DrawEllipse}).
}

\proc{int}{FitEllipticalSegmentMoments}{double moment[15],double
ell\_par[5],double line\_start[2],double line\_end[2]}
\descr{
Eingabe sind die 15 Flächenmomente $moment$ des Objektes. Es wird
an das Objekt ein Ellipsensegment angepaßt. Dies ist kein Bogensegment einer
Ellipse, sondern das gesamte Flächen-Objekt wird aufgefaßt als eine Ellipse, von
der ein Stück abgeschnitten wurde. Das Ellipsensegment wird daher beschrieben
durch die eigentlichen 5 Ellipsenparameter $ell\_par[5]$
(\see{DrawEllipse})  und durch eine Strecke, die die Ellipse
beschneidet. Dabei beschreibt $line\_start[2]$ den Anfangspunkt der Strecke und
$line\_end[2]$ den Endpunkt der Strecke. Als Algorithmus wird eine
Momenten-Normierungs-Methode verwendet.
}
\proc{int}{FitSuperEllipseMoments}{double moment[15],double
\&c1,double \&f1,double tr1[3][3],double \&c2,double \&f2,double tr2[3][3]}
\descr{
Eingabe sind die 15 Flächenmomente $moment$ des Objektes. Es wird
an das Objekt eine Super-Ellipse angepaßt. Da für bestimmte Situationen zwei
gute Lösungen existieren, werden diese alle beide berechnet. Der
Approximationswert ist $f1$ bzw. $f2$, es sollte die Lösung genommen werden,
für die dieser Wert kleiner ist. Die eigentlichen Super-Ellipse Parameter sind
dann der Exponent $c1$ bzw. $c2$ und eine affine Transformation $tr1[3][3]$
bzw. $tr2[3][3]$. Zur grafischen Darstellung der gefitteten Super-Ellipse kann
die Funktion DrawSuperEllipse (\see{DrawSuperEllipse}) benutzt werden. Als 
Algorithmus wird eine Momenten-Normierungs-Methode verwendet.
}

%%% Local Variables: 
%%% mode: latex
%%% TeX-master: icedoc_base.tex
%%% TeX-master: "icedoc_base"
%%% End: 
