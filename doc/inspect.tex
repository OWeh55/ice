\lsection{Inspektion von Szenenbestandteilen}

Für eine visuell unterstützte Greifplanung ist eine Untersuchung der
Zugänglichkeit von einzelnen Modellbestandteilen, insbesondere Flächen, für
den Greifer erforderlich. Dazu wird die Kamera so positioniert, daß das zu
untersuchende Modellbestandteil im Bild sichtbar sein müßte. Durch
Kantendetektion und Segmentierung wird eine Bildbeschreibung erzeugt. Ebenso
kann durch Abbildung der zu untersuchenden Objekte in die Bildebene unter
berücksichtigung verdeckter Kanten eine weitere Bildbeschreibung erzeugt
werden. Zu dieser ``Soll''-Bildbeschreibung werden aus der
``Ist''-Beschreibung alle die Kanten herausgesucht, die einen vorzugebenden
Maximalabstand zu der jeweiligen ``Soll''-Kante nicht überschreiten. Eine
Bildbeschreibung mit diesen Kanten wird als Ergebnis der Inspektion
zurückgegeben.

Vor der Inspektion von Szenenbestandteilen muß einmal bzw. erneut nach jeder
Positionsänderung der Kamera die Funktion ChangeView() aufgerufen werden, um
blickrichtungsabhängige Merkmale in der Szenenbeschreibung neu zu berechnen.

\proc{double}{DistanceLineseg}{double l0[2],double l1[2],double ls0[2],double
ls1[2]}
\descr{
Es wird ein Abstandsmaß zwischen den Geradensegmenten $l0,l1$ und $ls0,ls1$
bestimmt und zurückgegeben.
}
\proc{Edge}{SearchMatchingEdge}{Edge edgm, Description D, double *dmin}
\descr{
Aus der Bildbeschreibung $D$ wird die Kante mit dem geringsten Abstand zur
Kante $edgm$ gesucht und zurückgegeben. Wenn keine Kante gefunden wird, deren
Abstand kleiner als $dmin$ ist, wird der Nullpointer zurückgegeben. Der
Abstand einer gefundenen Kante wird auf $dmin$ bereitgestellt.
}
\proc{int}{MapPointToMesh}{double p[2],camera cam,Mesh msh,double pw[3]}
\descr{
Der Bildpunkt $p$ wird auf die Ebene projiziert, die durch die Masche $msh$
gegeben ist. Die Position von $msh$ wird durch das aktuelle Lageframe des
zugehörigen Modells bestimmt. Der projizierte Raumpunkt wird auf $pw$
zurückgegeben. 
}
\proc{Description}{InspectModel}{Model mod,camera cam,ModelList S,Description D}
\descr{
Aus der Bildbeschreibung $D$ werden alle Kanten herausgesucht, die sich bei der
durch die Kamera $cam$ gegebenen Blickrichtung und der aktuellen
Modellposition einer Modellkante des Modells $mod$ zuordnen lassen. 
}
\proc{Description}{InspectMesh}{Mesh msh,camera cam,ModelList S,Description D}
\descr{
Aus der Bildbeschreibung $D$ werden alle Kanten herausgesucht, die sich bei der
durch die Kamera $cam$ gegebenen Blickrichtung und der aktuellen
Modellposition einer Modellkante der Masche $msh$ zuordnen lassen. 
}
\proc{Description}{InspectEdge}{Edge medg,camera cam,ModelList S,Description D}
\descr{
Aus der Bildbeschreibung $D$ werden alle Kanten herausgesucht, die sich bei der
durch die Kamera $cam$ gegebenen Blickrichtung und der aktuellen
Modellposition der Modellkante $medg$ zuordnen lassen. 
}
